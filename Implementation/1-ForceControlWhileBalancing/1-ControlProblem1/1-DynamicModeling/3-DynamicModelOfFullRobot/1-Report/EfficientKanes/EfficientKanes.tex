\documentclass[a4paper,10pt]{article}
\usepackage[utf8]{inputenc}
\usepackage{graphicx}
\usepackage[thinlines]{easytable}
\usepackage{enumitem}
\usepackage{amsmath}
\usepackage{amsfonts}
\usepackage{graphicx}
\usepackage{bbm}
\usepackage{textcomp}
\usepackage{relsize}
\usepackage{listings}
\usepackage[usenames,dvipsnames]{color}
\usepackage{mathtools}
\usepackage{multicol}
\setcounter{MaxMatrixCols}{20}
\allowdisplaybreaks
%%%%%%%%%%%%%%%%%%%%%%%%%%%%%%%%%%%%%%%%%%%%%%%%%%%%%%%%%%%%%%%%%%%%%%%%%%%%%%%%%%%%%%%%%%%
% MATLAB code listing
%
% This is the color used for MATLAB comments below
\definecolor{MyDarkGreen}{rgb}{0.0,0.4,0.0}
 
% For faster processing, load Matlab syntax for listings
\lstloadlanguages{Matlab}%
\lstset{language=Matlab, % Use MATLAB
frame=single, % Single frame around code
basicstyle=\tiny\ttfamily, % Use small true type font
keywordstyle=[1]\color{Blue}\bfseries, % MATLAB functions bold and blue
keywordstyle=[2]\color{Purple}, % MATLAB function arguments purple
keywordstyle=[3]\color{Blue}\underbar, % User functions underlined and blue
identifierstyle=, % Nothing special about identifiers
% Comments small dark green courier
commentstyle=\usefont{T1}{pcr}{m}{sl}\color{MyDarkGreen}\tiny,
stringstyle=\color{Purple}, % Strings are purple
showstringspaces=false, % Don't put marks in string spaces
tabsize=5, % 5 spaces per tab
%
%%% Put standard MATLAB functions not included in the default
%%% language here
morekeywords={xlim,ylim,var,alpha,factorial,poissrnd,normpdf,normcdf},
%
%%% Put MATLAB function parameters here
morekeywords=[2]{on, off, interp},
%
%%% Put user defined functions here
morekeywords=[3]{FindESS, homework_example},
%
morecomment=[l][\color{Blue}]{...}, % Line continuation (...) like blue comment
numbers=left, % Line numbers on left
firstnumber=1, % Line numbers start with line 1
numberstyle=\tiny\color{Blue}, % Line numbers are blue
stepnumber=5 % Line numbers go in steps of 5
}
%%%%%%%%%%%%%%%%%%%%%%%%%%%%%%%%%%%%%%%%%%%%%%%%%%%%%%%%%%%%%%%5


\newcommand\scalemath[2]{\scalebox{#1}{\mbox{\ensuremath{\displaystyle #2}}}}
% \scalemath{1.0}


%opening
\title{Efficient Computation of the Dynamic Model for Krang}
\author{Munzir Zafar}

\begin{document}

\maketitle

In our earlier report we derived the model of the robot, Golem Krang, using Kane's formulation. Since the purpose of deriving the model
is either simulation or control, both of which require us to calculate the dynamics of the robot as fast as possible, we are stuck with a problem.
Due to large number of degrees of freedom, the final expression for the dynamics of the system is extremely long. Also, it contains a repetition
of same combination of terms. Thus the computation using the final form that will come out as a result of applying the results of the report
mentioned earlier, are going to be inefficient. Even if the final expression is not directly used but the resursive formulation is used to 
calculate the terms still suffers with the same problem. For this reason, it will be nice if we can reduce the amount of computation time
that is required to compute the dynamics of the system in run-time.

We use the ideas presented in \cite{kane1983use}. The basic idea is simple. As we derive the kinematics of the system recursively, we keep substituting the resulting expressions,
that are combinations of more than one variables, with one new variable. And we derive the kinematics of the succeeding links (in the serial structure) in terms 
of the substitute variables. And then we repeat this substitution process at each step. This allows us to express the final expression in simpler closed forms,
and alse if the substitutions are encoded, a potential recurring combination of variables will be evaluated/computed just once within an iteration thus
resulting the overall computation time significantly.




\tiny

\begin{align}
 \bar\omega_{1} &= {}^{1}A_{0} \bar\omega_{0} + \dot{q}_{1} \bar{e}_{1} 
 \nonumber \\ 
 \bar\omega_{1} &= \left[\begin{matrix} -\dot{q}_{imu} & \dot{\psi}c_{imu} & \dot{\psi}s_{imu} &  \end{matrix}\right] 
 \nonumber \\ 
K_{1} &= \dot{\psi}c_{imu} \nonumber \\
K_{2} &= \dot{\psi}s_{imu} \nonumber \\
 \bar\omega_{1} &= \left[\begin{matrix} -\dot{q}_{imu} & K_{1} & K_{2} &  \end{matrix}\right] 
 \nonumber \\ 
 \bar\omega_{1} &= \left[\begin{matrix} -\dot{q}_{imu} & \dot{\psi}c_{imu} & \dot{\psi}s_{imu} &  \end{matrix}\right] 
 \nonumber \\ 
 \bar\omega_{1} &= \left[\begin{matrix} -\dot{q}_{imu} & \dot{\psi}c_{imu} & \dot{\psi}s_{imu} &  \end{matrix}\right] 
 \nonumber \\ 
 \bar{v}_{1} &= {}^{1}A_{0} \left(\bar{v}_{0} + \bar\omega_{0} \times \bar{P}_{1}\right) 
 \nonumber \\ 
 \bar{v}_{1} &= \left[\begin{matrix} 0 & \dot{x}s_{imu} & -\dot{x}c_{imu} &  \end{matrix}\right] 
 \nonumber \\ 
K_{3} &= \dot{x}s_{imu} \nonumber \\
K_{4} &= -\dot{x}c_{imu} \nonumber \\
 \bar{v}_{1} &= \left[\begin{matrix} 0 & K_{3} & K_{4} &  \end{matrix}\right] 
 \nonumber \\ 
 \bar{v}_{1} &= \left[\begin{matrix} 0 & \dot{x}s_{imu} & -\dot{x}c_{imu} &  \end{matrix}\right] 
 \nonumber \\ 
 \bar{v}_{1} &= \left[\begin{matrix} 0 & \dot{x}s_{imu} & -\dot{x}c_{imu} &  \end{matrix}\right] 
 \nonumber \\ 
 \bar\alpha_{1} &= {}^{1}A_{0} \bar\alpha_{0} + \ddot{q}_{1} \bar{e}_{1} + \dot{q}_{1} \left(\bar\omega_{1} \times \bar{e}_{1}\right) 
 \nonumber \\ 
 \bar\alpha_{1} &= \left[\begin{matrix} -\ddot{q}_{imu} & \ddot{\psi}c_{imu} - K_{2}\dot{q}_{imu} & K_{1}\dot{q}_{imu} + \ddot{\psi}s_{imu} &  \end{matrix}\right] 
 \nonumber \\ 
K_{5} &= -K_{2}\dot{q}_{imu} \nonumber \\
K_{6} &= K_{1}\dot{q}_{imu} \nonumber \\
 \bar\alpha_{1} &= \left[\begin{matrix} -\ddot{q}_{imu} & K_{5} + \ddot{\psi}c_{imu} & K_{6} + \ddot{\psi}s_{imu} &  \end{matrix}\right] 
 \nonumber \\ 
 \bar{a}_{1} &= {}^{1}A_{0} \left(\bar{a}_{0} + \bar\alpha_{0} \times \bar{P}_{1} + \bar\omega_{0} \times \left(\bar\omega_{0} \times \bar{P}_{1}\right)\right) 
 \nonumber \\ 
 \bar\alpha_{1} &= \left[\begin{matrix} -\dot{\psi} & \ddot{x}s_{imu} & -\ddot{x}c_{imu} &  \end{matrix}\right] 
 \nonumber \\ 
 \bar{a}_{1} &= \left[\begin{matrix} -\dot{\psi} & \ddot{x}s_{imu} & -\ddot{x}c_{imu} &  \end{matrix}\right] 
 \nonumber \\ 
 \bar{g}_{1} &= {}^{1}A_{0} \bar{g}_{0} 
 \nonumber \\ 
 \bar{g}_{1} &= \left[\begin{matrix} 0 & -gc_{imu} & -gs_{imu} &  \end{matrix}\right] 
 \nonumber \\ 
 \bar{g}_{1} &= \left[\begin{matrix} 0 & -gc_{imu} & -gs_{imu} &  \end{matrix}\right] 
 \nonumber \\ 
 m_{1}\bar{S}_{1}^{\times}\bar{g}_{1} &= \mathbf{MS}_{1} \times \bar{g}_{1} 
 \nonumber \\ 
 m_{1}\bar{S}_{1}^{\times}\bar{g}_{1} &= \left[\begin{matrix} \mathbf{MZ}_1gc_{imu} - \mathbf{MY}_1gs_{imu} & \mathbf{MX}_1gs_{imu} & -\mathbf{MX}_1gc_{imu} &  \end{matrix}\right] 
 \nonumber \\ 
D_{1} &= \mathbf{MZ}_1c_{imu} - \mathbf{MY}_1s_{imu} \nonumber \\
D_{2} &= \mathbf{MX}_1s_{imu} \nonumber \\
D_{3} &= -\mathbf{MX}_1c_{imu} \nonumber \\
 m_{1}\bar{S}_{1}^{\times}\bar{g}_{1} &= \left[\begin{matrix} D_{1}g & D_{2}g & D_{3}g &  \end{matrix}\right] 
 \nonumber \\ 
 m_{1}\bar{a}_{G(1)} &= m_{1}\bar{a}_{1} + \bar\alpha_{1} \times \mathbf{MS}_{1} + \bar\omega_{1} \times \left(\bar\omega_{1} \times \mathbf{MS}_{1}\right) 
 \nonumber \\ 
 m_{1}\bar{a}_{G(1)} &= \left[\begin{matrix} \mathbf{MZ}_1(K_{5} + \ddot{\psi}c_{imu}) - K_{2}(\mathbf{MZ}_1\dot{q}_{imu} + K_{2}\mathbf{MX}_1) - \dot{\psi}m_1 - K_{1}(\mathbf{MY}_1\dot{q}_{imu} + K_{1}\mathbf{MX}_1) - \mathbf{MY}_1(K_{6} + \ddot{\psi}s_{imu}) & \mathbf{MZ}_1\ddot{q}_{imu} - \dot{q}_{imu}(\mathbf{MY}_1\dot{q}_{imu} + K_{1}\mathbf{MX}_1) - K_{2}(K_{2}\mathbf{MY}_1 - K_{1}\mathbf{MZ}_1) + \mathbf{MX}_1(K_{6} + \ddot{\psi}s_{imu}) + \ddot{x}m_1s_{imu} & K_{1}(K_{2}\mathbf{MY}_1 - K_{1}\mathbf{MZ}_1) - \dot{q}_{imu}(\mathbf{MZ}_1\dot{q}_{imu} + K_{2}\mathbf{MX}_1) - \mathbf{MX}_1(K_{5} + \ddot{\psi}c_{imu}) - \mathbf{MY}_1\ddot{q}_{imu} - \ddot{x}m_1c_{imu} &  \end{matrix}\right] 
 \nonumber \\ 
D_{4} &= \mathbf{MZ}_1c_{imu} - \mathbf{MY}_1s_{imu} \nonumber \\
D_{5} &= K_{5}\mathbf{MZ}_1 - K_{1}^2\mathbf{MX}_1 - K_{2}^2\mathbf{MX}_1  \nonumber \\
&- K_{6}\mathbf{MY}_1 - \dot{\psi}m_1 - K_{1}\mathbf{MY}_1\dot{q}_{imu}  \nonumber \\
&- K_{2}\mathbf{MZ}_1\dot{q}_{imu} \nonumber \\
D_{6} &= m_1s_{imu} \nonumber \\
D_{7} &= \mathbf{MX}_1s_{imu} \nonumber \\
D_{8} &= K_{6}\mathbf{MX}_1 - \mathbf{MY}_1\dot{q}_{imu}^2 - K_{2}^2\mathbf{MY}_1  \nonumber \\
&+ K_{1}K_{2}\mathbf{MZ}_1 - K_{1}\mathbf{MX}_1\dot{q}_{imu} \nonumber \\
D_{9} &= -m_1c_{imu} \nonumber \\
D_{10} &= -\mathbf{MX}_1c_{imu} \nonumber \\
D_{11} &= K_{1}K_{2}\mathbf{MY}_1 - \mathbf{MZ}_1\dot{q}_{imu}^2 - K_{5}\mathbf{MX}_1  \nonumber \\
&- K_{1}^2\mathbf{MZ}_1 - K_{2}\mathbf{MX}_1\dot{q}_{imu} \nonumber \\
 m_{1}\bar{a}_{G(1)} &= \left[\begin{matrix} D_{5} + D_{4}\ddot{\psi} & D_{8} + D_{7}\ddot{\psi} + D_{6}\ddot{x} + \mathbf{MZ}_1\ddot{q}_{imu} & D_{11} + D_{10}\ddot{\psi} + D_{9}\ddot{x} - \mathbf{MY}_1\ddot{q}_{imu} &  \end{matrix}\right] 
 \nonumber \\ 
 \dot{\bar{H}}_{1} &= \mathbf{MS}_{1} \times \bar{a}_{1} + J_{1}\bar{\alpha}_{1} + \bar\omega_{1} \times J_{1}\bar{\omega}_{1} 
 \nonumber \\ 
 \dot{\bar{H}}_{1} &= \left[\begin{matrix} \mathbf{XZ}_1(K_{6} + \ddot{\psi}s_{imu}) - \mathbf{XX}_1\ddot{q}_{imu} - K_{2}(K_{1}\mathbf{YY}_1 + K_{2}\mathbf{YZ}_1 - \mathbf{XY}_1\dot{q}_{imu}) + K_{1}(K_{1}\mathbf{YZ}_1 + K_{2}\mathbf{ZZ}_1 - \mathbf{XZ}_1\dot{q}_{imu}) + \mathbf{XY}_1(K_{5} + \ddot{\psi}c_{imu}) - \mathbf{MY}_1\ddot{x}c_{imu} - \mathbf{MZ}_1\ddot{x}s_{imu} & \mathbf{YZ}_1(K_{6} + \ddot{\psi}s_{imu}) - \mathbf{MZ}_1\dot{\psi} - \mathbf{XY}_1\ddot{q}_{imu} + K_{2}(K_{1}\mathbf{XY}_1 + K_{2}\mathbf{XZ}_1 - \mathbf{XX}_1\dot{q}_{imu}) + \dot{q}_{imu}(K_{1}\mathbf{YZ}_1 + K_{2}\mathbf{ZZ}_1 - \mathbf{XZ}_1\dot{q}_{imu}) + \mathbf{YY}_1(K_{5} + \ddot{\psi}c_{imu}) + \mathbf{MX}_1\ddot{x}c_{imu} & \mathbf{ZZ}_1(K_{6} + \ddot{\psi}s_{imu}) + \mathbf{MY}_1\dot{\psi} - \mathbf{XZ}_1\ddot{q}_{imu} - K_{1}(K_{1}\mathbf{XY}_1 + K_{2}\mathbf{XZ}_1 - \mathbf{XX}_1\dot{q}_{imu}) - \dot{q}_{imu}(K_{1}\mathbf{YY}_1 + K_{2}\mathbf{YZ}_1 - \mathbf{XY}_1\dot{q}_{imu}) + \mathbf{YZ}_1(K_{5} + \ddot{\psi}c_{imu}) + \mathbf{MX}_1\ddot{x}s_{imu} &  \end{matrix}\right] 
 \nonumber \\ 
D_{12} &= - \mathbf{MY}_1c_{imu} - \mathbf{MZ}_1s_{imu} \nonumber \\
D_{13} &= \mathbf{XY}_1c_{imu} + \mathbf{XZ}_1s_{imu} \nonumber \\
D_{14} &= K_{5}\mathbf{XY}_1 + K_{6}\mathbf{XZ}_1 + K_{1}^2\mathbf{YZ}_1  \nonumber \\
&- K_{2}^2\mathbf{YZ}_1 - K_{1}K_{2}\mathbf{YY}_1 + K_{1}K_{2}\mathbf{ZZ}_1  \nonumber \\
&+ K_{2}\mathbf{XY}_1\dot{q}_{imu} - K_{1}\mathbf{XZ}_1\dot{q}_{imu} \nonumber \\
D_{15} &= \mathbf{MX}_1c_{imu} \nonumber \\
D_{16} &= \mathbf{YY}_1c_{imu} + \mathbf{YZ}_1s_{imu} \nonumber \\
D_{17} &= K_{5}\mathbf{YY}_1 + K_{6}\mathbf{YZ}_1 - \mathbf{MZ}_1\dot{\psi}  \nonumber \\
&+ K_{2}^2\mathbf{XZ}_1 - \mathbf{XZ}_1\dot{q}_{imu}^2 + K_{1}K_{2}\mathbf{XY}_1  \nonumber \\
&- K_{2}\mathbf{XX}_1\dot{q}_{imu} + K_{1}\mathbf{YZ}_1\dot{q}_{imu}  \nonumber \\
&+ K_{2}\mathbf{ZZ}_1\dot{q}_{imu} \nonumber \\
D_{18} &= \mathbf{MX}_1s_{imu} \nonumber \\
D_{19} &= \mathbf{YZ}_1c_{imu} + \mathbf{ZZ}_1s_{imu} \nonumber \\
D_{20} &= K_{5}\mathbf{YZ}_1 + K_{6}\mathbf{ZZ}_1 + \mathbf{MY}_1\dot{\psi}  \nonumber \\
&- K_{1}^2\mathbf{XY}_1 + \mathbf{XY}_1\dot{q}_{imu}^2 - K_{1}K_{2}\mathbf{XZ}_1  \nonumber \\
&+ K_{1}\mathbf{XX}_1\dot{q}_{imu} - K_{1}\mathbf{YY}_1\dot{q}_{imu}  \nonumber \\
&- K_{2}\mathbf{YZ}_1\dot{q}_{imu} \nonumber \\
 \dot{\bar{H}}_{1} &= \left[\begin{matrix} D_{5} + D_{4}\ddot{\psi} & D_{8} + D_{7}\ddot{\psi} + D_{6}\ddot{x} + \mathbf{MZ}_1\ddot{q}_{imu} & D_{11} + D_{10}\ddot{\psi} + D_{9}\ddot{x} - \mathbf{MY}_1\ddot{q}_{imu} &  \end{matrix}\right] 
 \nonumber \\ 
 \bar\omega_{2} &= {}^{2}A_{1} \bar\omega_{1} + \dot{q}_{2} \bar{e}_{2} 
 \nonumber \\ 
 \bar\omega_{2} &= \left[\begin{matrix} - \dot{q}_{w} - \dot{q}_{imu} & K_{1}c_{w} - K_{2}s_{w} & K_{2}c_{w} + K_{1}s_{w} &  \end{matrix}\right] 
 \nonumber \\ 
K_{7} &= - \dot{q}_{w} - \dot{q}_{imu} \nonumber \\
K_{8} &= K_{1}c_{w} - K_{2}s_{w} \nonumber \\
K_{9} &= K_{2}c_{w} + K_{1}s_{w} \nonumber \\
 \bar\omega_{2} &= \left[\begin{matrix} K_{7} & K_{8} & K_{9} &  \end{matrix}\right] 
 \nonumber \\ 
 \bar\omega_{2} &= \left[\begin{matrix} - \dot{q}_{w} - \dot{q}_{imu} & \dot{\psi}c_{imu}c_{w} - \dot{\psi}s_{imu}s_{w} & \dot{\psi}c_{imu}s_{w} + \dot{\psi}c_{w}s_{imu} &  \end{matrix}\right] 
 \nonumber \\ 
K_{10} &= c_{imu}c_{w} - s_{imu}s_{w} \nonumber \\
K_{11} &= c_{imu}s_{w} + c_{w}s_{imu} \nonumber \\
 \bar\omega_{2} &= \left[\begin{matrix} - \dot{q}_{w} - \dot{q}_{imu} & K_{10}\dot{\psi} & K_{11}\dot{\psi} &  \end{matrix}\right] 
 \nonumber \\ 
 \bar{v}_{2} &= {}^{2}A_{1} \left(\bar{v}_{1} + \bar\omega_{1} \times \bar{P}_{2}\right) 
 \nonumber \\ 
 \bar{v}_{2} &= \left[\begin{matrix} - K_{1}L_2 - K_{2}L_1 & c_{w}(K_{3} - L_2\dot{q}_{imu}) - s_{w}(K_{4} - L_1\dot{q}_{imu}) & c_{w}(K_{4} - L_1\dot{q}_{imu}) + s_{w}(K_{3} - L_2\dot{q}_{imu}) &  \end{matrix}\right] 
 \nonumber \\ 
K_{12} &= - K_{1}L_2 - K_{2}L_1 \nonumber \\
K_{13} &= c_{w}(K_{3} - L_2\dot{q}_{imu}) - s_{w}(K_{4}  \nonumber \\
&- L_1\dot{q}_{imu}) \nonumber \\
K_{14} &= c_{w}(K_{4} - L_1\dot{q}_{imu}) + s_{w}(K_{3}  \nonumber \\
&- L_2\dot{q}_{imu}) \nonumber \\
 \bar{v}_{2} &= \left[\begin{matrix} K_{12} & K_{13} & K_{14} &  \end{matrix}\right] 
 \nonumber \\ 
 \bar{v}_{2} &= \left[\begin{matrix} - L_2\dot{\psi}c_{imu} - L_1\dot{\psi}s_{imu} & s_{w}(L_1\dot{q}_{imu} + \dot{x}c_{imu}) - c_{w}(L_2\dot{q}_{imu} - \dot{x}s_{imu}) & - c_{w}(L_1\dot{q}_{imu} + \dot{x}c_{imu}) - s_{w}(L_2\dot{q}_{imu} - \dot{x}s_{imu}) &  \end{matrix}\right] 
 \nonumber \\ 
K_{15} &= - L_2c_{imu} - L_1s_{imu} \nonumber \\
K_{16} &= L_1s_{w} - L_2c_{w} \nonumber \\
K_{17} &= s_{imu}s_{w} - c_{imu}c_{w} \nonumber \\
K_{18} &= - L_1c_{w} - L_2s_{w} \nonumber \\
 \bar{v}_{2} &= \left[\begin{matrix} K_{15}\dot{\psi} & K_{16}\dot{q}_{imu} + K_{11}\dot{x} & K_{18}\dot{q}_{imu} + K_{17}\dot{x} &  \end{matrix}\right] 
 \nonumber \\ 
 \bar\alpha_{2} &= {}^{2}A_{1} \bar\alpha_{1} + \ddot{q}_{2} \bar{e}_{2} + \dot{q}_{2} \left(\bar\omega_{2} \times \bar{e}_{2}\right) 
 \nonumber \\ 
 \bar\alpha_{2} &= \left[\begin{matrix} - \ddot{q}_{w} - \ddot{q}_{imu} & c_{w}(K_{5} + \ddot{\psi}c_{imu}) - K_{9}\dot{q}_{w} - s_{w}(K_{6} + \ddot{\psi}s_{imu}) & K_{8}\dot{q}_{w} + s_{w}(K_{5} + \ddot{\psi}c_{imu}) + c_{w}(K_{6} + \ddot{\psi}s_{imu}) &  \end{matrix}\right] 
 \nonumber \\ 
K_{19} &= K_{5}c_{w} - K_{9}\dot{q}_{w} - K_{6}s_{w} \nonumber \\
K_{20} &= K_{8}\dot{q}_{w} + K_{6}c_{w} + K_{5}s_{w} \nonumber \\
 \bar\alpha_{2} &= \left[\begin{matrix} - \ddot{q}_{w} - \ddot{q}_{imu} & K_{19} + K_{10}\ddot{\psi} & K_{20} + K_{11}\ddot{\psi} &  \end{matrix}\right] 
 \nonumber \\ 
 \bar{a}_{2} &= {}^{2}A_{1} \left(\bar{a}_{1} + \bar\alpha_{1} \times \bar{P}_{2} + \bar\omega_{1} \times \left(\bar\omega_{1} \times \bar{P}_{2}\right)\right) 
 \nonumber \\ 
 \bar\alpha_{2} &= \left[\begin{matrix} K_{2}L_2\dot{q}_{imu} - L_2(K_{5} + \ddot{\psi}c_{imu}) - L_1(K_{6} + \ddot{\psi}s_{imu}) - K_{1}L_1\dot{q}_{imu} - \dot{\psi} & s_{w}(L_1\ddot{q}_{imu} - L_2\dot{q}_{imu}^2 + \ddot{x}c_{imu} - K_{1}(K_{1}L_2 + K_{2}L_1)) - c_{w}(L_2\ddot{q}_{imu} + L_1\dot{q}_{imu}^2 - \ddot{x}s_{imu} + K_{2}(K_{1}L_2 + K_{2}L_1)) & - c_{w}(L_1\ddot{q}_{imu} - L_2\dot{q}_{imu}^2 + \ddot{x}c_{imu} - K_{1}(K_{1}L_2 + K_{2}L_1)) - s_{w}(L_2\ddot{q}_{imu} + L_1\dot{q}_{imu}^2 - \ddot{x}s_{imu} + K_{2}(K_{1}L_2 + K_{2}L_1)) &  \end{matrix}\right] 
 \nonumber \\ 
K_{21} &= K_{2}L_2\dot{q}_{imu} - K_{5}L_2 - K_{6}L_1  \nonumber \\
&- K_{1}L_1\dot{q}_{imu} - \dot{\psi} \nonumber \\
K_{22} &= - K_{2}^2L_1c_{w} - K_{1}^2L_2s_{w}  \nonumber \\
&- L_1\dot{q}_{imu}^2c_{w} - L_2\dot{q}_{imu}^2s_{w}  \nonumber \\
&- K_{1}K_{2}L_2c_{w} - K_{1}K_{2}L_1s_{w} \nonumber \\
K_{23} &= K_{1}^2L_2c_{w} - K_{2}^2L_1s_{w}  \nonumber \\
&+ L_2\dot{q}_{imu}^2c_{w} - L_1\dot{q}_{imu}^2s_{w}  \nonumber \\
&+ K_{1}K_{2}L_1c_{w} - K_{1}K_{2}L_2s_{w} \nonumber \\
 \bar{a}_{2} &= \left[\begin{matrix} K_{21} + K_{15}\ddot{\psi} & K_{22} + K_{16}\ddot{q}_{imu} + K_{11}\ddot{x} & K_{23} + K_{18}\ddot{q}_{imu} + K_{17}\ddot{x} &  \end{matrix}\right] 
 \nonumber \\ 
 \bar{g}_{2} &= {}^{2}A_{1} \bar{g}_{1} 
 \nonumber \\ 
 \bar{g}_{2} &= \left[\begin{matrix} 0 & gs_{imu}s_{w} - gc_{imu}c_{w} & - gc_{imu}s_{w} - gc_{w}s_{imu} &  \end{matrix}\right] 
 \nonumber \\ 
K_{24} &= - c_{imu}s_{w} - c_{w}s_{imu} \nonumber \\
 \bar{g}_{2} &= \left[\begin{matrix} 0 & K_{17}g & K_{24}g &  \end{matrix}\right] 
 \nonumber \\ 
 m_{2}\bar{S}_{2}^{\times}\bar{g}_{2} &= \mathbf{MS}_{2} \times \bar{g}_{2} 
 \nonumber \\ 
 m_{2}\bar{S}_{2}^{\times}\bar{g}_{2} &= \left[\begin{matrix} K_{24}\mathbf{MY}_2g - K_{17}\mathbf{MZ}_2g & -K_{24}\mathbf{MX}_2g & K_{17}\mathbf{MX}_2g &  \end{matrix}\right] 
 \nonumber \\ 
D_{21} &= K_{24}\mathbf{MY}_2 - K_{17}\mathbf{MZ}_2 \nonumber \\
D_{22} &= -K_{24}\mathbf{MX}_2 \nonumber \\
D_{23} &= K_{17}\mathbf{MX}_2 \nonumber \\
 m_{2}\bar{S}_{2}^{\times}\bar{g}_{2} &= \left[\begin{matrix} D_{21}g & D_{22}g & D_{23}g &  \end{matrix}\right] 
 \nonumber \\ 
 m_{2}\bar{a}_{G(2)} &= m_{2}\bar{a}_{2} + \bar\alpha_{2} \times \mathbf{MS}_{2} + \bar\omega_{2} \times \left(\bar\omega_{2} \times \mathbf{MS}_{2}\right) 
 \nonumber \\ 
 m_{2}\bar{a}_{G(2)} &= \left[\begin{matrix} \mathbf{MZ}_2(K_{19} + K_{10}\ddot{\psi}) - \mathbf{MY}_2(K_{20} + K_{11}\ddot{\psi}) + m_2(K_{21} + K_{15}\ddot{\psi}) - K_{8}(K_{8}\mathbf{MX}_2 - K_{7}\mathbf{MY}_2) - K_{9}(K_{9}\mathbf{MX}_2 - K_{7}\mathbf{MZ}_2) & \mathbf{MX}_2(K_{20} + K_{11}\ddot{\psi}) + m_2(K_{22} + K_{16}\ddot{q}_{imu} + K_{11}\ddot{x}) + \mathbf{MZ}_2(\ddot{q}_{w} + \ddot{q}_{imu}) + K_{7}(K_{8}\mathbf{MX}_2 - K_{7}\mathbf{MY}_2) - K_{9}(K_{9}\mathbf{MY}_2 - K_{8}\mathbf{MZ}_2) & m_2(K_{23} + K_{18}\ddot{q}_{imu} + K_{17}\ddot{x}) - \mathbf{MX}_2(K_{19} + K_{10}\ddot{\psi}) - \mathbf{MY}_2(\ddot{q}_{w} + \ddot{q}_{imu}) + K_{7}(K_{9}\mathbf{MX}_2 - K_{7}\mathbf{MZ}_2) + K_{8}(K_{9}\mathbf{MY}_2 - K_{8}\mathbf{MZ}_2) &  \end{matrix}\right] 
 \nonumber \\ 
D_{24} &= K_{15}m_2 - K_{11}\mathbf{MY}_2 + K_{10}\mathbf{MZ}_2 \nonumber \\
D_{25} &= K_{21}m_2 - K_{8}^2\mathbf{MX}_2 - K_{9}^2\mathbf{MX}_2  \nonumber \\
&- K_{20}\mathbf{MY}_2 + K_{19}\mathbf{MZ}_2 + K_{7}K_{8}\mathbf{MY}_2  \nonumber \\
&+ K_{7}K_{9}\mathbf{MZ}_2 \nonumber \\
D_{26} &= K_{11}m_2 \nonumber \\
D_{27} &= K_{11}\mathbf{MX}_2 \nonumber \\
D_{28} &= \mathbf{MZ}_2 + K_{16}m_2 \nonumber \\
D_{29} &= K_{22}m_2 - K_{7}^2\mathbf{MY}_2 - K_{9}^2\mathbf{MY}_2  \nonumber \\
&+ K_{20}\mathbf{MX}_2 + K_{7}K_{8}\mathbf{MX}_2 + K_{8}K_{9}\mathbf{MZ}_2 \nonumber \\
D_{30} &= K_{17}m_2 \nonumber \\
D_{31} &= -K_{10}\mathbf{MX}_2 \nonumber \\
D_{32} &= K_{18}m_2 - \mathbf{MY}_2 \nonumber \\
D_{33} &= K_{23}m_2 - K_{7}^2\mathbf{MZ}_2 - K_{8}^2\mathbf{MZ}_2  \nonumber \\
&- K_{19}\mathbf{MX}_2 + K_{7}K_{9}\mathbf{MX}_2 + K_{8}K_{9}\mathbf{MY}_2 \nonumber \\
 m_{2}\bar{a}_{G(2)} &= \left[\begin{matrix} D_{25} + D_{24}\ddot{\psi} & D_{29} + D_{27}\ddot{\psi} + D_{28}\ddot{q}_{imu} + D_{26}\ddot{x} + \mathbf{MZ}_2\ddot{q}_{w} & D_{33} + D_{31}\ddot{\psi} + D_{32}\ddot{q}_{imu} + D_{30}\ddot{x} - \mathbf{MY}_2\ddot{q}_{w} &  \end{matrix}\right] 
 \nonumber \\ 
 \dot{\bar{H}}_{2} &= \mathbf{MS}_{2} \times \bar{a}_{2} + J_{2}\bar{\alpha}_{2} + \bar\omega_{2} \times J_{2}\bar{\omega}_{2} 
 \nonumber \\ 
 \dot{\bar{H}}_{2} &= \left[\begin{matrix} K_{8}(K_{7}\mathbf{XZ}_2 + K_{8}\mathbf{YZ}_2 + K_{9}\mathbf{ZZ}_2) - K_{9}(K_{7}\mathbf{XY}_2 + K_{8}\mathbf{YY}_2 + K_{9}\mathbf{YZ}_2) + \mathbf{XY}_2(K_{19} + K_{10}\ddot{\psi}) + \mathbf{XZ}_2(K_{20} + K_{11}\ddot{\psi}) + \mathbf{MY}_2(K_{23} + K_{18}\ddot{q}_{imu} + K_{17}\ddot{x}) - \mathbf{MZ}_2(K_{22} + K_{16}\ddot{q}_{imu} + K_{11}\ddot{x}) - \mathbf{XX}_2(\ddot{q}_{w} + \ddot{q}_{imu}) & K_{9}(K_{7}\mathbf{XX}_2 + K_{8}\mathbf{XY}_2 + K_{9}\mathbf{XZ}_2) - K_{7}(K_{7}\mathbf{XZ}_2 + K_{8}\mathbf{YZ}_2 + K_{9}\mathbf{ZZ}_2) + \mathbf{MZ}_2(K_{21} + K_{15}\ddot{\psi}) + \mathbf{YY}_2(K_{19} + K_{10}\ddot{\psi}) + \mathbf{YZ}_2(K_{20} + K_{11}\ddot{\psi}) - \mathbf{MX}_2(K_{23} + K_{18}\ddot{q}_{imu} + K_{17}\ddot{x}) - \mathbf{XY}_2(\ddot{q}_{w} + \ddot{q}_{imu}) & K_{7}(K_{7}\mathbf{XY}_2 + K_{8}\mathbf{YY}_2 + K_{9}\mathbf{YZ}_2) - K_{8}(K_{7}\mathbf{XX}_2 + K_{8}\mathbf{XY}_2 + K_{9}\mathbf{XZ}_2) - \mathbf{MY}_2(K_{21} + K_{15}\ddot{\psi}) + \mathbf{YZ}_2(K_{19} + K_{10}\ddot{\psi}) + \mathbf{ZZ}_2(K_{20} + K_{11}\ddot{\psi}) + \mathbf{MX}_2(K_{22} + K_{16}\ddot{q}_{imu} + K_{11}\ddot{x}) - \mathbf{XZ}_2(\ddot{q}_{w} + \ddot{q}_{imu}) &  \end{matrix}\right] 
 \nonumber \\ 
D_{34} &= K_{17}\mathbf{MY}_2 - K_{11}\mathbf{MZ}_2 \nonumber \\
D_{35} &= K_{10}\mathbf{XY}_2 + K_{11}\mathbf{XZ}_2 \nonumber \\
D_{36} &= K_{18}\mathbf{MY}_2 - \mathbf{XX}_2 - K_{16}\mathbf{MZ}_2 \nonumber \\
D_{37} &= K_{19}\mathbf{XY}_2 + K_{20}\mathbf{XZ}_2 + K_{8}^2\mathbf{YZ}_2  \nonumber \\
&- K_{9}^2\mathbf{YZ}_2 + K_{23}\mathbf{MY}_2 - K_{22}\mathbf{MZ}_2  \nonumber \\
&- K_{7}K_{9}\mathbf{XY}_2 + K_{7}K_{8}\mathbf{XZ}_2  \nonumber \\
&- K_{8}K_{9}\mathbf{YY}_2 + K_{8}K_{9}\mathbf{ZZ}_2 \nonumber \\
D_{38} &= -K_{17}\mathbf{MX}_2 \nonumber \\
D_{39} &= K_{10}\mathbf{YY}_2 + K_{11}\mathbf{YZ}_2 + K_{15}\mathbf{MZ}_2 \nonumber \\
D_{40} &= - \mathbf{XY}_2 - K_{18}\mathbf{MX}_2 \nonumber \\
D_{41} &= K_{19}\mathbf{YY}_2 + K_{20}\mathbf{YZ}_2 - K_{7}^2\mathbf{XZ}_2  \nonumber \\
&+ K_{9}^2\mathbf{XZ}_2 - K_{23}\mathbf{MX}_2 + K_{21}\mathbf{MZ}_2  \nonumber \\
&+ K_{7}K_{9}\mathbf{XX}_2 + K_{8}K_{9}\mathbf{XY}_2  \nonumber \\
&- K_{7}K_{8}\mathbf{YZ}_2 - K_{7}K_{9}\mathbf{ZZ}_2 \nonumber \\
D_{42} &= K_{11}\mathbf{MX}_2 \nonumber \\
D_{43} &= K_{10}\mathbf{YZ}_2 + K_{11}\mathbf{ZZ}_2 - K_{15}\mathbf{MY}_2 \nonumber \\
D_{44} &= K_{16}\mathbf{MX}_2 - \mathbf{XZ}_2 \nonumber \\
D_{45} &= K_{19}\mathbf{YZ}_2 + K_{20}\mathbf{ZZ}_2 + K_{7}^2\mathbf{XY}_2  \nonumber \\
&- K_{8}^2\mathbf{XY}_2 + K_{22}\mathbf{MX}_2 - K_{21}\mathbf{MY}_2  \nonumber \\
&- K_{7}K_{8}\mathbf{XX}_2 - K_{8}K_{9}\mathbf{XZ}_2  \nonumber \\
&+ K_{7}K_{8}\mathbf{YY}_2 + K_{7}K_{9}\mathbf{YZ}_2 \nonumber \\
 \dot{\bar{H}}_{2} &= \left[\begin{matrix} D_{25} + D_{24}\ddot{\psi} & D_{29} + D_{27}\ddot{\psi} + D_{28}\ddot{q}_{imu} + D_{26}\ddot{x} + \mathbf{MZ}_2\ddot{q}_{w} & D_{33} + D_{31}\ddot{\psi} + D_{32}\ddot{q}_{imu} + D_{30}\ddot{x} - \mathbf{MY}_2\ddot{q}_{w} &  \end{matrix}\right] 
 \nonumber \\ 
 \bar\omega_{3} &= {}^{3}A_{2} \bar\omega_{2} + \dot{q}_{3} \bar{e}_{3} 
 \nonumber \\ 
 \bar\omega_{3} &= \left[\begin{matrix} - K_{7}c_{torso} - K_{9}s_{torso} & K_{8} - \dot{q}_{torso} & K_{7}s_{torso} - K_{9}c_{torso} &  \end{matrix}\right] 
 \nonumber \\ 
K_{25} &= - K_{7}c_{torso} - K_{9}s_{torso} \nonumber \\
K_{26} &= K_{8} - \dot{q}_{torso} \nonumber \\
K_{27} &= K_{7}s_{torso} - K_{9}c_{torso} \nonumber \\
 \bar\omega_{3} &= \left[\begin{matrix} K_{25} & K_{26} & K_{27} &  \end{matrix}\right] 
 \nonumber \\ 
 \bar\omega_{3} &= \left[\begin{matrix} c_{torso}(\dot{q}_{w} + \dot{q}_{imu}) - K_{11}\dot{\psi}s_{torso} & K_{10}\dot{\psi} - \dot{q}_{torso} & - s_{torso}(\dot{q}_{w} + \dot{q}_{imu}) - K_{11}\dot{\psi}c_{torso} &  \end{matrix}\right] 
 \nonumber \\ 
K_{28} &= -K_{11}s_{torso} \nonumber \\
K_{29} &= -K_{11}c_{torso} \nonumber \\
 \bar\omega_{3} &= \left[\begin{matrix} K_{28}\dot{\psi} + \dot{q}_{w}c_{torso} + \dot{q}_{imu}c_{torso} & K_{10}\dot{\psi} - \dot{q}_{torso} & K_{29}\dot{\psi} - \dot{q}_{w}s_{torso} - \dot{q}_{imu}s_{torso} &  \end{matrix}\right] 
 \nonumber \\ 
 \bar{v}_{3} &= {}^{3}A_{2} \left(\bar{v}_{2} + \bar\omega_{2} \times \bar{P}_{3}\right) 
 \nonumber \\ 
 \bar{v}_{3} &= \left[\begin{matrix} - c_{torso}(K_{12} + K_{8}L_4 - K_{9}L_3) - s_{torso}(K_{14} + K_{7}L_3) & K_{13} - K_{7}L_4 & s_{torso}(K_{12} + K_{8}L_4 - K_{9}L_3) - c_{torso}(K_{14} + K_{7}L_3) &  \end{matrix}\right] 
 \nonumber \\ 
K_{30} &= - c_{torso}(K_{12} + K_{8}L_4 - K_{9}L_3)  \nonumber \\
&- s_{torso}(K_{14} + K_{7}L_3) \nonumber \\
K_{31} &= K_{13} - K_{7}L_4 \nonumber \\
K_{32} &= s_{torso}(K_{12} + K_{8}L_4 - K_{9}L_3)  \nonumber \\
&- c_{torso}(K_{14} + K_{7}L_3) \nonumber \\
 \bar{v}_{3} &= \left[\begin{matrix} K_{30} & K_{31} & K_{32} &  \end{matrix}\right] 
 \nonumber \\ 
 \bar{v}_{3} &= \left[\begin{matrix} - c_{torso}(K_{15}\dot{\psi} + K_{10}L_4\dot{\psi} - K_{11}L_3\dot{\psi}) - s_{torso}(K_{18}\dot{q}_{imu} + K_{17}\dot{x} - L_3(\dot{q}_{w} + \dot{q}_{imu})) & K_{16}\dot{q}_{imu} + K_{11}\dot{x} + L_4(\dot{q}_{w} + \dot{q}_{imu}) & s_{torso}(K_{15}\dot{\psi} + K_{10}L_4\dot{\psi} - K_{11}L_3\dot{\psi}) - c_{torso}(K_{18}\dot{q}_{imu} + K_{17}\dot{x} - L_3(\dot{q}_{w} + \dot{q}_{imu})) &  \end{matrix}\right] 
 \nonumber \\ 
K_{33} &= -K_{17}s_{torso} \nonumber \\
K_{34} &= -c_{torso}(K_{15} + K_{10}L_4 - K_{11}L_3) \nonumber \\
K_{35} &= -s_{torso}(K_{18} - L_3) \nonumber \\
K_{36} &= L_3s_{torso} \nonumber \\
K_{37} &= K_{16} + L_4 \nonumber \\
K_{38} &= -K_{17}c_{torso} \nonumber \\
K_{39} &= s_{torso}(K_{15} + K_{10}L_4 - K_{11}L_3) \nonumber \\
K_{40} &= -c_{torso}(K_{18} - L_3) \nonumber \\
K_{41} &= L_3c_{torso} \nonumber \\
 \bar{v}_{3} &= \left[\begin{matrix} K_{34}\dot{\psi} + K_{36}\dot{q}_{w} + K_{35}\dot{q}_{imu} + K_{33}\dot{x} & K_{37}\dot{q}_{imu} + K_{11}\dot{x} + L_4\dot{q}_{w} & K_{39}\dot{\psi} + K_{41}\dot{q}_{w} + K_{40}\dot{q}_{imu} + K_{38}\dot{x} &  \end{matrix}\right] 
 \nonumber \\ 
 \bar\alpha_{3} &= {}^{3}A_{2} \bar\alpha_{2} + \ddot{q}_{3} \bar{e}_{3} + \dot{q}_{3} \left(\bar\omega_{3} \times \bar{e}_{3}\right) 
 \nonumber \\ 
 \bar\alpha_{3} &= \left[\begin{matrix} K_{27}\dot{q}_{torso} + c_{torso}(\ddot{q}_{w} + \ddot{q}_{imu}) - s_{torso}(K_{20} + K_{11}\ddot{\psi}) & K_{19} - \ddot{q}_{torso} + K_{10}\ddot{\psi} & - K_{25}\dot{q}_{torso} - s_{torso}(\ddot{q}_{w} + \ddot{q}_{imu}) - c_{torso}(K_{20} + K_{11}\ddot{\psi}) &  \end{matrix}\right] 
 \nonumber \\ 
K_{42} &= K_{27}\dot{q}_{torso} - K_{20}s_{torso} \nonumber \\
K_{43} &= - K_{25}\dot{q}_{torso} - K_{20}c_{torso} \nonumber \\
 \bar\alpha_{3} &= \left[\begin{matrix} K_{42} + K_{28}\ddot{\psi} + \ddot{q}_{w}c_{torso} + \ddot{q}_{imu}c_{torso} & K_{19} - \ddot{q}_{torso} + K_{10}\ddot{\psi} & K_{43} + K_{29}\ddot{\psi} - \ddot{q}_{w}s_{torso} - \ddot{q}_{imu}s_{torso} &  \end{matrix}\right] 
 \nonumber \\ 
 \bar{a}_{3} &= {}^{3}A_{2} \left(\bar{a}_{2} + \bar\alpha_{2} \times \bar{P}_{3} + \bar\omega_{2} \times \left(\bar\omega_{2} \times \bar{P}_{3}\right)\right) 
 \nonumber \\ 
 \bar\alpha_{3} &= \left[\begin{matrix} - s_{torso}(K_{23} + K_{18}\ddot{q}_{imu} + K_{17}\ddot{x} - K_{7}^2L_4 - L_3(\ddot{q}_{w} + \ddot{q}_{imu}) - K_{8}(K_{8}L_4 - K_{9}L_3)) - c_{torso}(K_{21} + K_{15}\ddot{\psi} + L_4(K_{19} + K_{10}\ddot{\psi}) - L_3(K_{20} + K_{11}\ddot{\psi}) + K_{7}K_{8}L_3 + K_{7}K_{9}L_4) & K_{22} + K_{16}\ddot{q}_{imu} + K_{11}\ddot{x} - K_{7}^2L_3 + L_4(\ddot{q}_{w} + \ddot{q}_{imu}) + K_{9}(K_{8}L_4 - K_{9}L_3) & s_{torso}(K_{21} + K_{15}\ddot{\psi} + L_4(K_{19} + K_{10}\ddot{\psi}) - L_3(K_{20} + K_{11}\ddot{\psi}) + K_{7}K_{8}L_3 + K_{7}K_{9}L_4) - c_{torso}(K_{23} + K_{18}\ddot{q}_{imu} + K_{17}\ddot{x} - K_{7}^2L_4 - L_3(\ddot{q}_{w} + \ddot{q}_{imu}) - K_{8}(K_{8}L_4 - K_{9}L_3)) &  \end{matrix}\right] 
 \nonumber \\ 
K_{44} &= K_{7}^2L_4s_{torso} - K_{23}s_{torso} - K_{21}c_{torso}  \nonumber \\
&+ K_{8}^2L_4s_{torso} - K_{19}L_4c_{torso}  \nonumber \\
&+ K_{20}L_3c_{torso} - K_{7}K_{8}L_3c_{torso}  \nonumber \\
&- K_{7}K_{9}L_4c_{torso} - K_{8}K_{9}L_3s_{torso} \nonumber \\
K_{45} &= K_{22} - K_{7}^2L_3 - K_{9}^2L_3 + K_{8}K_{9}L_4 \nonumber \\
K_{46} &= K_{21}s_{torso} - K_{23}c_{torso} + K_{7}^2L_4c_{torso}  \nonumber \\
&+ K_{8}^2L_4c_{torso} + K_{19}L_4s_{torso}  \nonumber \\
&- K_{20}L_3s_{torso} - K_{8}K_{9}L_3c_{torso}  \nonumber \\
&+ K_{7}K_{8}L_3s_{torso} + K_{7}K_{9}L_4s_{torso} \nonumber \\
 \bar{a}_{3} &= \left[\begin{matrix} K_{44} + K_{34}\ddot{\psi} + K_{36}\ddot{q}_{w} + K_{35}\ddot{q}_{imu} + K_{33}\ddot{x} & K_{45} + K_{37}\ddot{q}_{imu} + K_{11}\ddot{x} + L_4\ddot{q}_{w} & K_{46} + K_{39}\ddot{\psi} + K_{41}\ddot{q}_{w} + K_{40}\ddot{q}_{imu} + K_{38}\ddot{x} &  \end{matrix}\right] 
 \nonumber \\ 
 \bar{g}_{3} &= {}^{3}A_{2} \bar{g}_{2} 
 \nonumber \\ 
 \bar{g}_{3} &= \left[\begin{matrix} -K_{24}gs_{torso} & K_{17}g & -K_{24}gc_{torso} &  \end{matrix}\right] 
 \nonumber \\ 
K_{47} &= -K_{24}s_{torso} \nonumber \\
K_{48} &= -K_{24}c_{torso} \nonumber \\
 \bar{g}_{3} &= \left[\begin{matrix} K_{47}g & K_{17}g & K_{48}g &  \end{matrix}\right] 
 \nonumber \\ 
 m_{3}\bar{S}_{3}^{\times}\bar{g}_{3} &= \mathbf{MS}_{3} \times \bar{g}_{3} 
 \nonumber \\ 
 m_{3}\bar{S}_{3}^{\times}\bar{g}_{3} &= \left[\begin{matrix} K_{48}\mathbf{MY}_3g - K_{17}\mathbf{MZ}_3g & K_{47}\mathbf{MZ}_3g - K_{48}\mathbf{MX}_3g & K_{17}\mathbf{MX}_3g - K_{47}\mathbf{MY}_3g &  \end{matrix}\right] 
 \nonumber \\ 
D_{46} &= K_{48}\mathbf{MY}_3 - K_{17}\mathbf{MZ}_3 \nonumber \\
D_{47} &= K_{47}\mathbf{MZ}_3 - K_{48}\mathbf{MX}_3 \nonumber \\
D_{48} &= K_{17}\mathbf{MX}_3 - K_{47}\mathbf{MY}_3 \nonumber \\
 m_{3}\bar{S}_{3}^{\times}\bar{g}_{3} &= \left[\begin{matrix} D_{46}g & D_{47}g & D_{48}g &  \end{matrix}\right] 
 \nonumber \\ 
 m_{3}\bar{a}_{G(3)} &= m_{3}\bar{a}_{3} + \bar\alpha_{3} \times \mathbf{MS}_{3} + \bar\omega_{3} \times \left(\bar\omega_{3} \times \mathbf{MS}_{3}\right) 
 \nonumber \\ 
 m_{3}\bar{a}_{G(3)} &= \left[\begin{matrix} \mathbf{MZ}_3(K_{19} - \ddot{q}_{torso} + K_{10}\ddot{\psi}) + m_3(K_{44} + K_{34}\ddot{\psi} + K_{36}\ddot{q}_{w} + K_{35}\ddot{q}_{imu} + K_{33}\ddot{x}) - \mathbf{MY}_3(K_{43} + K_{29}\ddot{\psi} - \ddot{q}_{w}s_{torso} - \ddot{q}_{imu}s_{torso}) - K_{26}(K_{26}\mathbf{MX}_3 - K_{25}\mathbf{MY}_3) - K_{27}(K_{27}\mathbf{MX}_3 - K_{25}\mathbf{MZ}_3) & m_3(K_{45} + K_{37}\ddot{q}_{imu} + K_{11}\ddot{x} + L_4\ddot{q}_{w}) - \mathbf{MZ}_3(K_{42} + K_{28}\ddot{\psi} + \ddot{q}_{w}c_{torso} + \ddot{q}_{imu}c_{torso}) + \mathbf{MX}_3(K_{43} + K_{29}\ddot{\psi} - \ddot{q}_{w}s_{torso} - \ddot{q}_{imu}s_{torso}) + K_{25}(K_{26}\mathbf{MX}_3 - K_{25}\mathbf{MY}_3) - K_{27}(K_{27}\mathbf{MY}_3 - K_{26}\mathbf{MZ}_3) & \mathbf{MY}_3(K_{42} + K_{28}\ddot{\psi} + \ddot{q}_{w}c_{torso} + \ddot{q}_{imu}c_{torso}) - \mathbf{MX}_3(K_{19} - \ddot{q}_{torso} + K_{10}\ddot{\psi}) + m_3(K_{46} + K_{39}\ddot{\psi} + K_{41}\ddot{q}_{w} + K_{40}\ddot{q}_{imu} + K_{38}\ddot{x}) + K_{25}(K_{27}\mathbf{MX}_3 - K_{25}\mathbf{MZ}_3) + K_{26}(K_{27}\mathbf{MY}_3 - K_{26}\mathbf{MZ}_3) &  \end{matrix}\right] 
 \nonumber \\ 
D_{49} &= K_{33}m_3 \nonumber \\
D_{50} &= K_{34}m_3 - K_{29}\mathbf{MY}_3 + K_{10}\mathbf{MZ}_3 \nonumber \\
D_{51} &= K_{35}m_3 + \mathbf{MY}_3s_{torso} \nonumber \\
D_{52} &= K_{36}m_3 + \mathbf{MY}_3s_{torso} \nonumber \\
D_{53} &= K_{44}m_3 - K_{26}^2\mathbf{MX}_3 - K_{27}^2\mathbf{MX}_3  \nonumber \\
&- K_{43}\mathbf{MY}_3 + K_{19}\mathbf{MZ}_3 + K_{25}K_{26}\mathbf{MY}_3  \nonumber \\
&+ K_{25}K_{27}\mathbf{MZ}_3 \nonumber \\
D_{54} &= K_{11}m_3 \nonumber \\
D_{55} &= K_{29}\mathbf{MX}_3 - K_{28}\mathbf{MZ}_3 \nonumber \\
D_{56} &= K_{37}m_3 - \mathbf{MZ}_3c_{torso} - \mathbf{MX}_3s_{torso} \nonumber \\
D_{57} &= L_4m_3 - \mathbf{MZ}_3c_{torso} - \mathbf{MX}_3s_{torso} \nonumber \\
D_{58} &= K_{45}m_3 - K_{25}^2\mathbf{MY}_3 - K_{27}^2\mathbf{MY}_3  \nonumber \\
&+ K_{43}\mathbf{MX}_3 - K_{42}\mathbf{MZ}_3 + K_{25}K_{26}\mathbf{MX}_3  \nonumber \\
&+ K_{26}K_{27}\mathbf{MZ}_3 \nonumber \\
D_{59} &= K_{38}m_3 \nonumber \\
D_{60} &= K_{39}m_3 - K_{10}\mathbf{MX}_3 + K_{28}\mathbf{MY}_3 \nonumber \\
D_{61} &= K_{40}m_3 + \mathbf{MY}_3c_{torso} \nonumber \\
D_{62} &= K_{41}m_3 + \mathbf{MY}_3c_{torso} \nonumber \\
D_{63} &= K_{46}m_3 - K_{25}^2\mathbf{MZ}_3 - K_{26}^2\mathbf{MZ}_3  \nonumber \\
&- K_{19}\mathbf{MX}_3 + K_{42}\mathbf{MY}_3 + K_{25}K_{27}\mathbf{MX}_3  \nonumber \\
&+ K_{26}K_{27}\mathbf{MY}_3 \nonumber \\
 m_{3}\bar{a}_{G(3)} &= \left[\begin{matrix} D_{53} + D_{50}\ddot{\psi} + D_{52}\ddot{q}_{w} + D_{51}\ddot{q}_{imu} + D_{49}\ddot{x} - \mathbf{MZ}_3\ddot{q}_{torso} & D_{58} + D_{55}\ddot{\psi} + D_{57}\ddot{q}_{w} + D_{56}\ddot{q}_{imu} + D_{54}\ddot{x} & D_{63} + D_{60}\ddot{\psi} + D_{62}\ddot{q}_{w} + D_{61}\ddot{q}_{imu} + D_{59}\ddot{x} + \mathbf{MX}_3\ddot{q}_{torso} &  \end{matrix}\right] 
 \nonumber \\ 
 \dot{\bar{H}}_{3} &= \mathbf{MS}_{3} \times \bar{a}_{3} + J_{3}\bar{\alpha}_{3} + \bar\omega_{3} \times J_{3}\bar{\omega}_{3} 
 \nonumber \\ 
 \dot{\bar{H}}_{3} &= \left[\begin{matrix} K_{26}(K_{25}\mathbf{XZ}_3 + K_{26}\mathbf{YZ}_3 + K_{27}\mathbf{ZZ}_3) - K_{27}(K_{25}\mathbf{XY}_3 + K_{26}\mathbf{YY}_3 + K_{27}\mathbf{YZ}_3) + \mathbf{XY}_3(K_{19} - \ddot{q}_{torso} + K_{10}\ddot{\psi}) - \mathbf{MZ}_3(K_{45} + K_{37}\ddot{q}_{imu} + K_{11}\ddot{x} + L_4\ddot{q}_{w}) + \mathbf{MY}_3(K_{46} + K_{39}\ddot{\psi} + K_{41}\ddot{q}_{w} + K_{40}\ddot{q}_{imu} + K_{38}\ddot{x}) + \mathbf{XX}_3(K_{42} + K_{28}\ddot{\psi} + \ddot{q}_{w}c_{torso} + \ddot{q}_{imu}c_{torso}) + \mathbf{XZ}_3(K_{43} + K_{29}\ddot{\psi} - \ddot{q}_{w}s_{torso} - \ddot{q}_{imu}s_{torso}) & K_{27}(K_{25}\mathbf{XX}_3 + K_{26}\mathbf{XY}_3 + K_{27}\mathbf{XZ}_3) - K_{25}(K_{25}\mathbf{XZ}_3 + K_{26}\mathbf{YZ}_3 + K_{27}\mathbf{ZZ}_3) + \mathbf{YY}_3(K_{19} - \ddot{q}_{torso} + K_{10}\ddot{\psi}) - \mathbf{MX}_3(K_{46} + K_{39}\ddot{\psi} + K_{41}\ddot{q}_{w} + K_{40}\ddot{q}_{imu} + K_{38}\ddot{x}) + \mathbf{MZ}_3(K_{44} + K_{34}\ddot{\psi} + K_{36}\ddot{q}_{w} + K_{35}\ddot{q}_{imu} + K_{33}\ddot{x}) + \mathbf{XY}_3(K_{42} + K_{28}\ddot{\psi} + \ddot{q}_{w}c_{torso} + \ddot{q}_{imu}c_{torso}) + \mathbf{YZ}_3(K_{43} + K_{29}\ddot{\psi} - \ddot{q}_{w}s_{torso} - \ddot{q}_{imu}s_{torso}) & K_{25}(K_{25}\mathbf{XY}_3 + K_{26}\mathbf{YY}_3 + K_{27}\mathbf{YZ}_3) - K_{26}(K_{25}\mathbf{XX}_3 + K_{26}\mathbf{XY}_3 + K_{27}\mathbf{XZ}_3) + \mathbf{YZ}_3(K_{19} - \ddot{q}_{torso} + K_{10}\ddot{\psi}) + \mathbf{MX}_3(K_{45} + K_{37}\ddot{q}_{imu} + K_{11}\ddot{x} + L_4\ddot{q}_{w}) - \mathbf{MY}_3(K_{44} + K_{34}\ddot{\psi} + K_{36}\ddot{q}_{w} + K_{35}\ddot{q}_{imu} + K_{33}\ddot{x}) + \mathbf{XZ}_3(K_{42} + K_{28}\ddot{\psi} + \ddot{q}_{w}c_{torso} + \ddot{q}_{imu}c_{torso}) + \mathbf{ZZ}_3(K_{43} + K_{29}\ddot{\psi} - \ddot{q}_{w}s_{torso} - \ddot{q}_{imu}s_{torso}) &  \end{matrix}\right] 
 \nonumber \\ 
D_{64} &= K_{38}\mathbf{MY}_3 - K_{11}\mathbf{MZ}_3 \nonumber \\
D_{65} &= K_{28}\mathbf{XX}_3 + K_{10}\mathbf{XY}_3 + K_{29}\mathbf{XZ}_3  \nonumber \\
&+ K_{39}\mathbf{MY}_3 \nonumber \\
D_{66} &= \mathbf{XX}_3c_{torso} - \mathbf{XZ}_3s_{torso} + K_{40}\mathbf{MY}_3  \nonumber \\
&- K_{37}\mathbf{MZ}_3 \nonumber \\
D_{67} &= \mathbf{XX}_3c_{torso} - \mathbf{XZ}_3s_{torso} + K_{41}\mathbf{MY}_3  \nonumber \\
&- L_4\mathbf{MZ}_3 \nonumber \\
D_{68} &= K_{42}\mathbf{XX}_3 + K_{19}\mathbf{XY}_3 + K_{43}\mathbf{XZ}_3  \nonumber \\
&+ K_{26}^2\mathbf{YZ}_3 - K_{27}^2\mathbf{YZ}_3 + K_{46}\mathbf{MY}_3  \nonumber \\
&- K_{45}\mathbf{MZ}_3 - K_{25}K_{27}\mathbf{XY}_3 + K_{25}K_{26}\mathbf{XZ}_3  \nonumber \\
&- K_{26}K_{27}\mathbf{YY}_3 + K_{26}K_{27}\mathbf{ZZ}_3 \nonumber \\
D_{69} &= K_{33}\mathbf{MZ}_3 - K_{38}\mathbf{MX}_3 \nonumber \\
D_{70} &= K_{28}\mathbf{XY}_3 + K_{10}\mathbf{YY}_3 + K_{29}\mathbf{YZ}_3  \nonumber \\
&- K_{39}\mathbf{MX}_3 + K_{34}\mathbf{MZ}_3 \nonumber \\
D_{71} &= \mathbf{XY}_3c_{torso} - \mathbf{YZ}_3s_{torso} - K_{40}\mathbf{MX}_3  \nonumber \\
&+ K_{35}\mathbf{MZ}_3 \nonumber \\
D_{72} &= \mathbf{XY}_3c_{torso} - \mathbf{YZ}_3s_{torso} - K_{41}\mathbf{MX}_3  \nonumber \\
&+ K_{36}\mathbf{MZ}_3 \nonumber \\
D_{73} &= K_{42}\mathbf{XY}_3 + K_{19}\mathbf{YY}_3 + K_{43}\mathbf{YZ}_3  \nonumber \\
&- K_{25}^2\mathbf{XZ}_3 + K_{27}^2\mathbf{XZ}_3 - K_{46}\mathbf{MX}_3  \nonumber \\
&+ K_{44}\mathbf{MZ}_3 + K_{25}K_{27}\mathbf{XX}_3 + K_{26}K_{27}\mathbf{XY}_3  \nonumber \\
&- K_{25}K_{26}\mathbf{YZ}_3 - K_{25}K_{27}\mathbf{ZZ}_3 \nonumber \\
D_{74} &= K_{11}\mathbf{MX}_3 - K_{33}\mathbf{MY}_3 \nonumber \\
D_{75} &= K_{28}\mathbf{XZ}_3 + K_{10}\mathbf{YZ}_3 + K_{29}\mathbf{ZZ}_3  \nonumber \\
&- K_{34}\mathbf{MY}_3 \nonumber \\
D_{76} &= \mathbf{XZ}_3c_{torso} - \mathbf{ZZ}_3s_{torso} + K_{37}\mathbf{MX}_3  \nonumber \\
&- K_{35}\mathbf{MY}_3 \nonumber \\
D_{77} &= \mathbf{XZ}_3c_{torso} - \mathbf{ZZ}_3s_{torso} - K_{36}\mathbf{MY}_3  \nonumber \\
&+ L_4\mathbf{MX}_3 \nonumber \\
D_{78} &= K_{42}\mathbf{XZ}_3 + K_{19}\mathbf{YZ}_3 + K_{43}\mathbf{ZZ}_3  \nonumber \\
&+ K_{25}^2\mathbf{XY}_3 - K_{26}^2\mathbf{XY}_3 + K_{45}\mathbf{MX}_3  \nonumber \\
&- K_{44}\mathbf{MY}_3 - K_{25}K_{26}\mathbf{XX}_3 - K_{26}K_{27}\mathbf{XZ}_3  \nonumber \\
&+ K_{25}K_{26}\mathbf{YY}_3 + K_{25}K_{27}\mathbf{YZ}_3 \nonumber \\
 \dot{\bar{H}}_{3} &= \left[\begin{matrix} D_{53} + D_{50}\ddot{\psi} + D_{52}\ddot{q}_{w} + D_{51}\ddot{q}_{imu} + D_{49}\ddot{x} - \mathbf{MZ}_3\ddot{q}_{torso} & D_{58} + D_{55}\ddot{\psi} + D_{57}\ddot{q}_{w} + D_{56}\ddot{q}_{imu} + D_{54}\ddot{x} & D_{63} + D_{60}\ddot{\psi} + D_{62}\ddot{q}_{w} + D_{61}\ddot{q}_{imu} + D_{59}\ddot{x} + \mathbf{MX}_3\ddot{q}_{torso} &  \end{matrix}\right] 
 \nonumber \\ 
 \bar\omega_{4l} &= {}^{4l}A_{3} \bar\omega_{3} + \dot{q}_{4l} \bar{e}_{4l} 
 \nonumber \\ 
 \bar\omega_{4l} &= \left[\begin{matrix} K_{26}c_{1l} - K_{27}s_{1l} & K_{25} - \dot{q}_{1l} & - K_{27}c_{1l} - K_{26}s_{1l} &  \end{matrix}\right] 
 \nonumber \\ 
K_{1l} &= K_{26}c_{1l} - K_{27}s_{1l} \nonumber \\
K_{2l} &= K_{25} - \dot{q}_{1l} \nonumber \\
K_{3l} &= - K_{27}c_{1l} - K_{26}s_{1l} \nonumber \\
 \bar\omega_{4l} &= \left[\begin{matrix} K_{1l} & K_{2l} & K_{3l} &  \end{matrix}\right] 
 \nonumber \\ 
 \bar\omega_{4l} &= \left[\begin{matrix} s_{1l}(\dot{q}_{w}s_{torso} - K_{29}\dot{\psi} + \dot{q}_{imu}s_{torso}) - c_{1l}(\dot{q}_{torso} - K_{10}\dot{\psi}) & K_{28}\dot{\psi} - \dot{q}_{1l} + \dot{q}_{w}c_{torso} + \dot{q}_{imu}c_{torso} & s_{1l}(\dot{q}_{torso} - K_{10}\dot{\psi}) + c_{1l}(\dot{q}_{w}s_{torso} - K_{29}\dot{\psi} + \dot{q}_{imu}s_{torso}) &  \end{matrix}\right] 
 \nonumber \\ 
K_{4l} &= K_{10}c_{1l} - K_{29}s_{1l} \nonumber \\
K_{5l} &= s_{1l}s_{torso} \nonumber \\
K_{6l} &= - K_{29}c_{1l} - K_{10}s_{1l} \nonumber \\
K_{7l} &= c_{1l}s_{torso} \nonumber \\
 \bar\omega_{4l} &= \left[\begin{matrix} K_{4l}\dot{\psi} + K_{5l}\dot{q}_{w} + K_{5l}\dot{q}_{imu} - \dot{q}_{torso}c_{1l} & K_{28}\dot{\psi} - \dot{q}_{1l} + \dot{q}_{w}c_{torso} + \dot{q}_{imu}c_{torso} & K_{6l}\dot{\psi} + K_{7l}\dot{q}_{w} + K_{7l}\dot{q}_{imu} + \dot{q}_{torso}s_{1l} &  \end{matrix}\right] 
 \nonumber \\ 
 \bar{v}_{4l} &= {}^{4l}A_{3} \left(\bar{v}_{3} + \bar\omega_{3} \times \bar{P}_{4l}\right) 
 \nonumber \\ 
 \bar{v}_{4l} &= \left[\begin{matrix} c_{1l}(K_{31} + K_{27}L_6) - s_{1l}(K_{32} + K_{25}L_5 - K_{26}L_6) & K_{30} - K_{27}L_5 & - c_{1l}(K_{32} + K_{25}L_5 - K_{26}L_6) - s_{1l}(K_{31} + K_{27}L_6) &  \end{matrix}\right] 
 \nonumber \\ 
K_{8l} &= c_{1l}(K_{31} + K_{27}L_6) - s_{1l}(K_{32}  \nonumber \\
&+ K_{25}L_5 - K_{26}L_6) \nonumber \\
K_{9l} &= K_{30} - K_{27}L_5 \nonumber \\
K_{10l} &= - c_{1l}(K_{32} + K_{25}L_5 - K_{26}L_6)  \nonumber \\
&- s_{1l}(K_{31} + K_{27}L_6) \nonumber \\
 \bar{v}_{4l} &= \left[\begin{matrix} K_{8l} & K_{9l} & K_{10l} &  \end{matrix}\right] 
 \nonumber \\ 
 \bar{v}_{4l} &= \left[\begin{matrix} c_{1l}(K_{37}\dot{q}_{imu} + K_{11}\dot{x} + L_4\dot{q}_{w} - L_6(\dot{q}_{w}s_{torso} - K_{29}\dot{\psi} + \dot{q}_{imu}s_{torso})) - s_{1l}(K_{39}\dot{\psi} + K_{41}\dot{q}_{w} + K_{40}\dot{q}_{imu} + K_{38}\dot{x} + L_5(K_{28}\dot{\psi} + \dot{q}_{w}c_{torso} + \dot{q}_{imu}c_{torso}) + L_6(\dot{q}_{torso} - K_{10}\dot{\psi})) & K_{34}\dot{\psi} + K_{36}\dot{q}_{w} + K_{35}\dot{q}_{imu} + K_{33}\dot{x} + L_5(\dot{q}_{w}s_{torso} - K_{29}\dot{\psi} + \dot{q}_{imu}s_{torso}) & - s_{1l}(K_{37}\dot{q}_{imu} + K_{11}\dot{x} + L_4\dot{q}_{w} - L_6(\dot{q}_{w}s_{torso} - K_{29}\dot{\psi} + \dot{q}_{imu}s_{torso})) - c_{1l}(K_{39}\dot{\psi} + K_{41}\dot{q}_{w} + K_{40}\dot{q}_{imu} + K_{38}\dot{x} + L_5(K_{28}\dot{\psi} + \dot{q}_{w}c_{torso} + \dot{q}_{imu}c_{torso}) + L_6(\dot{q}_{torso} - K_{10}\dot{\psi})) &  \end{matrix}\right] 
 \nonumber \\ 
K_{11l} &= K_{11}c_{1l} - K_{38}s_{1l} \nonumber \\
K_{12l} &= K_{29}L_6c_{1l} - s_{1l}(K_{39} - K_{10}L_6  \nonumber \\
&+ K_{28}L_5) \nonumber \\
K_{13l} &= c_{1l}(K_{37} - L_6s_{torso}) - s_{1l}(K_{40}  \nonumber \\
&+ L_5c_{torso}) \nonumber \\
K_{14l} &= c_{1l}(L_4 - L_6s_{torso}) - s_{1l}(K_{41}  \nonumber \\
&+ L_5c_{torso}) \nonumber \\
K_{15l} &= -L_6s_{1l} \nonumber \\
K_{16l} &= K_{34} - K_{29}L_5 \nonumber \\
K_{17l} &= K_{35} + L_5s_{torso} \nonumber \\
K_{18l} &= K_{36} + L_5s_{torso} \nonumber \\
K_{19l} &= - K_{38}c_{1l} - K_{11}s_{1l} \nonumber \\
K_{20l} &= - c_{1l}(K_{39} - K_{10}L_6 + K_{28}L_5)  \nonumber \\
&- K_{29}L_6s_{1l} \nonumber \\
K_{21l} &= - c_{1l}(K_{40} + L_5c_{torso}) - s_{1l}(K_{37}  \nonumber \\
&- L_6s_{torso}) \nonumber \\
K_{22l} &= - c_{1l}(K_{41} + L_5c_{torso}) - s_{1l}(L_4  \nonumber \\
&- L_6s_{torso}) \nonumber \\
K_{23l} &= -L_6c_{1l} \nonumber \\
 \bar{v}_{4l} &= \left[\begin{matrix} K_{12l}\dot{\psi} + K_{14l}\dot{q}_{w} + K_{13l}\dot{q}_{imu} + K_{15l}\dot{q}_{torso} + K_{11l}\dot{x} & K_{16l}\dot{\psi} + K_{18l}\dot{q}_{w} + K_{17l}\dot{q}_{imu} + K_{33}\dot{x} & K_{20l}\dot{\psi} + K_{22l}\dot{q}_{w} + K_{21l}\dot{q}_{imu} + K_{23l}\dot{q}_{torso} + K_{19l}\dot{x} &  \end{matrix}\right] 
 \nonumber \\ 
 \bar\alpha_{4l} &= {}^{4l}A_{3} \bar\alpha_{3} + \ddot{q}_{4l} \bar{e}_{4l} + \dot{q}_{4l} \left(\bar\omega_{4l} \times \bar{e}_{4l}\right) 
 \nonumber \\ 
 \bar\alpha_{4l} &= \left[\begin{matrix} K_{3l}\dot{q}_{1l} - s_{1l}(K_{43} + K_{29}\ddot{\psi} - \ddot{q}_{w}s_{torso} - \ddot{q}_{imu}s_{torso}) + c_{1l}(K_{19} - \ddot{q}_{torso} + K_{10}\ddot{\psi}) & K_{42} - \ddot{q}_{1l} + K_{28}\ddot{\psi} + \ddot{q}_{w}c_{torso} + \ddot{q}_{imu}c_{torso} & - K_{1l}\dot{q}_{1l} - c_{1l}(K_{43} + K_{29}\ddot{\psi} - \ddot{q}_{w}s_{torso} - \ddot{q}_{imu}s_{torso}) - s_{1l}(K_{19} - \ddot{q}_{torso} + K_{10}\ddot{\psi}) &  \end{matrix}\right] 
 \nonumber \\ 
K_{24l} &= K_{3l}\dot{q}_{1l} + K_{19}c_{1l} - K_{43}s_{1l} \nonumber \\
K_{25l} &= - K_{1l}\dot{q}_{1l} - K_{43}c_{1l} - K_{19}s_{1l} \nonumber \\
 \bar\alpha_{4l} &= \left[\begin{matrix} K_{24l} + K_{4l}\ddot{\psi} + K_{5l}\ddot{q}_{w} + K_{5l}\ddot{q}_{imu} - \ddot{q}_{torso}c_{1l} & K_{42} - \ddot{q}_{1l} + K_{28}\ddot{\psi} + \ddot{q}_{w}c_{torso} + \ddot{q}_{imu}c_{torso} & K_{25l} + K_{6l}\ddot{\psi} + K_{7l}\ddot{q}_{w} + K_{7l}\ddot{q}_{imu} + \ddot{q}_{torso}s_{1l} &  \end{matrix}\right] 
 \nonumber \\ 
 \bar{a}_{4l} &= {}^{4l}A_{3} \left(\bar{a}_{3} + \bar\alpha_{3} \times \bar{P}_{4l} + \bar\omega_{3} \times \left(\bar\omega_{3} \times \bar{P}_{4l}\right)\right) 
 \nonumber \\ 
 \bar\alpha_{4l} &= \left[\begin{matrix} c_{1l}(K_{45} + K_{37}\ddot{q}_{imu} + K_{11}\ddot{x} + L_4\ddot{q}_{w} - K_{27}^2L_5 + L_6(K_{43} + K_{29}\ddot{\psi} - \ddot{q}_{w}s_{torso} - \ddot{q}_{imu}s_{torso}) - K_{25}(K_{25}L_5 - K_{26}L_6)) - s_{1l}(K_{46} + K_{39}\ddot{\psi} + K_{41}\ddot{q}_{w} + K_{40}\ddot{q}_{imu} + K_{38}\ddot{x} - L_6(K_{19} - \ddot{q}_{torso} + K_{10}\ddot{\psi}) + L_5(K_{42} + K_{28}\ddot{\psi} + \ddot{q}_{w}c_{torso} + \ddot{q}_{imu}c_{torso}) + K_{25}K_{27}L_6 + K_{26}K_{27}L_5) & K_{44} + K_{34}\ddot{\psi} + K_{36}\ddot{q}_{w} + K_{35}\ddot{q}_{imu} + K_{33}\ddot{x} - K_{27}^2L_6 - L_5(K_{43} + K_{29}\ddot{\psi} - \ddot{q}_{w}s_{torso} - \ddot{q}_{imu}s_{torso}) + K_{26}(K_{25}L_5 - K_{26}L_6) & - s_{1l}(K_{45} + K_{37}\ddot{q}_{imu} + K_{11}\ddot{x} + L_4\ddot{q}_{w} - K_{27}^2L_5 + L_6(K_{43} + K_{29}\ddot{\psi} - \ddot{q}_{w}s_{torso} - \ddot{q}_{imu}s_{torso}) - K_{25}(K_{25}L_5 - K_{26}L_6)) - c_{1l}(K_{46} + K_{39}\ddot{\psi} + K_{41}\ddot{q}_{w} + K_{40}\ddot{q}_{imu} + K_{38}\ddot{x} - L_6(K_{19} - \ddot{q}_{torso} + K_{10}\ddot{\psi}) + L_5(K_{42} + K_{28}\ddot{\psi} + \ddot{q}_{w}c_{torso} + \ddot{q}_{imu}c_{torso}) + K_{25}K_{27}L_6 + K_{26}K_{27}L_5) &  \end{matrix}\right] 
 \nonumber \\ 
K_{26l} &= K_{45}c_{1l} - K_{46}s_{1l} - K_{25}^2L_5c_{1l}  \nonumber \\
&- K_{27}^2L_5c_{1l} + K_{43}L_6c_{1l}  \nonumber \\
&+ K_{19}L_6s_{1l} - K_{42}L_5s_{1l}  \nonumber \\
&+ K_{25}K_{26}L_6c_{1l} - K_{25}K_{27}L_6s_{1l}  \nonumber \\
&- K_{26}K_{27}L_5s_{1l} \nonumber \\
K_{27l} &= K_{44} - K_{26}^2L_6 - K_{27}^2L_6 - K_{43}L_5  \nonumber \\
&+ K_{25}K_{26}L_5 \nonumber \\
K_{28l} &= K_{25}^2L_5s_{1l} - K_{45}s_{1l} - K_{46}c_{1l}  \nonumber \\
&+ K_{27}^2L_5s_{1l} + K_{19}L_6c_{1l}  \nonumber \\
&- K_{42}L_5c_{1l} - K_{43}L_6s_{1l}  \nonumber \\
&- K_{25}K_{27}L_6c_{1l} - K_{26}K_{27}L_5c_{1l}  \nonumber \\
&- K_{25}K_{26}L_6s_{1l} \nonumber \\
 \bar{a}_{4l} &= \left[\begin{matrix} K_{26l} + K_{12l}\ddot{\psi} + K_{14l}\ddot{q}_{w} + K_{13l}\ddot{q}_{imu} + K_{15l}\ddot{q}_{torso} + K_{11l}\ddot{x} & K_{27l} + K_{16l}\ddot{\psi} + K_{18l}\ddot{q}_{w} + K_{17l}\ddot{q}_{imu} + K_{33}\ddot{x} & K_{28l} + K_{20l}\ddot{\psi} + K_{22l}\ddot{q}_{w} + K_{21l}\ddot{q}_{imu} + K_{23l}\ddot{q}_{torso} + K_{19l}\ddot{x} &  \end{matrix}\right] 
 \nonumber \\ 
 \bar{g}_{4l} &= {}^{4l}A_{3} \bar{g}_{3} 
 \nonumber \\ 
 \bar{g}_{4l} &= \left[\begin{matrix} K_{17}gc_{1l} - K_{48}gs_{1l} & K_{47}g & - K_{48}gc_{1l} - K_{17}gs_{1l} &  \end{matrix}\right] 
 \nonumber \\ 
K_{29l} &= K_{17}c_{1l} - K_{48}s_{1l} \nonumber \\
K_{30l} &= - K_{48}c_{1l} - K_{17}s_{1l} \nonumber \\
 \bar{g}_{4l} &= \left[\begin{matrix} K_{29l}g & K_{47}g & K_{30l}g &  \end{matrix}\right] 
 \nonumber \\ 
 m_{4l}\bar{S}_{4l}^{\times}\bar{g}_{4l} &= \mathbf{MS}_{4l} \times \bar{g}_{4l} 
 \nonumber \\ 
 m_{4l}\bar{S}_{4l}^{\times}\bar{g}_{4l} &= \left[\begin{matrix} K_{30l}\mathbf{MY}_{4l}g - K_{47}\mathbf{MZ}_{4l}g & K_{29l}\mathbf{MZ}_{4l}g - K_{30l}\mathbf{MX}_{4l}g & K_{47}\mathbf{MX}_{4l}g - K_{29l}\mathbf{MY}_{4l}g &  \end{matrix}\right] 
 \nonumber \\ 
D_{1l} &= K_{30l}\mathbf{MY}_{4l} - K_{47}\mathbf{MZ}_{4l} \nonumber \\
D_{2l} &= K_{29l}\mathbf{MZ}_{4l} - K_{30l}\mathbf{MX}_{4l} \nonumber \\
D_{3l} &= K_{47}\mathbf{MX}_{4l} - K_{29l}\mathbf{MY}_{4l} \nonumber \\
 m_{4l}\bar{S}_{4l}^{\times}\bar{g}_{4l} &= \left[\begin{matrix} D_{1l}g & D_{2l}g & D_{3l}g &  \end{matrix}\right] 
 \nonumber \\ 
 m_{4l}\bar{a}_{G(4l)} &= m_{4l}\bar{a}_{4l} + \bar\alpha_{4l} \times \mathbf{MS}_{4l} + \bar\omega_{4l} \times \left(\bar\omega_{4l} \times \mathbf{MS}_{4l}\right) 
 \nonumber \\ 
 m_{4l}\bar{a}_{G(4l)} &= \left[\begin{matrix} \mathbf{MZ}_{4l}(K_{42} - \ddot{q}_{1l} + K_{28}\ddot{\psi} + \ddot{q}_{w}c_{torso} + \ddot{q}_{imu}c_{torso}) - \mathbf{MY}_{4l}(K_{25l} + K_{6l}\ddot{\psi} + K_{7l}\ddot{q}_{w} + K_{7l}\ddot{q}_{imu} + \ddot{q}_{torso}s_{1l}) + m_{4l}(K_{26l} + K_{12l}\ddot{\psi} + K_{14l}\ddot{q}_{w} + K_{13l}\ddot{q}_{imu} + K_{15l}\ddot{q}_{torso} + K_{11l}\ddot{x}) - K_{2l}(K_{2l}\mathbf{MX}_{4l} - K_{1l}\mathbf{MY}_{4l}) - K_{3l}(K_{3l}\mathbf{MX}_{4l} - K_{1l}\mathbf{MZ}_{4l}) & \mathbf{MX}_{4l}(K_{25l} + K_{6l}\ddot{\psi} + K_{7l}\ddot{q}_{w} + K_{7l}\ddot{q}_{imu} + \ddot{q}_{torso}s_{1l}) - \mathbf{MZ}_{4l}(K_{24l} + K_{4l}\ddot{\psi} + K_{5l}\ddot{q}_{w} + K_{5l}\ddot{q}_{imu} - \ddot{q}_{torso}c_{1l}) + m_{4l}(K_{27l} + K_{16l}\ddot{\psi} + K_{18l}\ddot{q}_{w} + K_{17l}\ddot{q}_{imu} + K_{33}\ddot{x}) + K_{1l}(K_{2l}\mathbf{MX}_{4l} - K_{1l}\mathbf{MY}_{4l}) - K_{3l}(K_{3l}\mathbf{MY}_{4l} - K_{2l}\mathbf{MZ}_{4l}) & \mathbf{MY}_{4l}(K_{24l} + K_{4l}\ddot{\psi} + K_{5l}\ddot{q}_{w} + K_{5l}\ddot{q}_{imu} - \ddot{q}_{torso}c_{1l}) - \mathbf{MX}_{4l}(K_{42} - \ddot{q}_{1l} + K_{28}\ddot{\psi} + \ddot{q}_{w}c_{torso} + \ddot{q}_{imu}c_{torso}) + m_{4l}(K_{28l} + K_{20l}\ddot{\psi} + K_{22l}\ddot{q}_{w} + K_{21l}\ddot{q}_{imu} + K_{23l}\ddot{q}_{torso} + K_{19l}\ddot{x}) + K_{1l}(K_{3l}\mathbf{MX}_{4l} - K_{1l}\mathbf{MZ}_{4l}) + K_{2l}(K_{3l}\mathbf{MY}_{4l} - K_{2l}\mathbf{MZ}_{4l}) &  \end{matrix}\right] 
 \nonumber \\ 
D_{4l} &= K_{11l}m_{4l} \nonumber \\
D_{5l} &= K_{12l}m_{4l} - K_{6l}\mathbf{MY}_{4l} + K_{28}\mathbf{MZ}_{4l} \nonumber \\
D_{6l} &= K_{13l}m_{4l} + \mathbf{MZ}_{4l}c_{torso} - K_{7l}\mathbf{MY}_{4l} \nonumber \\
D_{7l} &= K_{14l}m_{4l} + \mathbf{MZ}_{4l}c_{torso} - K_{7l}\mathbf{MY}_{4l} \nonumber \\
D_{8l} &= K_{15l}m_{4l} - \mathbf{MY}_{4l}s_{1l} \nonumber \\
D_{9l} &= K_{26l}m_{4l} - K_{2l}^2\mathbf{MX}_{4l} - K_{3l}^2\mathbf{MX}_{4l}  \nonumber \\
&- K_{25l}\mathbf{MY}_{4l} + K_{42}\mathbf{MZ}_{4l} + K_{1l}K_{2l}\mathbf{MY}_{4l}  \nonumber \\
&+ K_{1l}K_{3l}\mathbf{MZ}_{4l} \nonumber \\
D_{10l} &= K_{33}m_{4l} \nonumber \\
D_{11l} &= K_{16l}m_{4l} + K_{6l}\mathbf{MX}_{4l} - K_{4l}\mathbf{MZ}_{4l} \nonumber \\
D_{12l} &= K_{17l}m_{4l} + K_{7l}\mathbf{MX}_{4l} - K_{5l}\mathbf{MZ}_{4l} \nonumber \\
D_{13l} &= K_{18l}m_{4l} + K_{7l}\mathbf{MX}_{4l} - K_{5l}\mathbf{MZ}_{4l} \nonumber \\
D_{14l} &= \mathbf{MZ}_{4l}c_{1l} + \mathbf{MX}_{4l}s_{1l} \nonumber \\
D_{15l} &= K_{27l}m_{4l} - K_{1l}^2\mathbf{MY}_{4l} - K_{3l}^2\mathbf{MY}_{4l}  \nonumber \\
&+ K_{25l}\mathbf{MX}_{4l} - K_{24l}\mathbf{MZ}_{4l} + K_{1l}K_{2l}\mathbf{MX}_{4l}  \nonumber \\
&+ K_{2l}K_{3l}\mathbf{MZ}_{4l} \nonumber \\
D_{16l} &= K_{19l}m_{4l} \nonumber \\
D_{17l} &= K_{20l}m_{4l} - K_{28}\mathbf{MX}_{4l} + K_{4l}\mathbf{MY}_{4l} \nonumber \\
D_{18l} &= K_{21l}m_{4l} - \mathbf{MX}_{4l}c_{torso} + K_{5l}\mathbf{MY}_{4l} \nonumber \\
D_{19l} &= K_{22l}m_{4l} - \mathbf{MX}_{4l}c_{torso} + K_{5l}\mathbf{MY}_{4l} \nonumber \\
D_{20l} &= K_{23l}m_{4l} - \mathbf{MY}_{4l}c_{1l} \nonumber \\
D_{21l} &= K_{28l}m_{4l} - K_{1l}^2\mathbf{MZ}_{4l} - K_{2l}^2\mathbf{MZ}_{4l}  \nonumber \\
&- K_{42}\mathbf{MX}_{4l} + K_{24l}\mathbf{MY}_{4l} + K_{1l}K_{3l}\mathbf{MX}_{4l}  \nonumber \\
&+ K_{2l}K_{3l}\mathbf{MY}_{4l} \nonumber \\
 m_{4l}\bar{a}_{G(4l)} &= \left[\begin{matrix} D_{9l} + D_{5l}\ddot{\psi} + D_{7l}\ddot{q}_{w} + D_{6l}\ddot{q}_{imu} + D_{8l}\ddot{q}_{torso} + D_{4l}\ddot{x} - \mathbf{MZ}_{4l}\ddot{q}_{1l} & D_{15l} + D_{11l}\ddot{\psi} + D_{13l}\ddot{q}_{w} + D_{12l}\ddot{q}_{imu} + D_{14l}\ddot{q}_{torso} + D_{10l}\ddot{x} & D_{21l} + D_{17l}\ddot{\psi} + D_{19l}\ddot{q}_{w} + D_{18l}\ddot{q}_{imu} + D_{20l}\ddot{q}_{torso} + D_{16l}\ddot{x} + \mathbf{MX}_{4l}\ddot{q}_{1l} &  \end{matrix}\right] 
 \nonumber \\ 
 \dot{\bar{H}}_{4l} &= \mathbf{MS}_{4l} \times \bar{a}_{4l} + J_{4l}\bar{\alpha}_{4l} + \bar\omega_{4l} \times J_{4l}\bar{\omega}_{4l} 
 \nonumber \\ 
 \dot{\bar{H}}_{4l} &= \left[\begin{matrix} K_{2l}(K_{1l}\mathbf{XZ}_{4l} + K_{2l}\mathbf{YZ}_{4l} + K_{3l}\mathbf{ZZ}_{4l}) - K_{3l}(K_{1l}\mathbf{XY}_{4l} + K_{2l}\mathbf{YY}_{4l} + K_{3l}\mathbf{YZ}_{4l}) + \mathbf{XX}_{4l}(K_{24l} + K_{4l}\ddot{\psi} + K_{5l}\ddot{q}_{w} + K_{5l}\ddot{q}_{imu} - \ddot{q}_{torso}c_{1l}) + \mathbf{XZ}_{4l}(K_{25l} + K_{6l}\ddot{\psi} + K_{7l}\ddot{q}_{w} + K_{7l}\ddot{q}_{imu} + \ddot{q}_{torso}s_{1l}) + \mathbf{XY}_{4l}(K_{42} - \ddot{q}_{1l} + K_{28}\ddot{\psi} + \ddot{q}_{w}c_{torso} + \ddot{q}_{imu}c_{torso}) - \mathbf{MZ}_{4l}(K_{27l} + K_{16l}\ddot{\psi} + K_{18l}\ddot{q}_{w} + K_{17l}\ddot{q}_{imu} + K_{33}\ddot{x}) + \mathbf{MY}_{4l}(K_{28l} + K_{20l}\ddot{\psi} + K_{22l}\ddot{q}_{w} + K_{21l}\ddot{q}_{imu} + K_{23l}\ddot{q}_{torso} + K_{19l}\ddot{x}) & K_{3l}(K_{1l}\mathbf{XX}_{4l} + K_{2l}\mathbf{XY}_{4l} + K_{3l}\mathbf{XZ}_{4l}) - K_{1l}(K_{1l}\mathbf{XZ}_{4l} + K_{2l}\mathbf{YZ}_{4l} + K_{3l}\mathbf{ZZ}_{4l}) + \mathbf{XY}_{4l}(K_{24l} + K_{4l}\ddot{\psi} + K_{5l}\ddot{q}_{w} + K_{5l}\ddot{q}_{imu} - \ddot{q}_{torso}c_{1l}) + \mathbf{YZ}_{4l}(K_{25l} + K_{6l}\ddot{\psi} + K_{7l}\ddot{q}_{w} + K_{7l}\ddot{q}_{imu} + \ddot{q}_{torso}s_{1l}) + \mathbf{YY}_{4l}(K_{42} - \ddot{q}_{1l} + K_{28}\ddot{\psi} + \ddot{q}_{w}c_{torso} + \ddot{q}_{imu}c_{torso}) - \mathbf{MX}_{4l}(K_{28l} + K_{20l}\ddot{\psi} + K_{22l}\ddot{q}_{w} + K_{21l}\ddot{q}_{imu} + K_{23l}\ddot{q}_{torso} + K_{19l}\ddot{x}) + \mathbf{MZ}_{4l}(K_{26l} + K_{12l}\ddot{\psi} + K_{14l}\ddot{q}_{w} + K_{13l}\ddot{q}_{imu} + K_{15l}\ddot{q}_{torso} + K_{11l}\ddot{x}) & K_{1l}(K_{1l}\mathbf{XY}_{4l} + K_{2l}\mathbf{YY}_{4l} + K_{3l}\mathbf{YZ}_{4l}) - K_{2l}(K_{1l}\mathbf{XX}_{4l} + K_{2l}\mathbf{XY}_{4l} + K_{3l}\mathbf{XZ}_{4l}) + \mathbf{XZ}_{4l}(K_{24l} + K_{4l}\ddot{\psi} + K_{5l}\ddot{q}_{w} + K_{5l}\ddot{q}_{imu} - \ddot{q}_{torso}c_{1l}) + \mathbf{ZZ}_{4l}(K_{25l} + K_{6l}\ddot{\psi} + K_{7l}\ddot{q}_{w} + K_{7l}\ddot{q}_{imu} + \ddot{q}_{torso}s_{1l}) + \mathbf{YZ}_{4l}(K_{42} - \ddot{q}_{1l} + K_{28}\ddot{\psi} + \ddot{q}_{w}c_{torso} + \ddot{q}_{imu}c_{torso}) + \mathbf{MX}_{4l}(K_{27l} + K_{16l}\ddot{\psi} + K_{18l}\ddot{q}_{w} + K_{17l}\ddot{q}_{imu} + K_{33}\ddot{x}) - \mathbf{MY}_{4l}(K_{26l} + K_{12l}\ddot{\psi} + K_{14l}\ddot{q}_{w} + K_{13l}\ddot{q}_{imu} + K_{15l}\ddot{q}_{torso} + K_{11l}\ddot{x}) &  \end{matrix}\right] 
 \nonumber \\ 
D_{22l} &= K_{19l}\mathbf{MY}_{4l} - K_{33}\mathbf{MZ}_{4l} \nonumber \\
D_{23l} &= K_{4l}\mathbf{XX}_{4l} + K_{28}\mathbf{XY}_{4l} + K_{6l}\mathbf{XZ}_{4l}  \nonumber \\
&+ K_{20l}\mathbf{MY}_{4l} - K_{16l}\mathbf{MZ}_{4l} \nonumber \\
D_{24l} &= K_{5l}\mathbf{XX}_{4l} + K_{7l}\mathbf{XZ}_{4l} + \mathbf{XY}_{4l}c_{torso}  \nonumber \\
&+ K_{21l}\mathbf{MY}_{4l} - K_{17l}\mathbf{MZ}_{4l} \nonumber \\
D_{25l} &= K_{5l}\mathbf{XX}_{4l} + K_{7l}\mathbf{XZ}_{4l} + \mathbf{XY}_{4l}c_{torso}  \nonumber \\
&+ K_{22l}\mathbf{MY}_{4l} - K_{18l}\mathbf{MZ}_{4l} \nonumber \\
D_{26l} &= \mathbf{XZ}_{4l}s_{1l} - \mathbf{XX}_{4l}c_{1l} + K_{23l}\mathbf{MY}_{4l} \nonumber \\
D_{27l} &= K_{24l}\mathbf{XX}_{4l} + K_{42}\mathbf{XY}_{4l} + K_{25l}\mathbf{XZ}_{4l}  \nonumber \\
&+ K_{2l}^2\mathbf{YZ}_{4l} - K_{3l}^2\mathbf{YZ}_{4l} + K_{28l}\mathbf{MY}_{4l}  \nonumber \\
&- K_{27l}\mathbf{MZ}_{4l} - K_{1l}K_{3l}\mathbf{XY}_{4l} + K_{1l}K_{2l}\mathbf{XZ}_{4l}  \nonumber \\
&- K_{2l}K_{3l}\mathbf{YY}_{4l} + K_{2l}K_{3l}\mathbf{ZZ}_{4l} \nonumber \\
D_{28l} &= K_{11l}\mathbf{MZ}_{4l} - K_{19l}\mathbf{MX}_{4l} \nonumber \\
D_{29l} &= K_{4l}\mathbf{XY}_{4l} + K_{28}\mathbf{YY}_{4l} + K_{6l}\mathbf{YZ}_{4l}  \nonumber \\
&- K_{20l}\mathbf{MX}_{4l} + K_{12l}\mathbf{MZ}_{4l} \nonumber \\
D_{30l} &= K_{5l}\mathbf{XY}_{4l} + K_{7l}\mathbf{YZ}_{4l} + \mathbf{YY}_{4l}c_{torso}  \nonumber \\
&- K_{21l}\mathbf{MX}_{4l} + K_{13l}\mathbf{MZ}_{4l} \nonumber \\
D_{31l} &= K_{5l}\mathbf{XY}_{4l} + K_{7l}\mathbf{YZ}_{4l} + \mathbf{YY}_{4l}c_{torso}  \nonumber \\
&- K_{22l}\mathbf{MX}_{4l} + K_{14l}\mathbf{MZ}_{4l} \nonumber \\
D_{32l} &= \mathbf{YZ}_{4l}s_{1l} - \mathbf{XY}_{4l}c_{1l} - K_{23l}\mathbf{MX}_{4l}  \nonumber \\
&+ K_{15l}\mathbf{MZ}_{4l} \nonumber \\
D_{33l} &= K_{24l}\mathbf{XY}_{4l} + K_{42}\mathbf{YY}_{4l} + K_{25l}\mathbf{YZ}_{4l}  \nonumber \\
&- K_{1l}^2\mathbf{XZ}_{4l} + K_{3l}^2\mathbf{XZ}_{4l} - K_{28l}\mathbf{MX}_{4l}  \nonumber \\
&+ K_{26l}\mathbf{MZ}_{4l} + K_{1l}K_{3l}\mathbf{XX}_{4l} + K_{2l}K_{3l}\mathbf{XY}_{4l}  \nonumber \\
&- K_{1l}K_{2l}\mathbf{YZ}_{4l} - K_{1l}K_{3l}\mathbf{ZZ}_{4l} \nonumber \\
D_{34l} &= K_{33}\mathbf{MX}_{4l} - K_{11l}\mathbf{MY}_{4l} \nonumber \\
D_{35l} &= K_{4l}\mathbf{XZ}_{4l} + K_{28}\mathbf{YZ}_{4l} + K_{6l}\mathbf{ZZ}_{4l}  \nonumber \\
&+ K_{16l}\mathbf{MX}_{4l} - K_{12l}\mathbf{MY}_{4l} \nonumber \\
D_{36l} &= K_{5l}\mathbf{XZ}_{4l} + K_{7l}\mathbf{ZZ}_{4l} + \mathbf{YZ}_{4l}c_{torso}  \nonumber \\
&+ K_{17l}\mathbf{MX}_{4l} - K_{13l}\mathbf{MY}_{4l} \nonumber \\
D_{37l} &= K_{5l}\mathbf{XZ}_{4l} + K_{7l}\mathbf{ZZ}_{4l} + \mathbf{YZ}_{4l}c_{torso}  \nonumber \\
&+ K_{18l}\mathbf{MX}_{4l} - K_{14l}\mathbf{MY}_{4l} \nonumber \\
D_{38l} &= \mathbf{ZZ}_{4l}s_{1l} - \mathbf{XZ}_{4l}c_{1l} - K_{15l}\mathbf{MY}_{4l} \nonumber \\
D_{39l} &= K_{24l}\mathbf{XZ}_{4l} + K_{42}\mathbf{YZ}_{4l} + K_{25l}\mathbf{ZZ}_{4l}  \nonumber \\
&+ K_{1l}^2\mathbf{XY}_{4l} - K_{2l}^2\mathbf{XY}_{4l} + K_{27l}\mathbf{MX}_{4l}  \nonumber \\
&- K_{26l}\mathbf{MY}_{4l} - K_{1l}K_{2l}\mathbf{XX}_{4l} - K_{2l}K_{3l}\mathbf{XZ}_{4l}  \nonumber \\
&+ K_{1l}K_{2l}\mathbf{YY}_{4l} + K_{1l}K_{3l}\mathbf{YZ}_{4l} \nonumber \\
 \dot{\bar{H}}_{4l} &= \left[\begin{matrix} D_{9l} + D_{5l}\ddot{\psi} + D_{7l}\ddot{q}_{w} + D_{6l}\ddot{q}_{imu} + D_{8l}\ddot{q}_{torso} + D_{4l}\ddot{x} - \mathbf{MZ}_{4l}\ddot{q}_{1l} & D_{15l} + D_{11l}\ddot{\psi} + D_{13l}\ddot{q}_{w} + D_{12l}\ddot{q}_{imu} + D_{14l}\ddot{q}_{torso} + D_{10l}\ddot{x} & D_{21l} + D_{17l}\ddot{\psi} + D_{19l}\ddot{q}_{w} + D_{18l}\ddot{q}_{imu} + D_{20l}\ddot{q}_{torso} + D_{16l}\ddot{x} + \mathbf{MX}_{4l}\ddot{q}_{1l} &  \end{matrix}\right] 
 \nonumber \\ 
 \bar\omega_{5l} &= {}^{5l}A_{4l} \bar\omega_{4l} + \dot{q}_{5l} \bar{e}_{5l} 
 \nonumber \\ 
 \bar\omega_{5l} &= \left[\begin{matrix} - K_{1l} - \dot{q}_{2l} & - K_{2l}c_{2l} - K_{3l}s_{2l} & K_{3l}c_{2l} - K_{2l}s_{2l} &  \end{matrix}\right] 
 \nonumber \\ 
K_{31l} &= - K_{1l} - \dot{q}_{2l} \nonumber \\
K_{32l} &= - K_{2l}c_{2l} - K_{3l}s_{2l} \nonumber \\
K_{33l} &= K_{3l}c_{2l} - K_{2l}s_{2l} \nonumber \\
 \bar\omega_{5l} &= \left[\begin{matrix} K_{31l} & K_{32l} & K_{33l} &  \end{matrix}\right] 
 \nonumber \\ 
 \bar\omega_{5l} &= \left[\begin{matrix} \dot{q}_{torso}c_{1l} - K_{4l}\dot{\psi} - K_{5l}\dot{q}_{w} - K_{5l}\dot{q}_{imu} - \dot{q}_{2l} & - s_{2l}(K_{6l}\dot{\psi} + K_{7l}\dot{q}_{w} + K_{7l}\dot{q}_{imu} + \dot{q}_{torso}s_{1l}) - c_{2l}(K_{28}\dot{\psi} - \dot{q}_{1l} + \dot{q}_{w}c_{torso} + \dot{q}_{imu}c_{torso}) & c_{2l}(K_{6l}\dot{\psi} + K_{7l}\dot{q}_{w} + K_{7l}\dot{q}_{imu} + \dot{q}_{torso}s_{1l}) - s_{2l}(K_{28}\dot{\psi} - \dot{q}_{1l} + \dot{q}_{w}c_{torso} + \dot{q}_{imu}c_{torso}) &  \end{matrix}\right] 
 \nonumber \\ 
K_{34l} &= - K_{28}c_{2l} - K_{6l}s_{2l} \nonumber \\
K_{35l} &= - c_{2l}c_{torso} - K_{7l}s_{2l} \nonumber \\
K_{36l} &= -s_{1l}s_{2l} \nonumber \\
K_{37l} &= K_{6l}c_{2l} - K_{28}s_{2l} \nonumber \\
K_{38l} &= K_{7l}c_{2l} - c_{torso}s_{2l} \nonumber \\
K_{39l} &= c_{2l}s_{1l} \nonumber \\
 \bar\omega_{5l} &= \left[\begin{matrix} \dot{q}_{torso}c_{1l} - K_{4l}\dot{\psi} - K_{5l}\dot{q}_{w} - K_{5l}\dot{q}_{imu} - \dot{q}_{2l} & K_{34l}\dot{\psi} + K_{35l}\dot{q}_{w} + K_{35l}\dot{q}_{imu} + K_{36l}\dot{q}_{torso} + \dot{q}_{1l}c_{2l} & K_{37l}\dot{\psi} + K_{38l}\dot{q}_{w} + K_{38l}\dot{q}_{imu} + K_{39l}\dot{q}_{torso} + \dot{q}_{1l}s_{2l} &  \end{matrix}\right] 
 \nonumber \\ 
 \bar{v}_{5l} &= {}^{5l}A_{4l} \left(\bar{v}_{4l} + \bar\omega_{4l} \times \bar{P}_{5l}\right) 
 \nonumber \\ 
 \bar{v}_{5l} &= \left[\begin{matrix} -K_{8l} & - K_{9l}c_{2l} - K_{10l}s_{2l} & K_{10l}c_{2l} - K_{9l}s_{2l} &  \end{matrix}\right] 
 \nonumber \\ 
K_{40l} &= - K_{9l}c_{2l} - K_{10l}s_{2l} \nonumber \\
K_{41l} &= K_{10l}c_{2l} - K_{9l}s_{2l} \nonumber \\
 \bar{v}_{5l} &= \left[\begin{matrix} -K_{8l} & K_{40l} & K_{41l} &  \end{matrix}\right] 
 \nonumber \\ 
 \bar{v}_{5l} &= \left[\begin{matrix} - K_{12l}\dot{\psi} - K_{14l}\dot{q}_{w} - K_{13l}\dot{q}_{imu} - K_{15l}\dot{q}_{torso} - K_{11l}\dot{x} & - c_{2l}(K_{16l}\dot{\psi} + K_{18l}\dot{q}_{w} + K_{17l}\dot{q}_{imu} + K_{33}\dot{x}) - s_{2l}(K_{20l}\dot{\psi} + K_{22l}\dot{q}_{w} + K_{21l}\dot{q}_{imu} + K_{23l}\dot{q}_{torso} + K_{19l}\dot{x}) & c_{2l}(K_{20l}\dot{\psi} + K_{22l}\dot{q}_{w} + K_{21l}\dot{q}_{imu} + K_{23l}\dot{q}_{torso} + K_{19l}\dot{x}) - s_{2l}(K_{16l}\dot{\psi} + K_{18l}\dot{q}_{w} + K_{17l}\dot{q}_{imu} + K_{33}\dot{x}) &  \end{matrix}\right] 
 \nonumber \\ 
K_{42l} &= - K_{33}c_{2l} - K_{19l}s_{2l} \nonumber \\
K_{43l} &= - K_{16l}c_{2l} - K_{20l}s_{2l} \nonumber \\
K_{44l} &= - K_{17l}c_{2l} - K_{21l}s_{2l} \nonumber \\
K_{45l} &= - K_{18l}c_{2l} - K_{22l}s_{2l} \nonumber \\
K_{46l} &= -K_{23l}s_{2l} \nonumber \\
K_{47l} &= K_{19l}c_{2l} - K_{33}s_{2l} \nonumber \\
K_{48l} &= K_{20l}c_{2l} - K_{16l}s_{2l} \nonumber \\
K_{49l} &= K_{21l}c_{2l} - K_{17l}s_{2l} \nonumber \\
K_{50l} &= K_{22l}c_{2l} - K_{18l}s_{2l} \nonumber \\
K_{51l} &= K_{23l}c_{2l} \nonumber \\
 \bar{v}_{5l} &= \left[\begin{matrix} - K_{12l}\dot{\psi} - K_{14l}\dot{q}_{w} - K_{13l}\dot{q}_{imu} - K_{15l}\dot{q}_{torso} - K_{11l}\dot{x} & K_{43l}\dot{\psi} + K_{45l}\dot{q}_{w} + K_{44l}\dot{q}_{imu} + K_{46l}\dot{q}_{torso} + K_{42l}\dot{x} & K_{48l}\dot{\psi} + K_{50l}\dot{q}_{w} + K_{49l}\dot{q}_{imu} + K_{51l}\dot{q}_{torso} + K_{47l}\dot{x} &  \end{matrix}\right] 
 \nonumber \\ 
 \bar\alpha_{5l} &= {}^{5l}A_{4l} \bar\alpha_{4l} + \ddot{q}_{5l} \bar{e}_{5l} + \dot{q}_{5l} \left(\bar\omega_{5l} \times \bar{e}_{5l}\right) 
 \nonumber \\ 
 \bar\alpha_{5l} &= \left[\begin{matrix} \ddot{q}_{torso}c_{1l} - \ddot{q}_{2l} - K_{4l}\ddot{\psi} - K_{5l}\ddot{q}_{w} - K_{5l}\ddot{q}_{imu} - K_{24l} & - K_{33l}\dot{q}_{2l} - s_{2l}(K_{25l} + K_{6l}\ddot{\psi} + K_{7l}\ddot{q}_{w} + K_{7l}\ddot{q}_{imu} + \ddot{q}_{torso}s_{1l}) - c_{2l}(K_{42} - \ddot{q}_{1l} + K_{28}\ddot{\psi} + \ddot{q}_{w}c_{torso} + \ddot{q}_{imu}c_{torso}) & K_{32l}\dot{q}_{2l} + c_{2l}(K_{25l} + K_{6l}\ddot{\psi} + K_{7l}\ddot{q}_{w} + K_{7l}\ddot{q}_{imu} + \ddot{q}_{torso}s_{1l}) - s_{2l}(K_{42} - \ddot{q}_{1l} + K_{28}\ddot{\psi} + \ddot{q}_{w}c_{torso} + \ddot{q}_{imu}c_{torso}) &  \end{matrix}\right] 
 \nonumber \\ 
K_{52l} &= - K_{33l}\dot{q}_{2l} - K_{42}c_{2l} - K_{25l}s_{2l} \nonumber \\
K_{53l} &= K_{32l}\dot{q}_{2l} + K_{25l}c_{2l} - K_{42}s_{2l} \nonumber \\
 \bar\alpha_{5l} &= \left[\begin{matrix} \ddot{q}_{torso}c_{1l} - \ddot{q}_{2l} - K_{4l}\ddot{\psi} - K_{5l}\ddot{q}_{w} - K_{5l}\ddot{q}_{imu} - K_{24l} & K_{52l} + K_{34l}\ddot{\psi} + K_{35l}\ddot{q}_{w} + K_{35l}\ddot{q}_{imu} + K_{36l}\ddot{q}_{torso} + \ddot{q}_{1l}c_{2l} & K_{53l} + K_{37l}\ddot{\psi} + K_{38l}\ddot{q}_{w} + K_{38l}\ddot{q}_{imu} + K_{39l}\ddot{q}_{torso} + \ddot{q}_{1l}s_{2l} &  \end{matrix}\right] 
 \nonumber \\ 
 \bar{a}_{5l} &= {}^{5l}A_{4l} \left(\bar{a}_{4l} + \bar\alpha_{4l} \times \bar{P}_{5l} + \bar\omega_{4l} \times \left(\bar\omega_{4l} \times \bar{P}_{5l}\right)\right) 
 \nonumber \\ 
 \bar\alpha_{5l} &= \left[\begin{matrix} - K_{26l} - K_{12l}\ddot{\psi} - K_{14l}\ddot{q}_{w} - K_{13l}\ddot{q}_{imu} - K_{15l}\ddot{q}_{torso} - K_{11l}\ddot{x} & - c_{2l}(K_{27l} + K_{16l}\ddot{\psi} + K_{18l}\ddot{q}_{w} + K_{17l}\ddot{q}_{imu} + K_{33}\ddot{x}) - s_{2l}(K_{28l} + K_{20l}\ddot{\psi} + K_{22l}\ddot{q}_{w} + K_{21l}\ddot{q}_{imu} + K_{23l}\ddot{q}_{torso} + K_{19l}\ddot{x}) & c_{2l}(K_{28l} + K_{20l}\ddot{\psi} + K_{22l}\ddot{q}_{w} + K_{21l}\ddot{q}_{imu} + K_{23l}\ddot{q}_{torso} + K_{19l}\ddot{x}) - s_{2l}(K_{27l} + K_{16l}\ddot{\psi} + K_{18l}\ddot{q}_{w} + K_{17l}\ddot{q}_{imu} + K_{33}\ddot{x}) &  \end{matrix}\right] 
 \nonumber \\ 
K_{54l} &= - K_{27l}c_{2l} - K_{28l}s_{2l} \nonumber \\
K_{55l} &= K_{28l}c_{2l} - K_{27l}s_{2l} \nonumber \\
 \bar{a}_{5l} &= \left[\begin{matrix} - K_{26l} - K_{12l}\ddot{\psi} - K_{14l}\ddot{q}_{w} - K_{13l}\ddot{q}_{imu} - K_{15l}\ddot{q}_{torso} - K_{11l}\ddot{x} & K_{54l} + K_{43l}\ddot{\psi} + K_{45l}\ddot{q}_{w} + K_{44l}\ddot{q}_{imu} + K_{46l}\ddot{q}_{torso} + K_{42l}\ddot{x} & K_{55l} + K_{48l}\ddot{\psi} + K_{50l}\ddot{q}_{w} + K_{49l}\ddot{q}_{imu} + K_{51l}\ddot{q}_{torso} + K_{47l}\ddot{x} &  \end{matrix}\right] 
 \nonumber \\ 
 \bar{g}_{5l} &= {}^{5l}A_{4l} \bar{g}_{4l} 
 \nonumber \\ 
 \bar{g}_{5l} &= \left[\begin{matrix} -K_{29l}g & - K_{47}gc_{2l} - K_{30l}gs_{2l} & K_{30l}gc_{2l} - K_{47}gs_{2l} &  \end{matrix}\right] 
 \nonumber \\ 
K_{56l} &= - K_{47}c_{2l} - K_{30l}s_{2l} \nonumber \\
K_{57l} &= K_{30l}c_{2l} - K_{47}s_{2l} \nonumber \\
 \bar{g}_{5l} &= \left[\begin{matrix} -K_{29l}g & K_{56l}g & K_{57l}g &  \end{matrix}\right] 
 \nonumber \\ 
 m_{5l}\bar{S}_{5l}^{\times}\bar{g}_{5l} &= \mathbf{MS}_{5l} \times \bar{g}_{5l} 
 \nonumber \\ 
 m_{5l}\bar{S}_{5l}^{\times}\bar{g}_{5l} &= \left[\begin{matrix} K_{57l}\mathbf{MY}_{5l}g - K_{56l}\mathbf{MZ}_{5l}g & - K_{57l}\mathbf{MX}_{5l}g - K_{29l}\mathbf{MZ}_{5l}g & K_{56l}\mathbf{MX}_{5l}g + K_{29l}\mathbf{MY}_{5l}g &  \end{matrix}\right] 
 \nonumber \\ 
D_{40l} &= K_{57l}\mathbf{MY}_{5l} - K_{56l}\mathbf{MZ}_{5l} \nonumber \\
D_{41l} &= - K_{57l}\mathbf{MX}_{5l} - K_{29l}\mathbf{MZ}_{5l} \nonumber \\
D_{42l} &= K_{56l}\mathbf{MX}_{5l} + K_{29l}\mathbf{MY}_{5l} \nonumber \\
 m_{5l}\bar{S}_{5l}^{\times}\bar{g}_{5l} &= \left[\begin{matrix} D_{40l}g & D_{41l}g & D_{42l}g &  \end{matrix}\right] 
 \nonumber \\ 
 m_{5l}\bar{a}_{G(5l)} &= m_{5l}\bar{a}_{5l} + \bar\alpha_{5l} \times \mathbf{MS}_{5l} + \bar\omega_{5l} \times \left(\bar\omega_{5l} \times \mathbf{MS}_{5l}\right) 
 \nonumber \\ 
 m_{5l}\bar{a}_{G(5l)} &= \left[\begin{matrix} \mathbf{MZ}_{5l}(K_{52l} + K_{34l}\ddot{\psi} + K_{35l}\ddot{q}_{w} + K_{35l}\ddot{q}_{imu} + K_{36l}\ddot{q}_{torso} + \ddot{q}_{1l}c_{2l}) - \mathbf{MY}_{5l}(K_{53l} + K_{37l}\ddot{\psi} + K_{38l}\ddot{q}_{w} + K_{38l}\ddot{q}_{imu} + K_{39l}\ddot{q}_{torso} + \ddot{q}_{1l}s_{2l}) - m_{5l}(K_{26l} + K_{12l}\ddot{\psi} + K_{14l}\ddot{q}_{w} + K_{13l}\ddot{q}_{imu} + K_{15l}\ddot{q}_{torso} + K_{11l}\ddot{x}) - K_{32l}(K_{32l}\mathbf{MX}_{5l} - K_{31l}\mathbf{MY}_{5l}) - K_{33l}(K_{33l}\mathbf{MX}_{5l} - K_{31l}\mathbf{MZ}_{5l}) & \mathbf{MX}_{5l}(K_{53l} + K_{37l}\ddot{\psi} + K_{38l}\ddot{q}_{w} + K_{38l}\ddot{q}_{imu} + K_{39l}\ddot{q}_{torso} + \ddot{q}_{1l}s_{2l}) + \mathbf{MZ}_{5l}(K_{24l} + \ddot{q}_{2l} + K_{4l}\ddot{\psi} + K_{5l}\ddot{q}_{w} + K_{5l}\ddot{q}_{imu} - \ddot{q}_{torso}c_{1l}) + m_{5l}(K_{54l} + K_{43l}\ddot{\psi} + K_{45l}\ddot{q}_{w} + K_{44l}\ddot{q}_{imu} + K_{46l}\ddot{q}_{torso} + K_{42l}\ddot{x}) + K_{31l}(K_{32l}\mathbf{MX}_{5l} - K_{31l}\mathbf{MY}_{5l}) - K_{33l}(K_{33l}\mathbf{MY}_{5l} - K_{32l}\mathbf{MZ}_{5l}) & m_{5l}(K_{55l} + K_{48l}\ddot{\psi} + K_{50l}\ddot{q}_{w} + K_{49l}\ddot{q}_{imu} + K_{51l}\ddot{q}_{torso} + K_{47l}\ddot{x}) - \mathbf{MY}_{5l}(K_{24l} + \ddot{q}_{2l} + K_{4l}\ddot{\psi} + K_{5l}\ddot{q}_{w} + K_{5l}\ddot{q}_{imu} - \ddot{q}_{torso}c_{1l}) - \mathbf{MX}_{5l}(K_{52l} + K_{34l}\ddot{\psi} + K_{35l}\ddot{q}_{w} + K_{35l}\ddot{q}_{imu} + K_{36l}\ddot{q}_{torso} + \ddot{q}_{1l}c_{2l}) + K_{31l}(K_{33l}\mathbf{MX}_{5l} - K_{31l}\mathbf{MZ}_{5l}) + K_{32l}(K_{33l}\mathbf{MY}_{5l} - K_{32l}\mathbf{MZ}_{5l}) &  \end{matrix}\right] 
 \nonumber \\ 
D_{43l} &= -K_{11l}m_{5l} \nonumber \\
D_{44l} &= K_{34l}\mathbf{MZ}_{5l} - K_{37l}\mathbf{MY}_{5l} - K_{12l}m_{5l} \nonumber \\
D_{45l} &= K_{35l}\mathbf{MZ}_{5l} - K_{38l}\mathbf{MY}_{5l} - K_{13l}m_{5l} \nonumber \\
D_{46l} &= K_{35l}\mathbf{MZ}_{5l} - K_{38l}\mathbf{MY}_{5l} - K_{14l}m_{5l} \nonumber \\
D_{47l} &= K_{36l}\mathbf{MZ}_{5l} - K_{39l}\mathbf{MY}_{5l} - K_{15l}m_{5l} \nonumber \\
D_{48l} &= \mathbf{MZ}_{5l}c_{2l} - \mathbf{MY}_{5l}s_{2l} \nonumber \\
D_{49l} &= K_{52l}\mathbf{MZ}_{5l} - K_{32l}^2\mathbf{MX}_{5l} - K_{33l}^2\mathbf{MX}_{5l}  \nonumber \\
&- K_{53l}\mathbf{MY}_{5l} - K_{26l}m_{5l} + K_{31l}K_{32l}\mathbf{MY}_{5l}  \nonumber \\
&+ K_{31l}K_{33l}\mathbf{MZ}_{5l} \nonumber \\
D_{50l} &= K_{42l}m_{5l} \nonumber \\
D_{51l} &= K_{43l}m_{5l} + K_{37l}\mathbf{MX}_{5l} + K_{4l}\mathbf{MZ}_{5l} \nonumber \\
D_{52l} &= K_{44l}m_{5l} + K_{38l}\mathbf{MX}_{5l} + K_{5l}\mathbf{MZ}_{5l} \nonumber \\
D_{53l} &= K_{45l}m_{5l} + K_{38l}\mathbf{MX}_{5l} + K_{5l}\mathbf{MZ}_{5l} \nonumber \\
D_{54l} &= K_{46l}m_{5l} - \mathbf{MZ}_{5l}c_{1l} + K_{39l}\mathbf{MX}_{5l} \nonumber \\
D_{55l} &= \mathbf{MX}_{5l}s_{2l} \nonumber \\
D_{56l} &= K_{54l}m_{5l} - K_{31l}^2\mathbf{MY}_{5l} - K_{33l}^2\mathbf{MY}_{5l}  \nonumber \\
&+ K_{53l}\mathbf{MX}_{5l} + K_{24l}\mathbf{MZ}_{5l} + K_{31l}K_{32l}\mathbf{MX}_{5l}  \nonumber \\
&+ K_{32l}K_{33l}\mathbf{MZ}_{5l} \nonumber \\
D_{57l} &= K_{47l}m_{5l} \nonumber \\
D_{58l} &= K_{48l}m_{5l} - K_{34l}\mathbf{MX}_{5l} - K_{4l}\mathbf{MY}_{5l} \nonumber \\
D_{59l} &= K_{49l}m_{5l} - K_{35l}\mathbf{MX}_{5l} - K_{5l}\mathbf{MY}_{5l} \nonumber \\
D_{60l} &= K_{50l}m_{5l} - K_{35l}\mathbf{MX}_{5l} - K_{5l}\mathbf{MY}_{5l} \nonumber \\
D_{61l} &= K_{51l}m_{5l} + \mathbf{MY}_{5l}c_{1l} - K_{36l}\mathbf{MX}_{5l} \nonumber \\
D_{62l} &= -\mathbf{MX}_{5l}c_{2l} \nonumber \\
D_{63l} &= K_{55l}m_{5l} - K_{31l}^2\mathbf{MZ}_{5l} - K_{32l}^2\mathbf{MZ}_{5l}  \nonumber \\
&- K_{52l}\mathbf{MX}_{5l} - K_{24l}\mathbf{MY}_{5l} + K_{31l}K_{33l}\mathbf{MX}_{5l}  \nonumber \\
&+ K_{32l}K_{33l}\mathbf{MY}_{5l} \nonumber \\
 m_{5l}\bar{a}_{G(5l)} &= \left[\begin{matrix} D_{49l} + D_{44l}\ddot{\psi} + D_{48l}\ddot{q}_{1l} + D_{46l}\ddot{q}_{w} + D_{45l}\ddot{q}_{imu} + D_{47l}\ddot{q}_{torso} + D_{43l}\ddot{x} & D_{56l} + D_{51l}\ddot{\psi} + D_{55l}\ddot{q}_{1l} + D_{53l}\ddot{q}_{w} + D_{52l}\ddot{q}_{imu} + D_{54l}\ddot{q}_{torso} + D_{50l}\ddot{x} + \mathbf{MZ}_{5l}\ddot{q}_{2l} & D_{63l} + D_{58l}\ddot{\psi} + D_{62l}\ddot{q}_{1l} + D_{60l}\ddot{q}_{w} + D_{59l}\ddot{q}_{imu} + D_{61l}\ddot{q}_{torso} + D_{57l}\ddot{x} - \mathbf{MY}_{5l}\ddot{q}_{2l} &  \end{matrix}\right] 
 \nonumber \\ 
 \dot{\bar{H}}_{5l} &= \mathbf{MS}_{5l} \times \bar{a}_{5l} + J_{5l}\bar{\alpha}_{5l} + \bar\omega_{5l} \times J_{5l}\bar{\omega}_{5l} 
 \nonumber \\ 
 \dot{\bar{H}}_{5l} &= \left[\begin{matrix} K_{32l}(K_{31l}\mathbf{XZ}_{5l} + K_{32l}\mathbf{YZ}_{5l} + K_{33l}\mathbf{ZZ}_{5l}) - K_{33l}(K_{31l}\mathbf{XY}_{5l} + K_{32l}\mathbf{YY}_{5l} + K_{33l}\mathbf{YZ}_{5l}) + \mathbf{XY}_{5l}(K_{52l} + K_{34l}\ddot{\psi} + K_{35l}\ddot{q}_{w} + K_{35l}\ddot{q}_{imu} + K_{36l}\ddot{q}_{torso} + \ddot{q}_{1l}c_{2l}) + \mathbf{XZ}_{5l}(K_{53l} + K_{37l}\ddot{\psi} + K_{38l}\ddot{q}_{w} + K_{38l}\ddot{q}_{imu} + K_{39l}\ddot{q}_{torso} + \ddot{q}_{1l}s_{2l}) - \mathbf{XX}_{5l}(K_{24l} + \ddot{q}_{2l} + K_{4l}\ddot{\psi} + K_{5l}\ddot{q}_{w} + K_{5l}\ddot{q}_{imu} - \ddot{q}_{torso}c_{1l}) + \mathbf{MY}_{5l}(K_{55l} + K_{48l}\ddot{\psi} + K_{50l}\ddot{q}_{w} + K_{49l}\ddot{q}_{imu} + K_{51l}\ddot{q}_{torso} + K_{47l}\ddot{x}) - \mathbf{MZ}_{5l}(K_{54l} + K_{43l}\ddot{\psi} + K_{45l}\ddot{q}_{w} + K_{44l}\ddot{q}_{imu} + K_{46l}\ddot{q}_{torso} + K_{42l}\ddot{x}) & K_{33l}(K_{31l}\mathbf{XX}_{5l} + K_{32l}\mathbf{XY}_{5l} + K_{33l}\mathbf{XZ}_{5l}) - K_{31l}(K_{31l}\mathbf{XZ}_{5l} + K_{32l}\mathbf{YZ}_{5l} + K_{33l}\mathbf{ZZ}_{5l}) + \mathbf{YY}_{5l}(K_{52l} + K_{34l}\ddot{\psi} + K_{35l}\ddot{q}_{w} + K_{35l}\ddot{q}_{imu} + K_{36l}\ddot{q}_{torso} + \ddot{q}_{1l}c_{2l}) + \mathbf{YZ}_{5l}(K_{53l} + K_{37l}\ddot{\psi} + K_{38l}\ddot{q}_{w} + K_{38l}\ddot{q}_{imu} + K_{39l}\ddot{q}_{torso} + \ddot{q}_{1l}s_{2l}) - \mathbf{XY}_{5l}(K_{24l} + \ddot{q}_{2l} + K_{4l}\ddot{\psi} + K_{5l}\ddot{q}_{w} + K_{5l}\ddot{q}_{imu} - \ddot{q}_{torso}c_{1l}) - \mathbf{MX}_{5l}(K_{55l} + K_{48l}\ddot{\psi} + K_{50l}\ddot{q}_{w} + K_{49l}\ddot{q}_{imu} + K_{51l}\ddot{q}_{torso} + K_{47l}\ddot{x}) - \mathbf{MZ}_{5l}(K_{26l} + K_{12l}\ddot{\psi} + K_{14l}\ddot{q}_{w} + K_{13l}\ddot{q}_{imu} + K_{15l}\ddot{q}_{torso} + K_{11l}\ddot{x}) & K_{31l}(K_{31l}\mathbf{XY}_{5l} + K_{32l}\mathbf{YY}_{5l} + K_{33l}\mathbf{YZ}_{5l}) - K_{32l}(K_{31l}\mathbf{XX}_{5l} + K_{32l}\mathbf{XY}_{5l} + K_{33l}\mathbf{XZ}_{5l}) + \mathbf{YZ}_{5l}(K_{52l} + K_{34l}\ddot{\psi} + K_{35l}\ddot{q}_{w} + K_{35l}\ddot{q}_{imu} + K_{36l}\ddot{q}_{torso} + \ddot{q}_{1l}c_{2l}) + \mathbf{ZZ}_{5l}(K_{53l} + K_{37l}\ddot{\psi} + K_{38l}\ddot{q}_{w} + K_{38l}\ddot{q}_{imu} + K_{39l}\ddot{q}_{torso} + \ddot{q}_{1l}s_{2l}) - \mathbf{XZ}_{5l}(K_{24l} + \ddot{q}_{2l} + K_{4l}\ddot{\psi} + K_{5l}\ddot{q}_{w} + K_{5l}\ddot{q}_{imu} - \ddot{q}_{torso}c_{1l}) + \mathbf{MX}_{5l}(K_{54l} + K_{43l}\ddot{\psi} + K_{45l}\ddot{q}_{w} + K_{44l}\ddot{q}_{imu} + K_{46l}\ddot{q}_{torso} + K_{42l}\ddot{x}) + \mathbf{MY}_{5l}(K_{26l} + K_{12l}\ddot{\psi} + K_{14l}\ddot{q}_{w} + K_{13l}\ddot{q}_{imu} + K_{15l}\ddot{q}_{torso} + K_{11l}\ddot{x}) &  \end{matrix}\right] 
 \nonumber \\ 
D_{64l} &= K_{47l}\mathbf{MY}_{5l} - K_{42l}\mathbf{MZ}_{5l} \nonumber \\
D_{65l} &= K_{34l}\mathbf{XY}_{5l} - K_{4l}\mathbf{XX}_{5l} + K_{37l}\mathbf{XZ}_{5l}  \nonumber \\
&+ K_{48l}\mathbf{MY}_{5l} - K_{43l}\mathbf{MZ}_{5l} \nonumber \\
D_{66l} &= K_{35l}\mathbf{XY}_{5l} - K_{5l}\mathbf{XX}_{5l} + K_{38l}\mathbf{XZ}_{5l}  \nonumber \\
&+ K_{49l}\mathbf{MY}_{5l} - K_{44l}\mathbf{MZ}_{5l} \nonumber \\
D_{67l} &= K_{35l}\mathbf{XY}_{5l} - K_{5l}\mathbf{XX}_{5l} + K_{38l}\mathbf{XZ}_{5l}  \nonumber \\
&+ K_{50l}\mathbf{MY}_{5l} - K_{45l}\mathbf{MZ}_{5l} \nonumber \\
D_{68l} &= K_{36l}\mathbf{XY}_{5l} + K_{39l}\mathbf{XZ}_{5l} + \mathbf{XX}_{5l}c_{1l}  \nonumber \\
&+ K_{51l}\mathbf{MY}_{5l} - K_{46l}\mathbf{MZ}_{5l} \nonumber \\
D_{69l} &= \mathbf{XY}_{5l}c_{2l} + \mathbf{XZ}_{5l}s_{2l} \nonumber \\
D_{70l} &= K_{52l}\mathbf{XY}_{5l} - K_{24l}\mathbf{XX}_{5l} + K_{53l}\mathbf{XZ}_{5l}  \nonumber \\
&+ K_{32l}^2\mathbf{YZ}_{5l} - K_{33l}^2\mathbf{YZ}_{5l} + K_{55l}\mathbf{MY}_{5l}  \nonumber \\
&- K_{54l}\mathbf{MZ}_{5l} - K_{31l}K_{33l}\mathbf{XY}_{5l} + K_{31l}K_{32l}\mathbf{XZ}_{5l}  \nonumber \\
&- K_{32l}K_{33l}\mathbf{YY}_{5l} + K_{32l}K_{33l}\mathbf{ZZ}_{5l} \nonumber \\
D_{71l} &= - K_{47l}\mathbf{MX}_{5l} - K_{11l}\mathbf{MZ}_{5l} \nonumber \\
D_{72l} &= K_{34l}\mathbf{YY}_{5l} - K_{4l}\mathbf{XY}_{5l} + K_{37l}\mathbf{YZ}_{5l}  \nonumber \\
&- K_{48l}\mathbf{MX}_{5l} - K_{12l}\mathbf{MZ}_{5l} \nonumber \\
D_{73l} &= K_{35l}\mathbf{YY}_{5l} - K_{5l}\mathbf{XY}_{5l} + K_{38l}\mathbf{YZ}_{5l}  \nonumber \\
&- K_{49l}\mathbf{MX}_{5l} - K_{13l}\mathbf{MZ}_{5l} \nonumber \\
D_{74l} &= K_{35l}\mathbf{YY}_{5l} - K_{5l}\mathbf{XY}_{5l} + K_{38l}\mathbf{YZ}_{5l}  \nonumber \\
&- K_{50l}\mathbf{MX}_{5l} - K_{14l}\mathbf{MZ}_{5l} \nonumber \\
D_{75l} &= K_{36l}\mathbf{YY}_{5l} + K_{39l}\mathbf{YZ}_{5l} + \mathbf{XY}_{5l}c_{1l}  \nonumber \\
&- K_{51l}\mathbf{MX}_{5l} - K_{15l}\mathbf{MZ}_{5l} \nonumber \\
D_{76l} &= \mathbf{YY}_{5l}c_{2l} + \mathbf{YZ}_{5l}s_{2l} \nonumber \\
D_{77l} &= K_{52l}\mathbf{YY}_{5l} - K_{24l}\mathbf{XY}_{5l} + K_{53l}\mathbf{YZ}_{5l}  \nonumber \\
&- K_{31l}^2\mathbf{XZ}_{5l} + K_{33l}^2\mathbf{XZ}_{5l} - K_{55l}\mathbf{MX}_{5l}  \nonumber \\
&- K_{26l}\mathbf{MZ}_{5l} + K_{31l}K_{33l}\mathbf{XX}_{5l} + K_{32l}K_{33l}\mathbf{XY}_{5l}  \nonumber \\
&- K_{31l}K_{32l}\mathbf{YZ}_{5l} - K_{31l}K_{33l}\mathbf{ZZ}_{5l} \nonumber \\
D_{78l} &= K_{42l}\mathbf{MX}_{5l} + K_{11l}\mathbf{MY}_{5l} \nonumber \\
D_{79l} &= K_{34l}\mathbf{YZ}_{5l} - K_{4l}\mathbf{XZ}_{5l} + K_{37l}\mathbf{ZZ}_{5l}  \nonumber \\
&+ K_{43l}\mathbf{MX}_{5l} + K_{12l}\mathbf{MY}_{5l} \nonumber \\
D_{80l} &= K_{35l}\mathbf{YZ}_{5l} - K_{5l}\mathbf{XZ}_{5l} + K_{38l}\mathbf{ZZ}_{5l}  \nonumber \\
&+ K_{44l}\mathbf{MX}_{5l} + K_{13l}\mathbf{MY}_{5l} \nonumber \\
D_{81l} &= K_{35l}\mathbf{YZ}_{5l} - K_{5l}\mathbf{XZ}_{5l} + K_{38l}\mathbf{ZZ}_{5l}  \nonumber \\
&+ K_{45l}\mathbf{MX}_{5l} + K_{14l}\mathbf{MY}_{5l} \nonumber \\
D_{82l} &= K_{36l}\mathbf{YZ}_{5l} + K_{39l}\mathbf{ZZ}_{5l} + \mathbf{XZ}_{5l}c_{1l}  \nonumber \\
&+ K_{46l}\mathbf{MX}_{5l} + K_{15l}\mathbf{MY}_{5l} \nonumber \\
D_{83l} &= \mathbf{YZ}_{5l}c_{2l} + \mathbf{ZZ}_{5l}s_{2l} \nonumber \\
D_{84l} &= K_{52l}\mathbf{YZ}_{5l} - K_{24l}\mathbf{XZ}_{5l} + K_{53l}\mathbf{ZZ}_{5l}  \nonumber \\
&+ K_{31l}^2\mathbf{XY}_{5l} - K_{32l}^2\mathbf{XY}_{5l} + K_{54l}\mathbf{MX}_{5l}  \nonumber \\
&+ K_{26l}\mathbf{MY}_{5l} - K_{31l}K_{32l}\mathbf{XX}_{5l} - K_{32l}K_{33l}\mathbf{XZ}_{5l}  \nonumber \\
&+ K_{31l}K_{32l}\mathbf{YY}_{5l} + K_{31l}K_{33l}\mathbf{YZ}_{5l} \nonumber \\
 \dot{\bar{H}}_{5l} &= \left[\begin{matrix} D_{49l} + D_{44l}\ddot{\psi} + D_{48l}\ddot{q}_{1l} + D_{46l}\ddot{q}_{w} + D_{45l}\ddot{q}_{imu} + D_{47l}\ddot{q}_{torso} + D_{43l}\ddot{x} & D_{56l} + D_{51l}\ddot{\psi} + D_{55l}\ddot{q}_{1l} + D_{53l}\ddot{q}_{w} + D_{52l}\ddot{q}_{imu} + D_{54l}\ddot{q}_{torso} + D_{50l}\ddot{x} + \mathbf{MZ}_{5l}\ddot{q}_{2l} & D_{63l} + D_{58l}\ddot{\psi} + D_{62l}\ddot{q}_{1l} + D_{60l}\ddot{q}_{w} + D_{59l}\ddot{q}_{imu} + D_{61l}\ddot{q}_{torso} + D_{57l}\ddot{x} - \mathbf{MY}_{5l}\ddot{q}_{2l} &  \end{matrix}\right] 
 \nonumber \\ 
 \bar\omega_{6l} &= {}^{6l}A_{5l} \bar\omega_{5l} + \dot{q}_{6l} \bar{e}_{6l} 
 \nonumber \\ 
 \bar\omega_{6l} &= \left[\begin{matrix} K_{33l}s_{3l} - K_{31l}c_{3l} & - K_{32l} - \dot{q}_{3l} & K_{33l}c_{3l} + K_{31l}s_{3l} &  \end{matrix}\right] 
 \nonumber \\ 
K_{58l} &= K_{33l}s_{3l} - K_{31l}c_{3l} \nonumber \\
K_{59l} &= - K_{32l} - \dot{q}_{3l} \nonumber \\
K_{60l} &= K_{33l}c_{3l} + K_{31l}s_{3l} \nonumber \\
 \bar\omega_{6l} &= \left[\begin{matrix} K_{58l} & K_{59l} & K_{60l} &  \end{matrix}\right] 
 \nonumber \\ 
 \bar\omega_{6l} &= \left[\begin{matrix} s_{3l}(K_{37l}\dot{\psi} + K_{38l}\dot{q}_{w} + K_{38l}\dot{q}_{imu} + K_{39l}\dot{q}_{torso} + \dot{q}_{1l}s_{2l}) + c_{3l}(\dot{q}_{2l} + K_{4l}\dot{\psi} + K_{5l}\dot{q}_{w} + K_{5l}\dot{q}_{imu} - \dot{q}_{torso}c_{1l}) & - \dot{q}_{3l} - K_{34l}\dot{\psi} - K_{35l}\dot{q}_{w} - K_{35l}\dot{q}_{imu} - K_{36l}\dot{q}_{torso} - \dot{q}_{1l}c_{2l} & c_{3l}(K_{37l}\dot{\psi} + K_{38l}\dot{q}_{w} + K_{38l}\dot{q}_{imu} + K_{39l}\dot{q}_{torso} + \dot{q}_{1l}s_{2l}) - s_{3l}(\dot{q}_{2l} + K_{4l}\dot{\psi} + K_{5l}\dot{q}_{w} + K_{5l}\dot{q}_{imu} - \dot{q}_{torso}c_{1l}) &  \end{matrix}\right] 
 \nonumber \\ 
K_{61l} &= K_{4l}c_{3l} + K_{37l}s_{3l} \nonumber \\
K_{62l} &= K_{5l}c_{3l} + K_{38l}s_{3l} \nonumber \\
K_{63l} &= K_{39l}s_{3l} - c_{1l}c_{3l} \nonumber \\
K_{64l} &= s_{2l}s_{3l} \nonumber \\
K_{65l} &= K_{37l}c_{3l} - K_{4l}s_{3l} \nonumber \\
K_{66l} &= K_{38l}c_{3l} - K_{5l}s_{3l} \nonumber \\
K_{67l} &= c_{1l}s_{3l} + K_{39l}c_{3l} \nonumber \\
K_{68l} &= c_{3l}s_{2l} \nonumber \\
 \bar\omega_{6l} &= \left[\begin{matrix} K_{61l}\dot{\psi} + K_{64l}\dot{q}_{1l} + K_{62l}\dot{q}_{w} + K_{62l}\dot{q}_{imu} + K_{63l}\dot{q}_{torso} + \dot{q}_{2l}c_{3l} & - \dot{q}_{3l} - K_{34l}\dot{\psi} - K_{35l}\dot{q}_{w} - K_{35l}\dot{q}_{imu} - K_{36l}\dot{q}_{torso} - \dot{q}_{1l}c_{2l} & K_{65l}\dot{\psi} + K_{68l}\dot{q}_{1l} + K_{66l}\dot{q}_{w} + K_{66l}\dot{q}_{imu} + K_{67l}\dot{q}_{torso} - \dot{q}_{2l}s_{3l} &  \end{matrix}\right] 
 \nonumber \\ 
 \bar{v}_{6l} &= {}^{6l}A_{5l} \left(\bar{v}_{5l} + \bar\omega_{5l} \times \bar{P}_{6l}\right) 
 \nonumber \\ 
 \bar{v}_{6l} &= \left[\begin{matrix} c_{3l}(K_{8l} - K_{33l}L_7) + s_{3l}(K_{41l} - K_{31l}L_7) & -K_{40l} & c_{3l}(K_{41l} - K_{31l}L_7) - s_{3l}(K_{8l} - K_{33l}L_7) &  \end{matrix}\right] 
 \nonumber \\ 
K_{69l} &= c_{3l}(K_{8l} - K_{33l}L_7) + s_{3l}(K_{41l}  \nonumber \\
&- K_{31l}L_7) \nonumber \\
K_{70l} &= c_{3l}(K_{41l} - K_{31l}L_7) - s_{3l}(K_{8l}  \nonumber \\
&- K_{33l}L_7) \nonumber \\
 \bar{v}_{6l} &= \left[\begin{matrix} K_{69l} & -K_{40l} & K_{70l} &  \end{matrix}\right] 
 \nonumber \\ 
 \bar{v}_{6l} &= \left[\begin{matrix} s_{3l}(K_{48l}\dot{\psi} + K_{50l}\dot{q}_{w} + K_{49l}\dot{q}_{imu} + K_{51l}\dot{q}_{torso} + K_{47l}\dot{x} + L_7(\dot{q}_{2l} + K_{4l}\dot{\psi} + K_{5l}\dot{q}_{w} + K_{5l}\dot{q}_{imu} - \dot{q}_{torso}c_{1l})) + c_{3l}(K_{12l}\dot{\psi} + K_{14l}\dot{q}_{w} + K_{13l}\dot{q}_{imu} + K_{15l}\dot{q}_{torso} + K_{11l}\dot{x} - L_7(K_{37l}\dot{\psi} + K_{38l}\dot{q}_{w} + K_{38l}\dot{q}_{imu} + K_{39l}\dot{q}_{torso} + \dot{q}_{1l}s_{2l})) & - K_{43l}\dot{\psi} - K_{45l}\dot{q}_{w} - K_{44l}\dot{q}_{imu} - K_{46l}\dot{q}_{torso} - K_{42l}\dot{x} & c_{3l}(K_{48l}\dot{\psi} + K_{50l}\dot{q}_{w} + K_{49l}\dot{q}_{imu} + K_{51l}\dot{q}_{torso} + K_{47l}\dot{x} + L_7(\dot{q}_{2l} + K_{4l}\dot{\psi} + K_{5l}\dot{q}_{w} + K_{5l}\dot{q}_{imu} - \dot{q}_{torso}c_{1l})) - s_{3l}(K_{12l}\dot{\psi} + K_{14l}\dot{q}_{w} + K_{13l}\dot{q}_{imu} + K_{15l}\dot{q}_{torso} + K_{11l}\dot{x} - L_7(K_{37l}\dot{\psi} + K_{38l}\dot{q}_{w} + K_{38l}\dot{q}_{imu} + K_{39l}\dot{q}_{torso} + \dot{q}_{1l}s_{2l})) &  \end{matrix}\right] 
 \nonumber \\ 
K_{71l} &= K_{11l}c_{3l} + K_{47l}s_{3l} \nonumber \\
K_{72l} &= c_{3l}(K_{12l} - K_{37l}L_7) + s_{3l}(K_{48l}  \nonumber \\
&+ K_{4l}L_7) \nonumber \\
K_{73l} &= c_{3l}(K_{13l} - K_{38l}L_7) + s_{3l}(K_{49l}  \nonumber \\
&+ K_{5l}L_7) \nonumber \\
K_{74l} &= c_{3l}(K_{14l} - K_{38l}L_7) + s_{3l}(K_{50l}  \nonumber \\
&+ K_{5l}L_7) \nonumber \\
K_{75l} &= s_{3l}(K_{51l} - L_7c_{1l}) + c_{3l}(K_{15l}  \nonumber \\
&- K_{39l}L_7) \nonumber \\
K_{76l} &= -L_7c_{3l}s_{2l} \nonumber \\
K_{77l} &= L_7s_{3l} \nonumber \\
K_{78l} &= K_{47l}c_{3l} - K_{11l}s_{3l} \nonumber \\
K_{79l} &= c_{3l}(K_{48l} + K_{4l}L_7) - s_{3l}(K_{12l}  \nonumber \\
&- K_{37l}L_7) \nonumber \\
K_{80l} &= c_{3l}(K_{49l} + K_{5l}L_7) - s_{3l}(K_{13l}  \nonumber \\
&- K_{38l}L_7) \nonumber \\
K_{81l} &= c_{3l}(K_{50l} + K_{5l}L_7) - s_{3l}(K_{14l}  \nonumber \\
&- K_{38l}L_7) \nonumber \\
K_{82l} &= c_{3l}(K_{51l} - L_7c_{1l}) - s_{3l}(K_{15l}  \nonumber \\
&- K_{39l}L_7) \nonumber \\
K_{83l} &= L_7s_{2l}s_{3l} \nonumber \\
K_{84l} &= L_7c_{3l} \nonumber \\
 \bar{v}_{6l} &= \left[\begin{matrix} K_{72l}\dot{\psi} + K_{76l}\dot{q}_{1l} + K_{77l}\dot{q}_{2l} + K_{74l}\dot{q}_{w} + K_{73l}\dot{q}_{imu} + K_{75l}\dot{q}_{torso} + K_{71l}\dot{x} & - K_{43l}\dot{\psi} - K_{45l}\dot{q}_{w} - K_{44l}\dot{q}_{imu} - K_{46l}\dot{q}_{torso} - K_{42l}\dot{x} & K_{79l}\dot{\psi} + K_{83l}\dot{q}_{1l} + K_{84l}\dot{q}_{2l} + K_{81l}\dot{q}_{w} + K_{80l}\dot{q}_{imu} + K_{82l}\dot{q}_{torso} + K_{78l}\dot{x} &  \end{matrix}\right] 
 \nonumber \\ 
 \bar\alpha_{6l} &= {}^{6l}A_{5l} \bar\alpha_{5l} + \ddot{q}_{6l} \bar{e}_{6l} + \dot{q}_{6l} \left(\bar\omega_{6l} \times \bar{e}_{6l}\right) 
 \nonumber \\ 
 \bar\alpha_{6l} &= \left[\begin{matrix} c_{3l}(K_{24l} + \ddot{q}_{2l} + K_{4l}\ddot{\psi} + K_{5l}\ddot{q}_{w} + K_{5l}\ddot{q}_{imu} - \ddot{q}_{torso}c_{1l}) + K_{60l}\dot{q}_{3l} + s_{3l}(K_{53l} + K_{37l}\ddot{\psi} + K_{38l}\ddot{q}_{w} + K_{38l}\ddot{q}_{imu} + K_{39l}\ddot{q}_{torso} + \ddot{q}_{1l}s_{2l}) & - K_{52l} - \ddot{q}_{3l} - K_{34l}\ddot{\psi} - K_{35l}\ddot{q}_{w} - K_{35l}\ddot{q}_{imu} - K_{36l}\ddot{q}_{torso} - \ddot{q}_{1l}c_{2l} & c_{3l}(K_{53l} + K_{37l}\ddot{\psi} + K_{38l}\ddot{q}_{w} + K_{38l}\ddot{q}_{imu} + K_{39l}\ddot{q}_{torso} + \ddot{q}_{1l}s_{2l}) - s_{3l}(K_{24l} + \ddot{q}_{2l} + K_{4l}\ddot{\psi} + K_{5l}\ddot{q}_{w} + K_{5l}\ddot{q}_{imu} - \ddot{q}_{torso}c_{1l}) - K_{58l}\dot{q}_{3l} &  \end{matrix}\right] 
 \nonumber \\ 
K_{85l} &= K_{60l}\dot{q}_{3l} + K_{24l}c_{3l} + K_{53l}s_{3l} \nonumber \\
K_{86l} &= K_{53l}c_{3l} - K_{58l}\dot{q}_{3l} - K_{24l}s_{3l} \nonumber \\
 \bar\alpha_{6l} &= \left[\begin{matrix} K_{85l} + K_{61l}\ddot{\psi} + K_{64l}\ddot{q}_{1l} + K_{62l}\ddot{q}_{w} + K_{62l}\ddot{q}_{imu} + K_{63l}\ddot{q}_{torso} + \ddot{q}_{2l}c_{3l} & - K_{52l} - \ddot{q}_{3l} - K_{34l}\ddot{\psi} - K_{35l}\ddot{q}_{w} - K_{35l}\ddot{q}_{imu} - K_{36l}\ddot{q}_{torso} - \ddot{q}_{1l}c_{2l} & K_{86l} + K_{65l}\ddot{\psi} + K_{68l}\ddot{q}_{1l} + K_{66l}\ddot{q}_{w} + K_{66l}\ddot{q}_{imu} + K_{67l}\ddot{q}_{torso} - \ddot{q}_{2l}s_{3l} &  \end{matrix}\right] 
 \nonumber \\ 
 \bar{a}_{6l} &= {}^{6l}A_{5l} \left(\bar{a}_{5l} + \bar\alpha_{5l} \times \bar{P}_{6l} + \bar\omega_{5l} \times \left(\bar\omega_{5l} \times \bar{P}_{6l}\right)\right) 
 \nonumber \\ 
 \bar\alpha_{6l} &= \left[\begin{matrix} c_{3l}(K_{26l} + K_{12l}\ddot{\psi} + K_{14l}\ddot{q}_{w} + K_{13l}\ddot{q}_{imu} + K_{15l}\ddot{q}_{torso} + K_{11l}\ddot{x} - L_7(K_{53l} + K_{37l}\ddot{\psi} + K_{38l}\ddot{q}_{w} + K_{38l}\ddot{q}_{imu} + K_{39l}\ddot{q}_{torso} + \ddot{q}_{1l}s_{2l}) + K_{31l}K_{32l}L_7) + s_{3l}(K_{55l} + K_{48l}\ddot{\psi} + K_{50l}\ddot{q}_{w} + K_{49l}\ddot{q}_{imu} + K_{51l}\ddot{q}_{torso} + K_{47l}\ddot{x} + L_7(K_{24l} + \ddot{q}_{2l} + K_{4l}\ddot{\psi} + K_{5l}\ddot{q}_{w} + K_{5l}\ddot{q}_{imu} - \ddot{q}_{torso}c_{1l}) - K_{32l}K_{33l}L_7) & - K_{54l} - K_{43l}\ddot{\psi} - K_{45l}\ddot{q}_{w} - K_{44l}\ddot{q}_{imu} - K_{46l}\ddot{q}_{torso} - K_{42l}\ddot{x} - K_{31l}^2L_7 - K_{33l}^2L_7 & c_{3l}(K_{55l} + K_{48l}\ddot{\psi} + K_{50l}\ddot{q}_{w} + K_{49l}\ddot{q}_{imu} + K_{51l}\ddot{q}_{torso} + K_{47l}\ddot{x} + L_7(K_{24l} + \ddot{q}_{2l} + K_{4l}\ddot{\psi} + K_{5l}\ddot{q}_{w} + K_{5l}\ddot{q}_{imu} - \ddot{q}_{torso}c_{1l}) - K_{32l}K_{33l}L_7) - s_{3l}(K_{26l} + K_{12l}\ddot{\psi} + K_{14l}\ddot{q}_{w} + K_{13l}\ddot{q}_{imu} + K_{15l}\ddot{q}_{torso} + K_{11l}\ddot{x} - L_7(K_{53l} + K_{37l}\ddot{\psi} + K_{38l}\ddot{q}_{w} + K_{38l}\ddot{q}_{imu} + K_{39l}\ddot{q}_{torso} + \ddot{q}_{1l}s_{2l}) + K_{31l}K_{32l}L_7) &  \end{matrix}\right] 
 \nonumber \\ 
K_{87l} &= K_{26l}c_{3l} + K_{55l}s_{3l} - K_{53l}L_7c_{3l}  \nonumber \\
&+ K_{24l}L_7s_{3l} + K_{31l}K_{32l}L_7c_{3l}  \nonumber \\
&- K_{32l}K_{33l}L_7s_{3l} \nonumber \\
K_{88l} &= - K_{54l} - K_{31l}^2L_7 - K_{33l}^2L_7 \nonumber \\
K_{89l} &= K_{55l}c_{3l} - K_{26l}s_{3l} + K_{24l}L_7c_{3l}  \nonumber \\
&+ K_{53l}L_7s_{3l} - K_{32l}K_{33l}L_7c_{3l}  \nonumber \\
&- K_{31l}K_{32l}L_7s_{3l} \nonumber \\
 \bar{a}_{6l} &= \left[\begin{matrix} K_{87l} + K_{72l}\ddot{\psi} + K_{76l}\ddot{q}_{1l} + K_{77l}\ddot{q}_{2l} + K_{74l}\ddot{q}_{w} + K_{73l}\ddot{q}_{imu} + K_{75l}\ddot{q}_{torso} + K_{71l}\ddot{x} & K_{88l} - K_{43l}\ddot{\psi} - K_{45l}\ddot{q}_{w} - K_{44l}\ddot{q}_{imu} - K_{46l}\ddot{q}_{torso} - K_{42l}\ddot{x} & K_{89l} + K_{79l}\ddot{\psi} + K_{83l}\ddot{q}_{1l} + K_{84l}\ddot{q}_{2l} + K_{81l}\ddot{q}_{w} + K_{80l}\ddot{q}_{imu} + K_{82l}\ddot{q}_{torso} + K_{78l}\ddot{x} &  \end{matrix}\right] 
 \nonumber \\ 
 \bar{g}_{6l} &= {}^{6l}A_{5l} \bar{g}_{5l} 
 \nonumber \\ 
 \bar{g}_{6l} &= \left[\begin{matrix} K_{29l}gc_{3l} + K_{57l}gs_{3l} & -K_{56l}g & K_{57l}gc_{3l} - K_{29l}gs_{3l} &  \end{matrix}\right] 
 \nonumber \\ 
K_{90l} &= K_{29l}c_{3l} + K_{57l}s_{3l} \nonumber \\
K_{91l} &= K_{57l}c_{3l} - K_{29l}s_{3l} \nonumber \\
 \bar{g}_{6l} &= \left[\begin{matrix} K_{90l}g & -K_{56l}g & K_{91l}g &  \end{matrix}\right] 
 \nonumber \\ 
 m_{6l}\bar{S}_{6l}^{\times}\bar{g}_{6l} &= \mathbf{MS}_{6l} \times \bar{g}_{6l} 
 \nonumber \\ 
 m_{6l}\bar{S}_{6l}^{\times}\bar{g}_{6l} &= \left[\begin{matrix} K_{91l}\mathbf{MY}_{6l}g + K_{56l}\mathbf{MZ}_{6l}g & K_{90l}\mathbf{MZ}_{6l}g - K_{91l}\mathbf{MX}_{6l}g & - K_{56l}\mathbf{MX}_{6l}g - K_{90l}\mathbf{MY}_{6l}g &  \end{matrix}\right] 
 \nonumber \\ 
D_{85l} &= K_{91l}\mathbf{MY}_{6l} + K_{56l}\mathbf{MZ}_{6l} \nonumber \\
D_{86l} &= K_{90l}\mathbf{MZ}_{6l} - K_{91l}\mathbf{MX}_{6l} \nonumber \\
D_{87l} &= - K_{56l}\mathbf{MX}_{6l} - K_{90l}\mathbf{MY}_{6l} \nonumber \\
 m_{6l}\bar{S}_{6l}^{\times}\bar{g}_{6l} &= \left[\begin{matrix} D_{85l}g & D_{86l}g & D_{87l}g &  \end{matrix}\right] 
 \nonumber \\ 
 m_{6l}\bar{a}_{G(6l)} &= m_{6l}\bar{a}_{6l} + \bar\alpha_{6l} \times \mathbf{MS}_{6l} + \bar\omega_{6l} \times \left(\bar\omega_{6l} \times \mathbf{MS}_{6l}\right) 
 \nonumber \\ 
 m_{6l}\bar{a}_{G(6l)} &= \left[\begin{matrix} m_{6l}(K_{87l} + K_{72l}\ddot{\psi} + K_{76l}\ddot{q}_{1l} + K_{77l}\ddot{q}_{2l} + K_{74l}\ddot{q}_{w} + K_{73l}\ddot{q}_{imu} + K_{75l}\ddot{q}_{torso} + K_{71l}\ddot{x}) - \mathbf{MZ}_{6l}(K_{52l} + \ddot{q}_{3l} + K_{34l}\ddot{\psi} + K_{35l}\ddot{q}_{w} + K_{35l}\ddot{q}_{imu} + K_{36l}\ddot{q}_{torso} + \ddot{q}_{1l}c_{2l}) - \mathbf{MY}_{6l}(K_{86l} + K_{65l}\ddot{\psi} + K_{68l}\ddot{q}_{1l} + K_{66l}\ddot{q}_{w} + K_{66l}\ddot{q}_{imu} + K_{67l}\ddot{q}_{torso} - \ddot{q}_{2l}s_{3l}) - K_{59l}(K_{59l}\mathbf{MX}_{6l} - K_{58l}\mathbf{MY}_{6l}) - K_{60l}(K_{60l}\mathbf{MX}_{6l} - K_{58l}\mathbf{MZ}_{6l}) & \mathbf{MX}_{6l}(K_{86l} + K_{65l}\ddot{\psi} + K_{68l}\ddot{q}_{1l} + K_{66l}\ddot{q}_{w} + K_{66l}\ddot{q}_{imu} + K_{67l}\ddot{q}_{torso} - \ddot{q}_{2l}s_{3l}) - \mathbf{MZ}_{6l}(K_{85l} + K_{61l}\ddot{\psi} + K_{64l}\ddot{q}_{1l} + K_{62l}\ddot{q}_{w} + K_{62l}\ddot{q}_{imu} + K_{63l}\ddot{q}_{torso} + \ddot{q}_{2l}c_{3l}) - m_{6l}(K_{43l}\ddot{\psi} - K_{88l} + K_{45l}\ddot{q}_{w} + K_{44l}\ddot{q}_{imu} + K_{46l}\ddot{q}_{torso} + K_{42l}\ddot{x}) + K_{58l}(K_{59l}\mathbf{MX}_{6l} - K_{58l}\mathbf{MY}_{6l}) - K_{60l}(K_{60l}\mathbf{MY}_{6l} - K_{59l}\mathbf{MZ}_{6l}) & \mathbf{MY}_{6l}(K_{85l} + K_{61l}\ddot{\psi} + K_{64l}\ddot{q}_{1l} + K_{62l}\ddot{q}_{w} + K_{62l}\ddot{q}_{imu} + K_{63l}\ddot{q}_{torso} + \ddot{q}_{2l}c_{3l}) + \mathbf{MX}_{6l}(K_{52l} + \ddot{q}_{3l} + K_{34l}\ddot{\psi} + K_{35l}\ddot{q}_{w} + K_{35l}\ddot{q}_{imu} + K_{36l}\ddot{q}_{torso} + \ddot{q}_{1l}c_{2l}) + m_{6l}(K_{89l} + K_{79l}\ddot{\psi} + K_{83l}\ddot{q}_{1l} + K_{84l}\ddot{q}_{2l} + K_{81l}\ddot{q}_{w} + K_{80l}\ddot{q}_{imu} + K_{82l}\ddot{q}_{torso} + K_{78l}\ddot{x}) + K_{58l}(K_{60l}\mathbf{MX}_{6l} - K_{58l}\mathbf{MZ}_{6l}) + K_{59l}(K_{60l}\mathbf{MY}_{6l} - K_{59l}\mathbf{MZ}_{6l}) &  \end{matrix}\right] 
 \nonumber \\ 
D_{88l} &= K_{71l}m_{6l} \nonumber \\
D_{89l} &= K_{72l}m_{6l} - K_{65l}\mathbf{MY}_{6l} - K_{34l}\mathbf{MZ}_{6l} \nonumber \\
D_{90l} &= K_{73l}m_{6l} - K_{66l}\mathbf{MY}_{6l} - K_{35l}\mathbf{MZ}_{6l} \nonumber \\
D_{91l} &= K_{74l}m_{6l} - K_{66l}\mathbf{MY}_{6l} - K_{35l}\mathbf{MZ}_{6l} \nonumber \\
D_{92l} &= K_{75l}m_{6l} - K_{67l}\mathbf{MY}_{6l} - K_{36l}\mathbf{MZ}_{6l} \nonumber \\
D_{93l} &= K_{76l}m_{6l} - \mathbf{MZ}_{6l}c_{2l} - K_{68l}\mathbf{MY}_{6l} \nonumber \\
D_{94l} &= K_{77l}m_{6l} + \mathbf{MY}_{6l}s_{3l} \nonumber \\
D_{95l} &= K_{87l}m_{6l} - K_{59l}^2\mathbf{MX}_{6l} - K_{60l}^2\mathbf{MX}_{6l}  \nonumber \\
&- K_{86l}\mathbf{MY}_{6l} - K_{52l}\mathbf{MZ}_{6l} + K_{58l}K_{59l}\mathbf{MY}_{6l}  \nonumber \\
&+ K_{58l}K_{60l}\mathbf{MZ}_{6l} \nonumber \\
D_{96l} &= -K_{42l}m_{6l} \nonumber \\
D_{97l} &= K_{65l}\mathbf{MX}_{6l} - K_{43l}m_{6l} - K_{61l}\mathbf{MZ}_{6l} \nonumber \\
D_{98l} &= K_{66l}\mathbf{MX}_{6l} - K_{44l}m_{6l} - K_{62l}\mathbf{MZ}_{6l} \nonumber \\
D_{99l} &= K_{66l}\mathbf{MX}_{6l} - K_{45l}m_{6l} - K_{62l}\mathbf{MZ}_{6l} \nonumber \\
D_{100l} &= K_{67l}\mathbf{MX}_{6l} - K_{46l}m_{6l} - K_{63l}\mathbf{MZ}_{6l} \nonumber \\
D_{101l} &= K_{68l}\mathbf{MX}_{6l} - K_{64l}\mathbf{MZ}_{6l} \nonumber \\
D_{102l} &= - \mathbf{MZ}_{6l}c_{3l} - \mathbf{MX}_{6l}s_{3l} \nonumber \\
D_{103l} &= K_{88l}m_{6l} - K_{58l}^2\mathbf{MY}_{6l} - K_{60l}^2\mathbf{MY}_{6l}  \nonumber \\
&+ K_{86l}\mathbf{MX}_{6l} - K_{85l}\mathbf{MZ}_{6l} + K_{58l}K_{59l}\mathbf{MX}_{6l}  \nonumber \\
&+ K_{59l}K_{60l}\mathbf{MZ}_{6l} \nonumber \\
D_{104l} &= K_{78l}m_{6l} \nonumber \\
D_{105l} &= K_{79l}m_{6l} + K_{34l}\mathbf{MX}_{6l} + K_{61l}\mathbf{MY}_{6l} \nonumber \\
D_{106l} &= K_{80l}m_{6l} + K_{35l}\mathbf{MX}_{6l} + K_{62l}\mathbf{MY}_{6l} \nonumber \\
D_{107l} &= K_{81l}m_{6l} + K_{35l}\mathbf{MX}_{6l} + K_{62l}\mathbf{MY}_{6l} \nonumber \\
D_{108l} &= K_{82l}m_{6l} + K_{36l}\mathbf{MX}_{6l} + K_{63l}\mathbf{MY}_{6l} \nonumber \\
D_{109l} &= K_{83l}m_{6l} + \mathbf{MX}_{6l}c_{2l} + K_{64l}\mathbf{MY}_{6l} \nonumber \\
D_{110l} &= K_{84l}m_{6l} + \mathbf{MY}_{6l}c_{3l} \nonumber \\
D_{111l} &= K_{89l}m_{6l} - K_{58l}^2\mathbf{MZ}_{6l} - K_{59l}^2\mathbf{MZ}_{6l}  \nonumber \\
&+ K_{52l}\mathbf{MX}_{6l} + K_{85l}\mathbf{MY}_{6l} + K_{58l}K_{60l}\mathbf{MX}_{6l}  \nonumber \\
&+ K_{59l}K_{60l}\mathbf{MY}_{6l} \nonumber \\
 m_{6l}\bar{a}_{G(6l)} &= \left[\begin{matrix} D_{95l} + D_{89l}\ddot{\psi} + D_{93l}\ddot{q}_{1l} + D_{94l}\ddot{q}_{2l} + D_{91l}\ddot{q}_{w} + D_{90l}\ddot{q}_{imu} + D_{92l}\ddot{q}_{torso} + D_{88l}\ddot{x} - \mathbf{MZ}_{6l}\ddot{q}_{3l} & D_{103l} + D_{97l}\ddot{\psi} + D_{101l}\ddot{q}_{1l} + D_{102l}\ddot{q}_{2l} + D_{99l}\ddot{q}_{w} + D_{98l}\ddot{q}_{imu} + D_{100l}\ddot{q}_{torso} + D_{96l}\ddot{x} & D_{111l} + D_{105l}\ddot{\psi} + D_{109l}\ddot{q}_{1l} + D_{110l}\ddot{q}_{2l} + D_{107l}\ddot{q}_{w} + D_{106l}\ddot{q}_{imu} + D_{108l}\ddot{q}_{torso} + D_{104l}\ddot{x} + \mathbf{MX}_{6l}\ddot{q}_{3l} &  \end{matrix}\right] 
 \nonumber \\ 
 \dot{\bar{H}}_{6l} &= \mathbf{MS}_{6l} \times \bar{a}_{6l} + J_{6l}\bar{\alpha}_{6l} + \bar\omega_{6l} \times J_{6l}\bar{\omega}_{6l} 
 \nonumber \\ 
 \dot{\bar{H}}_{6l} &= \left[\begin{matrix} \mathbf{MZ}_{6l}(K_{43l}\ddot{\psi} - K_{88l} + K_{45l}\ddot{q}_{w} + K_{44l}\ddot{q}_{imu} + K_{46l}\ddot{q}_{torso} + K_{42l}\ddot{x}) - K_{60l}(K_{58l}\mathbf{XY}_{6l} + K_{59l}\mathbf{YY}_{6l} + K_{60l}\mathbf{YZ}_{6l}) + K_{59l}(K_{58l}\mathbf{XZ}_{6l} + K_{59l}\mathbf{YZ}_{6l} + K_{60l}\mathbf{ZZ}_{6l}) + \mathbf{XX}_{6l}(K_{85l} + K_{61l}\ddot{\psi} + K_{64l}\ddot{q}_{1l} + K_{62l}\ddot{q}_{w} + K_{62l}\ddot{q}_{imu} + K_{63l}\ddot{q}_{torso} + \ddot{q}_{2l}c_{3l}) + \mathbf{XZ}_{6l}(K_{86l} + K_{65l}\ddot{\psi} + K_{68l}\ddot{q}_{1l} + K_{66l}\ddot{q}_{w} + K_{66l}\ddot{q}_{imu} + K_{67l}\ddot{q}_{torso} - \ddot{q}_{2l}s_{3l}) - \mathbf{XY}_{6l}(K_{52l} + \ddot{q}_{3l} + K_{34l}\ddot{\psi} + K_{35l}\ddot{q}_{w} + K_{35l}\ddot{q}_{imu} + K_{36l}\ddot{q}_{torso} + \ddot{q}_{1l}c_{2l}) + \mathbf{MY}_{6l}(K_{89l} + K_{79l}\ddot{\psi} + K_{83l}\ddot{q}_{1l} + K_{84l}\ddot{q}_{2l} + K_{81l}\ddot{q}_{w} + K_{80l}\ddot{q}_{imu} + K_{82l}\ddot{q}_{torso} + K_{78l}\ddot{x}) & K_{60l}(K_{58l}\mathbf{XX}_{6l} + K_{59l}\mathbf{XY}_{6l} + K_{60l}\mathbf{XZ}_{6l}) - K_{58l}(K_{58l}\mathbf{XZ}_{6l} + K_{59l}\mathbf{YZ}_{6l} + K_{60l}\mathbf{ZZ}_{6l}) + \mathbf{XY}_{6l}(K_{85l} + K_{61l}\ddot{\psi} + K_{64l}\ddot{q}_{1l} + K_{62l}\ddot{q}_{w} + K_{62l}\ddot{q}_{imu} + K_{63l}\ddot{q}_{torso} + \ddot{q}_{2l}c_{3l}) + \mathbf{YZ}_{6l}(K_{86l} + K_{65l}\ddot{\psi} + K_{68l}\ddot{q}_{1l} + K_{66l}\ddot{q}_{w} + K_{66l}\ddot{q}_{imu} + K_{67l}\ddot{q}_{torso} - \ddot{q}_{2l}s_{3l}) - \mathbf{YY}_{6l}(K_{52l} + \ddot{q}_{3l} + K_{34l}\ddot{\psi} + K_{35l}\ddot{q}_{w} + K_{35l}\ddot{q}_{imu} + K_{36l}\ddot{q}_{torso} + \ddot{q}_{1l}c_{2l}) - \mathbf{MX}_{6l}(K_{89l} + K_{79l}\ddot{\psi} + K_{83l}\ddot{q}_{1l} + K_{84l}\ddot{q}_{2l} + K_{81l}\ddot{q}_{w} + K_{80l}\ddot{q}_{imu} + K_{82l}\ddot{q}_{torso} + K_{78l}\ddot{x}) + \mathbf{MZ}_{6l}(K_{87l} + K_{72l}\ddot{\psi} + K_{76l}\ddot{q}_{1l} + K_{77l}\ddot{q}_{2l} + K_{74l}\ddot{q}_{w} + K_{73l}\ddot{q}_{imu} + K_{75l}\ddot{q}_{torso} + K_{71l}\ddot{x}) & K_{58l}(K_{58l}\mathbf{XY}_{6l} + K_{59l}\mathbf{YY}_{6l} + K_{60l}\mathbf{YZ}_{6l}) - K_{59l}(K_{58l}\mathbf{XX}_{6l} + K_{59l}\mathbf{XY}_{6l} + K_{60l}\mathbf{XZ}_{6l}) - \mathbf{MX}_{6l}(K_{43l}\ddot{\psi} - K_{88l} + K_{45l}\ddot{q}_{w} + K_{44l}\ddot{q}_{imu} + K_{46l}\ddot{q}_{torso} + K_{42l}\ddot{x}) + \mathbf{XZ}_{6l}(K_{85l} + K_{61l}\ddot{\psi} + K_{64l}\ddot{q}_{1l} + K_{62l}\ddot{q}_{w} + K_{62l}\ddot{q}_{imu} + K_{63l}\ddot{q}_{torso} + \ddot{q}_{2l}c_{3l}) + \mathbf{ZZ}_{6l}(K_{86l} + K_{65l}\ddot{\psi} + K_{68l}\ddot{q}_{1l} + K_{66l}\ddot{q}_{w} + K_{66l}\ddot{q}_{imu} + K_{67l}\ddot{q}_{torso} - \ddot{q}_{2l}s_{3l}) - \mathbf{YZ}_{6l}(K_{52l} + \ddot{q}_{3l} + K_{34l}\ddot{\psi} + K_{35l}\ddot{q}_{w} + K_{35l}\ddot{q}_{imu} + K_{36l}\ddot{q}_{torso} + \ddot{q}_{1l}c_{2l}) - \mathbf{MY}_{6l}(K_{87l} + K_{72l}\ddot{\psi} + K_{76l}\ddot{q}_{1l} + K_{77l}\ddot{q}_{2l} + K_{74l}\ddot{q}_{w} + K_{73l}\ddot{q}_{imu} + K_{75l}\ddot{q}_{torso} + K_{71l}\ddot{x}) &  \end{matrix}\right] 
 \nonumber \\ 
D_{112l} &= K_{78l}\mathbf{MY}_{6l} + K_{42l}\mathbf{MZ}_{6l} \nonumber \\
D_{113l} &= K_{61l}\mathbf{XX}_{6l} - K_{34l}\mathbf{XY}_{6l} + K_{65l}\mathbf{XZ}_{6l}  \nonumber \\
&+ K_{79l}\mathbf{MY}_{6l} + K_{43l}\mathbf{MZ}_{6l} \nonumber \\
D_{114l} &= K_{62l}\mathbf{XX}_{6l} - K_{35l}\mathbf{XY}_{6l} + K_{66l}\mathbf{XZ}_{6l}  \nonumber \\
&+ K_{80l}\mathbf{MY}_{6l} + K_{44l}\mathbf{MZ}_{6l} \nonumber \\
D_{115l} &= K_{62l}\mathbf{XX}_{6l} - K_{35l}\mathbf{XY}_{6l} + K_{66l}\mathbf{XZ}_{6l}  \nonumber \\
&+ K_{81l}\mathbf{MY}_{6l} + K_{45l}\mathbf{MZ}_{6l} \nonumber \\
D_{116l} &= K_{63l}\mathbf{XX}_{6l} - K_{36l}\mathbf{XY}_{6l} + K_{67l}\mathbf{XZ}_{6l}  \nonumber \\
&+ K_{82l}\mathbf{MY}_{6l} + K_{46l}\mathbf{MZ}_{6l} \nonumber \\
D_{117l} &= K_{64l}\mathbf{XX}_{6l} + K_{68l}\mathbf{XZ}_{6l} - \mathbf{XY}_{6l}c_{2l}  \nonumber \\
&+ K_{83l}\mathbf{MY}_{6l} \nonumber \\
D_{118l} &= \mathbf{XX}_{6l}c_{3l} - \mathbf{XZ}_{6l}s_{3l} + K_{84l}\mathbf{MY}_{6l} \nonumber \\
D_{119l} &= K_{85l}\mathbf{XX}_{6l} - K_{52l}\mathbf{XY}_{6l} + K_{86l}\mathbf{XZ}_{6l}  \nonumber \\
&+ K_{59l}^2\mathbf{YZ}_{6l} - K_{60l}^2\mathbf{YZ}_{6l} + K_{89l}\mathbf{MY}_{6l}  \nonumber \\
&- K_{88l}\mathbf{MZ}_{6l} - K_{58l}K_{60l}\mathbf{XY}_{6l} + K_{58l}K_{59l}\mathbf{XZ}_{6l}  \nonumber \\
&- K_{59l}K_{60l}\mathbf{YY}_{6l} + K_{59l}K_{60l}\mathbf{ZZ}_{6l} \nonumber \\
D_{120l} &= K_{71l}\mathbf{MZ}_{6l} - K_{78l}\mathbf{MX}_{6l} \nonumber \\
D_{121l} &= K_{61l}\mathbf{XY}_{6l} - K_{34l}\mathbf{YY}_{6l} + K_{65l}\mathbf{YZ}_{6l}  \nonumber \\
&- K_{79l}\mathbf{MX}_{6l} + K_{72l}\mathbf{MZ}_{6l} \nonumber \\
D_{122l} &= K_{62l}\mathbf{XY}_{6l} - K_{35l}\mathbf{YY}_{6l} + K_{66l}\mathbf{YZ}_{6l}  \nonumber \\
&- K_{80l}\mathbf{MX}_{6l} + K_{73l}\mathbf{MZ}_{6l} \nonumber \\
D_{123l} &= K_{62l}\mathbf{XY}_{6l} - K_{35l}\mathbf{YY}_{6l} + K_{66l}\mathbf{YZ}_{6l}  \nonumber \\
&- K_{81l}\mathbf{MX}_{6l} + K_{74l}\mathbf{MZ}_{6l} \nonumber \\
D_{124l} &= K_{63l}\mathbf{XY}_{6l} - K_{36l}\mathbf{YY}_{6l} + K_{67l}\mathbf{YZ}_{6l}  \nonumber \\
&- K_{82l}\mathbf{MX}_{6l} + K_{75l}\mathbf{MZ}_{6l} \nonumber \\
D_{125l} &= K_{64l}\mathbf{XY}_{6l} + K_{68l}\mathbf{YZ}_{6l} - \mathbf{YY}_{6l}c_{2l}  \nonumber \\
&- K_{83l}\mathbf{MX}_{6l} + K_{76l}\mathbf{MZ}_{6l} \nonumber \\
D_{126l} &= \mathbf{XY}_{6l}c_{3l} - \mathbf{YZ}_{6l}s_{3l} - K_{84l}\mathbf{MX}_{6l}  \nonumber \\
&+ K_{77l}\mathbf{MZ}_{6l} \nonumber \\
D_{127l} &= K_{85l}\mathbf{XY}_{6l} - K_{52l}\mathbf{YY}_{6l} + K_{86l}\mathbf{YZ}_{6l}  \nonumber \\
&- K_{58l}^2\mathbf{XZ}_{6l} + K_{60l}^2\mathbf{XZ}_{6l} - K_{89l}\mathbf{MX}_{6l}  \nonumber \\
&+ K_{87l}\mathbf{MZ}_{6l} + K_{58l}K_{60l}\mathbf{XX}_{6l} + K_{59l}K_{60l}\mathbf{XY}_{6l}  \nonumber \\
&- K_{58l}K_{59l}\mathbf{YZ}_{6l} - K_{58l}K_{60l}\mathbf{ZZ}_{6l} \nonumber \\
D_{128l} &= - K_{42l}\mathbf{MX}_{6l} - K_{71l}\mathbf{MY}_{6l} \nonumber \\
D_{129l} &= K_{61l}\mathbf{XZ}_{6l} - K_{34l}\mathbf{YZ}_{6l} + K_{65l}\mathbf{ZZ}_{6l}  \nonumber \\
&- K_{43l}\mathbf{MX}_{6l} - K_{72l}\mathbf{MY}_{6l} \nonumber \\
D_{130l} &= K_{62l}\mathbf{XZ}_{6l} - K_{35l}\mathbf{YZ}_{6l} + K_{66l}\mathbf{ZZ}_{6l}  \nonumber \\
&- K_{44l}\mathbf{MX}_{6l} - K_{73l}\mathbf{MY}_{6l} \nonumber \\
D_{131l} &= K_{62l}\mathbf{XZ}_{6l} - K_{35l}\mathbf{YZ}_{6l} + K_{66l}\mathbf{ZZ}_{6l}  \nonumber \\
&- K_{45l}\mathbf{MX}_{6l} - K_{74l}\mathbf{MY}_{6l} \nonumber \\
D_{132l} &= K_{63l}\mathbf{XZ}_{6l} - K_{36l}\mathbf{YZ}_{6l} + K_{67l}\mathbf{ZZ}_{6l}  \nonumber \\
&- K_{46l}\mathbf{MX}_{6l} - K_{75l}\mathbf{MY}_{6l} \nonumber \\
D_{133l} &= K_{64l}\mathbf{XZ}_{6l} + K_{68l}\mathbf{ZZ}_{6l} - \mathbf{YZ}_{6l}c_{2l}  \nonumber \\
&- K_{76l}\mathbf{MY}_{6l} \nonumber \\
D_{134l} &= \mathbf{XZ}_{6l}c_{3l} - \mathbf{ZZ}_{6l}s_{3l} - K_{77l}\mathbf{MY}_{6l} \nonumber \\
D_{135l} &= K_{85l}\mathbf{XZ}_{6l} - K_{52l}\mathbf{YZ}_{6l} + K_{86l}\mathbf{ZZ}_{6l}  \nonumber \\
&+ K_{58l}^2\mathbf{XY}_{6l} - K_{59l}^2\mathbf{XY}_{6l} + K_{88l}\mathbf{MX}_{6l}  \nonumber \\
&- K_{87l}\mathbf{MY}_{6l} - K_{58l}K_{59l}\mathbf{XX}_{6l} - K_{59l}K_{60l}\mathbf{XZ}_{6l}  \nonumber \\
&+ K_{58l}K_{59l}\mathbf{YY}_{6l} + K_{58l}K_{60l}\mathbf{YZ}_{6l} \nonumber \\
 \dot{\bar{H}}_{6l} &= \left[\begin{matrix} D_{95l} + D_{89l}\ddot{\psi} + D_{93l}\ddot{q}_{1l} + D_{94l}\ddot{q}_{2l} + D_{91l}\ddot{q}_{w} + D_{90l}\ddot{q}_{imu} + D_{92l}\ddot{q}_{torso} + D_{88l}\ddot{x} - \mathbf{MZ}_{6l}\ddot{q}_{3l} & D_{103l} + D_{97l}\ddot{\psi} + D_{101l}\ddot{q}_{1l} + D_{102l}\ddot{q}_{2l} + D_{99l}\ddot{q}_{w} + D_{98l}\ddot{q}_{imu} + D_{100l}\ddot{q}_{torso} + D_{96l}\ddot{x} & D_{111l} + D_{105l}\ddot{\psi} + D_{109l}\ddot{q}_{1l} + D_{110l}\ddot{q}_{2l} + D_{107l}\ddot{q}_{w} + D_{106l}\ddot{q}_{imu} + D_{108l}\ddot{q}_{torso} + D_{104l}\ddot{x} + \mathbf{MX}_{6l}\ddot{q}_{3l} &  \end{matrix}\right] 
 \nonumber \\ 
 \bar\omega_{7l} &= {}^{7l}A_{6l} \bar\omega_{6l} + \dot{q}_{7l} \bar{e}_{7l} 
 \nonumber \\ 
 \bar\omega_{7l} &= \left[\begin{matrix} - K_{58l} - \dot{q}_{4l} & - K_{59l}c_{4l} - K_{60l}s_{4l} & K_{60l}c_{4l} - K_{59l}s_{4l} &  \end{matrix}\right] 
 \nonumber \\ 
K_{92l} &= - K_{58l} - \dot{q}_{4l} \nonumber \\
K_{93l} &= - K_{59l}c_{4l} - K_{60l}s_{4l} \nonumber \\
K_{94l} &= K_{60l}c_{4l} - K_{59l}s_{4l} \nonumber \\
 \bar\omega_{7l} &= \left[\begin{matrix} K_{92l} & K_{93l} & K_{94l} &  \end{matrix}\right] 
 \nonumber \\ 
 \bar\omega_{7l} &= \left[\begin{matrix} - \dot{q}_{4l} - K_{61l}\dot{\psi} - K_{64l}\dot{q}_{1l} - K_{62l}\dot{q}_{w} - K_{62l}\dot{q}_{imu} - K_{63l}\dot{q}_{torso} - \dot{q}_{2l}c_{3l} & c_{4l}(\dot{q}_{3l} + K_{34l}\dot{\psi} + K_{35l}\dot{q}_{w} + K_{35l}\dot{q}_{imu} + K_{36l}\dot{q}_{torso} + \dot{q}_{1l}c_{2l}) - s_{4l}(K_{65l}\dot{\psi} + K_{68l}\dot{q}_{1l} + K_{66l}\dot{q}_{w} + K_{66l}\dot{q}_{imu} + K_{67l}\dot{q}_{torso} - \dot{q}_{2l}s_{3l}) & c_{4l}(K_{65l}\dot{\psi} + K_{68l}\dot{q}_{1l} + K_{66l}\dot{q}_{w} + K_{66l}\dot{q}_{imu} + K_{67l}\dot{q}_{torso} - \dot{q}_{2l}s_{3l}) + s_{4l}(\dot{q}_{3l} + K_{34l}\dot{\psi} + K_{35l}\dot{q}_{w} + K_{35l}\dot{q}_{imu} + K_{36l}\dot{q}_{torso} + \dot{q}_{1l}c_{2l}) &  \end{matrix}\right] 
 \nonumber \\ 
K_{95l} &= K_{34l}c_{4l} - K_{65l}s_{4l} \nonumber \\
K_{96l} &= K_{35l}c_{4l} - K_{66l}s_{4l} \nonumber \\
K_{97l} &= K_{36l}c_{4l} - K_{67l}s_{4l} \nonumber \\
K_{98l} &= c_{2l}c_{4l} - K_{68l}s_{4l} \nonumber \\
K_{99l} &= s_{3l}s_{4l} \nonumber \\
K_{100l} &= K_{65l}c_{4l} + K_{34l}s_{4l} \nonumber \\
K_{101l} &= K_{66l}c_{4l} + K_{35l}s_{4l} \nonumber \\
K_{102l} &= K_{67l}c_{4l} + K_{36l}s_{4l} \nonumber \\
K_{103l} &= c_{2l}s_{4l} + K_{68l}c_{4l} \nonumber \\
K_{104l} &= -c_{4l}s_{3l} \nonumber \\
 \bar\omega_{7l} &= \left[\begin{matrix} - \dot{q}_{4l} - K_{61l}\dot{\psi} - K_{64l}\dot{q}_{1l} - K_{62l}\dot{q}_{w} - K_{62l}\dot{q}_{imu} - K_{63l}\dot{q}_{torso} - \dot{q}_{2l}c_{3l} & K_{95l}\dot{\psi} + K_{98l}\dot{q}_{1l} + K_{99l}\dot{q}_{2l} + K_{96l}\dot{q}_{w} + K_{96l}\dot{q}_{imu} + K_{97l}\dot{q}_{torso} + \dot{q}_{3l}c_{4l} & K_{100l}\dot{\psi} + K_{103l}\dot{q}_{1l} + K_{104l}\dot{q}_{2l} + K_{101l}\dot{q}_{w} + K_{101l}\dot{q}_{imu} + K_{102l}\dot{q}_{torso} + \dot{q}_{3l}s_{4l} &  \end{matrix}\right] 
 \nonumber \\ 
 \bar{v}_{7l} &= {}^{7l}A_{6l} \left(\bar{v}_{6l} + \bar\omega_{6l} \times \bar{P}_{7l}\right) 
 \nonumber \\ 
 \bar{v}_{7l} &= \left[\begin{matrix} -K_{69l} & K_{40l}c_{4l} - K_{70l}s_{4l} & K_{70l}c_{4l} + K_{40l}s_{4l} &  \end{matrix}\right] 
 \nonumber \\ 
K_{105l} &= K_{40l}c_{4l} - K_{70l}s_{4l} \nonumber \\
K_{106l} &= K_{70l}c_{4l} + K_{40l}s_{4l} \nonumber \\
 \bar{v}_{7l} &= \left[\begin{matrix} -K_{69l} & K_{105l} & K_{106l} &  \end{matrix}\right] 
 \nonumber \\ 
 \bar{v}_{7l} &= \left[\begin{matrix} - K_{72l}\dot{\psi} - K_{76l}\dot{q}_{1l} - K_{77l}\dot{q}_{2l} - K_{74l}\dot{q}_{w} - K_{73l}\dot{q}_{imu} - K_{75l}\dot{q}_{torso} - K_{71l}\dot{x} & c_{4l}(K_{43l}\dot{\psi} + K_{45l}\dot{q}_{w} + K_{44l}\dot{q}_{imu} + K_{46l}\dot{q}_{torso} + K_{42l}\dot{x}) - s_{4l}(K_{79l}\dot{\psi} + K_{83l}\dot{q}_{1l} + K_{84l}\dot{q}_{2l} + K_{81l}\dot{q}_{w} + K_{80l}\dot{q}_{imu} + K_{82l}\dot{q}_{torso} + K_{78l}\dot{x}) & s_{4l}(K_{43l}\dot{\psi} + K_{45l}\dot{q}_{w} + K_{44l}\dot{q}_{imu} + K_{46l}\dot{q}_{torso} + K_{42l}\dot{x}) + c_{4l}(K_{79l}\dot{\psi} + K_{83l}\dot{q}_{1l} + K_{84l}\dot{q}_{2l} + K_{81l}\dot{q}_{w} + K_{80l}\dot{q}_{imu} + K_{82l}\dot{q}_{torso} + K_{78l}\dot{x}) &  \end{matrix}\right] 
 \nonumber \\ 
K_{107l} &= K_{42l}c_{4l} - K_{78l}s_{4l} \nonumber \\
K_{108l} &= K_{43l}c_{4l} - K_{79l}s_{4l} \nonumber \\
K_{109l} &= K_{44l}c_{4l} - K_{80l}s_{4l} \nonumber \\
K_{110l} &= K_{45l}c_{4l} - K_{81l}s_{4l} \nonumber \\
K_{111l} &= K_{46l}c_{4l} - K_{82l}s_{4l} \nonumber \\
K_{112l} &= -K_{83l}s_{4l} \nonumber \\
K_{113l} &= -K_{84l}s_{4l} \nonumber \\
K_{114l} &= K_{78l}c_{4l} + K_{42l}s_{4l} \nonumber \\
K_{115l} &= K_{79l}c_{4l} + K_{43l}s_{4l} \nonumber \\
K_{116l} &= K_{80l}c_{4l} + K_{44l}s_{4l} \nonumber \\
K_{117l} &= K_{81l}c_{4l} + K_{45l}s_{4l} \nonumber \\
K_{118l} &= K_{82l}c_{4l} + K_{46l}s_{4l} \nonumber \\
K_{119l} &= K_{83l}c_{4l} \nonumber \\
K_{120l} &= K_{84l}c_{4l} \nonumber \\
 \bar{v}_{7l} &= \left[\begin{matrix} - K_{72l}\dot{\psi} - K_{76l}\dot{q}_{1l} - K_{77l}\dot{q}_{2l} - K_{74l}\dot{q}_{w} - K_{73l}\dot{q}_{imu} - K_{75l}\dot{q}_{torso} - K_{71l}\dot{x} & K_{108l}\dot{\psi} + K_{112l}\dot{q}_{1l} + K_{113l}\dot{q}_{2l} + K_{110l}\dot{q}_{w} + K_{109l}\dot{q}_{imu} + K_{111l}\dot{q}_{torso} + K_{107l}\dot{x} & K_{115l}\dot{\psi} + K_{119l}\dot{q}_{1l} + K_{120l}\dot{q}_{2l} + K_{117l}\dot{q}_{w} + K_{116l}\dot{q}_{imu} + K_{118l}\dot{q}_{torso} + K_{114l}\dot{x} &  \end{matrix}\right] 
 \nonumber \\ 
 \bar\alpha_{7l} &= {}^{7l}A_{6l} \bar\alpha_{6l} + \ddot{q}_{7l} \bar{e}_{7l} + \dot{q}_{7l} \left(\bar\omega_{7l} \times \bar{e}_{7l}\right) 
 \nonumber \\ 
 \bar\alpha_{7l} &= \left[\begin{matrix} - K_{85l} - \ddot{q}_{4l} - K_{61l}\ddot{\psi} - K_{64l}\ddot{q}_{1l} - K_{62l}\ddot{q}_{w} - K_{62l}\ddot{q}_{imu} - K_{63l}\ddot{q}_{torso} - \ddot{q}_{2l}c_{3l} & c_{4l}(K_{52l} + \ddot{q}_{3l} + K_{34l}\ddot{\psi} + K_{35l}\ddot{q}_{w} + K_{35l}\ddot{q}_{imu} + K_{36l}\ddot{q}_{torso} + \ddot{q}_{1l}c_{2l}) - K_{94l}\dot{q}_{4l} - s_{4l}(K_{86l} + K_{65l}\ddot{\psi} + K_{68l}\ddot{q}_{1l} + K_{66l}\ddot{q}_{w} + K_{66l}\ddot{q}_{imu} + K_{67l}\ddot{q}_{torso} - \ddot{q}_{2l}s_{3l}) & K_{93l}\dot{q}_{4l} + s_{4l}(K_{52l} + \ddot{q}_{3l} + K_{34l}\ddot{\psi} + K_{35l}\ddot{q}_{w} + K_{35l}\ddot{q}_{imu} + K_{36l}\ddot{q}_{torso} + \ddot{q}_{1l}c_{2l}) + c_{4l}(K_{86l} + K_{65l}\ddot{\psi} + K_{68l}\ddot{q}_{1l} + K_{66l}\ddot{q}_{w} + K_{66l}\ddot{q}_{imu} + K_{67l}\ddot{q}_{torso} - \ddot{q}_{2l}s_{3l}) &  \end{matrix}\right] 
 \nonumber \\ 
K_{121l} &= K_{52l}c_{4l} - K_{94l}\dot{q}_{4l} - K_{86l}s_{4l} \nonumber \\
K_{122l} &= K_{93l}\dot{q}_{4l} + K_{86l}c_{4l} + K_{52l}s_{4l} \nonumber \\
 \bar\alpha_{7l} &= \left[\begin{matrix} - K_{85l} - \ddot{q}_{4l} - K_{61l}\ddot{\psi} - K_{64l}\ddot{q}_{1l} - K_{62l}\ddot{q}_{w} - K_{62l}\ddot{q}_{imu} - K_{63l}\ddot{q}_{torso} - \ddot{q}_{2l}c_{3l} & K_{121l} + K_{95l}\ddot{\psi} + K_{98l}\ddot{q}_{1l} + K_{99l}\ddot{q}_{2l} + K_{96l}\ddot{q}_{w} + K_{96l}\ddot{q}_{imu} + K_{97l}\ddot{q}_{torso} + \ddot{q}_{3l}c_{4l} & K_{122l} + K_{100l}\ddot{\psi} + K_{103l}\ddot{q}_{1l} + K_{104l}\ddot{q}_{2l} + K_{101l}\ddot{q}_{w} + K_{101l}\ddot{q}_{imu} + K_{102l}\ddot{q}_{torso} + \ddot{q}_{3l}s_{4l} &  \end{matrix}\right] 
 \nonumber \\ 
 \bar{a}_{7l} &= {}^{7l}A_{6l} \left(\bar{a}_{6l} + \bar\alpha_{6l} \times \bar{P}_{7l} + \bar\omega_{6l} \times \left(\bar\omega_{6l} \times \bar{P}_{7l}\right)\right) 
 \nonumber \\ 
 \bar\alpha_{7l} &= \left[\begin{matrix} - K_{87l} - K_{72l}\ddot{\psi} - K_{76l}\ddot{q}_{1l} - K_{77l}\ddot{q}_{2l} - K_{74l}\ddot{q}_{w} - K_{73l}\ddot{q}_{imu} - K_{75l}\ddot{q}_{torso} - K_{71l}\ddot{x} & c_{4l}(K_{43l}\ddot{\psi} - K_{88l} + K_{45l}\ddot{q}_{w} + K_{44l}\ddot{q}_{imu} + K_{46l}\ddot{q}_{torso} + K_{42l}\ddot{x}) - s_{4l}(K_{89l} + K_{79l}\ddot{\psi} + K_{83l}\ddot{q}_{1l} + K_{84l}\ddot{q}_{2l} + K_{81l}\ddot{q}_{w} + K_{80l}\ddot{q}_{imu} + K_{82l}\ddot{q}_{torso} + K_{78l}\ddot{x}) & c_{4l}(K_{89l} + K_{79l}\ddot{\psi} + K_{83l}\ddot{q}_{1l} + K_{84l}\ddot{q}_{2l} + K_{81l}\ddot{q}_{w} + K_{80l}\ddot{q}_{imu} + K_{82l}\ddot{q}_{torso} + K_{78l}\ddot{x}) + s_{4l}(K_{43l}\ddot{\psi} - K_{88l} + K_{45l}\ddot{q}_{w} + K_{44l}\ddot{q}_{imu} + K_{46l}\ddot{q}_{torso} + K_{42l}\ddot{x}) &  \end{matrix}\right] 
 \nonumber \\ 
K_{123l} &= - K_{88l}c_{4l} - K_{89l}s_{4l} \nonumber \\
K_{124l} &= K_{89l}c_{4l} - K_{88l}s_{4l} \nonumber \\
 \bar{a}_{7l} &= \left[\begin{matrix} - K_{87l} - K_{72l}\ddot{\psi} - K_{76l}\ddot{q}_{1l} - K_{77l}\ddot{q}_{2l} - K_{74l}\ddot{q}_{w} - K_{73l}\ddot{q}_{imu} - K_{75l}\ddot{q}_{torso} - K_{71l}\ddot{x} & K_{123l} + K_{108l}\ddot{\psi} + K_{112l}\ddot{q}_{1l} + K_{113l}\ddot{q}_{2l} + K_{110l}\ddot{q}_{w} + K_{109l}\ddot{q}_{imu} + K_{111l}\ddot{q}_{torso} + K_{107l}\ddot{x} & K_{124l} + K_{115l}\ddot{\psi} + K_{119l}\ddot{q}_{1l} + K_{120l}\ddot{q}_{2l} + K_{117l}\ddot{q}_{w} + K_{116l}\ddot{q}_{imu} + K_{118l}\ddot{q}_{torso} + K_{114l}\ddot{x} &  \end{matrix}\right] 
 \nonumber \\ 
 \bar{g}_{7l} &= {}^{7l}A_{6l} \bar{g}_{6l} 
 \nonumber \\ 
 \bar{g}_{7l} &= \left[\begin{matrix} -K_{90l}g & K_{56l}gc_{4l} - K_{91l}gs_{4l} & K_{91l}gc_{4l} + K_{56l}gs_{4l} &  \end{matrix}\right] 
 \nonumber \\ 
K_{125l} &= K_{56l}c_{4l} - K_{91l}s_{4l} \nonumber \\
K_{126l} &= K_{91l}c_{4l} + K_{56l}s_{4l} \nonumber \\
 \bar{g}_{7l} &= \left[\begin{matrix} -K_{90l}g & K_{125l}g & K_{126l}g &  \end{matrix}\right] 
 \nonumber \\ 
 m_{7l}\bar{S}_{7l}^{\times}\bar{g}_{7l} &= \mathbf{MS}_{7l} \times \bar{g}_{7l} 
 \nonumber \\ 
 m_{7l}\bar{S}_{7l}^{\times}\bar{g}_{7l} &= \left[\begin{matrix} K_{126l}\mathbf{MY}_{7l}g - K_{125l}\mathbf{MZ}_{7l}g & - K_{126l}\mathbf{MX}_{7l}g - K_{90l}\mathbf{MZ}_{7l}g & K_{125l}\mathbf{MX}_{7l}g + K_{90l}\mathbf{MY}_{7l}g &  \end{matrix}\right] 
 \nonumber \\ 
D_{136l} &= K_{126l}\mathbf{MY}_{7l} - K_{125l}\mathbf{MZ}_{7l} \nonumber \\
D_{137l} &= - K_{126l}\mathbf{MX}_{7l} - K_{90l}\mathbf{MZ}_{7l} \nonumber \\
D_{138l} &= K_{125l}\mathbf{MX}_{7l} + K_{90l}\mathbf{MY}_{7l} \nonumber \\
 m_{7l}\bar{S}_{7l}^{\times}\bar{g}_{7l} &= \left[\begin{matrix} D_{136l}g & D_{137l}g & D_{138l}g &  \end{matrix}\right] 
 \nonumber \\ 
 m_{7l}\bar{a}_{G(7l)} &= m_{7l}\bar{a}_{7l} + \bar\alpha_{7l} \times \mathbf{MS}_{7l} + \bar\omega_{7l} \times \left(\bar\omega_{7l} \times \mathbf{MS}_{7l}\right) 
 \nonumber \\ 
 m_{7l}\bar{a}_{G(7l)} &= \left[\begin{matrix} \mathbf{MZ}_{7l}(K_{121l} + K_{95l}\ddot{\psi} + K_{98l}\ddot{q}_{1l} + K_{99l}\ddot{q}_{2l} + K_{96l}\ddot{q}_{w} + K_{96l}\ddot{q}_{imu} + K_{97l}\ddot{q}_{torso} + \ddot{q}_{3l}c_{4l}) - \mathbf{MY}_{7l}(K_{122l} + K_{100l}\ddot{\psi} + K_{103l}\ddot{q}_{1l} + K_{104l}\ddot{q}_{2l} + K_{101l}\ddot{q}_{w} + K_{101l}\ddot{q}_{imu} + K_{102l}\ddot{q}_{torso} + \ddot{q}_{3l}s_{4l}) - m_{7l}(K_{87l} + K_{72l}\ddot{\psi} + K_{76l}\ddot{q}_{1l} + K_{77l}\ddot{q}_{2l} + K_{74l}\ddot{q}_{w} + K_{73l}\ddot{q}_{imu} + K_{75l}\ddot{q}_{torso} + K_{71l}\ddot{x}) - K_{93l}(K_{93l}\mathbf{MX}_{7l} - K_{92l}\mathbf{MY}_{7l}) - K_{94l}(K_{94l}\mathbf{MX}_{7l} - K_{92l}\mathbf{MZ}_{7l}) & \mathbf{MX}_{7l}(K_{122l} + K_{100l}\ddot{\psi} + K_{103l}\ddot{q}_{1l} + K_{104l}\ddot{q}_{2l} + K_{101l}\ddot{q}_{w} + K_{101l}\ddot{q}_{imu} + K_{102l}\ddot{q}_{torso} + \ddot{q}_{3l}s_{4l}) + \mathbf{MZ}_{7l}(K_{85l} + \ddot{q}_{4l} + K_{61l}\ddot{\psi} + K_{64l}\ddot{q}_{1l} + K_{62l}\ddot{q}_{w} + K_{62l}\ddot{q}_{imu} + K_{63l}\ddot{q}_{torso} + \ddot{q}_{2l}c_{3l}) + m_{7l}(K_{123l} + K_{108l}\ddot{\psi} + K_{112l}\ddot{q}_{1l} + K_{113l}\ddot{q}_{2l} + K_{110l}\ddot{q}_{w} + K_{109l}\ddot{q}_{imu} + K_{111l}\ddot{q}_{torso} + K_{107l}\ddot{x}) + K_{92l}(K_{93l}\mathbf{MX}_{7l} - K_{92l}\mathbf{MY}_{7l}) - K_{94l}(K_{94l}\mathbf{MY}_{7l} - K_{93l}\mathbf{MZ}_{7l}) & m_{7l}(K_{124l} + K_{115l}\ddot{\psi} + K_{119l}\ddot{q}_{1l} + K_{120l}\ddot{q}_{2l} + K_{117l}\ddot{q}_{w} + K_{116l}\ddot{q}_{imu} + K_{118l}\ddot{q}_{torso} + K_{114l}\ddot{x}) - \mathbf{MY}_{7l}(K_{85l} + \ddot{q}_{4l} + K_{61l}\ddot{\psi} + K_{64l}\ddot{q}_{1l} + K_{62l}\ddot{q}_{w} + K_{62l}\ddot{q}_{imu} + K_{63l}\ddot{q}_{torso} + \ddot{q}_{2l}c_{3l}) - \mathbf{MX}_{7l}(K_{121l} + K_{95l}\ddot{\psi} + K_{98l}\ddot{q}_{1l} + K_{99l}\ddot{q}_{2l} + K_{96l}\ddot{q}_{w} + K_{96l}\ddot{q}_{imu} + K_{97l}\ddot{q}_{torso} + \ddot{q}_{3l}c_{4l}) + K_{92l}(K_{94l}\mathbf{MX}_{7l} - K_{92l}\mathbf{MZ}_{7l}) + K_{93l}(K_{94l}\mathbf{MY}_{7l} - K_{93l}\mathbf{MZ}_{7l}) &  \end{matrix}\right] 
 \nonumber \\ 
D_{139l} &= -K_{71l}m_{7l} \nonumber \\
D_{140l} &= K_{95l}\mathbf{MZ}_{7l} - K_{100l}\mathbf{MY}_{7l} - K_{72l}m_{7l} \nonumber \\
D_{141l} &= K_{96l}\mathbf{MZ}_{7l} - K_{101l}\mathbf{MY}_{7l} - K_{73l}m_{7l} \nonumber \\
D_{142l} &= K_{96l}\mathbf{MZ}_{7l} - K_{101l}\mathbf{MY}_{7l} - K_{74l}m_{7l} \nonumber \\
D_{143l} &= K_{97l}\mathbf{MZ}_{7l} - K_{102l}\mathbf{MY}_{7l} - K_{75l}m_{7l} \nonumber \\
D_{144l} &= K_{98l}\mathbf{MZ}_{7l} - K_{103l}\mathbf{MY}_{7l} - K_{76l}m_{7l} \nonumber \\
D_{145l} &= K_{99l}\mathbf{MZ}_{7l} - K_{104l}\mathbf{MY}_{7l} - K_{77l}m_{7l} \nonumber \\
D_{146l} &= \mathbf{MZ}_{7l}c_{4l} - \mathbf{MY}_{7l}s_{4l} \nonumber \\
D_{147l} &= K_{121l}\mathbf{MZ}_{7l} - K_{93l}^2\mathbf{MX}_{7l} - K_{94l}^2\mathbf{MX}_{7l}  \nonumber \\
&- K_{122l}\mathbf{MY}_{7l} - K_{87l}m_{7l} + K_{92l}K_{93l}\mathbf{MY}_{7l}  \nonumber \\
&+ K_{92l}K_{94l}\mathbf{MZ}_{7l} \nonumber \\
D_{148l} &= K_{107l}m_{7l} \nonumber \\
D_{149l} &= K_{108l}m_{7l} + K_{100l}\mathbf{MX}_{7l} + K_{61l}\mathbf{MZ}_{7l} \nonumber \\
D_{150l} &= K_{109l}m_{7l} + K_{101l}\mathbf{MX}_{7l} + K_{62l}\mathbf{MZ}_{7l} \nonumber \\
D_{151l} &= K_{110l}m_{7l} + K_{101l}\mathbf{MX}_{7l} + K_{62l}\mathbf{MZ}_{7l} \nonumber \\
D_{152l} &= K_{111l}m_{7l} + K_{102l}\mathbf{MX}_{7l} + K_{63l}\mathbf{MZ}_{7l} \nonumber \\
D_{153l} &= K_{112l}m_{7l} + K_{103l}\mathbf{MX}_{7l} + K_{64l}\mathbf{MZ}_{7l} \nonumber \\
D_{154l} &= K_{113l}m_{7l} + \mathbf{MZ}_{7l}c_{3l} + K_{104l}\mathbf{MX}_{7l} \nonumber \\
D_{155l} &= \mathbf{MX}_{7l}s_{4l} \nonumber \\
D_{156l} &= K_{123l}m_{7l} - K_{92l}^2\mathbf{MY}_{7l} - K_{94l}^2\mathbf{MY}_{7l}  \nonumber \\
&+ K_{122l}\mathbf{MX}_{7l} + K_{85l}\mathbf{MZ}_{7l} + K_{92l}K_{93l}\mathbf{MX}_{7l}  \nonumber \\
&+ K_{93l}K_{94l}\mathbf{MZ}_{7l} \nonumber \\
D_{157l} &= K_{114l}m_{7l} \nonumber \\
D_{158l} &= K_{115l}m_{7l} - K_{95l}\mathbf{MX}_{7l} - K_{61l}\mathbf{MY}_{7l} \nonumber \\
D_{159l} &= K_{116l}m_{7l} - K_{96l}\mathbf{MX}_{7l} - K_{62l}\mathbf{MY}_{7l} \nonumber \\
D_{160l} &= K_{117l}m_{7l} - K_{96l}\mathbf{MX}_{7l} - K_{62l}\mathbf{MY}_{7l} \nonumber \\
D_{161l} &= K_{118l}m_{7l} - K_{97l}\mathbf{MX}_{7l} - K_{63l}\mathbf{MY}_{7l} \nonumber \\
D_{162l} &= K_{119l}m_{7l} - K_{98l}\mathbf{MX}_{7l} - K_{64l}\mathbf{MY}_{7l} \nonumber \\
D_{163l} &= K_{120l}m_{7l} - \mathbf{MY}_{7l}c_{3l} - K_{99l}\mathbf{MX}_{7l} \nonumber \\
D_{164l} &= -\mathbf{MX}_{7l}c_{4l} \nonumber \\
D_{165l} &= K_{124l}m_{7l} - K_{92l}^2\mathbf{MZ}_{7l} - K_{93l}^2\mathbf{MZ}_{7l}  \nonumber \\
&- K_{121l}\mathbf{MX}_{7l} - K_{85l}\mathbf{MY}_{7l} + K_{92l}K_{94l}\mathbf{MX}_{7l}  \nonumber \\
&+ K_{93l}K_{94l}\mathbf{MY}_{7l} \nonumber \\
 m_{7l}\bar{a}_{G(7l)} &= \left[\begin{matrix} D_{147l} + D_{140l}\ddot{\psi} + D_{144l}\ddot{q}_{1l} + D_{145l}\ddot{q}_{2l} + D_{146l}\ddot{q}_{3l} + D_{142l}\ddot{q}_{w} + D_{141l}\ddot{q}_{imu} + D_{143l}\ddot{q}_{torso} + D_{139l}\ddot{x} & D_{156l} + D_{149l}\ddot{\psi} + D_{153l}\ddot{q}_{1l} + D_{154l}\ddot{q}_{2l} + D_{155l}\ddot{q}_{3l} + D_{151l}\ddot{q}_{w} + D_{150l}\ddot{q}_{imu} + D_{152l}\ddot{q}_{torso} + D_{148l}\ddot{x} + \mathbf{MZ}_{7l}\ddot{q}_{4l} & D_{165l} + D_{158l}\ddot{\psi} + D_{162l}\ddot{q}_{1l} + D_{163l}\ddot{q}_{2l} + D_{164l}\ddot{q}_{3l} + D_{160l}\ddot{q}_{w} + D_{159l}\ddot{q}_{imu} + D_{161l}\ddot{q}_{torso} + D_{157l}\ddot{x} - \mathbf{MY}_{7l}\ddot{q}_{4l} &  \end{matrix}\right] 
 \nonumber \\ 
 \dot{\bar{H}}_{7l} &= \mathbf{MS}_{7l} \times \bar{a}_{7l} + J_{7l}\bar{\alpha}_{7l} + \bar\omega_{7l} \times J_{7l}\bar{\omega}_{7l} 
 \nonumber \\ 
 \dot{\bar{H}}_{7l} &= \left[\begin{matrix} K_{93l}(K_{92l}\mathbf{XZ}_{7l} + K_{93l}\mathbf{YZ}_{7l} + K_{94l}\mathbf{ZZ}_{7l}) - K_{94l}(K_{92l}\mathbf{XY}_{7l} + K_{93l}\mathbf{YY}_{7l} + K_{94l}\mathbf{YZ}_{7l}) + \mathbf{XY}_{7l}(K_{121l} + K_{95l}\ddot{\psi} + K_{98l}\ddot{q}_{1l} + K_{99l}\ddot{q}_{2l} + K_{96l}\ddot{q}_{w} + K_{96l}\ddot{q}_{imu} + K_{97l}\ddot{q}_{torso} + \ddot{q}_{3l}c_{4l}) + \mathbf{XZ}_{7l}(K_{122l} + K_{100l}\ddot{\psi} + K_{103l}\ddot{q}_{1l} + K_{104l}\ddot{q}_{2l} + K_{101l}\ddot{q}_{w} + K_{101l}\ddot{q}_{imu} + K_{102l}\ddot{q}_{torso} + \ddot{q}_{3l}s_{4l}) - \mathbf{XX}_{7l}(K_{85l} + \ddot{q}_{4l} + K_{61l}\ddot{\psi} + K_{64l}\ddot{q}_{1l} + K_{62l}\ddot{q}_{w} + K_{62l}\ddot{q}_{imu} + K_{63l}\ddot{q}_{torso} + \ddot{q}_{2l}c_{3l}) + \mathbf{MY}_{7l}(K_{124l} + K_{115l}\ddot{\psi} + K_{119l}\ddot{q}_{1l} + K_{120l}\ddot{q}_{2l} + K_{117l}\ddot{q}_{w} + K_{116l}\ddot{q}_{imu} + K_{118l}\ddot{q}_{torso} + K_{114l}\ddot{x}) - \mathbf{MZ}_{7l}(K_{123l} + K_{108l}\ddot{\psi} + K_{112l}\ddot{q}_{1l} + K_{113l}\ddot{q}_{2l} + K_{110l}\ddot{q}_{w} + K_{109l}\ddot{q}_{imu} + K_{111l}\ddot{q}_{torso} + K_{107l}\ddot{x}) & K_{94l}(K_{92l}\mathbf{XX}_{7l} + K_{93l}\mathbf{XY}_{7l} + K_{94l}\mathbf{XZ}_{7l}) - K_{92l}(K_{92l}\mathbf{XZ}_{7l} + K_{93l}\mathbf{YZ}_{7l} + K_{94l}\mathbf{ZZ}_{7l}) + \mathbf{YY}_{7l}(K_{121l} + K_{95l}\ddot{\psi} + K_{98l}\ddot{q}_{1l} + K_{99l}\ddot{q}_{2l} + K_{96l}\ddot{q}_{w} + K_{96l}\ddot{q}_{imu} + K_{97l}\ddot{q}_{torso} + \ddot{q}_{3l}c_{4l}) + \mathbf{YZ}_{7l}(K_{122l} + K_{100l}\ddot{\psi} + K_{103l}\ddot{q}_{1l} + K_{104l}\ddot{q}_{2l} + K_{101l}\ddot{q}_{w} + K_{101l}\ddot{q}_{imu} + K_{102l}\ddot{q}_{torso} + \ddot{q}_{3l}s_{4l}) - \mathbf{XY}_{7l}(K_{85l} + \ddot{q}_{4l} + K_{61l}\ddot{\psi} + K_{64l}\ddot{q}_{1l} + K_{62l}\ddot{q}_{w} + K_{62l}\ddot{q}_{imu} + K_{63l}\ddot{q}_{torso} + \ddot{q}_{2l}c_{3l}) - \mathbf{MX}_{7l}(K_{124l} + K_{115l}\ddot{\psi} + K_{119l}\ddot{q}_{1l} + K_{120l}\ddot{q}_{2l} + K_{117l}\ddot{q}_{w} + K_{116l}\ddot{q}_{imu} + K_{118l}\ddot{q}_{torso} + K_{114l}\ddot{x}) - \mathbf{MZ}_{7l}(K_{87l} + K_{72l}\ddot{\psi} + K_{76l}\ddot{q}_{1l} + K_{77l}\ddot{q}_{2l} + K_{74l}\ddot{q}_{w} + K_{73l}\ddot{q}_{imu} + K_{75l}\ddot{q}_{torso} + K_{71l}\ddot{x}) & K_{92l}(K_{92l}\mathbf{XY}_{7l} + K_{93l}\mathbf{YY}_{7l} + K_{94l}\mathbf{YZ}_{7l}) - K_{93l}(K_{92l}\mathbf{XX}_{7l} + K_{93l}\mathbf{XY}_{7l} + K_{94l}\mathbf{XZ}_{7l}) + \mathbf{YZ}_{7l}(K_{121l} + K_{95l}\ddot{\psi} + K_{98l}\ddot{q}_{1l} + K_{99l}\ddot{q}_{2l} + K_{96l}\ddot{q}_{w} + K_{96l}\ddot{q}_{imu} + K_{97l}\ddot{q}_{torso} + \ddot{q}_{3l}c_{4l}) + \mathbf{ZZ}_{7l}(K_{122l} + K_{100l}\ddot{\psi} + K_{103l}\ddot{q}_{1l} + K_{104l}\ddot{q}_{2l} + K_{101l}\ddot{q}_{w} + K_{101l}\ddot{q}_{imu} + K_{102l}\ddot{q}_{torso} + \ddot{q}_{3l}s_{4l}) - \mathbf{XZ}_{7l}(K_{85l} + \ddot{q}_{4l} + K_{61l}\ddot{\psi} + K_{64l}\ddot{q}_{1l} + K_{62l}\ddot{q}_{w} + K_{62l}\ddot{q}_{imu} + K_{63l}\ddot{q}_{torso} + \ddot{q}_{2l}c_{3l}) + \mathbf{MX}_{7l}(K_{123l} + K_{108l}\ddot{\psi} + K_{112l}\ddot{q}_{1l} + K_{113l}\ddot{q}_{2l} + K_{110l}\ddot{q}_{w} + K_{109l}\ddot{q}_{imu} + K_{111l}\ddot{q}_{torso} + K_{107l}\ddot{x}) + \mathbf{MY}_{7l}(K_{87l} + K_{72l}\ddot{\psi} + K_{76l}\ddot{q}_{1l} + K_{77l}\ddot{q}_{2l} + K_{74l}\ddot{q}_{w} + K_{73l}\ddot{q}_{imu} + K_{75l}\ddot{q}_{torso} + K_{71l}\ddot{x}) &  \end{matrix}\right] 
 \nonumber \\ 
D_{166l} &= K_{114l}\mathbf{MY}_{7l} - K_{107l}\mathbf{MZ}_{7l} \nonumber \\
D_{167l} &= K_{95l}\mathbf{XY}_{7l} - K_{61l}\mathbf{XX}_{7l} + K_{100l}\mathbf{XZ}_{7l}  \nonumber \\
&+ K_{115l}\mathbf{MY}_{7l} - K_{108l}\mathbf{MZ}_{7l} \nonumber \\
D_{168l} &= K_{96l}\mathbf{XY}_{7l} - K_{62l}\mathbf{XX}_{7l} + K_{101l}\mathbf{XZ}_{7l}  \nonumber \\
&+ K_{116l}\mathbf{MY}_{7l} - K_{109l}\mathbf{MZ}_{7l} \nonumber \\
D_{169l} &= K_{96l}\mathbf{XY}_{7l} - K_{62l}\mathbf{XX}_{7l} + K_{101l}\mathbf{XZ}_{7l}  \nonumber \\
&+ K_{117l}\mathbf{MY}_{7l} - K_{110l}\mathbf{MZ}_{7l} \nonumber \\
D_{170l} &= K_{97l}\mathbf{XY}_{7l} - K_{63l}\mathbf{XX}_{7l} + K_{102l}\mathbf{XZ}_{7l}  \nonumber \\
&+ K_{118l}\mathbf{MY}_{7l} - K_{111l}\mathbf{MZ}_{7l} \nonumber \\
D_{171l} &= K_{98l}\mathbf{XY}_{7l} - K_{64l}\mathbf{XX}_{7l} + K_{103l}\mathbf{XZ}_{7l}  \nonumber \\
&+ K_{119l}\mathbf{MY}_{7l} - K_{112l}\mathbf{MZ}_{7l} \nonumber \\
D_{172l} &= K_{99l}\mathbf{XY}_{7l} + K_{104l}\mathbf{XZ}_{7l} - \mathbf{XX}_{7l}c_{3l}  \nonumber \\
&+ K_{120l}\mathbf{MY}_{7l} - K_{113l}\mathbf{MZ}_{7l} \nonumber \\
D_{173l} &= \mathbf{XY}_{7l}c_{4l} + \mathbf{XZ}_{7l}s_{4l} \nonumber \\
D_{174l} &= K_{121l}\mathbf{XY}_{7l} - K_{85l}\mathbf{XX}_{7l} + K_{122l}\mathbf{XZ}_{7l}  \nonumber \\
&+ K_{93l}^2\mathbf{YZ}_{7l} - K_{94l}^2\mathbf{YZ}_{7l} + K_{124l}\mathbf{MY}_{7l}  \nonumber \\
&- K_{123l}\mathbf{MZ}_{7l} - K_{92l}K_{94l}\mathbf{XY}_{7l} + K_{92l}K_{93l}\mathbf{XZ}_{7l}  \nonumber \\
&- K_{93l}K_{94l}\mathbf{YY}_{7l} + K_{93l}K_{94l}\mathbf{ZZ}_{7l} \nonumber \\
D_{175l} &= - K_{114l}\mathbf{MX}_{7l} - K_{71l}\mathbf{MZ}_{7l} \nonumber \\
D_{176l} &= K_{95l}\mathbf{YY}_{7l} - K_{61l}\mathbf{XY}_{7l} + K_{100l}\mathbf{YZ}_{7l}  \nonumber \\
&- K_{115l}\mathbf{MX}_{7l} - K_{72l}\mathbf{MZ}_{7l} \nonumber \\
D_{177l} &= K_{96l}\mathbf{YY}_{7l} - K_{62l}\mathbf{XY}_{7l} + K_{101l}\mathbf{YZ}_{7l}  \nonumber \\
&- K_{116l}\mathbf{MX}_{7l} - K_{73l}\mathbf{MZ}_{7l} \nonumber \\
D_{178l} &= K_{96l}\mathbf{YY}_{7l} - K_{62l}\mathbf{XY}_{7l} + K_{101l}\mathbf{YZ}_{7l}  \nonumber \\
&- K_{117l}\mathbf{MX}_{7l} - K_{74l}\mathbf{MZ}_{7l} \nonumber \\
D_{179l} &= K_{97l}\mathbf{YY}_{7l} - K_{63l}\mathbf{XY}_{7l} + K_{102l}\mathbf{YZ}_{7l}  \nonumber \\
&- K_{118l}\mathbf{MX}_{7l} - K_{75l}\mathbf{MZ}_{7l} \nonumber \\
D_{180l} &= K_{98l}\mathbf{YY}_{7l} - K_{64l}\mathbf{XY}_{7l} + K_{103l}\mathbf{YZ}_{7l}  \nonumber \\
&- K_{119l}\mathbf{MX}_{7l} - K_{76l}\mathbf{MZ}_{7l} \nonumber \\
D_{181l} &= K_{99l}\mathbf{YY}_{7l} + K_{104l}\mathbf{YZ}_{7l} - \mathbf{XY}_{7l}c_{3l}  \nonumber \\
&- K_{120l}\mathbf{MX}_{7l} - K_{77l}\mathbf{MZ}_{7l} \nonumber \\
D_{182l} &= \mathbf{YY}_{7l}c_{4l} + \mathbf{YZ}_{7l}s_{4l} \nonumber \\
D_{183l} &= K_{121l}\mathbf{YY}_{7l} - K_{85l}\mathbf{XY}_{7l} + K_{122l}\mathbf{YZ}_{7l}  \nonumber \\
&- K_{92l}^2\mathbf{XZ}_{7l} + K_{94l}^2\mathbf{XZ}_{7l} - K_{124l}\mathbf{MX}_{7l}  \nonumber \\
&- K_{87l}\mathbf{MZ}_{7l} + K_{92l}K_{94l}\mathbf{XX}_{7l} + K_{93l}K_{94l}\mathbf{XY}_{7l}  \nonumber \\
&- K_{92l}K_{93l}\mathbf{YZ}_{7l} - K_{92l}K_{94l}\mathbf{ZZ}_{7l} \nonumber \\
D_{184l} &= K_{107l}\mathbf{MX}_{7l} + K_{71l}\mathbf{MY}_{7l} \nonumber \\
D_{185l} &= K_{95l}\mathbf{YZ}_{7l} - K_{61l}\mathbf{XZ}_{7l} + K_{100l}\mathbf{ZZ}_{7l}  \nonumber \\
&+ K_{108l}\mathbf{MX}_{7l} + K_{72l}\mathbf{MY}_{7l} \nonumber \\
D_{186l} &= K_{96l}\mathbf{YZ}_{7l} - K_{62l}\mathbf{XZ}_{7l} + K_{101l}\mathbf{ZZ}_{7l}  \nonumber \\
&+ K_{109l}\mathbf{MX}_{7l} + K_{73l}\mathbf{MY}_{7l} \nonumber \\
D_{187l} &= K_{96l}\mathbf{YZ}_{7l} - K_{62l}\mathbf{XZ}_{7l} + K_{101l}\mathbf{ZZ}_{7l}  \nonumber \\
&+ K_{110l}\mathbf{MX}_{7l} + K_{74l}\mathbf{MY}_{7l} \nonumber \\
D_{188l} &= K_{97l}\mathbf{YZ}_{7l} - K_{63l}\mathbf{XZ}_{7l} + K_{102l}\mathbf{ZZ}_{7l}  \nonumber \\
&+ K_{111l}\mathbf{MX}_{7l} + K_{75l}\mathbf{MY}_{7l} \nonumber \\
D_{189l} &= K_{98l}\mathbf{YZ}_{7l} - K_{64l}\mathbf{XZ}_{7l} + K_{103l}\mathbf{ZZ}_{7l}  \nonumber \\
&+ K_{112l}\mathbf{MX}_{7l} + K_{76l}\mathbf{MY}_{7l} \nonumber \\
D_{190l} &= K_{99l}\mathbf{YZ}_{7l} + K_{104l}\mathbf{ZZ}_{7l} - \mathbf{XZ}_{7l}c_{3l}  \nonumber \\
&+ K_{113l}\mathbf{MX}_{7l} + K_{77l}\mathbf{MY}_{7l} \nonumber \\
D_{191l} &= \mathbf{YZ}_{7l}c_{4l} + \mathbf{ZZ}_{7l}s_{4l} \nonumber \\
D_{192l} &= K_{121l}\mathbf{YZ}_{7l} - K_{85l}\mathbf{XZ}_{7l} + K_{122l}\mathbf{ZZ}_{7l}  \nonumber \\
&+ K_{92l}^2\mathbf{XY}_{7l} - K_{93l}^2\mathbf{XY}_{7l} + K_{123l}\mathbf{MX}_{7l}  \nonumber \\
&+ K_{87l}\mathbf{MY}_{7l} - K_{92l}K_{93l}\mathbf{XX}_{7l} - K_{93l}K_{94l}\mathbf{XZ}_{7l}  \nonumber \\
&+ K_{92l}K_{93l}\mathbf{YY}_{7l} + K_{92l}K_{94l}\mathbf{YZ}_{7l} \nonumber \\
 \dot{\bar{H}}_{7l} &= \left[\begin{matrix} D_{147l} + D_{140l}\ddot{\psi} + D_{144l}\ddot{q}_{1l} + D_{145l}\ddot{q}_{2l} + D_{146l}\ddot{q}_{3l} + D_{142l}\ddot{q}_{w} + D_{141l}\ddot{q}_{imu} + D_{143l}\ddot{q}_{torso} + D_{139l}\ddot{x} & D_{156l} + D_{149l}\ddot{\psi} + D_{153l}\ddot{q}_{1l} + D_{154l}\ddot{q}_{2l} + D_{155l}\ddot{q}_{3l} + D_{151l}\ddot{q}_{w} + D_{150l}\ddot{q}_{imu} + D_{152l}\ddot{q}_{torso} + D_{148l}\ddot{x} + \mathbf{MZ}_{7l}\ddot{q}_{4l} & D_{165l} + D_{158l}\ddot{\psi} + D_{162l}\ddot{q}_{1l} + D_{163l}\ddot{q}_{2l} + D_{164l}\ddot{q}_{3l} + D_{160l}\ddot{q}_{w} + D_{159l}\ddot{q}_{imu} + D_{161l}\ddot{q}_{torso} + D_{157l}\ddot{x} - \mathbf{MY}_{7l}\ddot{q}_{4l} &  \end{matrix}\right] 
 \nonumber \\ 
 \bar\omega_{8l} &= {}^{8l}A_{7l} \bar\omega_{7l} + \dot{q}_{8l} \bar{e}_{8l} 
 \nonumber \\ 
 \bar\omega_{8l} &= \left[\begin{matrix} K_{94l}s_{5l} - K_{92l}c_{5l} & - K_{93l} - \dot{q}_{5l} & K_{94l}c_{5l} + K_{92l}s_{5l} &  \end{matrix}\right] 
 \nonumber \\ 
K_{127l} &= K_{94l}s_{5l} - K_{92l}c_{5l} \nonumber \\
K_{128l} &= - K_{93l} - \dot{q}_{5l} \nonumber \\
K_{129l} &= K_{94l}c_{5l} + K_{92l}s_{5l} \nonumber \\
 \bar\omega_{8l} &= \left[\begin{matrix} K_{127l} & K_{128l} & K_{129l} &  \end{matrix}\right] 
 \nonumber \\ 
 \bar\omega_{8l} &= \left[\begin{matrix} s_{5l}(K_{100l}\dot{\psi} + K_{103l}\dot{q}_{1l} + K_{104l}\dot{q}_{2l} + K_{101l}\dot{q}_{w} + K_{101l}\dot{q}_{imu} + K_{102l}\dot{q}_{torso} + \dot{q}_{3l}s_{4l}) + c_{5l}(\dot{q}_{4l} + K_{61l}\dot{\psi} + K_{64l}\dot{q}_{1l} + K_{62l}\dot{q}_{w} + K_{62l}\dot{q}_{imu} + K_{63l}\dot{q}_{torso} + \dot{q}_{2l}c_{3l}) & - \dot{q}_{5l} - K_{95l}\dot{\psi} - K_{98l}\dot{q}_{1l} - K_{99l}\dot{q}_{2l} - K_{96l}\dot{q}_{w} - K_{96l}\dot{q}_{imu} - K_{97l}\dot{q}_{torso} - \dot{q}_{3l}c_{4l} & c_{5l}(K_{100l}\dot{\psi} + K_{103l}\dot{q}_{1l} + K_{104l}\dot{q}_{2l} + K_{101l}\dot{q}_{w} + K_{101l}\dot{q}_{imu} + K_{102l}\dot{q}_{torso} + \dot{q}_{3l}s_{4l}) - s_{5l}(\dot{q}_{4l} + K_{61l}\dot{\psi} + K_{64l}\dot{q}_{1l} + K_{62l}\dot{q}_{w} + K_{62l}\dot{q}_{imu} + K_{63l}\dot{q}_{torso} + \dot{q}_{2l}c_{3l}) &  \end{matrix}\right] 
 \nonumber \\ 
K_{130l} &= K_{61l}c_{5l} + K_{100l}s_{5l} \nonumber \\
K_{131l} &= K_{62l}c_{5l} + K_{101l}s_{5l} \nonumber \\
K_{132l} &= K_{63l}c_{5l} + K_{102l}s_{5l} \nonumber \\
K_{133l} &= K_{64l}c_{5l} + K_{103l}s_{5l} \nonumber \\
K_{134l} &= c_{3l}c_{5l} + K_{104l}s_{5l} \nonumber \\
K_{135l} &= s_{4l}s_{5l} \nonumber \\
K_{136l} &= K_{100l}c_{5l} - K_{61l}s_{5l} \nonumber \\
K_{137l} &= K_{101l}c_{5l} - K_{62l}s_{5l} \nonumber \\
K_{138l} &= K_{102l}c_{5l} - K_{63l}s_{5l} \nonumber \\
K_{139l} &= K_{103l}c_{5l} - K_{64l}s_{5l} \nonumber \\
K_{140l} &= K_{104l}c_{5l} - c_{3l}s_{5l} \nonumber \\
K_{141l} &= c_{5l}s_{4l} \nonumber \\
 \bar\omega_{8l} &= \left[\begin{matrix} K_{130l}\dot{\psi} + K_{133l}\dot{q}_{1l} + K_{134l}\dot{q}_{2l} + K_{135l}\dot{q}_{3l} + K_{131l}\dot{q}_{w} + K_{131l}\dot{q}_{imu} + K_{132l}\dot{q}_{torso} + \dot{q}_{4l}c_{5l} & - \dot{q}_{5l} - K_{95l}\dot{\psi} - K_{98l}\dot{q}_{1l} - K_{99l}\dot{q}_{2l} - K_{96l}\dot{q}_{w} - K_{96l}\dot{q}_{imu} - K_{97l}\dot{q}_{torso} - \dot{q}_{3l}c_{4l} & K_{136l}\dot{\psi} + K_{139l}\dot{q}_{1l} + K_{140l}\dot{q}_{2l} + K_{141l}\dot{q}_{3l} + K_{137l}\dot{q}_{w} + K_{137l}\dot{q}_{imu} + K_{138l}\dot{q}_{torso} - \dot{q}_{4l}s_{5l} &  \end{matrix}\right] 
 \nonumber \\ 
 \bar{v}_{8l} &= {}^{8l}A_{7l} \left(\bar{v}_{7l} + \bar\omega_{7l} \times \bar{P}_{8l}\right) 
 \nonumber \\ 
 \bar{v}_{8l} &= \left[\begin{matrix} c_{5l}(K_{69l} - K_{94l}L_8) + s_{5l}(K_{106l} - K_{92l}L_8) & -K_{105l} & c_{5l}(K_{106l} - K_{92l}L_8) - s_{5l}(K_{69l} - K_{94l}L_8) &  \end{matrix}\right] 
 \nonumber \\ 
K_{142l} &= c_{5l}(K_{69l} - K_{94l}L_8) + s_{5l}(K_{106l}  \nonumber \\
&- K_{92l}L_8) \nonumber \\
K_{143l} &= c_{5l}(K_{106l} - K_{92l}L_8) - s_{5l}(K_{69l}  \nonumber \\
&- K_{94l}L_8) \nonumber \\
 \bar{v}_{8l} &= \left[\begin{matrix} K_{142l} & -K_{105l} & K_{143l} &  \end{matrix}\right] 
 \nonumber \\ 
 \bar{v}_{8l} &= \left[\begin{matrix} c_{5l}(K_{72l}\dot{\psi} - L_8(K_{100l}\dot{\psi} + K_{103l}\dot{q}_{1l} + K_{104l}\dot{q}_{2l} + K_{101l}\dot{q}_{w} + K_{101l}\dot{q}_{imu} + K_{102l}\dot{q}_{torso} + \dot{q}_{3l}s_{4l}) + K_{76l}\dot{q}_{1l} + K_{77l}\dot{q}_{2l} + K_{74l}\dot{q}_{w} + K_{73l}\dot{q}_{imu} + K_{75l}\dot{q}_{torso} + K_{71l}\dot{x}) + s_{5l}(K_{115l}\dot{\psi} + K_{119l}\dot{q}_{1l} + K_{120l}\dot{q}_{2l} + K_{117l}\dot{q}_{w} + K_{116l}\dot{q}_{imu} + K_{118l}\dot{q}_{torso} + K_{114l}\dot{x} + L_8(\dot{q}_{4l} + K_{61l}\dot{\psi} + K_{64l}\dot{q}_{1l} + K_{62l}\dot{q}_{w} + K_{62l}\dot{q}_{imu} + K_{63l}\dot{q}_{torso} + \dot{q}_{2l}c_{3l})) & - K_{108l}\dot{\psi} - K_{112l}\dot{q}_{1l} - K_{113l}\dot{q}_{2l} - K_{110l}\dot{q}_{w} - K_{109l}\dot{q}_{imu} - K_{111l}\dot{q}_{torso} - K_{107l}\dot{x} & c_{5l}(K_{115l}\dot{\psi} + K_{119l}\dot{q}_{1l} + K_{120l}\dot{q}_{2l} + K_{117l}\dot{q}_{w} + K_{116l}\dot{q}_{imu} + K_{118l}\dot{q}_{torso} + K_{114l}\dot{x} + L_8(\dot{q}_{4l} + K_{61l}\dot{\psi} + K_{64l}\dot{q}_{1l} + K_{62l}\dot{q}_{w} + K_{62l}\dot{q}_{imu} + K_{63l}\dot{q}_{torso} + \dot{q}_{2l}c_{3l})) - s_{5l}(K_{72l}\dot{\psi} - L_8(K_{100l}\dot{\psi} + K_{103l}\dot{q}_{1l} + K_{104l}\dot{q}_{2l} + K_{101l}\dot{q}_{w} + K_{101l}\dot{q}_{imu} + K_{102l}\dot{q}_{torso} + \dot{q}_{3l}s_{4l}) + K_{76l}\dot{q}_{1l} + K_{77l}\dot{q}_{2l} + K_{74l}\dot{q}_{w} + K_{73l}\dot{q}_{imu} + K_{75l}\dot{q}_{torso} + K_{71l}\dot{x}) &  \end{matrix}\right] 
 \nonumber \\ 
K_{144l} &= K_{71l}c_{5l} + K_{114l}s_{5l} \nonumber \\
K_{145l} &= c_{5l}(K_{72l} - K_{100l}L_8) + s_{5l}(K_{115l}  \nonumber \\
&+ K_{61l}L_8) \nonumber \\
K_{146l} &= c_{5l}(K_{73l} - K_{101l}L_8) + s_{5l}(K_{116l}  \nonumber \\
&+ K_{62l}L_8) \nonumber \\
K_{147l} &= c_{5l}(K_{74l} - K_{101l}L_8) + s_{5l}(K_{117l}  \nonumber \\
&+ K_{62l}L_8) \nonumber \\
K_{148l} &= c_{5l}(K_{75l} - K_{102l}L_8) + s_{5l}(K_{118l}  \nonumber \\
&+ K_{63l}L_8) \nonumber \\
K_{149l} &= c_{5l}(K_{76l} - K_{103l}L_8) + s_{5l}(K_{119l}  \nonumber \\
&+ K_{64l}L_8) \nonumber \\
K_{150l} &= s_{5l}(K_{120l} + L_8c_{3l}) + c_{5l}(K_{77l}  \nonumber \\
&- K_{104l}L_8) \nonumber \\
K_{151l} &= -L_8c_{5l}s_{4l} \nonumber \\
K_{152l} &= L_8s_{5l} \nonumber \\
K_{153l} &= K_{114l}c_{5l} - K_{71l}s_{5l} \nonumber \\
K_{154l} &= c_{5l}(K_{115l} + K_{61l}L_8) - s_{5l}(K_{72l}  \nonumber \\
&- K_{100l}L_8) \nonumber \\
K_{155l} &= c_{5l}(K_{116l} + K_{62l}L_8) - s_{5l}(K_{73l}  \nonumber \\
&- K_{101l}L_8) \nonumber \\
K_{156l} &= c_{5l}(K_{117l} + K_{62l}L_8) - s_{5l}(K_{74l}  \nonumber \\
&- K_{101l}L_8) \nonumber \\
K_{157l} &= c_{5l}(K_{118l} + K_{63l}L_8) - s_{5l}(K_{75l}  \nonumber \\
&- K_{102l}L_8) \nonumber \\
K_{158l} &= c_{5l}(K_{119l} + K_{64l}L_8) - s_{5l}(K_{76l}  \nonumber \\
&- K_{103l}L_8) \nonumber \\
K_{159l} &= c_{5l}(K_{120l} + L_8c_{3l}) - s_{5l}(K_{77l}  \nonumber \\
&- K_{104l}L_8) \nonumber \\
K_{160l} &= L_8s_{4l}s_{5l} \nonumber \\
K_{161l} &= L_8c_{5l} \nonumber \\
 \bar{v}_{8l} &= \left[\begin{matrix} K_{145l}\dot{\psi} + K_{149l}\dot{q}_{1l} + K_{150l}\dot{q}_{2l} + K_{151l}\dot{q}_{3l} + K_{152l}\dot{q}_{4l} + K_{147l}\dot{q}_{w} + K_{146l}\dot{q}_{imu} + K_{148l}\dot{q}_{torso} + K_{144l}\dot{x} & - K_{108l}\dot{\psi} - K_{112l}\dot{q}_{1l} - K_{113l}\dot{q}_{2l} - K_{110l}\dot{q}_{w} - K_{109l}\dot{q}_{imu} - K_{111l}\dot{q}_{torso} - K_{107l}\dot{x} & K_{154l}\dot{\psi} + K_{158l}\dot{q}_{1l} + K_{159l}\dot{q}_{2l} + K_{160l}\dot{q}_{3l} + K_{161l}\dot{q}_{4l} + K_{156l}\dot{q}_{w} + K_{155l}\dot{q}_{imu} + K_{157l}\dot{q}_{torso} + K_{153l}\dot{x} &  \end{matrix}\right] 
 \nonumber \\ 
 \bar\alpha_{8l} &= {}^{8l}A_{7l} \bar\alpha_{7l} + \ddot{q}_{8l} \bar{e}_{8l} + \dot{q}_{8l} \left(\bar\omega_{8l} \times \bar{e}_{8l}\right) 
 \nonumber \\ 
 \bar\alpha_{8l} &= \left[\begin{matrix} K_{129l}\dot{q}_{5l} + c_{5l}(K_{85l} + \ddot{q}_{4l} + K_{61l}\ddot{\psi} + K_{64l}\ddot{q}_{1l} + K_{62l}\ddot{q}_{w} + K_{62l}\ddot{q}_{imu} + K_{63l}\ddot{q}_{torso} + \ddot{q}_{2l}c_{3l}) + s_{5l}(K_{122l} + K_{100l}\ddot{\psi} + K_{103l}\ddot{q}_{1l} + K_{104l}\ddot{q}_{2l} + K_{101l}\ddot{q}_{w} + K_{101l}\ddot{q}_{imu} + K_{102l}\ddot{q}_{torso} + \ddot{q}_{3l}s_{4l}) & - K_{121l} - \ddot{q}_{5l} - K_{95l}\ddot{\psi} - K_{98l}\ddot{q}_{1l} - K_{99l}\ddot{q}_{2l} - K_{96l}\ddot{q}_{w} - K_{96l}\ddot{q}_{imu} - K_{97l}\ddot{q}_{torso} - \ddot{q}_{3l}c_{4l} & c_{5l}(K_{122l} + K_{100l}\ddot{\psi} + K_{103l}\ddot{q}_{1l} + K_{104l}\ddot{q}_{2l} + K_{101l}\ddot{q}_{w} + K_{101l}\ddot{q}_{imu} + K_{102l}\ddot{q}_{torso} + \ddot{q}_{3l}s_{4l}) - s_{5l}(K_{85l} + \ddot{q}_{4l} + K_{61l}\ddot{\psi} + K_{64l}\ddot{q}_{1l} + K_{62l}\ddot{q}_{w} + K_{62l}\ddot{q}_{imu} + K_{63l}\ddot{q}_{torso} + \ddot{q}_{2l}c_{3l}) - K_{127l}\dot{q}_{5l} &  \end{matrix}\right] 
 \nonumber \\ 
K_{162l} &= K_{129l}\dot{q}_{5l} + K_{85l}c_{5l} + K_{122l}s_{5l} \nonumber \\
K_{163l} &= K_{122l}c_{5l} - K_{127l}\dot{q}_{5l} - K_{85l}s_{5l} \nonumber \\
 \bar\alpha_{8l} &= \left[\begin{matrix} K_{162l} + K_{130l}\ddot{\psi} + K_{133l}\ddot{q}_{1l} + K_{134l}\ddot{q}_{2l} + K_{135l}\ddot{q}_{3l} + K_{131l}\ddot{q}_{w} + K_{131l}\ddot{q}_{imu} + K_{132l}\ddot{q}_{torso} + \ddot{q}_{4l}c_{5l} & - K_{121l} - \ddot{q}_{5l} - K_{95l}\ddot{\psi} - K_{98l}\ddot{q}_{1l} - K_{99l}\ddot{q}_{2l} - K_{96l}\ddot{q}_{w} - K_{96l}\ddot{q}_{imu} - K_{97l}\ddot{q}_{torso} - \ddot{q}_{3l}c_{4l} & K_{163l} + K_{136l}\ddot{\psi} + K_{139l}\ddot{q}_{1l} + K_{140l}\ddot{q}_{2l} + K_{141l}\ddot{q}_{3l} + K_{137l}\ddot{q}_{w} + K_{137l}\ddot{q}_{imu} + K_{138l}\ddot{q}_{torso} - \ddot{q}_{4l}s_{5l} &  \end{matrix}\right] 
 \nonumber \\ 
 \bar{a}_{8l} &= {}^{8l}A_{7l} \left(\bar{a}_{7l} + \bar\alpha_{7l} \times \bar{P}_{8l} + \bar\omega_{7l} \times \left(\bar\omega_{7l} \times \bar{P}_{8l}\right)\right) 
 \nonumber \\ 
 \bar\alpha_{8l} &= \left[\begin{matrix} s_{5l}(K_{124l} + K_{115l}\ddot{\psi} + K_{119l}\ddot{q}_{1l} + K_{120l}\ddot{q}_{2l} + K_{117l}\ddot{q}_{w} + K_{116l}\ddot{q}_{imu} + K_{118l}\ddot{q}_{torso} + K_{114l}\ddot{x} + L_8(K_{85l} + \ddot{q}_{4l} + K_{61l}\ddot{\psi} + K_{64l}\ddot{q}_{1l} + K_{62l}\ddot{q}_{w} + K_{62l}\ddot{q}_{imu} + K_{63l}\ddot{q}_{torso} + \ddot{q}_{2l}c_{3l}) - K_{93l}K_{94l}L_8) + c_{5l}(K_{87l} + K_{72l}\ddot{\psi} + K_{76l}\ddot{q}_{1l} + K_{77l}\ddot{q}_{2l} + K_{74l}\ddot{q}_{w} + K_{73l}\ddot{q}_{imu} + K_{75l}\ddot{q}_{torso} + K_{71l}\ddot{x} - L_8(K_{122l} + K_{100l}\ddot{\psi} + K_{103l}\ddot{q}_{1l} + K_{104l}\ddot{q}_{2l} + K_{101l}\ddot{q}_{w} + K_{101l}\ddot{q}_{imu} + K_{102l}\ddot{q}_{torso} + \ddot{q}_{3l}s_{4l}) + K_{92l}K_{93l}L_8) & - K_{123l} - K_{108l}\ddot{\psi} - K_{112l}\ddot{q}_{1l} - K_{113l}\ddot{q}_{2l} - K_{110l}\ddot{q}_{w} - K_{109l}\ddot{q}_{imu} - K_{111l}\ddot{q}_{torso} - K_{107l}\ddot{x} - K_{92l}^2L_8 - K_{94l}^2L_8 & c_{5l}(K_{124l} + K_{115l}\ddot{\psi} + K_{119l}\ddot{q}_{1l} + K_{120l}\ddot{q}_{2l} + K_{117l}\ddot{q}_{w} + K_{116l}\ddot{q}_{imu} + K_{118l}\ddot{q}_{torso} + K_{114l}\ddot{x} + L_8(K_{85l} + \ddot{q}_{4l} + K_{61l}\ddot{\psi} + K_{64l}\ddot{q}_{1l} + K_{62l}\ddot{q}_{w} + K_{62l}\ddot{q}_{imu} + K_{63l}\ddot{q}_{torso} + \ddot{q}_{2l}c_{3l}) - K_{93l}K_{94l}L_8) - s_{5l}(K_{87l} + K_{72l}\ddot{\psi} + K_{76l}\ddot{q}_{1l} + K_{77l}\ddot{q}_{2l} + K_{74l}\ddot{q}_{w} + K_{73l}\ddot{q}_{imu} + K_{75l}\ddot{q}_{torso} + K_{71l}\ddot{x} - L_8(K_{122l} + K_{100l}\ddot{\psi} + K_{103l}\ddot{q}_{1l} + K_{104l}\ddot{q}_{2l} + K_{101l}\ddot{q}_{w} + K_{101l}\ddot{q}_{imu} + K_{102l}\ddot{q}_{torso} + \ddot{q}_{3l}s_{4l}) + K_{92l}K_{93l}L_8) &  \end{matrix}\right] 
 \nonumber \\ 
K_{164l} &= K_{87l}c_{5l} + K_{124l}s_{5l} - K_{122l}L_8c_{5l}  \nonumber \\
&+ K_{85l}L_8s_{5l} + K_{92l}K_{93l}L_8c_{5l}  \nonumber \\
&- K_{93l}K_{94l}L_8s_{5l} \nonumber \\
K_{165l} &= - K_{123l} - K_{92l}^2L_8 - K_{94l}^2L_8 \nonumber \\
K_{166l} &= K_{124l}c_{5l} - K_{87l}s_{5l} + K_{85l}L_8c_{5l}  \nonumber \\
&+ K_{122l}L_8s_{5l} - K_{93l}K_{94l}L_8c_{5l}  \nonumber \\
&- K_{92l}K_{93l}L_8s_{5l} \nonumber \\
 \bar{a}_{8l} &= \left[\begin{matrix} K_{164l} + K_{145l}\ddot{\psi} + K_{149l}\ddot{q}_{1l} + K_{150l}\ddot{q}_{2l} + K_{151l}\ddot{q}_{3l} + K_{152l}\ddot{q}_{4l} + K_{147l}\ddot{q}_{w} + K_{146l}\ddot{q}_{imu} + K_{148l}\ddot{q}_{torso} + K_{144l}\ddot{x} & K_{165l} - K_{108l}\ddot{\psi} - K_{112l}\ddot{q}_{1l} - K_{113l}\ddot{q}_{2l} - K_{110l}\ddot{q}_{w} - K_{109l}\ddot{q}_{imu} - K_{111l}\ddot{q}_{torso} - K_{107l}\ddot{x} & K_{166l} + K_{154l}\ddot{\psi} + K_{158l}\ddot{q}_{1l} + K_{159l}\ddot{q}_{2l} + K_{160l}\ddot{q}_{3l} + K_{161l}\ddot{q}_{4l} + K_{156l}\ddot{q}_{w} + K_{155l}\ddot{q}_{imu} + K_{157l}\ddot{q}_{torso} + K_{153l}\ddot{x} &  \end{matrix}\right] 
 \nonumber \\ 
 \bar{g}_{8l} &= {}^{8l}A_{7l} \bar{g}_{7l} 
 \nonumber \\ 
 \bar{g}_{8l} &= \left[\begin{matrix} K_{90l}gc_{5l} + K_{126l}gs_{5l} & -K_{125l}g & K_{126l}gc_{5l} - K_{90l}gs_{5l} &  \end{matrix}\right] 
 \nonumber \\ 
K_{167l} &= K_{90l}c_{5l} + K_{126l}s_{5l} \nonumber \\
K_{168l} &= K_{126l}c_{5l} - K_{90l}s_{5l} \nonumber \\
 \bar{g}_{8l} &= \left[\begin{matrix} K_{167l}g & -K_{125l}g & K_{168l}g &  \end{matrix}\right] 
 \nonumber \\ 
 m_{8l}\bar{S}_{8l}^{\times}\bar{g}_{8l} &= \mathbf{MS}_{8l} \times \bar{g}_{8l} 
 \nonumber \\ 
 m_{8l}\bar{S}_{8l}^{\times}\bar{g}_{8l} &= \left[\begin{matrix} K_{168l}\mathbf{MY}_{8r}g + K_{125l}\mathbf{MZ}_{8r}g & K_{167l}\mathbf{MZ}_{8r}g - K_{168l}\mathbf{MX}_{8r}g & - K_{125l}\mathbf{MX}_{8r}g - K_{167l}\mathbf{MY}_{8r}g &  \end{matrix}\right] 
 \nonumber \\ 
D_{193l} &= K_{168l}\mathbf{MY}_{8r} + K_{125l}\mathbf{MZ}_{8r} \nonumber \\
D_{194l} &= K_{167l}\mathbf{MZ}_{8r} - K_{168l}\mathbf{MX}_{8r} \nonumber \\
D_{195l} &= - K_{125l}\mathbf{MX}_{8r} - K_{167l}\mathbf{MY}_{8r} \nonumber \\
 m_{8l}\bar{S}_{8l}^{\times}\bar{g}_{8l} &= \left[\begin{matrix} D_{193l}g & D_{194l}g & D_{195l}g &  \end{matrix}\right] 
 \nonumber \\ 
 m_{8l}\bar{a}_{G(8l)} &= m_{8l}\bar{a}_{8l} + \bar\alpha_{8l} \times \mathbf{MS}_{8l} + \bar\omega_{8l} \times \left(\bar\omega_{8l} \times \mathbf{MS}_{8l}\right) 
 \nonumber \\ 
 m_{8l}\bar{a}_{G(8l)} &= \left[\begin{matrix} m_{8r}(K_{164l} + K_{145l}\ddot{\psi} + K_{149l}\ddot{q}_{1l} + K_{150l}\ddot{q}_{2l} + K_{151l}\ddot{q}_{3l} + K_{152l}\ddot{q}_{4l} + K_{147l}\ddot{q}_{w} + K_{146l}\ddot{q}_{imu} + K_{148l}\ddot{q}_{torso} + K_{144l}\ddot{x}) - \mathbf{MZ}_{8r}(K_{121l} + \ddot{q}_{5l} + K_{95l}\ddot{\psi} + K_{98l}\ddot{q}_{1l} + K_{99l}\ddot{q}_{2l} + K_{96l}\ddot{q}_{w} + K_{96l}\ddot{q}_{imu} + K_{97l}\ddot{q}_{torso} + \ddot{q}_{3l}c_{4l}) - K_{128l}(K_{128l}\mathbf{MX}_{8r} - K_{127l}\mathbf{MY}_{8r}) - K_{129l}(K_{129l}\mathbf{MX}_{8r} - K_{127l}\mathbf{MZ}_{8r}) - \mathbf{MY}_{8r}(K_{163l} + K_{136l}\ddot{\psi} + K_{139l}\ddot{q}_{1l} + K_{140l}\ddot{q}_{2l} + K_{141l}\ddot{q}_{3l} + K_{137l}\ddot{q}_{w} + K_{137l}\ddot{q}_{imu} + K_{138l}\ddot{q}_{torso} - \ddot{q}_{4l}s_{5l}) & \mathbf{MX}_{8r}(K_{163l} + K_{136l}\ddot{\psi} + K_{139l}\ddot{q}_{1l} + K_{140l}\ddot{q}_{2l} + K_{141l}\ddot{q}_{3l} + K_{137l}\ddot{q}_{w} + K_{137l}\ddot{q}_{imu} + K_{138l}\ddot{q}_{torso} - \ddot{q}_{4l}s_{5l}) - \mathbf{MZ}_{8r}(K_{162l} + K_{130l}\ddot{\psi} + K_{133l}\ddot{q}_{1l} + K_{134l}\ddot{q}_{2l} + K_{135l}\ddot{q}_{3l} + K_{131l}\ddot{q}_{w} + K_{131l}\ddot{q}_{imu} + K_{132l}\ddot{q}_{torso} + \ddot{q}_{4l}c_{5l}) - m_{8r}(K_{108l}\ddot{\psi} - K_{165l} + K_{112l}\ddot{q}_{1l} + K_{113l}\ddot{q}_{2l} + K_{110l}\ddot{q}_{w} + K_{109l}\ddot{q}_{imu} + K_{111l}\ddot{q}_{torso} + K_{107l}\ddot{x}) + K_{127l}(K_{128l}\mathbf{MX}_{8r} - K_{127l}\mathbf{MY}_{8r}) - K_{129l}(K_{129l}\mathbf{MY}_{8r} - K_{128l}\mathbf{MZ}_{8r}) & \mathbf{MY}_{8r}(K_{162l} + K_{130l}\ddot{\psi} + K_{133l}\ddot{q}_{1l} + K_{134l}\ddot{q}_{2l} + K_{135l}\ddot{q}_{3l} + K_{131l}\ddot{q}_{w} + K_{131l}\ddot{q}_{imu} + K_{132l}\ddot{q}_{torso} + \ddot{q}_{4l}c_{5l}) + \mathbf{MX}_{8r}(K_{121l} + \ddot{q}_{5l} + K_{95l}\ddot{\psi} + K_{98l}\ddot{q}_{1l} + K_{99l}\ddot{q}_{2l} + K_{96l}\ddot{q}_{w} + K_{96l}\ddot{q}_{imu} + K_{97l}\ddot{q}_{torso} + \ddot{q}_{3l}c_{4l}) + K_{127l}(K_{129l}\mathbf{MX}_{8r} - K_{127l}\mathbf{MZ}_{8r}) + K_{128l}(K_{129l}\mathbf{MY}_{8r} - K_{128l}\mathbf{MZ}_{8r}) + m_{8r}(K_{166l} + K_{154l}\ddot{\psi} + K_{158l}\ddot{q}_{1l} + K_{159l}\ddot{q}_{2l} + K_{160l}\ddot{q}_{3l} + K_{161l}\ddot{q}_{4l} + K_{156l}\ddot{q}_{w} + K_{155l}\ddot{q}_{imu} + K_{157l}\ddot{q}_{torso} + K_{153l}\ddot{x}) &  \end{matrix}\right] 
 \nonumber \\ 
D_{196l} &= K_{144l}m_{8r} \nonumber \\
D_{197l} &= K_{145l}m_{8r} - K_{136l}\mathbf{MY}_{8r} - K_{95l}\mathbf{MZ}_{8r} \nonumber \\
D_{198l} &= K_{146l}m_{8r} - K_{137l}\mathbf{MY}_{8r} - K_{96l}\mathbf{MZ}_{8r} \nonumber \\
D_{199l} &= K_{147l}m_{8r} - K_{137l}\mathbf{MY}_{8r} - K_{96l}\mathbf{MZ}_{8r} \nonumber \\
D_{200l} &= K_{148l}m_{8r} - K_{138l}\mathbf{MY}_{8r} - K_{97l}\mathbf{MZ}_{8r} \nonumber \\
D_{201l} &= K_{149l}m_{8r} - K_{139l}\mathbf{MY}_{8r} - K_{98l}\mathbf{MZ}_{8r} \nonumber \\
D_{202l} &= K_{150l}m_{8r} - K_{140l}\mathbf{MY}_{8r} - K_{99l}\mathbf{MZ}_{8r} \nonumber \\
D_{203l} &= K_{151l}m_{8r} - \mathbf{MZ}_{8r}c_{4l} - K_{141l}\mathbf{MY}_{8r} \nonumber \\
D_{204l} &= K_{152l}m_{8r} + \mathbf{MY}_{8r}s_{5l} \nonumber \\
D_{205l} &= K_{164l}m_{8r} - K_{128l}^2\mathbf{MX}_{8r} - K_{129l}^2\mathbf{MX}_{8r}  \nonumber \\
&- K_{163l}\mathbf{MY}_{8r} - K_{121l}\mathbf{MZ}_{8r} + K_{127l}K_{128l}\mathbf{MY}_{8r}  \nonumber \\
&+ K_{127l}K_{129l}\mathbf{MZ}_{8r} \nonumber \\
D_{206l} &= -K_{107l}m_{8r} \nonumber \\
D_{207l} &= K_{136l}\mathbf{MX}_{8r} - K_{108l}m_{8r} - K_{130l}\mathbf{MZ}_{8r} \nonumber \\
D_{208l} &= K_{137l}\mathbf{MX}_{8r} - K_{109l}m_{8r} - K_{131l}\mathbf{MZ}_{8r} \nonumber \\
D_{209l} &= K_{137l}\mathbf{MX}_{8r} - K_{110l}m_{8r} - K_{131l}\mathbf{MZ}_{8r} \nonumber \\
D_{210l} &= K_{138l}\mathbf{MX}_{8r} - K_{111l}m_{8r} - K_{132l}\mathbf{MZ}_{8r} \nonumber \\
D_{211l} &= K_{139l}\mathbf{MX}_{8r} - K_{112l}m_{8r} - K_{133l}\mathbf{MZ}_{8r} \nonumber \\
D_{212l} &= K_{140l}\mathbf{MX}_{8r} - K_{113l}m_{8r} - K_{134l}\mathbf{MZ}_{8r} \nonumber \\
D_{213l} &= K_{141l}\mathbf{MX}_{8r} - K_{135l}\mathbf{MZ}_{8r} \nonumber \\
D_{214l} &= - \mathbf{MZ}_{8r}c_{5l} - \mathbf{MX}_{8r}s_{5l} \nonumber \\
D_{215l} &= K_{165l}m_{8r} - K_{127l}^2\mathbf{MY}_{8r} - K_{129l}^2\mathbf{MY}_{8r}  \nonumber \\
&+ K_{163l}\mathbf{MX}_{8r} - K_{162l}\mathbf{MZ}_{8r} + K_{127l}K_{128l}\mathbf{MX}_{8r}  \nonumber \\
&+ K_{128l}K_{129l}\mathbf{MZ}_{8r} \nonumber \\
D_{216l} &= K_{153l}m_{8r} \nonumber \\
D_{217l} &= K_{154l}m_{8r} + K_{95l}\mathbf{MX}_{8r} + K_{130l}\mathbf{MY}_{8r} \nonumber \\
D_{218l} &= K_{155l}m_{8r} + K_{96l}\mathbf{MX}_{8r} + K_{131l}\mathbf{MY}_{8r} \nonumber \\
D_{219l} &= K_{156l}m_{8r} + K_{96l}\mathbf{MX}_{8r} + K_{131l}\mathbf{MY}_{8r} \nonumber \\
D_{220l} &= K_{157l}m_{8r} + K_{97l}\mathbf{MX}_{8r} + K_{132l}\mathbf{MY}_{8r} \nonumber \\
D_{221l} &= K_{158l}m_{8r} + K_{98l}\mathbf{MX}_{8r} + K_{133l}\mathbf{MY}_{8r} \nonumber \\
D_{222l} &= K_{159l}m_{8r} + K_{99l}\mathbf{MX}_{8r} + K_{134l}\mathbf{MY}_{8r} \nonumber \\
D_{223l} &= K_{160l}m_{8r} + \mathbf{MX}_{8r}c_{4l} + K_{135l}\mathbf{MY}_{8r} \nonumber \\
D_{224l} &= K_{161l}m_{8r} + \mathbf{MY}_{8r}c_{5l} \nonumber \\
D_{225l} &= K_{166l}m_{8r} - K_{127l}^2\mathbf{MZ}_{8r} - K_{128l}^2\mathbf{MZ}_{8r}  \nonumber \\
&+ K_{121l}\mathbf{MX}_{8r} + K_{162l}\mathbf{MY}_{8r} + K_{127l}K_{129l}\mathbf{MX}_{8r}  \nonumber \\
&+ K_{128l}K_{129l}\mathbf{MY}_{8r} \nonumber \\
 m_{8l}\bar{a}_{G(8l)} &= \left[\begin{matrix} D_{205l} + D_{197l}\ddot{\psi} + D_{201l}\ddot{q}_{1l} + D_{202l}\ddot{q}_{2l} + D_{203l}\ddot{q}_{3l} + D_{204l}\ddot{q}_{4l} + D_{199l}\ddot{q}_{w} + D_{198l}\ddot{q}_{imu} + D_{200l}\ddot{q}_{torso} + D_{196l}\ddot{x} - \mathbf{MZ}_{8r}\ddot{q}_{5l} & D_{215l} + D_{207l}\ddot{\psi} + D_{211l}\ddot{q}_{1l} + D_{212l}\ddot{q}_{2l} + D_{213l}\ddot{q}_{3l} + D_{214l}\ddot{q}_{4l} + D_{209l}\ddot{q}_{w} + D_{208l}\ddot{q}_{imu} + D_{210l}\ddot{q}_{torso} + D_{206l}\ddot{x} & D_{225l} + D_{217l}\ddot{\psi} + D_{221l}\ddot{q}_{1l} + D_{222l}\ddot{q}_{2l} + D_{223l}\ddot{q}_{3l} + D_{224l}\ddot{q}_{4l} + D_{219l}\ddot{q}_{w} + D_{218l}\ddot{q}_{imu} + D_{220l}\ddot{q}_{torso} + D_{216l}\ddot{x} + \mathbf{MX}_{8r}\ddot{q}_{5l} &  \end{matrix}\right] 
 \nonumber \\ 
 \dot{\bar{H}}_{8l} &= \mathbf{MS}_{8l} \times \bar{a}_{8l} + J_{8l}\bar{\alpha}_{8l} + \bar\omega_{8l} \times J_{8l}\bar{\omega}_{8l} 
 \nonumber \\ 
 \dot{\bar{H}}_{8l} &= \left[\begin{matrix} K_{128l}(K_{127l}\mathbf{XZ}_{8r} + K_{128l}\mathbf{YZ}_{8r} + K_{129l}\mathbf{ZZ}_{8r}) - K_{129l}(K_{127l}\mathbf{XY}_{8r} + K_{128l}\mathbf{YY}_{8r} + K_{129l}\mathbf{YZ}_{8r}) + \mathbf{MZ}_{8r}(K_{108l}\ddot{\psi} - K_{165l} + K_{112l}\ddot{q}_{1l} + K_{113l}\ddot{q}_{2l} + K_{110l}\ddot{q}_{w} + K_{109l}\ddot{q}_{imu} + K_{111l}\ddot{q}_{torso} + K_{107l}\ddot{x}) + \mathbf{XX}_{8r}(K_{162l} + K_{130l}\ddot{\psi} + K_{133l}\ddot{q}_{1l} + K_{134l}\ddot{q}_{2l} + K_{135l}\ddot{q}_{3l} + K_{131l}\ddot{q}_{w} + K_{131l}\ddot{q}_{imu} + K_{132l}\ddot{q}_{torso} + \ddot{q}_{4l}c_{5l}) + \mathbf{XZ}_{8r}(K_{163l} + K_{136l}\ddot{\psi} + K_{139l}\ddot{q}_{1l} + K_{140l}\ddot{q}_{2l} + K_{141l}\ddot{q}_{3l} + K_{137l}\ddot{q}_{w} + K_{137l}\ddot{q}_{imu} + K_{138l}\ddot{q}_{torso} - \ddot{q}_{4l}s_{5l}) - \mathbf{XY}_{8r}(K_{121l} + \ddot{q}_{5l} + K_{95l}\ddot{\psi} + K_{98l}\ddot{q}_{1l} + K_{99l}\ddot{q}_{2l} + K_{96l}\ddot{q}_{w} + K_{96l}\ddot{q}_{imu} + K_{97l}\ddot{q}_{torso} + \ddot{q}_{3l}c_{4l}) + \mathbf{MY}_{8r}(K_{166l} + K_{154l}\ddot{\psi} + K_{158l}\ddot{q}_{1l} + K_{159l}\ddot{q}_{2l} + K_{160l}\ddot{q}_{3l} + K_{161l}\ddot{q}_{4l} + K_{156l}\ddot{q}_{w} + K_{155l}\ddot{q}_{imu} + K_{157l}\ddot{q}_{torso} + K_{153l}\ddot{x}) & K_{129l}(K_{127l}\mathbf{XX}_{8r} + K_{128l}\mathbf{XY}_{8r} + K_{129l}\mathbf{XZ}_{8r}) - K_{127l}(K_{127l}\mathbf{XZ}_{8r} + K_{128l}\mathbf{YZ}_{8r} + K_{129l}\mathbf{ZZ}_{8r}) + \mathbf{XY}_{8r}(K_{162l} + K_{130l}\ddot{\psi} + K_{133l}\ddot{q}_{1l} + K_{134l}\ddot{q}_{2l} + K_{135l}\ddot{q}_{3l} + K_{131l}\ddot{q}_{w} + K_{131l}\ddot{q}_{imu} + K_{132l}\ddot{q}_{torso} + \ddot{q}_{4l}c_{5l}) + \mathbf{YZ}_{8r}(K_{163l} + K_{136l}\ddot{\psi} + K_{139l}\ddot{q}_{1l} + K_{140l}\ddot{q}_{2l} + K_{141l}\ddot{q}_{3l} + K_{137l}\ddot{q}_{w} + K_{137l}\ddot{q}_{imu} + K_{138l}\ddot{q}_{torso} - \ddot{q}_{4l}s_{5l}) - \mathbf{YY}_{8r}(K_{121l} + \ddot{q}_{5l} + K_{95l}\ddot{\psi} + K_{98l}\ddot{q}_{1l} + K_{99l}\ddot{q}_{2l} + K_{96l}\ddot{q}_{w} + K_{96l}\ddot{q}_{imu} + K_{97l}\ddot{q}_{torso} + \ddot{q}_{3l}c_{4l}) - \mathbf{MX}_{8r}(K_{166l} + K_{154l}\ddot{\psi} + K_{158l}\ddot{q}_{1l} + K_{159l}\ddot{q}_{2l} + K_{160l}\ddot{q}_{3l} + K_{161l}\ddot{q}_{4l} + K_{156l}\ddot{q}_{w} + K_{155l}\ddot{q}_{imu} + K_{157l}\ddot{q}_{torso} + K_{153l}\ddot{x}) + \mathbf{MZ}_{8r}(K_{164l} + K_{145l}\ddot{\psi} + K_{149l}\ddot{q}_{1l} + K_{150l}\ddot{q}_{2l} + K_{151l}\ddot{q}_{3l} + K_{152l}\ddot{q}_{4l} + K_{147l}\ddot{q}_{w} + K_{146l}\ddot{q}_{imu} + K_{148l}\ddot{q}_{torso} + K_{144l}\ddot{x}) & K_{127l}(K_{127l}\mathbf{XY}_{8r} + K_{128l}\mathbf{YY}_{8r} + K_{129l}\mathbf{YZ}_{8r}) - K_{128l}(K_{127l}\mathbf{XX}_{8r} + K_{128l}\mathbf{XY}_{8r} + K_{129l}\mathbf{XZ}_{8r}) - \mathbf{MX}_{8r}(K_{108l}\ddot{\psi} - K_{165l} + K_{112l}\ddot{q}_{1l} + K_{113l}\ddot{q}_{2l} + K_{110l}\ddot{q}_{w} + K_{109l}\ddot{q}_{imu} + K_{111l}\ddot{q}_{torso} + K_{107l}\ddot{x}) + \mathbf{XZ}_{8r}(K_{162l} + K_{130l}\ddot{\psi} + K_{133l}\ddot{q}_{1l} + K_{134l}\ddot{q}_{2l} + K_{135l}\ddot{q}_{3l} + K_{131l}\ddot{q}_{w} + K_{131l}\ddot{q}_{imu} + K_{132l}\ddot{q}_{torso} + \ddot{q}_{4l}c_{5l}) + \mathbf{ZZ}_{8r}(K_{163l} + K_{136l}\ddot{\psi} + K_{139l}\ddot{q}_{1l} + K_{140l}\ddot{q}_{2l} + K_{141l}\ddot{q}_{3l} + K_{137l}\ddot{q}_{w} + K_{137l}\ddot{q}_{imu} + K_{138l}\ddot{q}_{torso} - \ddot{q}_{4l}s_{5l}) - \mathbf{YZ}_{8r}(K_{121l} + \ddot{q}_{5l} + K_{95l}\ddot{\psi} + K_{98l}\ddot{q}_{1l} + K_{99l}\ddot{q}_{2l} + K_{96l}\ddot{q}_{w} + K_{96l}\ddot{q}_{imu} + K_{97l}\ddot{q}_{torso} + \ddot{q}_{3l}c_{4l}) - \mathbf{MY}_{8r}(K_{164l} + K_{145l}\ddot{\psi} + K_{149l}\ddot{q}_{1l} + K_{150l}\ddot{q}_{2l} + K_{151l}\ddot{q}_{3l} + K_{152l}\ddot{q}_{4l} + K_{147l}\ddot{q}_{w} + K_{146l}\ddot{q}_{imu} + K_{148l}\ddot{q}_{torso} + K_{144l}\ddot{x}) &  \end{matrix}\right] 
 \nonumber \\ 
D_{226l} &= K_{153l}\mathbf{MY}_{8r} + K_{107l}\mathbf{MZ}_{8r} \nonumber \\
D_{227l} &= K_{130l}\mathbf{XX}_{8r} - K_{95l}\mathbf{XY}_{8r} + K_{136l}\mathbf{XZ}_{8r}  \nonumber \\
&+ K_{154l}\mathbf{MY}_{8r} + K_{108l}\mathbf{MZ}_{8r} \nonumber \\
D_{228l} &= K_{131l}\mathbf{XX}_{8r} - K_{96l}\mathbf{XY}_{8r} + K_{137l}\mathbf{XZ}_{8r}  \nonumber \\
&+ K_{155l}\mathbf{MY}_{8r} + K_{109l}\mathbf{MZ}_{8r} \nonumber \\
D_{229l} &= K_{131l}\mathbf{XX}_{8r} - K_{96l}\mathbf{XY}_{8r} + K_{137l}\mathbf{XZ}_{8r}  \nonumber \\
&+ K_{156l}\mathbf{MY}_{8r} + K_{110l}\mathbf{MZ}_{8r} \nonumber \\
D_{230l} &= K_{132l}\mathbf{XX}_{8r} - K_{97l}\mathbf{XY}_{8r} + K_{138l}\mathbf{XZ}_{8r}  \nonumber \\
&+ K_{157l}\mathbf{MY}_{8r} + K_{111l}\mathbf{MZ}_{8r} \nonumber \\
D_{231l} &= K_{133l}\mathbf{XX}_{8r} - K_{98l}\mathbf{XY}_{8r} + K_{139l}\mathbf{XZ}_{8r}  \nonumber \\
&+ K_{158l}\mathbf{MY}_{8r} + K_{112l}\mathbf{MZ}_{8r} \nonumber \\
D_{232l} &= K_{134l}\mathbf{XX}_{8r} - K_{99l}\mathbf{XY}_{8r} + K_{140l}\mathbf{XZ}_{8r}  \nonumber \\
&+ K_{159l}\mathbf{MY}_{8r} + K_{113l}\mathbf{MZ}_{8r} \nonumber \\
D_{233l} &= K_{135l}\mathbf{XX}_{8r} + K_{141l}\mathbf{XZ}_{8r} - \mathbf{XY}_{8r}c_{4l}  \nonumber \\
&+ K_{160l}\mathbf{MY}_{8r} \nonumber \\
D_{234l} &= \mathbf{XX}_{8r}c_{5l} - \mathbf{XZ}_{8r}s_{5l} + K_{161l}\mathbf{MY}_{8r} \nonumber \\
D_{235l} &= K_{162l}\mathbf{XX}_{8r} - K_{121l}\mathbf{XY}_{8r} + K_{163l}\mathbf{XZ}_{8r}  \nonumber \\
&+ K_{128l}^2\mathbf{YZ}_{8r} - K_{129l}^2\mathbf{YZ}_{8r} + K_{166l}\mathbf{MY}_{8r}  \nonumber \\
&- K_{165l}\mathbf{MZ}_{8r} - K_{127l}K_{129l}\mathbf{XY}_{8r} + K_{127l}K_{128l}\mathbf{XZ}_{8r}  \nonumber \\
&- K_{128l}K_{129l}\mathbf{YY}_{8r} + K_{128l}K_{129l}\mathbf{ZZ}_{8r} \nonumber \\
D_{236l} &= K_{144l}\mathbf{MZ}_{8r} - K_{153l}\mathbf{MX}_{8r} \nonumber \\
D_{237l} &= K_{130l}\mathbf{XY}_{8r} - K_{95l}\mathbf{YY}_{8r} + K_{136l}\mathbf{YZ}_{8r}  \nonumber \\
&- K_{154l}\mathbf{MX}_{8r} + K_{145l}\mathbf{MZ}_{8r} \nonumber \\
D_{238l} &= K_{131l}\mathbf{XY}_{8r} - K_{96l}\mathbf{YY}_{8r} + K_{137l}\mathbf{YZ}_{8r}  \nonumber \\
&- K_{155l}\mathbf{MX}_{8r} + K_{146l}\mathbf{MZ}_{8r} \nonumber \\
D_{239l} &= K_{131l}\mathbf{XY}_{8r} - K_{96l}\mathbf{YY}_{8r} + K_{137l}\mathbf{YZ}_{8r}  \nonumber \\
&- K_{156l}\mathbf{MX}_{8r} + K_{147l}\mathbf{MZ}_{8r} \nonumber \\
D_{240l} &= K_{132l}\mathbf{XY}_{8r} - K_{97l}\mathbf{YY}_{8r} + K_{138l}\mathbf{YZ}_{8r}  \nonumber \\
&- K_{157l}\mathbf{MX}_{8r} + K_{148l}\mathbf{MZ}_{8r} \nonumber \\
D_{241l} &= K_{133l}\mathbf{XY}_{8r} - K_{98l}\mathbf{YY}_{8r} + K_{139l}\mathbf{YZ}_{8r}  \nonumber \\
&- K_{158l}\mathbf{MX}_{8r} + K_{149l}\mathbf{MZ}_{8r} \nonumber \\
D_{242l} &= K_{134l}\mathbf{XY}_{8r} - K_{99l}\mathbf{YY}_{8r} + K_{140l}\mathbf{YZ}_{8r}  \nonumber \\
&- K_{159l}\mathbf{MX}_{8r} + K_{150l}\mathbf{MZ}_{8r} \nonumber \\
D_{243l} &= K_{135l}\mathbf{XY}_{8r} + K_{141l}\mathbf{YZ}_{8r} - \mathbf{YY}_{8r}c_{4l}  \nonumber \\
&- K_{160l}\mathbf{MX}_{8r} + K_{151l}\mathbf{MZ}_{8r} \nonumber \\
D_{244l} &= \mathbf{XY}_{8r}c_{5l} - \mathbf{YZ}_{8r}s_{5l} - K_{161l}\mathbf{MX}_{8r}  \nonumber \\
&+ K_{152l}\mathbf{MZ}_{8r} \nonumber \\
D_{245l} &= K_{162l}\mathbf{XY}_{8r} - K_{121l}\mathbf{YY}_{8r} + K_{163l}\mathbf{YZ}_{8r}  \nonumber \\
&- K_{127l}^2\mathbf{XZ}_{8r} + K_{129l}^2\mathbf{XZ}_{8r} - K_{166l}\mathbf{MX}_{8r}  \nonumber \\
&+ K_{164l}\mathbf{MZ}_{8r} + K_{127l}K_{129l}\mathbf{XX}_{8r} + K_{128l}K_{129l}\mathbf{XY}_{8r}  \nonumber \\
&- K_{127l}K_{128l}\mathbf{YZ}_{8r} - K_{127l}K_{129l}\mathbf{ZZ}_{8r} \nonumber \\
D_{246l} &= - K_{107l}\mathbf{MX}_{8r} - K_{144l}\mathbf{MY}_{8r} \nonumber \\
D_{247l} &= K_{130l}\mathbf{XZ}_{8r} - K_{95l}\mathbf{YZ}_{8r} + K_{136l}\mathbf{ZZ}_{8r}  \nonumber \\
&- K_{108l}\mathbf{MX}_{8r} - K_{145l}\mathbf{MY}_{8r} \nonumber \\
D_{248l} &= K_{131l}\mathbf{XZ}_{8r} - K_{96l}\mathbf{YZ}_{8r} + K_{137l}\mathbf{ZZ}_{8r}  \nonumber \\
&- K_{109l}\mathbf{MX}_{8r} - K_{146l}\mathbf{MY}_{8r} \nonumber \\
D_{249l} &= K_{131l}\mathbf{XZ}_{8r} - K_{96l}\mathbf{YZ}_{8r} + K_{137l}\mathbf{ZZ}_{8r}  \nonumber \\
&- K_{110l}\mathbf{MX}_{8r} - K_{147l}\mathbf{MY}_{8r} \nonumber \\
D_{250l} &= K_{132l}\mathbf{XZ}_{8r} - K_{97l}\mathbf{YZ}_{8r} + K_{138l}\mathbf{ZZ}_{8r}  \nonumber \\
&- K_{111l}\mathbf{MX}_{8r} - K_{148l}\mathbf{MY}_{8r} \nonumber \\
D_{251l} &= K_{133l}\mathbf{XZ}_{8r} - K_{98l}\mathbf{YZ}_{8r} + K_{139l}\mathbf{ZZ}_{8r}  \nonumber \\
&- K_{112l}\mathbf{MX}_{8r} - K_{149l}\mathbf{MY}_{8r} \nonumber \\
D_{252l} &= K_{134l}\mathbf{XZ}_{8r} - K_{99l}\mathbf{YZ}_{8r} + K_{140l}\mathbf{ZZ}_{8r}  \nonumber \\
&- K_{113l}\mathbf{MX}_{8r} - K_{150l}\mathbf{MY}_{8r} \nonumber \\
D_{253l} &= K_{135l}\mathbf{XZ}_{8r} + K_{141l}\mathbf{ZZ}_{8r} - \mathbf{YZ}_{8r}c_{4l}  \nonumber \\
&- K_{151l}\mathbf{MY}_{8r} \nonumber \\
D_{254l} &= \mathbf{XZ}_{8r}c_{5l} - \mathbf{ZZ}_{8r}s_{5l} - K_{152l}\mathbf{MY}_{8r} \nonumber \\
D_{255l} &= K_{162l}\mathbf{XZ}_{8r} - K_{121l}\mathbf{YZ}_{8r} + K_{163l}\mathbf{ZZ}_{8r}  \nonumber \\
&+ K_{127l}^2\mathbf{XY}_{8r} - K_{128l}^2\mathbf{XY}_{8r} + K_{165l}\mathbf{MX}_{8r}  \nonumber \\
&- K_{164l}\mathbf{MY}_{8r} - K_{127l}K_{128l}\mathbf{XX}_{8r} - K_{128l}K_{129l}\mathbf{XZ}_{8r}  \nonumber \\
&+ K_{127l}K_{128l}\mathbf{YY}_{8r} + K_{127l}K_{129l}\mathbf{YZ}_{8r} \nonumber \\
 \dot{\bar{H}}_{8l} &= \left[\begin{matrix} D_{205l} + D_{197l}\ddot{\psi} + D_{201l}\ddot{q}_{1l} + D_{202l}\ddot{q}_{2l} + D_{203l}\ddot{q}_{3l} + D_{204l}\ddot{q}_{4l} + D_{199l}\ddot{q}_{w} + D_{198l}\ddot{q}_{imu} + D_{200l}\ddot{q}_{torso} + D_{196l}\ddot{x} - \mathbf{MZ}_{8r}\ddot{q}_{5l} & D_{215l} + D_{207l}\ddot{\psi} + D_{211l}\ddot{q}_{1l} + D_{212l}\ddot{q}_{2l} + D_{213l}\ddot{q}_{3l} + D_{214l}\ddot{q}_{4l} + D_{209l}\ddot{q}_{w} + D_{208l}\ddot{q}_{imu} + D_{210l}\ddot{q}_{torso} + D_{206l}\ddot{x} & D_{225l} + D_{217l}\ddot{\psi} + D_{221l}\ddot{q}_{1l} + D_{222l}\ddot{q}_{2l} + D_{223l}\ddot{q}_{3l} + D_{224l}\ddot{q}_{4l} + D_{219l}\ddot{q}_{w} + D_{218l}\ddot{q}_{imu} + D_{220l}\ddot{q}_{torso} + D_{216l}\ddot{x} + \mathbf{MX}_{8r}\ddot{q}_{5l} &  \end{matrix}\right] 
 \nonumber \\ 
 \bar\omega_{9l} &= {}^{9l}A_{8l} \bar\omega_{8l} + \dot{q}_{9l} \bar{e}_{9l} 
 \nonumber \\ 
 \bar\omega_{9l} &= \left[\begin{matrix} - K_{127l} - \dot{q}_{6l} & - K_{128l}c_{6l} - K_{129l}s_{6l} & K_{129l}c_{6l} - K_{128l}s_{6l} &  \end{matrix}\right] 
 \nonumber \\ 
K_{169l} &= - K_{127l} - \dot{q}_{6l} \nonumber \\
K_{170l} &= - K_{128l}c_{6l} - K_{129l}s_{6l} \nonumber \\
K_{171l} &= K_{129l}c_{6l} - K_{128l}s_{6l} \nonumber \\
 \bar\omega_{9l} &= \left[\begin{matrix} K_{169l} & K_{170l} & K_{171l} &  \end{matrix}\right] 
 \nonumber \\ 
 \bar\omega_{9l} &= \left[\begin{matrix} - \dot{q}_{6l} - K_{130l}\dot{\psi} - K_{133l}\dot{q}_{1l} - K_{134l}\dot{q}_{2l} - K_{135l}\dot{q}_{3l} - K_{131l}\dot{q}_{w} - K_{131l}\dot{q}_{imu} - K_{132l}\dot{q}_{torso} - \dot{q}_{4l}c_{5l} & c_{6l}(\dot{q}_{5l} + K_{95l}\dot{\psi} + K_{98l}\dot{q}_{1l} + K_{99l}\dot{q}_{2l} + K_{96l}\dot{q}_{w} + K_{96l}\dot{q}_{imu} + K_{97l}\dot{q}_{torso} + \dot{q}_{3l}c_{4l}) - s_{6l}(K_{136l}\dot{\psi} + K_{139l}\dot{q}_{1l} + K_{140l}\dot{q}_{2l} + K_{141l}\dot{q}_{3l} + K_{137l}\dot{q}_{w} + K_{137l}\dot{q}_{imu} + K_{138l}\dot{q}_{torso} - \dot{q}_{4l}s_{5l}) & c_{6l}(K_{136l}\dot{\psi} + K_{139l}\dot{q}_{1l} + K_{140l}\dot{q}_{2l} + K_{141l}\dot{q}_{3l} + K_{137l}\dot{q}_{w} + K_{137l}\dot{q}_{imu} + K_{138l}\dot{q}_{torso} - \dot{q}_{4l}s_{5l}) + s_{6l}(\dot{q}_{5l} + K_{95l}\dot{\psi} + K_{98l}\dot{q}_{1l} + K_{99l}\dot{q}_{2l} + K_{96l}\dot{q}_{w} + K_{96l}\dot{q}_{imu} + K_{97l}\dot{q}_{torso} + \dot{q}_{3l}c_{4l}) &  \end{matrix}\right] 
 \nonumber \\ 
K_{172l} &= K_{95l}c_{6l} - K_{136l}s_{6l} \nonumber \\
K_{173l} &= K_{96l}c_{6l} - K_{137l}s_{6l} \nonumber \\
K_{174l} &= K_{97l}c_{6l} - K_{138l}s_{6l} \nonumber \\
K_{175l} &= K_{98l}c_{6l} - K_{139l}s_{6l} \nonumber \\
K_{176l} &= K_{99l}c_{6l} - K_{140l}s_{6l} \nonumber \\
K_{177l} &= c_{4l}c_{6l} - K_{141l}s_{6l} \nonumber \\
K_{178l} &= s_{5l}s_{6l} \nonumber \\
K_{179l} &= K_{136l}c_{6l} + K_{95l}s_{6l} \nonumber \\
K_{180l} &= K_{137l}c_{6l} + K_{96l}s_{6l} \nonumber \\
K_{181l} &= K_{138l}c_{6l} + K_{97l}s_{6l} \nonumber \\
K_{182l} &= K_{139l}c_{6l} + K_{98l}s_{6l} \nonumber \\
K_{183l} &= K_{140l}c_{6l} + K_{99l}s_{6l} \nonumber \\
K_{184l} &= c_{4l}s_{6l} + K_{141l}c_{6l} \nonumber \\
K_{185l} &= -c_{6l}s_{5l} \nonumber \\
 \bar\omega_{9l} &= \left[\begin{matrix} - \dot{q}_{6l} - K_{130l}\dot{\psi} - K_{133l}\dot{q}_{1l} - K_{134l}\dot{q}_{2l} - K_{135l}\dot{q}_{3l} - K_{131l}\dot{q}_{w} - K_{131l}\dot{q}_{imu} - K_{132l}\dot{q}_{torso} - \dot{q}_{4l}c_{5l} & K_{172l}\dot{\psi} + K_{175l}\dot{q}_{1l} + K_{176l}\dot{q}_{2l} + K_{177l}\dot{q}_{3l} + K_{178l}\dot{q}_{4l} + K_{173l}\dot{q}_{w} + K_{173l}\dot{q}_{imu} + K_{174l}\dot{q}_{torso} + \dot{q}_{5l}c_{6l} & K_{179l}\dot{\psi} + K_{182l}\dot{q}_{1l} + K_{183l}\dot{q}_{2l} + K_{184l}\dot{q}_{3l} + K_{185l}\dot{q}_{4l} + K_{180l}\dot{q}_{w} + K_{180l}\dot{q}_{imu} + K_{181l}\dot{q}_{torso} + \dot{q}_{5l}s_{6l} &  \end{matrix}\right] 
 \nonumber \\ 
 \bar{v}_{9l} &= {}^{9l}A_{8l} \left(\bar{v}_{8l} + \bar\omega_{8l} \times \bar{P}_{9l}\right) 
 \nonumber \\ 
 \bar{v}_{9l} &= \left[\begin{matrix} -K_{142l} & K_{105l}c_{6l} - K_{143l}s_{6l} & K_{143l}c_{6l} + K_{105l}s_{6l} &  \end{matrix}\right] 
 \nonumber \\ 
K_{186l} &= K_{105l}c_{6l} - K_{143l}s_{6l} \nonumber \\
K_{187l} &= K_{143l}c_{6l} + K_{105l}s_{6l} \nonumber \\
 \bar{v}_{9l} &= \left[\begin{matrix} -K_{142l} & K_{186l} & K_{187l} &  \end{matrix}\right] 
 \nonumber \\ 
 \bar{v}_{9l} &= \left[\begin{matrix} - K_{145l}\dot{\psi} - K_{149l}\dot{q}_{1l} - K_{150l}\dot{q}_{2l} - K_{151l}\dot{q}_{3l} - K_{152l}\dot{q}_{4l} - K_{147l}\dot{q}_{w} - K_{146l}\dot{q}_{imu} - K_{148l}\dot{q}_{torso} - K_{144l}\dot{x} & c_{6l}(K_{108l}\dot{\psi} + K_{112l}\dot{q}_{1l} + K_{113l}\dot{q}_{2l} + K_{110l}\dot{q}_{w} + K_{109l}\dot{q}_{imu} + K_{111l}\dot{q}_{torso} + K_{107l}\dot{x}) - s_{6l}(K_{154l}\dot{\psi} + K_{158l}\dot{q}_{1l} + K_{159l}\dot{q}_{2l} + K_{160l}\dot{q}_{3l} + K_{161l}\dot{q}_{4l} + K_{156l}\dot{q}_{w} + K_{155l}\dot{q}_{imu} + K_{157l}\dot{q}_{torso} + K_{153l}\dot{x}) & s_{6l}(K_{108l}\dot{\psi} + K_{112l}\dot{q}_{1l} + K_{113l}\dot{q}_{2l} + K_{110l}\dot{q}_{w} + K_{109l}\dot{q}_{imu} + K_{111l}\dot{q}_{torso} + K_{107l}\dot{x}) + c_{6l}(K_{154l}\dot{\psi} + K_{158l}\dot{q}_{1l} + K_{159l}\dot{q}_{2l} + K_{160l}\dot{q}_{3l} + K_{161l}\dot{q}_{4l} + K_{156l}\dot{q}_{w} + K_{155l}\dot{q}_{imu} + K_{157l}\dot{q}_{torso} + K_{153l}\dot{x}) &  \end{matrix}\right] 
 \nonumber \\ 
K_{188l} &= K_{107l}c_{6l} - K_{153l}s_{6l} \nonumber \\
K_{189l} &= K_{108l}c_{6l} - K_{154l}s_{6l} \nonumber \\
K_{190l} &= K_{109l}c_{6l} - K_{155l}s_{6l} \nonumber \\
K_{191l} &= K_{110l}c_{6l} - K_{156l}s_{6l} \nonumber \\
K_{192l} &= K_{111l}c_{6l} - K_{157l}s_{6l} \nonumber \\
K_{193l} &= K_{112l}c_{6l} - K_{158l}s_{6l} \nonumber \\
K_{194l} &= K_{113l}c_{6l} - K_{159l}s_{6l} \nonumber \\
K_{195l} &= -K_{160l}s_{6l} \nonumber \\
K_{196l} &= -K_{161l}s_{6l} \nonumber \\
K_{197l} &= K_{153l}c_{6l} + K_{107l}s_{6l} \nonumber \\
K_{198l} &= K_{154l}c_{6l} + K_{108l}s_{6l} \nonumber \\
K_{199l} &= K_{155l}c_{6l} + K_{109l}s_{6l} \nonumber \\
K_{200l} &= K_{156l}c_{6l} + K_{110l}s_{6l} \nonumber \\
K_{201l} &= K_{157l}c_{6l} + K_{111l}s_{6l} \nonumber \\
K_{202l} &= K_{158l}c_{6l} + K_{112l}s_{6l} \nonumber \\
K_{203l} &= K_{159l}c_{6l} + K_{113l}s_{6l} \nonumber \\
K_{204l} &= K_{160l}c_{6l} \nonumber \\
K_{205l} &= K_{161l}c_{6l} \nonumber \\
 \bar{v}_{9l} &= \left[\begin{matrix} - K_{145l}\dot{\psi} - K_{149l}\dot{q}_{1l} - K_{150l}\dot{q}_{2l} - K_{151l}\dot{q}_{3l} - K_{152l}\dot{q}_{4l} - K_{147l}\dot{q}_{w} - K_{146l}\dot{q}_{imu} - K_{148l}\dot{q}_{torso} - K_{144l}\dot{x} & K_{189l}\dot{\psi} + K_{193l}\dot{q}_{1l} + K_{194l}\dot{q}_{2l} + K_{195l}\dot{q}_{3l} + K_{196l}\dot{q}_{4l} + K_{191l}\dot{q}_{w} + K_{190l}\dot{q}_{imu} + K_{192l}\dot{q}_{torso} + K_{188l}\dot{x} & K_{198l}\dot{\psi} + K_{202l}\dot{q}_{1l} + K_{203l}\dot{q}_{2l} + K_{204l}\dot{q}_{3l} + K_{205l}\dot{q}_{4l} + K_{200l}\dot{q}_{w} + K_{199l}\dot{q}_{imu} + K_{201l}\dot{q}_{torso} + K_{197l}\dot{x} &  \end{matrix}\right] 
 \nonumber \\ 
 \bar\alpha_{9l} &= {}^{9l}A_{8l} \bar\alpha_{8l} + \ddot{q}_{9l} \bar{e}_{9l} + \dot{q}_{9l} \left(\bar\omega_{9l} \times \bar{e}_{9l}\right) 
 \nonumber \\ 
 \bar\alpha_{9l} &= \left[\begin{matrix} - K_{162l} - \ddot{q}_{6l} - K_{130l}\ddot{\psi} - K_{133l}\ddot{q}_{1l} - K_{134l}\ddot{q}_{2l} - K_{135l}\ddot{q}_{3l} - K_{131l}\ddot{q}_{w} - K_{131l}\ddot{q}_{imu} - K_{132l}\ddot{q}_{torso} - \ddot{q}_{4l}c_{5l} & c_{6l}(K_{121l} + \ddot{q}_{5l} + K_{95l}\ddot{\psi} + K_{98l}\ddot{q}_{1l} + K_{99l}\ddot{q}_{2l} + K_{96l}\ddot{q}_{w} + K_{96l}\ddot{q}_{imu} + K_{97l}\ddot{q}_{torso} + \ddot{q}_{3l}c_{4l}) - K_{171l}\dot{q}_{6l} - s_{6l}(K_{163l} + K_{136l}\ddot{\psi} + K_{139l}\ddot{q}_{1l} + K_{140l}\ddot{q}_{2l} + K_{141l}\ddot{q}_{3l} + K_{137l}\ddot{q}_{w} + K_{137l}\ddot{q}_{imu} + K_{138l}\ddot{q}_{torso} - \ddot{q}_{4l}s_{5l}) & K_{170l}\dot{q}_{6l} + s_{6l}(K_{121l} + \ddot{q}_{5l} + K_{95l}\ddot{\psi} + K_{98l}\ddot{q}_{1l} + K_{99l}\ddot{q}_{2l} + K_{96l}\ddot{q}_{w} + K_{96l}\ddot{q}_{imu} + K_{97l}\ddot{q}_{torso} + \ddot{q}_{3l}c_{4l}) + c_{6l}(K_{163l} + K_{136l}\ddot{\psi} + K_{139l}\ddot{q}_{1l} + K_{140l}\ddot{q}_{2l} + K_{141l}\ddot{q}_{3l} + K_{137l}\ddot{q}_{w} + K_{137l}\ddot{q}_{imu} + K_{138l}\ddot{q}_{torso} - \ddot{q}_{4l}s_{5l}) &  \end{matrix}\right] 
 \nonumber \\ 
K_{206l} &= K_{121l}c_{6l} - K_{171l}\dot{q}_{6l} - K_{163l}s_{6l} \nonumber \\
K_{207l} &= K_{170l}\dot{q}_{6l} + K_{163l}c_{6l} + K_{121l}s_{6l} \nonumber \\
 \bar\alpha_{9l} &= \left[\begin{matrix} - K_{162l} - \ddot{q}_{6l} - K_{130l}\ddot{\psi} - K_{133l}\ddot{q}_{1l} - K_{134l}\ddot{q}_{2l} - K_{135l}\ddot{q}_{3l} - K_{131l}\ddot{q}_{w} - K_{131l}\ddot{q}_{imu} - K_{132l}\ddot{q}_{torso} - \ddot{q}_{4l}c_{5l} & K_{206l} + K_{172l}\ddot{\psi} + K_{175l}\ddot{q}_{1l} + K_{176l}\ddot{q}_{2l} + K_{177l}\ddot{q}_{3l} + K_{178l}\ddot{q}_{4l} + K_{173l}\ddot{q}_{w} + K_{173l}\ddot{q}_{imu} + K_{174l}\ddot{q}_{torso} + \ddot{q}_{5l}c_{6l} & K_{207l} + K_{179l}\ddot{\psi} + K_{182l}\ddot{q}_{1l} + K_{183l}\ddot{q}_{2l} + K_{184l}\ddot{q}_{3l} + K_{185l}\ddot{q}_{4l} + K_{180l}\ddot{q}_{w} + K_{180l}\ddot{q}_{imu} + K_{181l}\ddot{q}_{torso} + \ddot{q}_{5l}s_{6l} &  \end{matrix}\right] 
 \nonumber \\ 
 \bar{a}_{9l} &= {}^{9l}A_{8l} \left(\bar{a}_{8l} + \bar\alpha_{8l} \times \bar{P}_{9l} + \bar\omega_{8l} \times \left(\bar\omega_{8l} \times \bar{P}_{9l}\right)\right) 
 \nonumber \\ 
 \bar\alpha_{9l} &= \left[\begin{matrix} - K_{164l} - K_{145l}\ddot{\psi} - K_{149l}\ddot{q}_{1l} - K_{150l}\ddot{q}_{2l} - K_{151l}\ddot{q}_{3l} - K_{152l}\ddot{q}_{4l} - K_{147l}\ddot{q}_{w} - K_{146l}\ddot{q}_{imu} - K_{148l}\ddot{q}_{torso} - K_{144l}\ddot{x} & c_{6l}(K_{108l}\ddot{\psi} - K_{165l} + K_{112l}\ddot{q}_{1l} + K_{113l}\ddot{q}_{2l} + K_{110l}\ddot{q}_{w} + K_{109l}\ddot{q}_{imu} + K_{111l}\ddot{q}_{torso} + K_{107l}\ddot{x}) - s_{6l}(K_{166l} + K_{154l}\ddot{\psi} + K_{158l}\ddot{q}_{1l} + K_{159l}\ddot{q}_{2l} + K_{160l}\ddot{q}_{3l} + K_{161l}\ddot{q}_{4l} + K_{156l}\ddot{q}_{w} + K_{155l}\ddot{q}_{imu} + K_{157l}\ddot{q}_{torso} + K_{153l}\ddot{x}) & c_{6l}(K_{166l} + K_{154l}\ddot{\psi} + K_{158l}\ddot{q}_{1l} + K_{159l}\ddot{q}_{2l} + K_{160l}\ddot{q}_{3l} + K_{161l}\ddot{q}_{4l} + K_{156l}\ddot{q}_{w} + K_{155l}\ddot{q}_{imu} + K_{157l}\ddot{q}_{torso} + K_{153l}\ddot{x}) + s_{6l}(K_{108l}\ddot{\psi} - K_{165l} + K_{112l}\ddot{q}_{1l} + K_{113l}\ddot{q}_{2l} + K_{110l}\ddot{q}_{w} + K_{109l}\ddot{q}_{imu} + K_{111l}\ddot{q}_{torso} + K_{107l}\ddot{x}) &  \end{matrix}\right] 
 \nonumber \\ 
K_{208l} &= - K_{165l}c_{6l} - K_{166l}s_{6l} \nonumber \\
K_{209l} &= K_{166l}c_{6l} - K_{165l}s_{6l} \nonumber \\
 \bar{a}_{9l} &= \left[\begin{matrix} - K_{164l} - K_{145l}\ddot{\psi} - K_{149l}\ddot{q}_{1l} - K_{150l}\ddot{q}_{2l} - K_{151l}\ddot{q}_{3l} - K_{152l}\ddot{q}_{4l} - K_{147l}\ddot{q}_{w} - K_{146l}\ddot{q}_{imu} - K_{148l}\ddot{q}_{torso} - K_{144l}\ddot{x} & K_{208l} + K_{189l}\ddot{\psi} + K_{193l}\ddot{q}_{1l} + K_{194l}\ddot{q}_{2l} + K_{195l}\ddot{q}_{3l} + K_{196l}\ddot{q}_{4l} + K_{191l}\ddot{q}_{w} + K_{190l}\ddot{q}_{imu} + K_{192l}\ddot{q}_{torso} + K_{188l}\ddot{x} & K_{209l} + K_{198l}\ddot{\psi} + K_{202l}\ddot{q}_{1l} + K_{203l}\ddot{q}_{2l} + K_{204l}\ddot{q}_{3l} + K_{205l}\ddot{q}_{4l} + K_{200l}\ddot{q}_{w} + K_{199l}\ddot{q}_{imu} + K_{201l}\ddot{q}_{torso} + K_{197l}\ddot{x} &  \end{matrix}\right] 
 \nonumber \\ 
 \bar{g}_{9l} &= {}^{9l}A_{8l} \bar{g}_{8l} 
 \nonumber \\ 
 \bar{g}_{9l} &= \left[\begin{matrix} -K_{167l}g & K_{125l}gc_{6l} - K_{168l}gs_{6l} & K_{168l}gc_{6l} + K_{125l}gs_{6l} &  \end{matrix}\right] 
 \nonumber \\ 
K_{210l} &= K_{125l}c_{6l} - K_{168l}s_{6l} \nonumber \\
K_{211l} &= K_{168l}c_{6l} + K_{125l}s_{6l} \nonumber \\
 \bar{g}_{9l} &= \left[\begin{matrix} -K_{167l}g & K_{210l}g & K_{211l}g &  \end{matrix}\right] 
 \nonumber \\ 
 m_{9l}\bar{S}_{9l}^{\times}\bar{g}_{9l} &= \mathbf{MS}_{9l} \times \bar{g}_{9l} 
 \nonumber \\ 
 m_{9l}\bar{S}_{9l}^{\times}\bar{g}_{9l} &= \left[\begin{matrix} K_{211l}\mathbf{MY}_{9r}g - K_{210l}\mathbf{MZ}_{9r}g & - K_{211l}\mathbf{MX}_{9r}g - K_{167l}\mathbf{MZ}_{9r}g & K_{210l}\mathbf{MX}_{9r}g + K_{167l}\mathbf{MY}_{9r}g &  \end{matrix}\right] 
 \nonumber \\ 
D_{256l} &= K_{211l}\mathbf{MY}_{9r} - K_{210l}\mathbf{MZ}_{9r} \nonumber \\
D_{257l} &= - K_{211l}\mathbf{MX}_{9r} - K_{167l}\mathbf{MZ}_{9r} \nonumber \\
D_{258l} &= K_{210l}\mathbf{MX}_{9r} + K_{167l}\mathbf{MY}_{9r} \nonumber \\
 m_{9l}\bar{S}_{9l}^{\times}\bar{g}_{9l} &= \left[\begin{matrix} D_{256l}g & D_{257l}g & D_{258l}g &  \end{matrix}\right] 
 \nonumber \\ 
 m_{9l}\bar{a}_{G(9l)} &= m_{9l}\bar{a}_{9l} + \bar\alpha_{9l} \times \mathbf{MS}_{9l} + \bar\omega_{9l} \times \left(\bar\omega_{9l} \times \mathbf{MS}_{9l}\right) 
 \nonumber \\ 
 m_{9l}\bar{a}_{G(9l)} &= \left[\begin{matrix} \mathbf{MZ}_{9r}(K_{206l} + K_{172l}\ddot{\psi} + K_{175l}\ddot{q}_{1l} + K_{176l}\ddot{q}_{2l} + K_{177l}\ddot{q}_{3l} + K_{178l}\ddot{q}_{4l} + K_{173l}\ddot{q}_{w} + K_{173l}\ddot{q}_{imu} + K_{174l}\ddot{q}_{torso} + \ddot{q}_{5l}c_{6l}) - \mathbf{MY}_{9r}(K_{207l} + K_{179l}\ddot{\psi} + K_{182l}\ddot{q}_{1l} + K_{183l}\ddot{q}_{2l} + K_{184l}\ddot{q}_{3l} + K_{185l}\ddot{q}_{4l} + K_{180l}\ddot{q}_{w} + K_{180l}\ddot{q}_{imu} + K_{181l}\ddot{q}_{torso} + \ddot{q}_{5l}s_{6l}) - K_{170l}(K_{170l}\mathbf{MX}_{9r} - K_{169l}\mathbf{MY}_{9r}) - K_{171l}(K_{171l}\mathbf{MX}_{9r} - K_{169l}\mathbf{MZ}_{9r}) - m_{9r}(K_{164l} + K_{145l}\ddot{\psi} + K_{149l}\ddot{q}_{1l} + K_{150l}\ddot{q}_{2l} + K_{151l}\ddot{q}_{3l} + K_{152l}\ddot{q}_{4l} + K_{147l}\ddot{q}_{w} + K_{146l}\ddot{q}_{imu} + K_{148l}\ddot{q}_{torso} + K_{144l}\ddot{x}) & \mathbf{MX}_{9r}(K_{207l} + K_{179l}\ddot{\psi} + K_{182l}\ddot{q}_{1l} + K_{183l}\ddot{q}_{2l} + K_{184l}\ddot{q}_{3l} + K_{185l}\ddot{q}_{4l} + K_{180l}\ddot{q}_{w} + K_{180l}\ddot{q}_{imu} + K_{181l}\ddot{q}_{torso} + \ddot{q}_{5l}s_{6l}) + \mathbf{MZ}_{9r}(K_{162l} + \ddot{q}_{6l} + K_{130l}\ddot{\psi} + K_{133l}\ddot{q}_{1l} + K_{134l}\ddot{q}_{2l} + K_{135l}\ddot{q}_{3l} + K_{131l}\ddot{q}_{w} + K_{131l}\ddot{q}_{imu} + K_{132l}\ddot{q}_{torso} + \ddot{q}_{4l}c_{5l}) + K_{169l}(K_{170l}\mathbf{MX}_{9r} - K_{169l}\mathbf{MY}_{9r}) - K_{171l}(K_{171l}\mathbf{MY}_{9r} - K_{170l}\mathbf{MZ}_{9r}) + m_{9r}(K_{208l} + K_{189l}\ddot{\psi} + K_{193l}\ddot{q}_{1l} + K_{194l}\ddot{q}_{2l} + K_{195l}\ddot{q}_{3l} + K_{196l}\ddot{q}_{4l} + K_{191l}\ddot{q}_{w} + K_{190l}\ddot{q}_{imu} + K_{192l}\ddot{q}_{torso} + K_{188l}\ddot{x}) & K_{169l}(K_{171l}\mathbf{MX}_{9r} - K_{169l}\mathbf{MZ}_{9r}) - \mathbf{MY}_{9r}(K_{162l} + \ddot{q}_{6l} + K_{130l}\ddot{\psi} + K_{133l}\ddot{q}_{1l} + K_{134l}\ddot{q}_{2l} + K_{135l}\ddot{q}_{3l} + K_{131l}\ddot{q}_{w} + K_{131l}\ddot{q}_{imu} + K_{132l}\ddot{q}_{torso} + \ddot{q}_{4l}c_{5l}) - \mathbf{MX}_{9r}(K_{206l} + K_{172l}\ddot{\psi} + K_{175l}\ddot{q}_{1l} + K_{176l}\ddot{q}_{2l} + K_{177l}\ddot{q}_{3l} + K_{178l}\ddot{q}_{4l} + K_{173l}\ddot{q}_{w} + K_{173l}\ddot{q}_{imu} + K_{174l}\ddot{q}_{torso} + \ddot{q}_{5l}c_{6l}) + K_{170l}(K_{171l}\mathbf{MY}_{9r} - K_{170l}\mathbf{MZ}_{9r}) + m_{9r}(K_{209l} + K_{198l}\ddot{\psi} + K_{202l}\ddot{q}_{1l} + K_{203l}\ddot{q}_{2l} + K_{204l}\ddot{q}_{3l} + K_{205l}\ddot{q}_{4l} + K_{200l}\ddot{q}_{w} + K_{199l}\ddot{q}_{imu} + K_{201l}\ddot{q}_{torso} + K_{197l}\ddot{x}) &  \end{matrix}\right] 
 \nonumber \\ 
D_{259l} &= -K_{144l}m_{9r} \nonumber \\
D_{260l} &= K_{172l}\mathbf{MZ}_{9r} - K_{179l}\mathbf{MY}_{9r} - K_{145l}m_{9r} \nonumber \\
D_{261l} &= K_{173l}\mathbf{MZ}_{9r} - K_{180l}\mathbf{MY}_{9r} - K_{146l}m_{9r} \nonumber \\
D_{262l} &= K_{173l}\mathbf{MZ}_{9r} - K_{180l}\mathbf{MY}_{9r} - K_{147l}m_{9r} \nonumber \\
D_{263l} &= K_{174l}\mathbf{MZ}_{9r} - K_{181l}\mathbf{MY}_{9r} - K_{148l}m_{9r} \nonumber \\
D_{264l} &= K_{175l}\mathbf{MZ}_{9r} - K_{182l}\mathbf{MY}_{9r} - K_{149l}m_{9r} \nonumber \\
D_{265l} &= K_{176l}\mathbf{MZ}_{9r} - K_{183l}\mathbf{MY}_{9r} - K_{150l}m_{9r} \nonumber \\
D_{266l} &= K_{177l}\mathbf{MZ}_{9r} - K_{184l}\mathbf{MY}_{9r} - K_{151l}m_{9r} \nonumber \\
D_{267l} &= K_{178l}\mathbf{MZ}_{9r} - K_{185l}\mathbf{MY}_{9r} - K_{152l}m_{9r} \nonumber \\
D_{268l} &= \mathbf{MZ}_{9r}c_{6l} - \mathbf{MY}_{9r}s_{6l} \nonumber \\
D_{269l} &= K_{206l}\mathbf{MZ}_{9r} - K_{170l}^2\mathbf{MX}_{9r} - K_{171l}^2\mathbf{MX}_{9r}  \nonumber \\
&- K_{207l}\mathbf{MY}_{9r} - K_{164l}m_{9r} + K_{169l}K_{170l}\mathbf{MY}_{9r}  \nonumber \\
&+ K_{169l}K_{171l}\mathbf{MZ}_{9r} \nonumber \\
D_{270l} &= K_{188l}m_{9r} \nonumber \\
D_{271l} &= K_{189l}m_{9r} + K_{179l}\mathbf{MX}_{9r} + K_{130l}\mathbf{MZ}_{9r} \nonumber \\
D_{272l} &= K_{190l}m_{9r} + K_{180l}\mathbf{MX}_{9r} + K_{131l}\mathbf{MZ}_{9r} \nonumber \\
D_{273l} &= K_{191l}m_{9r} + K_{180l}\mathbf{MX}_{9r} + K_{131l}\mathbf{MZ}_{9r} \nonumber \\
D_{274l} &= K_{192l}m_{9r} + K_{181l}\mathbf{MX}_{9r} + K_{132l}\mathbf{MZ}_{9r} \nonumber \\
D_{275l} &= K_{193l}m_{9r} + K_{182l}\mathbf{MX}_{9r} + K_{133l}\mathbf{MZ}_{9r} \nonumber \\
D_{276l} &= K_{194l}m_{9r} + K_{183l}\mathbf{MX}_{9r} + K_{134l}\mathbf{MZ}_{9r} \nonumber \\
D_{277l} &= K_{195l}m_{9r} + K_{184l}\mathbf{MX}_{9r} + K_{135l}\mathbf{MZ}_{9r} \nonumber \\
D_{278l} &= K_{196l}m_{9r} + \mathbf{MZ}_{9r}c_{5l} + K_{185l}\mathbf{MX}_{9r} \nonumber \\
D_{279l} &= \mathbf{MX}_{9r}s_{6l} \nonumber \\
D_{280l} &= K_{208l}m_{9r} - K_{169l}^2\mathbf{MY}_{9r} - K_{171l}^2\mathbf{MY}_{9r}  \nonumber \\
&+ K_{207l}\mathbf{MX}_{9r} + K_{162l}\mathbf{MZ}_{9r} + K_{169l}K_{170l}\mathbf{MX}_{9r}  \nonumber \\
&+ K_{170l}K_{171l}\mathbf{MZ}_{9r} \nonumber \\
D_{281l} &= K_{197l}m_{9r} \nonumber \\
D_{282l} &= K_{198l}m_{9r} - K_{172l}\mathbf{MX}_{9r} - K_{130l}\mathbf{MY}_{9r} \nonumber \\
D_{283l} &= K_{199l}m_{9r} - K_{173l}\mathbf{MX}_{9r} - K_{131l}\mathbf{MY}_{9r} \nonumber \\
D_{284l} &= K_{200l}m_{9r} - K_{173l}\mathbf{MX}_{9r} - K_{131l}\mathbf{MY}_{9r} \nonumber \\
D_{285l} &= K_{201l}m_{9r} - K_{174l}\mathbf{MX}_{9r} - K_{132l}\mathbf{MY}_{9r} \nonumber \\
D_{286l} &= K_{202l}m_{9r} - K_{175l}\mathbf{MX}_{9r} - K_{133l}\mathbf{MY}_{9r} \nonumber \\
D_{287l} &= K_{203l}m_{9r} - K_{176l}\mathbf{MX}_{9r} - K_{134l}\mathbf{MY}_{9r} \nonumber \\
D_{288l} &= K_{204l}m_{9r} - K_{177l}\mathbf{MX}_{9r} - K_{135l}\mathbf{MY}_{9r} \nonumber \\
D_{289l} &= K_{205l}m_{9r} - \mathbf{MY}_{9r}c_{5l} - K_{178l}\mathbf{MX}_{9r} \nonumber \\
D_{290l} &= -\mathbf{MX}_{9r}c_{6l} \nonumber \\
D_{291l} &= K_{209l}m_{9r} - K_{169l}^2\mathbf{MZ}_{9r} - K_{170l}^2\mathbf{MZ}_{9r}  \nonumber \\
&- K_{206l}\mathbf{MX}_{9r} - K_{162l}\mathbf{MY}_{9r} + K_{169l}K_{171l}\mathbf{MX}_{9r}  \nonumber \\
&+ K_{170l}K_{171l}\mathbf{MY}_{9r} \nonumber \\
 m_{9l}\bar{a}_{G(9l)} &= \left[\begin{matrix} D_{269l} + D_{260l}\ddot{\psi} + D_{264l}\ddot{q}_{1l} + D_{265l}\ddot{q}_{2l} + D_{266l}\ddot{q}_{3l} + D_{267l}\ddot{q}_{4l} + D_{262l}\ddot{q}_{w} + D_{268l}\ddot{q}_{5l} + D_{261l}\ddot{q}_{imu} + D_{263l}\ddot{q}_{torso} + D_{259l}\ddot{x} & D_{280l} + D_{271l}\ddot{\psi} + D_{275l}\ddot{q}_{1l} + D_{276l}\ddot{q}_{2l} + D_{277l}\ddot{q}_{3l} + D_{278l}\ddot{q}_{4l} + D_{273l}\ddot{q}_{w} + D_{279l}\ddot{q}_{5l} + D_{272l}\ddot{q}_{imu} + D_{274l}\ddot{q}_{torso} + D_{270l}\ddot{x} + \mathbf{MZ}_{9r}\ddot{q}_{6l} & D_{291l} + D_{282l}\ddot{\psi} + D_{286l}\ddot{q}_{1l} + D_{287l}\ddot{q}_{2l} + D_{288l}\ddot{q}_{3l} + D_{289l}\ddot{q}_{4l} + D_{284l}\ddot{q}_{w} + D_{290l}\ddot{q}_{5l} + D_{283l}\ddot{q}_{imu} + D_{285l}\ddot{q}_{torso} + D_{281l}\ddot{x} - \mathbf{MY}_{9r}\ddot{q}_{6l} &  \end{matrix}\right] 
 \nonumber \\ 
 \dot{\bar{H}}_{9l} &= \mathbf{MS}_{9l} \times \bar{a}_{9l} + J_{9l}\bar{\alpha}_{9l} + \bar\omega_{9l} \times J_{9l}\bar{\omega}_{9l} 
 \nonumber \\ 
 \dot{\bar{H}}_{9l} &= \left[\begin{matrix} K_{170l}(K_{169l}\mathbf{XZ}_{9r} + K_{170l}\mathbf{YZ}_{9r} + K_{171l}\mathbf{ZZ}_{9r}) - K_{171l}(K_{169l}\mathbf{XY}_{9r} + K_{170l}\mathbf{YY}_{9r} + K_{171l}\mathbf{YZ}_{9r}) + \mathbf{XY}_{9r}(K_{206l} + K_{172l}\ddot{\psi} + K_{175l}\ddot{q}_{1l} + K_{176l}\ddot{q}_{2l} + K_{177l}\ddot{q}_{3l} + K_{178l}\ddot{q}_{4l} + K_{173l}\ddot{q}_{w} + K_{173l}\ddot{q}_{imu} + K_{174l}\ddot{q}_{torso} + \ddot{q}_{5l}c_{6l}) + \mathbf{XZ}_{9r}(K_{207l} + K_{179l}\ddot{\psi} + K_{182l}\ddot{q}_{1l} + K_{183l}\ddot{q}_{2l} + K_{184l}\ddot{q}_{3l} + K_{185l}\ddot{q}_{4l} + K_{180l}\ddot{q}_{w} + K_{180l}\ddot{q}_{imu} + K_{181l}\ddot{q}_{torso} + \ddot{q}_{5l}s_{6l}) - \mathbf{XX}_{9r}(K_{162l} + \ddot{q}_{6l} + K_{130l}\ddot{\psi} + K_{133l}\ddot{q}_{1l} + K_{134l}\ddot{q}_{2l} + K_{135l}\ddot{q}_{3l} + K_{131l}\ddot{q}_{w} + K_{131l}\ddot{q}_{imu} + K_{132l}\ddot{q}_{torso} + \ddot{q}_{4l}c_{5l}) + \mathbf{MY}_{9r}(K_{209l} + K_{198l}\ddot{\psi} + K_{202l}\ddot{q}_{1l} + K_{203l}\ddot{q}_{2l} + K_{204l}\ddot{q}_{3l} + K_{205l}\ddot{q}_{4l} + K_{200l}\ddot{q}_{w} + K_{199l}\ddot{q}_{imu} + K_{201l}\ddot{q}_{torso} + K_{197l}\ddot{x}) - \mathbf{MZ}_{9r}(K_{208l} + K_{189l}\ddot{\psi} + K_{193l}\ddot{q}_{1l} + K_{194l}\ddot{q}_{2l} + K_{195l}\ddot{q}_{3l} + K_{196l}\ddot{q}_{4l} + K_{191l}\ddot{q}_{w} + K_{190l}\ddot{q}_{imu} + K_{192l}\ddot{q}_{torso} + K_{188l}\ddot{x}) & K_{171l}(K_{169l}\mathbf{XX}_{9r} + K_{170l}\mathbf{XY}_{9r} + K_{171l}\mathbf{XZ}_{9r}) - K_{169l}(K_{169l}\mathbf{XZ}_{9r} + K_{170l}\mathbf{YZ}_{9r} + K_{171l}\mathbf{ZZ}_{9r}) + \mathbf{YY}_{9r}(K_{206l} + K_{172l}\ddot{\psi} + K_{175l}\ddot{q}_{1l} + K_{176l}\ddot{q}_{2l} + K_{177l}\ddot{q}_{3l} + K_{178l}\ddot{q}_{4l} + K_{173l}\ddot{q}_{w} + K_{173l}\ddot{q}_{imu} + K_{174l}\ddot{q}_{torso} + \ddot{q}_{5l}c_{6l}) + \mathbf{YZ}_{9r}(K_{207l} + K_{179l}\ddot{\psi} + K_{182l}\ddot{q}_{1l} + K_{183l}\ddot{q}_{2l} + K_{184l}\ddot{q}_{3l} + K_{185l}\ddot{q}_{4l} + K_{180l}\ddot{q}_{w} + K_{180l}\ddot{q}_{imu} + K_{181l}\ddot{q}_{torso} + \ddot{q}_{5l}s_{6l}) - \mathbf{XY}_{9r}(K_{162l} + \ddot{q}_{6l} + K_{130l}\ddot{\psi} + K_{133l}\ddot{q}_{1l} + K_{134l}\ddot{q}_{2l} + K_{135l}\ddot{q}_{3l} + K_{131l}\ddot{q}_{w} + K_{131l}\ddot{q}_{imu} + K_{132l}\ddot{q}_{torso} + \ddot{q}_{4l}c_{5l}) - \mathbf{MX}_{9r}(K_{209l} + K_{198l}\ddot{\psi} + K_{202l}\ddot{q}_{1l} + K_{203l}\ddot{q}_{2l} + K_{204l}\ddot{q}_{3l} + K_{205l}\ddot{q}_{4l} + K_{200l}\ddot{q}_{w} + K_{199l}\ddot{q}_{imu} + K_{201l}\ddot{q}_{torso} + K_{197l}\ddot{x}) - \mathbf{MZ}_{9r}(K_{164l} + K_{145l}\ddot{\psi} + K_{149l}\ddot{q}_{1l} + K_{150l}\ddot{q}_{2l} + K_{151l}\ddot{q}_{3l} + K_{152l}\ddot{q}_{4l} + K_{147l}\ddot{q}_{w} + K_{146l}\ddot{q}_{imu} + K_{148l}\ddot{q}_{torso} + K_{144l}\ddot{x}) & K_{169l}(K_{169l}\mathbf{XY}_{9r} + K_{170l}\mathbf{YY}_{9r} + K_{171l}\mathbf{YZ}_{9r}) - K_{170l}(K_{169l}\mathbf{XX}_{9r} + K_{170l}\mathbf{XY}_{9r} + K_{171l}\mathbf{XZ}_{9r}) + \mathbf{YZ}_{9r}(K_{206l} + K_{172l}\ddot{\psi} + K_{175l}\ddot{q}_{1l} + K_{176l}\ddot{q}_{2l} + K_{177l}\ddot{q}_{3l} + K_{178l}\ddot{q}_{4l} + K_{173l}\ddot{q}_{w} + K_{173l}\ddot{q}_{imu} + K_{174l}\ddot{q}_{torso} + \ddot{q}_{5l}c_{6l}) + \mathbf{ZZ}_{9r}(K_{207l} + K_{179l}\ddot{\psi} + K_{182l}\ddot{q}_{1l} + K_{183l}\ddot{q}_{2l} + K_{184l}\ddot{q}_{3l} + K_{185l}\ddot{q}_{4l} + K_{180l}\ddot{q}_{w} + K_{180l}\ddot{q}_{imu} + K_{181l}\ddot{q}_{torso} + \ddot{q}_{5l}s_{6l}) - \mathbf{XZ}_{9r}(K_{162l} + \ddot{q}_{6l} + K_{130l}\ddot{\psi} + K_{133l}\ddot{q}_{1l} + K_{134l}\ddot{q}_{2l} + K_{135l}\ddot{q}_{3l} + K_{131l}\ddot{q}_{w} + K_{131l}\ddot{q}_{imu} + K_{132l}\ddot{q}_{torso} + \ddot{q}_{4l}c_{5l}) + \mathbf{MX}_{9r}(K_{208l} + K_{189l}\ddot{\psi} + K_{193l}\ddot{q}_{1l} + K_{194l}\ddot{q}_{2l} + K_{195l}\ddot{q}_{3l} + K_{196l}\ddot{q}_{4l} + K_{191l}\ddot{q}_{w} + K_{190l}\ddot{q}_{imu} + K_{192l}\ddot{q}_{torso} + K_{188l}\ddot{x}) + \mathbf{MY}_{9r}(K_{164l} + K_{145l}\ddot{\psi} + K_{149l}\ddot{q}_{1l} + K_{150l}\ddot{q}_{2l} + K_{151l}\ddot{q}_{3l} + K_{152l}\ddot{q}_{4l} + K_{147l}\ddot{q}_{w} + K_{146l}\ddot{q}_{imu} + K_{148l}\ddot{q}_{torso} + K_{144l}\ddot{x}) &  \end{matrix}\right] 
 \nonumber \\ 
D_{292l} &= K_{197l}\mathbf{MY}_{9r} - K_{188l}\mathbf{MZ}_{9r} \nonumber \\
D_{293l} &= K_{172l}\mathbf{XY}_{9r} - K_{130l}\mathbf{XX}_{9r} + K_{179l}\mathbf{XZ}_{9r}  \nonumber \\
&+ K_{198l}\mathbf{MY}_{9r} - K_{189l}\mathbf{MZ}_{9r} \nonumber \\
D_{294l} &= K_{173l}\mathbf{XY}_{9r} - K_{131l}\mathbf{XX}_{9r} + K_{180l}\mathbf{XZ}_{9r}  \nonumber \\
&+ K_{199l}\mathbf{MY}_{9r} - K_{190l}\mathbf{MZ}_{9r} \nonumber \\
D_{295l} &= K_{173l}\mathbf{XY}_{9r} - K_{131l}\mathbf{XX}_{9r} + K_{180l}\mathbf{XZ}_{9r}  \nonumber \\
&+ K_{200l}\mathbf{MY}_{9r} - K_{191l}\mathbf{MZ}_{9r} \nonumber \\
D_{296l} &= K_{174l}\mathbf{XY}_{9r} - K_{132l}\mathbf{XX}_{9r} + K_{181l}\mathbf{XZ}_{9r}  \nonumber \\
&+ K_{201l}\mathbf{MY}_{9r} - K_{192l}\mathbf{MZ}_{9r} \nonumber \\
D_{297l} &= K_{175l}\mathbf{XY}_{9r} - K_{133l}\mathbf{XX}_{9r} + K_{182l}\mathbf{XZ}_{9r}  \nonumber \\
&+ K_{202l}\mathbf{MY}_{9r} - K_{193l}\mathbf{MZ}_{9r} \nonumber \\
D_{298l} &= K_{176l}\mathbf{XY}_{9r} - K_{134l}\mathbf{XX}_{9r} + K_{183l}\mathbf{XZ}_{9r}  \nonumber \\
&+ K_{203l}\mathbf{MY}_{9r} - K_{194l}\mathbf{MZ}_{9r} \nonumber \\
D_{299l} &= K_{177l}\mathbf{XY}_{9r} - K_{135l}\mathbf{XX}_{9r} + K_{184l}\mathbf{XZ}_{9r}  \nonumber \\
&+ K_{204l}\mathbf{MY}_{9r} - K_{195l}\mathbf{MZ}_{9r} \nonumber \\
D_{300l} &= K_{178l}\mathbf{XY}_{9r} + K_{185l}\mathbf{XZ}_{9r} - \mathbf{XX}_{9r}c_{5l}  \nonumber \\
&+ K_{205l}\mathbf{MY}_{9r} - K_{196l}\mathbf{MZ}_{9r} \nonumber \\
D_{301l} &= \mathbf{XY}_{9r}c_{6l} + \mathbf{XZ}_{9r}s_{6l} \nonumber \\
D_{302l} &= K_{206l}\mathbf{XY}_{9r} - K_{162l}\mathbf{XX}_{9r} + K_{207l}\mathbf{XZ}_{9r}  \nonumber \\
&+ K_{170l}^2\mathbf{YZ}_{9r} - K_{171l}^2\mathbf{YZ}_{9r} + K_{209l}\mathbf{MY}_{9r}  \nonumber \\
&- K_{208l}\mathbf{MZ}_{9r} - K_{169l}K_{171l}\mathbf{XY}_{9r} + K_{169l}K_{170l}\mathbf{XZ}_{9r}  \nonumber \\
&- K_{170l}K_{171l}\mathbf{YY}_{9r} + K_{170l}K_{171l}\mathbf{ZZ}_{9r} \nonumber \\
D_{303l} &= - K_{197l}\mathbf{MX}_{9r} - K_{144l}\mathbf{MZ}_{9r} \nonumber \\
D_{304l} &= K_{172l}\mathbf{YY}_{9r} - K_{130l}\mathbf{XY}_{9r} + K_{179l}\mathbf{YZ}_{9r}  \nonumber \\
&- K_{198l}\mathbf{MX}_{9r} - K_{145l}\mathbf{MZ}_{9r} \nonumber \\
D_{305l} &= K_{173l}\mathbf{YY}_{9r} - K_{131l}\mathbf{XY}_{9r} + K_{180l}\mathbf{YZ}_{9r}  \nonumber \\
&- K_{199l}\mathbf{MX}_{9r} - K_{146l}\mathbf{MZ}_{9r} \nonumber \\
D_{306l} &= K_{173l}\mathbf{YY}_{9r} - K_{131l}\mathbf{XY}_{9r} + K_{180l}\mathbf{YZ}_{9r}  \nonumber \\
&- K_{200l}\mathbf{MX}_{9r} - K_{147l}\mathbf{MZ}_{9r} \nonumber \\
D_{307l} &= K_{174l}\mathbf{YY}_{9r} - K_{132l}\mathbf{XY}_{9r} + K_{181l}\mathbf{YZ}_{9r}  \nonumber \\
&- K_{201l}\mathbf{MX}_{9r} - K_{148l}\mathbf{MZ}_{9r} \nonumber \\
D_{308l} &= K_{175l}\mathbf{YY}_{9r} - K_{133l}\mathbf{XY}_{9r} + K_{182l}\mathbf{YZ}_{9r}  \nonumber \\
&- K_{202l}\mathbf{MX}_{9r} - K_{149l}\mathbf{MZ}_{9r} \nonumber \\
D_{309l} &= K_{176l}\mathbf{YY}_{9r} - K_{134l}\mathbf{XY}_{9r} + K_{183l}\mathbf{YZ}_{9r}  \nonumber \\
&- K_{203l}\mathbf{MX}_{9r} - K_{150l}\mathbf{MZ}_{9r} \nonumber \\
D_{310l} &= K_{177l}\mathbf{YY}_{9r} - K_{135l}\mathbf{XY}_{9r} + K_{184l}\mathbf{YZ}_{9r}  \nonumber \\
&- K_{204l}\mathbf{MX}_{9r} - K_{151l}\mathbf{MZ}_{9r} \nonumber \\
D_{311l} &= K_{178l}\mathbf{YY}_{9r} + K_{185l}\mathbf{YZ}_{9r} - \mathbf{XY}_{9r}c_{5l}  \nonumber \\
&- K_{205l}\mathbf{MX}_{9r} - K_{152l}\mathbf{MZ}_{9r} \nonumber \\
D_{312l} &= \mathbf{YY}_{9r}c_{6l} + \mathbf{YZ}_{9r}s_{6l} \nonumber \\
D_{313l} &= K_{206l}\mathbf{YY}_{9r} - K_{162l}\mathbf{XY}_{9r} + K_{207l}\mathbf{YZ}_{9r}  \nonumber \\
&- K_{169l}^2\mathbf{XZ}_{9r} + K_{171l}^2\mathbf{XZ}_{9r} - K_{209l}\mathbf{MX}_{9r}  \nonumber \\
&- K_{164l}\mathbf{MZ}_{9r} + K_{169l}K_{171l}\mathbf{XX}_{9r} + K_{170l}K_{171l}\mathbf{XY}_{9r}  \nonumber \\
&- K_{169l}K_{170l}\mathbf{YZ}_{9r} - K_{169l}K_{171l}\mathbf{ZZ}_{9r} \nonumber \\
D_{314l} &= K_{188l}\mathbf{MX}_{9r} + K_{144l}\mathbf{MY}_{9r} \nonumber \\
D_{315l} &= K_{172l}\mathbf{YZ}_{9r} - K_{130l}\mathbf{XZ}_{9r} + K_{179l}\mathbf{ZZ}_{9r}  \nonumber \\
&+ K_{189l}\mathbf{MX}_{9r} + K_{145l}\mathbf{MY}_{9r} \nonumber \\
D_{316l} &= K_{173l}\mathbf{YZ}_{9r} - K_{131l}\mathbf{XZ}_{9r} + K_{180l}\mathbf{ZZ}_{9r}  \nonumber \\
&+ K_{190l}\mathbf{MX}_{9r} + K_{146l}\mathbf{MY}_{9r} \nonumber \\
D_{317l} &= K_{173l}\mathbf{YZ}_{9r} - K_{131l}\mathbf{XZ}_{9r} + K_{180l}\mathbf{ZZ}_{9r}  \nonumber \\
&+ K_{191l}\mathbf{MX}_{9r} + K_{147l}\mathbf{MY}_{9r} \nonumber \\
D_{318l} &= K_{174l}\mathbf{YZ}_{9r} - K_{132l}\mathbf{XZ}_{9r} + K_{181l}\mathbf{ZZ}_{9r}  \nonumber \\
&+ K_{192l}\mathbf{MX}_{9r} + K_{148l}\mathbf{MY}_{9r} \nonumber \\
D_{319l} &= K_{175l}\mathbf{YZ}_{9r} - K_{133l}\mathbf{XZ}_{9r} + K_{182l}\mathbf{ZZ}_{9r}  \nonumber \\
&+ K_{193l}\mathbf{MX}_{9r} + K_{149l}\mathbf{MY}_{9r} \nonumber \\
D_{320l} &= K_{176l}\mathbf{YZ}_{9r} - K_{134l}\mathbf{XZ}_{9r} + K_{183l}\mathbf{ZZ}_{9r}  \nonumber \\
&+ K_{194l}\mathbf{MX}_{9r} + K_{150l}\mathbf{MY}_{9r} \nonumber \\
D_{321l} &= K_{177l}\mathbf{YZ}_{9r} - K_{135l}\mathbf{XZ}_{9r} + K_{184l}\mathbf{ZZ}_{9r}  \nonumber \\
&+ K_{195l}\mathbf{MX}_{9r} + K_{151l}\mathbf{MY}_{9r} \nonumber \\
D_{322l} &= K_{178l}\mathbf{YZ}_{9r} + K_{185l}\mathbf{ZZ}_{9r} - \mathbf{XZ}_{9r}c_{5l}  \nonumber \\
&+ K_{196l}\mathbf{MX}_{9r} + K_{152l}\mathbf{MY}_{9r} \nonumber \\
D_{323l} &= \mathbf{YZ}_{9r}c_{6l} + \mathbf{ZZ}_{9r}s_{6l} \nonumber \\
D_{324l} &= K_{206l}\mathbf{YZ}_{9r} - K_{162l}\mathbf{XZ}_{9r} + K_{207l}\mathbf{ZZ}_{9r}  \nonumber \\
&+ K_{169l}^2\mathbf{XY}_{9r} - K_{170l}^2\mathbf{XY}_{9r} + K_{208l}\mathbf{MX}_{9r}  \nonumber \\
&+ K_{164l}\mathbf{MY}_{9r} - K_{169l}K_{170l}\mathbf{XX}_{9r} - K_{170l}K_{171l}\mathbf{XZ}_{9r}  \nonumber \\
&+ K_{169l}K_{170l}\mathbf{YY}_{9r} + K_{169l}K_{171l}\mathbf{YZ}_{9r} \nonumber \\
 \dot{\bar{H}}_{9l} &= \left[\begin{matrix} D_{269l} + D_{260l}\ddot{\psi} + D_{264l}\ddot{q}_{1l} + D_{265l}\ddot{q}_{2l} + D_{266l}\ddot{q}_{3l} + D_{267l}\ddot{q}_{4l} + D_{262l}\ddot{q}_{w} + D_{268l}\ddot{q}_{5l} + D_{261l}\ddot{q}_{imu} + D_{263l}\ddot{q}_{torso} + D_{259l}\ddot{x} & D_{280l} + D_{271l}\ddot{\psi} + D_{275l}\ddot{q}_{1l} + D_{276l}\ddot{q}_{2l} + D_{277l}\ddot{q}_{3l} + D_{278l}\ddot{q}_{4l} + D_{273l}\ddot{q}_{w} + D_{279l}\ddot{q}_{5l} + D_{272l}\ddot{q}_{imu} + D_{274l}\ddot{q}_{torso} + D_{270l}\ddot{x} + \mathbf{MZ}_{9r}\ddot{q}_{6l} & D_{291l} + D_{282l}\ddot{\psi} + D_{286l}\ddot{q}_{1l} + D_{287l}\ddot{q}_{2l} + D_{288l}\ddot{q}_{3l} + D_{289l}\ddot{q}_{4l} + D_{284l}\ddot{q}_{w} + D_{290l}\ddot{q}_{5l} + D_{283l}\ddot{q}_{imu} + D_{285l}\ddot{q}_{torso} + D_{281l}\ddot{x} - \mathbf{MY}_{9r}\ddot{q}_{6l} &  \end{matrix}\right] 
 \nonumber \\ 
 \bar\omega_{10l} &= {}^{10l}A_{9l} \bar\omega_{9l} + \dot{q}_{10l} \bar{e}_{10l} 
 \nonumber \\ 
 \bar\omega_{10l} &= \left[\begin{matrix} K_{171l}s_{7l} - K_{169l}c_{7l} & - K_{171l}c_{7l} - K_{169l}s_{7l} & - K_{170l} - \dot{q}_{7l} &  \end{matrix}\right] 
 \nonumber \\ 
K_{212l} &= K_{171l}s_{7l} - K_{169l}c_{7l} \nonumber \\
K_{213l} &= - K_{171l}c_{7l} - K_{169l}s_{7l} \nonumber \\
K_{214l} &= - K_{170l} - \dot{q}_{7l} \nonumber \\
 \bar\omega_{10l} &= \left[\begin{matrix} K_{212l} & K_{213l} & K_{214l} &  \end{matrix}\right] 
 \nonumber \\ 
 \bar\omega_{10l} &= \left[\begin{matrix} s_{7l}(K_{179l}\dot{\psi} + K_{182l}\dot{q}_{1l} + K_{183l}\dot{q}_{2l} + K_{184l}\dot{q}_{3l} + K_{185l}\dot{q}_{4l} + K_{180l}\dot{q}_{w} + K_{180l}\dot{q}_{imu} + K_{181l}\dot{q}_{torso} + \dot{q}_{5l}s_{6l}) + c_{7l}(\dot{q}_{6l} + K_{130l}\dot{\psi} + K_{133l}\dot{q}_{1l} + K_{134l}\dot{q}_{2l} + K_{135l}\dot{q}_{3l} + K_{131l}\dot{q}_{w} + K_{131l}\dot{q}_{imu} + K_{132l}\dot{q}_{torso} + \dot{q}_{4l}c_{5l}) & s_{7l}(\dot{q}_{6l} + K_{130l}\dot{\psi} + K_{133l}\dot{q}_{1l} + K_{134l}\dot{q}_{2l} + K_{135l}\dot{q}_{3l} + K_{131l}\dot{q}_{w} + K_{131l}\dot{q}_{imu} + K_{132l}\dot{q}_{torso} + \dot{q}_{4l}c_{5l}) - c_{7l}(K_{179l}\dot{\psi} + K_{182l}\dot{q}_{1l} + K_{183l}\dot{q}_{2l} + K_{184l}\dot{q}_{3l} + K_{185l}\dot{q}_{4l} + K_{180l}\dot{q}_{w} + K_{180l}\dot{q}_{imu} + K_{181l}\dot{q}_{torso} + \dot{q}_{5l}s_{6l}) & - \dot{q}_{7l} - K_{172l}\dot{\psi} - K_{175l}\dot{q}_{1l} - K_{176l}\dot{q}_{2l} - K_{177l}\dot{q}_{3l} - K_{178l}\dot{q}_{4l} - K_{173l}\dot{q}_{w} - K_{173l}\dot{q}_{imu} - K_{174l}\dot{q}_{torso} - \dot{q}_{5l}c_{6l} &  \end{matrix}\right] 
 \nonumber \\ 
K_{215l} &= K_{130l}c_{7l} + K_{179l}s_{7l} \nonumber \\
K_{216l} &= K_{131l}c_{7l} + K_{180l}s_{7l} \nonumber \\
K_{217l} &= K_{132l}c_{7l} + K_{181l}s_{7l} \nonumber \\
K_{218l} &= K_{133l}c_{7l} + K_{182l}s_{7l} \nonumber \\
K_{219l} &= K_{134l}c_{7l} + K_{183l}s_{7l} \nonumber \\
K_{220l} &= K_{135l}c_{7l} + K_{184l}s_{7l} \nonumber \\
K_{221l} &= c_{5l}c_{7l} + K_{185l}s_{7l} \nonumber \\
K_{222l} &= s_{6l}s_{7l} \nonumber \\
K_{223l} &= K_{130l}s_{7l} - K_{179l}c_{7l} \nonumber \\
K_{224l} &= K_{131l}s_{7l} - K_{180l}c_{7l} \nonumber \\
K_{225l} &= K_{132l}s_{7l} - K_{181l}c_{7l} \nonumber \\
K_{226l} &= K_{133l}s_{7l} - K_{182l}c_{7l} \nonumber \\
K_{227l} &= K_{134l}s_{7l} - K_{183l}c_{7l} \nonumber \\
K_{228l} &= K_{135l}s_{7l} - K_{184l}c_{7l} \nonumber \\
K_{229l} &= c_{5l}s_{7l} - K_{185l}c_{7l} \nonumber \\
K_{230l} &= -c_{7l}s_{6l} \nonumber \\
 \bar\omega_{10l} &= \left[\begin{matrix} K_{215l}\dot{\psi} + K_{218l}\dot{q}_{1l} + K_{219l}\dot{q}_{2l} + K_{220l}\dot{q}_{3l} + K_{221l}\dot{q}_{4l} + K_{216l}\dot{q}_{w} + K_{222l}\dot{q}_{5l} + K_{216l}\dot{q}_{imu} + K_{217l}\dot{q}_{torso} + \dot{q}_{6l}c_{7l} & K_{223l}\dot{\psi} + K_{226l}\dot{q}_{1l} + K_{227l}\dot{q}_{2l} + K_{228l}\dot{q}_{3l} + K_{229l}\dot{q}_{4l} + K_{224l}\dot{q}_{w} + K_{230l}\dot{q}_{5l} + K_{224l}\dot{q}_{imu} + K_{225l}\dot{q}_{torso} + \dot{q}_{6l}s_{7l} & - \dot{q}_{7l} - K_{172l}\dot{\psi} - K_{175l}\dot{q}_{1l} - K_{176l}\dot{q}_{2l} - K_{177l}\dot{q}_{3l} - K_{178l}\dot{q}_{4l} - K_{173l}\dot{q}_{w} - K_{173l}\dot{q}_{imu} - K_{174l}\dot{q}_{torso} - \dot{q}_{5l}c_{6l} &  \end{matrix}\right] 
 \nonumber \\ 
 \bar{v}_{10l} &= {}^{10l}A_{9l} \left(\bar{v}_{9l} + \bar\omega_{9l} \times \bar{P}_{10l}\right) 
 \nonumber \\ 
 \bar{v}_{10l} &= \left[\begin{matrix} c_{7l}(K_{142l} - K_{171l}L_9) + s_{7l}(K_{187l} - K_{169l}L_9) & s_{7l}(K_{142l} - K_{171l}L_9) - c_{7l}(K_{187l} - K_{169l}L_9) & -K_{186l} &  \end{matrix}\right] 
 \nonumber \\ 
K_{231l} &= c_{7l}(K_{142l} - K_{171l}L_9) + s_{7l}(K_{187l}  \nonumber \\
&- K_{169l}L_9) \nonumber \\
K_{232l} &= s_{7l}(K_{142l} - K_{171l}L_9) - c_{7l}(K_{187l}  \nonumber \\
&- K_{169l}L_9) \nonumber \\
 \bar{v}_{10l} &= \left[\begin{matrix} K_{231l} & K_{232l} & -K_{186l} &  \end{matrix}\right] 
 \nonumber \\ 
 \bar{v}_{10l} &= \left[\begin{matrix} c_{7l}(K_{145l}\dot{\psi} + K_{149l}\dot{q}_{1l} + K_{150l}\dot{q}_{2l} + K_{151l}\dot{q}_{3l} + K_{152l}\dot{q}_{4l} + K_{147l}\dot{q}_{w} + K_{146l}\dot{q}_{imu} + K_{148l}\dot{q}_{torso} + K_{144l}\dot{x} - L_9(K_{179l}\dot{\psi} + K_{182l}\dot{q}_{1l} + K_{183l}\dot{q}_{2l} + K_{184l}\dot{q}_{3l} + K_{185l}\dot{q}_{4l} + K_{180l}\dot{q}_{w} + K_{180l}\dot{q}_{imu} + K_{181l}\dot{q}_{torso} + \dot{q}_{5l}s_{6l})) + s_{7l}(K_{198l}\dot{\psi} + K_{202l}\dot{q}_{1l} + K_{203l}\dot{q}_{2l} + K_{204l}\dot{q}_{3l} + K_{205l}\dot{q}_{4l} + K_{200l}\dot{q}_{w} + K_{199l}\dot{q}_{imu} + K_{201l}\dot{q}_{torso} + K_{197l}\dot{x} + L_9(\dot{q}_{6l} + K_{130l}\dot{\psi} + K_{133l}\dot{q}_{1l} + K_{134l}\dot{q}_{2l} + K_{135l}\dot{q}_{3l} + K_{131l}\dot{q}_{w} + K_{131l}\dot{q}_{imu} + K_{132l}\dot{q}_{torso} + \dot{q}_{4l}c_{5l})) & s_{7l}(K_{145l}\dot{\psi} + K_{149l}\dot{q}_{1l} + K_{150l}\dot{q}_{2l} + K_{151l}\dot{q}_{3l} + K_{152l}\dot{q}_{4l} + K_{147l}\dot{q}_{w} + K_{146l}\dot{q}_{imu} + K_{148l}\dot{q}_{torso} + K_{144l}\dot{x} - L_9(K_{179l}\dot{\psi} + K_{182l}\dot{q}_{1l} + K_{183l}\dot{q}_{2l} + K_{184l}\dot{q}_{3l} + K_{185l}\dot{q}_{4l} + K_{180l}\dot{q}_{w} + K_{180l}\dot{q}_{imu} + K_{181l}\dot{q}_{torso} + \dot{q}_{5l}s_{6l})) - c_{7l}(K_{198l}\dot{\psi} + K_{202l}\dot{q}_{1l} + K_{203l}\dot{q}_{2l} + K_{204l}\dot{q}_{3l} + K_{205l}\dot{q}_{4l} + K_{200l}\dot{q}_{w} + K_{199l}\dot{q}_{imu} + K_{201l}\dot{q}_{torso} + K_{197l}\dot{x} + L_9(\dot{q}_{6l} + K_{130l}\dot{\psi} + K_{133l}\dot{q}_{1l} + K_{134l}\dot{q}_{2l} + K_{135l}\dot{q}_{3l} + K_{131l}\dot{q}_{w} + K_{131l}\dot{q}_{imu} + K_{132l}\dot{q}_{torso} + \dot{q}_{4l}c_{5l})) & - K_{189l}\dot{\psi} - K_{193l}\dot{q}_{1l} - K_{194l}\dot{q}_{2l} - K_{195l}\dot{q}_{3l} - K_{196l}\dot{q}_{4l} - K_{191l}\dot{q}_{w} - K_{190l}\dot{q}_{imu} - K_{192l}\dot{q}_{torso} - K_{188l}\dot{x} &  \end{matrix}\right] 
 \nonumber \\ 
K_{233l} &= K_{144l}c_{7l} + K_{197l}s_{7l} \nonumber \\
K_{234l} &= c_{7l}(K_{145l} - K_{179l}L_9) + s_{7l}(K_{198l}  \nonumber \\
&+ K_{130l}L_9) \nonumber \\
K_{235l} &= c_{7l}(K_{146l} - K_{180l}L_9) + s_{7l}(K_{199l}  \nonumber \\
&+ K_{131l}L_9) \nonumber \\
K_{236l} &= c_{7l}(K_{147l} - K_{180l}L_9) + s_{7l}(K_{200l}  \nonumber \\
&+ K_{131l}L_9) \nonumber \\
K_{237l} &= c_{7l}(K_{148l} - K_{181l}L_9) + s_{7l}(K_{201l}  \nonumber \\
&+ K_{132l}L_9) \nonumber \\
K_{238l} &= c_{7l}(K_{149l} - K_{182l}L_9) + s_{7l}(K_{202l}  \nonumber \\
&+ K_{133l}L_9) \nonumber \\
K_{239l} &= c_{7l}(K_{150l} - K_{183l}L_9) + s_{7l}(K_{203l}  \nonumber \\
&+ K_{134l}L_9) \nonumber \\
K_{240l} &= c_{7l}(K_{151l} - K_{184l}L_9) + s_{7l}(K_{204l}  \nonumber \\
&+ K_{135l}L_9) \nonumber \\
K_{241l} &= s_{7l}(K_{205l} + L_9c_{5l}) + c_{7l}(K_{152l}  \nonumber \\
&- K_{185l}L_9) \nonumber \\
K_{242l} &= -L_9c_{7l}s_{6l} \nonumber \\
K_{243l} &= L_9s_{7l} \nonumber \\
K_{244l} &= K_{144l}s_{7l} - K_{197l}c_{7l} \nonumber \\
K_{245l} &= s_{7l}(K_{145l} - K_{179l}L_9) - c_{7l}(K_{198l}  \nonumber \\
&+ K_{130l}L_9) \nonumber \\
K_{246l} &= s_{7l}(K_{146l} - K_{180l}L_9) - c_{7l}(K_{199l}  \nonumber \\
&+ K_{131l}L_9) \nonumber \\
K_{247l} &= s_{7l}(K_{147l} - K_{180l}L_9) - c_{7l}(K_{200l}  \nonumber \\
&+ K_{131l}L_9) \nonumber \\
K_{248l} &= s_{7l}(K_{148l} - K_{181l}L_9) - c_{7l}(K_{201l}  \nonumber \\
&+ K_{132l}L_9) \nonumber \\
K_{249l} &= s_{7l}(K_{149l} - K_{182l}L_9) - c_{7l}(K_{202l}  \nonumber \\
&+ K_{133l}L_9) \nonumber \\
K_{250l} &= s_{7l}(K_{150l} - K_{183l}L_9) - c_{7l}(K_{203l}  \nonumber \\
&+ K_{134l}L_9) \nonumber \\
K_{251l} &= s_{7l}(K_{151l} - K_{184l}L_9) - c_{7l}(K_{204l}  \nonumber \\
&+ K_{135l}L_9) \nonumber \\
K_{252l} &= s_{7l}(K_{152l} - K_{185l}L_9) - c_{7l}(K_{205l}  \nonumber \\
&+ L_9c_{5l}) \nonumber \\
K_{253l} &= -L_9s_{6l}s_{7l} \nonumber \\
K_{254l} &= -L_9c_{7l} \nonumber \\
 \bar{v}_{10l} &= \left[\begin{matrix} K_{234l}\dot{\psi} + K_{238l}\dot{q}_{1l} + K_{239l}\dot{q}_{2l} + K_{240l}\dot{q}_{3l} + K_{241l}\dot{q}_{4l} + K_{236l}\dot{q}_{w} + K_{242l}\dot{q}_{5l} + K_{243l}\dot{q}_{6l} + K_{235l}\dot{q}_{imu} + K_{237l}\dot{q}_{torso} + K_{233l}\dot{x} & K_{245l}\dot{\psi} + K_{249l}\dot{q}_{1l} + K_{250l}\dot{q}_{2l} + K_{251l}\dot{q}_{3l} + K_{252l}\dot{q}_{4l} + K_{247l}\dot{q}_{w} + K_{253l}\dot{q}_{5l} + K_{254l}\dot{q}_{6l} + K_{246l}\dot{q}_{imu} + K_{248l}\dot{q}_{torso} + K_{244l}\dot{x} & - K_{189l}\dot{\psi} - K_{193l}\dot{q}_{1l} - K_{194l}\dot{q}_{2l} - K_{195l}\dot{q}_{3l} - K_{196l}\dot{q}_{4l} - K_{191l}\dot{q}_{w} - K_{190l}\dot{q}_{imu} - K_{192l}\dot{q}_{torso} - K_{188l}\dot{x} &  \end{matrix}\right] 
 \nonumber \\ 
 \bar\alpha_{10l} &= {}^{10l}A_{9l} \bar\alpha_{9l} + \ddot{q}_{10l} \bar{e}_{10l} + \dot{q}_{10l} \left(\bar\omega_{10l} \times \bar{e}_{10l}\right) 
 \nonumber \\ 
 \bar\alpha_{10l} &= \left[\begin{matrix} s_{7l}(K_{207l} + K_{179l}\ddot{\psi} + K_{182l}\ddot{q}_{1l} + K_{183l}\ddot{q}_{2l} + K_{184l}\ddot{q}_{3l} + K_{185l}\ddot{q}_{4l} + K_{180l}\ddot{q}_{w} + K_{180l}\ddot{q}_{imu} + K_{181l}\ddot{q}_{torso} + \ddot{q}_{5l}s_{6l}) - K_{213l}\dot{q}_{7l} + c_{7l}(K_{162l} + \ddot{q}_{6l} + K_{130l}\ddot{\psi} + K_{133l}\ddot{q}_{1l} + K_{134l}\ddot{q}_{2l} + K_{135l}\ddot{q}_{3l} + K_{131l}\ddot{q}_{w} + K_{131l}\ddot{q}_{imu} + K_{132l}\ddot{q}_{torso} + \ddot{q}_{4l}c_{5l}) & K_{212l}\dot{q}_{7l} + s_{7l}(K_{162l} + \ddot{q}_{6l} + K_{130l}\ddot{\psi} + K_{133l}\ddot{q}_{1l} + K_{134l}\ddot{q}_{2l} + K_{135l}\ddot{q}_{3l} + K_{131l}\ddot{q}_{w} + K_{131l}\ddot{q}_{imu} + K_{132l}\ddot{q}_{torso} + \ddot{q}_{4l}c_{5l}) - c_{7l}(K_{207l} + K_{179l}\ddot{\psi} + K_{182l}\ddot{q}_{1l} + K_{183l}\ddot{q}_{2l} + K_{184l}\ddot{q}_{3l} + K_{185l}\ddot{q}_{4l} + K_{180l}\ddot{q}_{w} + K_{180l}\ddot{q}_{imu} + K_{181l}\ddot{q}_{torso} + \ddot{q}_{5l}s_{6l}) & - K_{206l} - \ddot{q}_{7l} - K_{172l}\ddot{\psi} - K_{175l}\ddot{q}_{1l} - K_{176l}\ddot{q}_{2l} - K_{177l}\ddot{q}_{3l} - K_{178l}\ddot{q}_{4l} - K_{173l}\ddot{q}_{w} - K_{173l}\ddot{q}_{imu} - K_{174l}\ddot{q}_{torso} - \ddot{q}_{5l}c_{6l} &  \end{matrix}\right] 
 \nonumber \\ 
K_{255l} &= K_{162l}c_{7l} - K_{213l}\dot{q}_{7l} + K_{207l}s_{7l} \nonumber \\
K_{256l} &= K_{212l}\dot{q}_{7l} - K_{207l}c_{7l} + K_{162l}s_{7l} \nonumber \\
 \bar\alpha_{10l} &= \left[\begin{matrix} K_{255l} + K_{215l}\ddot{\psi} + K_{218l}\ddot{q}_{1l} + K_{219l}\ddot{q}_{2l} + K_{220l}\ddot{q}_{3l} + K_{221l}\ddot{q}_{4l} + K_{216l}\ddot{q}_{w} + K_{222l}\ddot{q}_{5l} + K_{216l}\ddot{q}_{imu} + K_{217l}\ddot{q}_{torso} + \ddot{q}_{6l}c_{7l} & K_{256l} + K_{223l}\ddot{\psi} + K_{226l}\ddot{q}_{1l} + K_{227l}\ddot{q}_{2l} + K_{228l}\ddot{q}_{3l} + K_{229l}\ddot{q}_{4l} + K_{224l}\ddot{q}_{w} + K_{230l}\ddot{q}_{5l} + K_{224l}\ddot{q}_{imu} + K_{225l}\ddot{q}_{torso} + \ddot{q}_{6l}s_{7l} & - K_{206l} - \ddot{q}_{7l} - K_{172l}\ddot{\psi} - K_{175l}\ddot{q}_{1l} - K_{176l}\ddot{q}_{2l} - K_{177l}\ddot{q}_{3l} - K_{178l}\ddot{q}_{4l} - K_{173l}\ddot{q}_{w} - K_{173l}\ddot{q}_{imu} - K_{174l}\ddot{q}_{torso} - \ddot{q}_{5l}c_{6l} &  \end{matrix}\right] 
 \nonumber \\ 
 \bar{a}_{10l} &= {}^{10l}A_{9l} \left(\bar{a}_{9l} + \bar\alpha_{9l} \times \bar{P}_{10l} + \bar\omega_{9l} \times \left(\bar\omega_{9l} \times \bar{P}_{10l}\right)\right) 
 \nonumber \\ 
 \bar\alpha_{10l} &= \left[\begin{matrix} c_{7l}(K_{164l} + K_{145l}\ddot{\psi} + K_{149l}\ddot{q}_{1l} + K_{150l}\ddot{q}_{2l} + K_{151l}\ddot{q}_{3l} + K_{152l}\ddot{q}_{4l} + K_{147l}\ddot{q}_{w} + K_{146l}\ddot{q}_{imu} + K_{148l}\ddot{q}_{torso} + K_{144l}\ddot{x} - L_9(K_{207l} + K_{179l}\ddot{\psi} + K_{182l}\ddot{q}_{1l} + K_{183l}\ddot{q}_{2l} + K_{184l}\ddot{q}_{3l} + K_{185l}\ddot{q}_{4l} + K_{180l}\ddot{q}_{w} + K_{180l}\ddot{q}_{imu} + K_{181l}\ddot{q}_{torso} + \ddot{q}_{5l}s_{6l}) + K_{169l}K_{170l}L_9) + s_{7l}(K_{209l} + K_{198l}\ddot{\psi} + K_{202l}\ddot{q}_{1l} + K_{203l}\ddot{q}_{2l} + K_{204l}\ddot{q}_{3l} + K_{205l}\ddot{q}_{4l} + K_{200l}\ddot{q}_{w} + K_{199l}\ddot{q}_{imu} + K_{201l}\ddot{q}_{torso} + K_{197l}\ddot{x} + L_9(K_{162l} + \ddot{q}_{6l} + K_{130l}\ddot{\psi} + K_{133l}\ddot{q}_{1l} + K_{134l}\ddot{q}_{2l} + K_{135l}\ddot{q}_{3l} + K_{131l}\ddot{q}_{w} + K_{131l}\ddot{q}_{imu} + K_{132l}\ddot{q}_{torso} + \ddot{q}_{4l}c_{5l}) - K_{170l}K_{171l}L_9) & s_{7l}(K_{164l} + K_{145l}\ddot{\psi} + K_{149l}\ddot{q}_{1l} + K_{150l}\ddot{q}_{2l} + K_{151l}\ddot{q}_{3l} + K_{152l}\ddot{q}_{4l} + K_{147l}\ddot{q}_{w} + K_{146l}\ddot{q}_{imu} + K_{148l}\ddot{q}_{torso} + K_{144l}\ddot{x} - L_9(K_{207l} + K_{179l}\ddot{\psi} + K_{182l}\ddot{q}_{1l} + K_{183l}\ddot{q}_{2l} + K_{184l}\ddot{q}_{3l} + K_{185l}\ddot{q}_{4l} + K_{180l}\ddot{q}_{w} + K_{180l}\ddot{q}_{imu} + K_{181l}\ddot{q}_{torso} + \ddot{q}_{5l}s_{6l}) + K_{169l}K_{170l}L_9) - c_{7l}(K_{209l} + K_{198l}\ddot{\psi} + K_{202l}\ddot{q}_{1l} + K_{203l}\ddot{q}_{2l} + K_{204l}\ddot{q}_{3l} + K_{205l}\ddot{q}_{4l} + K_{200l}\ddot{q}_{w} + K_{199l}\ddot{q}_{imu} + K_{201l}\ddot{q}_{torso} + K_{197l}\ddot{x} + L_9(K_{162l} + \ddot{q}_{6l} + K_{130l}\ddot{\psi} + K_{133l}\ddot{q}_{1l} + K_{134l}\ddot{q}_{2l} + K_{135l}\ddot{q}_{3l} + K_{131l}\ddot{q}_{w} + K_{131l}\ddot{q}_{imu} + K_{132l}\ddot{q}_{torso} + \ddot{q}_{4l}c_{5l}) - K_{170l}K_{171l}L_9) & - K_{208l} - K_{189l}\ddot{\psi} - K_{193l}\ddot{q}_{1l} - K_{194l}\ddot{q}_{2l} - K_{195l}\ddot{q}_{3l} - K_{196l}\ddot{q}_{4l} - K_{191l}\ddot{q}_{w} - K_{190l}\ddot{q}_{imu} - K_{192l}\ddot{q}_{torso} - K_{188l}\ddot{x} - K_{169l}^2L_9 - K_{171l}^2L_9 &  \end{matrix}\right] 
 \nonumber \\ 
K_{257l} &= K_{164l}c_{7l} + K_{209l}s_{7l} - K_{207l}L_9c_{7l}  \nonumber \\
&+ K_{162l}L_9s_{7l} + K_{169l}K_{170l}L_9c_{7l}  \nonumber \\
&- K_{170l}K_{171l}L_9s_{7l} \nonumber \\
K_{258l} &= K_{164l}s_{7l} - K_{209l}c_{7l} - K_{162l}L_9c_{7l}  \nonumber \\
&- K_{207l}L_9s_{7l} + K_{170l}K_{171l}L_9c_{7l}  \nonumber \\
&+ K_{169l}K_{170l}L_9s_{7l} \nonumber \\
K_{259l} &= - K_{208l} - K_{169l}^2L_9 - K_{171l}^2L_9 \nonumber \\
 \bar{a}_{10l} &= \left[\begin{matrix} K_{257l} + K_{234l}\ddot{\psi} + K_{238l}\ddot{q}_{1l} + K_{239l}\ddot{q}_{2l} + K_{240l}\ddot{q}_{3l} + K_{241l}\ddot{q}_{4l} + K_{236l}\ddot{q}_{w} + K_{242l}\ddot{q}_{5l} + K_{243l}\ddot{q}_{6l} + K_{235l}\ddot{q}_{imu} + K_{237l}\ddot{q}_{torso} + K_{233l}\ddot{x} & K_{258l} + K_{245l}\ddot{\psi} + K_{249l}\ddot{q}_{1l} + K_{250l}\ddot{q}_{2l} + K_{251l}\ddot{q}_{3l} + K_{252l}\ddot{q}_{4l} + K_{247l}\ddot{q}_{w} + K_{253l}\ddot{q}_{5l} + K_{254l}\ddot{q}_{6l} + K_{246l}\ddot{q}_{imu} + K_{248l}\ddot{q}_{torso} + K_{244l}\ddot{x} & K_{259l} - K_{189l}\ddot{\psi} - K_{193l}\ddot{q}_{1l} - K_{194l}\ddot{q}_{2l} - K_{195l}\ddot{q}_{3l} - K_{196l}\ddot{q}_{4l} - K_{191l}\ddot{q}_{w} - K_{190l}\ddot{q}_{imu} - K_{192l}\ddot{q}_{torso} - K_{188l}\ddot{x} &  \end{matrix}\right] 
 \nonumber \\ 
 \bar{g}_{10l} &= {}^{10l}A_{9l} \bar{g}_{9l} 
 \nonumber \\ 
 \bar{g}_{10l} &= \left[\begin{matrix} K_{167l}gc_{7l} + K_{211l}gs_{7l} & K_{167l}gs_{7l} - K_{211l}gc_{7l} & -K_{210l}g &  \end{matrix}\right] 
 \nonumber \\ 
K_{260l} &= K_{167l}c_{7l} + K_{211l}s_{7l} \nonumber \\
K_{261l} &= K_{167l}s_{7l} - K_{211l}c_{7l} \nonumber \\
 \bar{g}_{10l} &= \left[\begin{matrix} K_{260l}g & K_{261l}g & -K_{210l}g &  \end{matrix}\right] 
 \nonumber \\ 
 m_{10l}\bar{S}_{10l}^{\times}\bar{g}_{10l} &= \mathbf{MS}_{10l} \times \bar{g}_{10l} 
 \nonumber \\ 
 m_{10l}\bar{S}_{10l}^{\times}\bar{g}_{10l} &= \left[\begin{matrix} - K_{210l}\mathbf{MY}_{10r}g - K_{261l}\mathbf{MZ}_{10r}g & K_{210l}\mathbf{MX}_{10r}g + K_{260l}\mathbf{MZ}_{10r}g & K_{261l}\mathbf{MX}_{10r}g - K_{260l}\mathbf{MY}_{10r}g &  \end{matrix}\right] 
 \nonumber \\ 
D_{325l} &= - K_{210l}\mathbf{MY}_{10r} - K_{261l}\mathbf{MZ}_{10r} \nonumber \\
D_{326l} &= K_{210l}\mathbf{MX}_{10r} + K_{260l}\mathbf{MZ}_{10r} \nonumber \\
D_{327l} &= K_{261l}\mathbf{MX}_{10r} - K_{260l}\mathbf{MY}_{10r} \nonumber \\
 m_{10l}\bar{S}_{10l}^{\times}\bar{g}_{10l} &= \left[\begin{matrix} D_{325l}g & D_{326l}g & D_{327l}g &  \end{matrix}\right] 
 \nonumber \\ 
 m_{10l}\bar{a}_{G(10l)} &= m_{10l}\bar{a}_{10l} + \bar\alpha_{10l} \times \mathbf{MS}_{10l} + \bar\omega_{10l} \times \left(\bar\omega_{10l} \times \mathbf{MS}_{10l}\right) 
 \nonumber \\ 
 m_{10l}\bar{a}_{G(10l)} &= \left[\begin{matrix} m_{10r}(K_{257l} + K_{234l}\ddot{\psi} + K_{238l}\ddot{q}_{1l} + K_{239l}\ddot{q}_{2l} + K_{240l}\ddot{q}_{3l} + K_{241l}\ddot{q}_{4l} + K_{236l}\ddot{q}_{w} + K_{242l}\ddot{q}_{5l} + K_{243l}\ddot{q}_{6l} + K_{235l}\ddot{q}_{imu} + K_{237l}\ddot{q}_{torso} + K_{233l}\ddot{x}) + \mathbf{MZ}_{10r}(K_{256l} + K_{223l}\ddot{\psi} + K_{226l}\ddot{q}_{1l} + K_{227l}\ddot{q}_{2l} + K_{228l}\ddot{q}_{3l} + K_{229l}\ddot{q}_{4l} + K_{224l}\ddot{q}_{w} + K_{230l}\ddot{q}_{5l} + K_{224l}\ddot{q}_{imu} + K_{225l}\ddot{q}_{torso} + \ddot{q}_{6l}s_{7l}) + \mathbf{MY}_{10r}(K_{206l} + \ddot{q}_{7l} + K_{172l}\ddot{\psi} + K_{175l}\ddot{q}_{1l} + K_{176l}\ddot{q}_{2l} + K_{177l}\ddot{q}_{3l} + K_{178l}\ddot{q}_{4l} + K_{173l}\ddot{q}_{w} + K_{173l}\ddot{q}_{imu} + K_{174l}\ddot{q}_{torso} + \ddot{q}_{5l}c_{6l}) - K_{213l}(K_{213l}\mathbf{MX}_{10r} - K_{212l}\mathbf{MY}_{10r}) - K_{214l}(K_{214l}\mathbf{MX}_{10r} - K_{212l}\mathbf{MZ}_{10r}) & m_{10r}(K_{258l} + K_{245l}\ddot{\psi} + K_{249l}\ddot{q}_{1l} + K_{250l}\ddot{q}_{2l} + K_{251l}\ddot{q}_{3l} + K_{252l}\ddot{q}_{4l} + K_{247l}\ddot{q}_{w} + K_{253l}\ddot{q}_{5l} + K_{254l}\ddot{q}_{6l} + K_{246l}\ddot{q}_{imu} + K_{248l}\ddot{q}_{torso} + K_{244l}\ddot{x}) - \mathbf{MZ}_{10r}(K_{255l} + K_{215l}\ddot{\psi} + K_{218l}\ddot{q}_{1l} + K_{219l}\ddot{q}_{2l} + K_{220l}\ddot{q}_{3l} + K_{221l}\ddot{q}_{4l} + K_{216l}\ddot{q}_{w} + K_{222l}\ddot{q}_{5l} + K_{216l}\ddot{q}_{imu} + K_{217l}\ddot{q}_{torso} + \ddot{q}_{6l}c_{7l}) - \mathbf{MX}_{10r}(K_{206l} + \ddot{q}_{7l} + K_{172l}\ddot{\psi} + K_{175l}\ddot{q}_{1l} + K_{176l}\ddot{q}_{2l} + K_{177l}\ddot{q}_{3l} + K_{178l}\ddot{q}_{4l} + K_{173l}\ddot{q}_{w} + K_{173l}\ddot{q}_{imu} + K_{174l}\ddot{q}_{torso} + \ddot{q}_{5l}c_{6l}) + K_{212l}(K_{213l}\mathbf{MX}_{10r} - K_{212l}\mathbf{MY}_{10r}) - K_{214l}(K_{214l}\mathbf{MY}_{10r} - K_{213l}\mathbf{MZ}_{10r}) & \mathbf{MY}_{10r}(K_{255l} + K_{215l}\ddot{\psi} + K_{218l}\ddot{q}_{1l} + K_{219l}\ddot{q}_{2l} + K_{220l}\ddot{q}_{3l} + K_{221l}\ddot{q}_{4l} + K_{216l}\ddot{q}_{w} + K_{222l}\ddot{q}_{5l} + K_{216l}\ddot{q}_{imu} + K_{217l}\ddot{q}_{torso} + \ddot{q}_{6l}c_{7l}) - m_{10r}(K_{189l}\ddot{\psi} - K_{259l} + K_{193l}\ddot{q}_{1l} + K_{194l}\ddot{q}_{2l} + K_{195l}\ddot{q}_{3l} + K_{196l}\ddot{q}_{4l} + K_{191l}\ddot{q}_{w} + K_{190l}\ddot{q}_{imu} + K_{192l}\ddot{q}_{torso} + K_{188l}\ddot{x}) - \mathbf{MX}_{10r}(K_{256l} + K_{223l}\ddot{\psi} + K_{226l}\ddot{q}_{1l} + K_{227l}\ddot{q}_{2l} + K_{228l}\ddot{q}_{3l} + K_{229l}\ddot{q}_{4l} + K_{224l}\ddot{q}_{w} + K_{230l}\ddot{q}_{5l} + K_{224l}\ddot{q}_{imu} + K_{225l}\ddot{q}_{torso} + \ddot{q}_{6l}s_{7l}) + K_{212l}(K_{214l}\mathbf{MX}_{10r} - K_{212l}\mathbf{MZ}_{10r}) + K_{213l}(K_{214l}\mathbf{MY}_{10r} - K_{213l}\mathbf{MZ}_{10r}) &  \end{matrix}\right] 
 \nonumber \\ 
D_{328l} &= K_{233l}m_{10r} \nonumber \\
D_{329l} &= K_{234l}m_{10r} + K_{172l}\mathbf{MY}_{10r} + K_{223l}\mathbf{MZ}_{10r} \nonumber \\
D_{330l} &= K_{235l}m_{10r} + K_{173l}\mathbf{MY}_{10r} + K_{224l}\mathbf{MZ}_{10r} \nonumber \\
D_{331l} &= K_{236l}m_{10r} + K_{173l}\mathbf{MY}_{10r} + K_{224l}\mathbf{MZ}_{10r} \nonumber \\
D_{332l} &= K_{237l}m_{10r} + K_{174l}\mathbf{MY}_{10r} + K_{225l}\mathbf{MZ}_{10r} \nonumber \\
D_{333l} &= K_{238l}m_{10r} + K_{175l}\mathbf{MY}_{10r} + K_{226l}\mathbf{MZ}_{10r} \nonumber \\
D_{334l} &= K_{239l}m_{10r} + K_{176l}\mathbf{MY}_{10r} + K_{227l}\mathbf{MZ}_{10r} \nonumber \\
D_{335l} &= K_{240l}m_{10r} + K_{177l}\mathbf{MY}_{10r} + K_{228l}\mathbf{MZ}_{10r} \nonumber \\
D_{336l} &= K_{241l}m_{10r} + K_{178l}\mathbf{MY}_{10r} + K_{229l}\mathbf{MZ}_{10r} \nonumber \\
D_{337l} &= K_{242l}m_{10r} + \mathbf{MY}_{10r}c_{6l} + K_{230l}\mathbf{MZ}_{10r} \nonumber \\
D_{338l} &= K_{243l}m_{10r} + \mathbf{MZ}_{10r}s_{7l} \nonumber \\
D_{339l} &= K_{257l}m_{10r} - K_{213l}^2\mathbf{MX}_{10r} - K_{214l}^2\mathbf{MX}_{10r}  \nonumber \\
&+ K_{206l}\mathbf{MY}_{10r} + K_{256l}\mathbf{MZ}_{10r} + K_{212l}K_{213l}\mathbf{MY}_{10r}  \nonumber \\
&+ K_{212l}K_{214l}\mathbf{MZ}_{10r} \nonumber \\
D_{340l} &= K_{244l}m_{10r} \nonumber \\
D_{341l} &= K_{245l}m_{10r} - K_{172l}\mathbf{MX}_{10r} - K_{215l}\mathbf{MZ}_{10r} \nonumber \\
D_{342l} &= K_{246l}m_{10r} - K_{173l}\mathbf{MX}_{10r} - K_{216l}\mathbf{MZ}_{10r} \nonumber \\
D_{343l} &= K_{247l}m_{10r} - K_{173l}\mathbf{MX}_{10r} - K_{216l}\mathbf{MZ}_{10r} \nonumber \\
D_{344l} &= K_{248l}m_{10r} - K_{174l}\mathbf{MX}_{10r} - K_{217l}\mathbf{MZ}_{10r} \nonumber \\
D_{345l} &= K_{249l}m_{10r} - K_{175l}\mathbf{MX}_{10r} - K_{218l}\mathbf{MZ}_{10r} \nonumber \\
D_{346l} &= K_{250l}m_{10r} - K_{176l}\mathbf{MX}_{10r} - K_{219l}\mathbf{MZ}_{10r} \nonumber \\
D_{347l} &= K_{251l}m_{10r} - K_{177l}\mathbf{MX}_{10r} - K_{220l}\mathbf{MZ}_{10r} \nonumber \\
D_{348l} &= K_{252l}m_{10r} - K_{178l}\mathbf{MX}_{10r} - K_{221l}\mathbf{MZ}_{10r} \nonumber \\
D_{349l} &= K_{253l}m_{10r} - \mathbf{MX}_{10r}c_{6l} - K_{222l}\mathbf{MZ}_{10r} \nonumber \\
D_{350l} &= K_{254l}m_{10r} - \mathbf{MZ}_{10r}c_{7l} \nonumber \\
D_{351l} &= K_{258l}m_{10r} - K_{212l}^2\mathbf{MY}_{10r} - K_{214l}^2\mathbf{MY}_{10r}  \nonumber \\
&- K_{206l}\mathbf{MX}_{10r} - K_{255l}\mathbf{MZ}_{10r} + K_{212l}K_{213l}\mathbf{MX}_{10r}  \nonumber \\
&+ K_{213l}K_{214l}\mathbf{MZ}_{10r} \nonumber \\
D_{352l} &= -K_{188l}m_{10r} \nonumber \\
D_{353l} &= K_{215l}\mathbf{MY}_{10r} - K_{223l}\mathbf{MX}_{10r} - K_{189l}m_{10r} \nonumber \\
D_{354l} &= K_{216l}\mathbf{MY}_{10r} - K_{224l}\mathbf{MX}_{10r} - K_{190l}m_{10r} \nonumber \\
D_{355l} &= K_{216l}\mathbf{MY}_{10r} - K_{224l}\mathbf{MX}_{10r} - K_{191l}m_{10r} \nonumber \\
D_{356l} &= K_{217l}\mathbf{MY}_{10r} - K_{225l}\mathbf{MX}_{10r} - K_{192l}m_{10r} \nonumber \\
D_{357l} &= K_{218l}\mathbf{MY}_{10r} - K_{226l}\mathbf{MX}_{10r} - K_{193l}m_{10r} \nonumber \\
D_{358l} &= K_{219l}\mathbf{MY}_{10r} - K_{227l}\mathbf{MX}_{10r} - K_{194l}m_{10r} \nonumber \\
D_{359l} &= K_{220l}\mathbf{MY}_{10r} - K_{228l}\mathbf{MX}_{10r} - K_{195l}m_{10r} \nonumber \\
D_{360l} &= K_{221l}\mathbf{MY}_{10r} - K_{229l}\mathbf{MX}_{10r} - K_{196l}m_{10r} \nonumber \\
D_{361l} &= K_{222l}\mathbf{MY}_{10r} - K_{230l}\mathbf{MX}_{10r} \nonumber \\
D_{362l} &= \mathbf{MY}_{10r}c_{7l} - \mathbf{MX}_{10r}s_{7l} \nonumber \\
D_{363l} &= K_{259l}m_{10r} - K_{212l}^2\mathbf{MZ}_{10r} - K_{213l}^2\mathbf{MZ}_{10r}  \nonumber \\
&- K_{256l}\mathbf{MX}_{10r} + K_{255l}\mathbf{MY}_{10r} + K_{212l}K_{214l}\mathbf{MX}_{10r}  \nonumber \\
&+ K_{213l}K_{214l}\mathbf{MY}_{10r} \nonumber \\
 m_{10l}\bar{a}_{G(10l)} &= \left[\begin{matrix} D_{339l} + D_{329l}\ddot{\psi} + D_{333l}\ddot{q}_{1l} + D_{334l}\ddot{q}_{2l} + D_{335l}\ddot{q}_{3l} + D_{336l}\ddot{q}_{4l} + D_{331l}\ddot{q}_{w} + D_{337l}\ddot{q}_{5l} + D_{338l}\ddot{q}_{6l} + D_{330l}\ddot{q}_{imu} + D_{332l}\ddot{q}_{torso} + D_{328l}\ddot{x} + \mathbf{MY}_{10r}\ddot{q}_{7l} & D_{351l} + D_{341l}\ddot{\psi} + D_{345l}\ddot{q}_{1l} + D_{346l}\ddot{q}_{2l} + D_{347l}\ddot{q}_{3l} + D_{348l}\ddot{q}_{4l} + D_{343l}\ddot{q}_{w} + D_{349l}\ddot{q}_{5l} + D_{350l}\ddot{q}_{6l} + D_{342l}\ddot{q}_{imu} + D_{344l}\ddot{q}_{torso} + D_{340l}\ddot{x} - \mathbf{MX}_{10r}\ddot{q}_{7l} & D_{363l} + D_{353l}\ddot{\psi} + D_{357l}\ddot{q}_{1l} + D_{358l}\ddot{q}_{2l} + D_{359l}\ddot{q}_{3l} + D_{360l}\ddot{q}_{4l} + D_{355l}\ddot{q}_{w} + D_{361l}\ddot{q}_{5l} + D_{362l}\ddot{q}_{6l} + D_{354l}\ddot{q}_{imu} + D_{356l}\ddot{q}_{torso} + D_{352l}\ddot{x} &  \end{matrix}\right] 
 \nonumber \\ 
 \dot{\bar{H}}_{10l} &= \mathbf{MS}_{10l} \times \bar{a}_{10l} + J_{10l}\bar{\alpha}_{10l} + \bar\omega_{10l} \times J_{10l}\bar{\omega}_{10l} 
 \nonumber \\ 
 \dot{\bar{H}}_{10l} &= \left[\begin{matrix} K_{213l}(K_{212l}\mathbf{XZ}_{10r} + K_{213l}\mathbf{YZ}_{10r} + K_{214l}\mathbf{ZZ}_{10r}) - K_{214l}(K_{212l}\mathbf{XY}_{10r} + K_{213l}\mathbf{YY}_{10r} + K_{214l}\mathbf{YZ}_{10r}) - \mathbf{MY}_{10r}(K_{189l}\ddot{\psi} - K_{259l} + K_{193l}\ddot{q}_{1l} + K_{194l}\ddot{q}_{2l} + K_{195l}\ddot{q}_{3l} + K_{196l}\ddot{q}_{4l} + K_{191l}\ddot{q}_{w} + K_{190l}\ddot{q}_{imu} + K_{192l}\ddot{q}_{torso} + K_{188l}\ddot{x}) + \mathbf{XX}_{10r}(K_{255l} + K_{215l}\ddot{\psi} + K_{218l}\ddot{q}_{1l} + K_{219l}\ddot{q}_{2l} + K_{220l}\ddot{q}_{3l} + K_{221l}\ddot{q}_{4l} + K_{216l}\ddot{q}_{w} + K_{222l}\ddot{q}_{5l} + K_{216l}\ddot{q}_{imu} + K_{217l}\ddot{q}_{torso} + \ddot{q}_{6l}c_{7l}) + \mathbf{XY}_{10r}(K_{256l} + K_{223l}\ddot{\psi} + K_{226l}\ddot{q}_{1l} + K_{227l}\ddot{q}_{2l} + K_{228l}\ddot{q}_{3l} + K_{229l}\ddot{q}_{4l} + K_{224l}\ddot{q}_{w} + K_{230l}\ddot{q}_{5l} + K_{224l}\ddot{q}_{imu} + K_{225l}\ddot{q}_{torso} + \ddot{q}_{6l}s_{7l}) - \mathbf{XZ}_{10r}(K_{206l} + \ddot{q}_{7l} + K_{172l}\ddot{\psi} + K_{175l}\ddot{q}_{1l} + K_{176l}\ddot{q}_{2l} + K_{177l}\ddot{q}_{3l} + K_{178l}\ddot{q}_{4l} + K_{173l}\ddot{q}_{w} + K_{173l}\ddot{q}_{imu} + K_{174l}\ddot{q}_{torso} + \ddot{q}_{5l}c_{6l}) - \mathbf{MZ}_{10r}(K_{258l} + K_{245l}\ddot{\psi} + K_{249l}\ddot{q}_{1l} + K_{250l}\ddot{q}_{2l} + K_{251l}\ddot{q}_{3l} + K_{252l}\ddot{q}_{4l} + K_{247l}\ddot{q}_{w} + K_{253l}\ddot{q}_{5l} + K_{254l}\ddot{q}_{6l} + K_{246l}\ddot{q}_{imu} + K_{248l}\ddot{q}_{torso} + K_{244l}\ddot{x}) & K_{214l}(K_{212l}\mathbf{XX}_{10r} + K_{213l}\mathbf{XY}_{10r} + K_{214l}\mathbf{XZ}_{10r}) - K_{212l}(K_{212l}\mathbf{XZ}_{10r} + K_{213l}\mathbf{YZ}_{10r} + K_{214l}\mathbf{ZZ}_{10r}) + \mathbf{MX}_{10r}(K_{189l}\ddot{\psi} - K_{259l} + K_{193l}\ddot{q}_{1l} + K_{194l}\ddot{q}_{2l} + K_{195l}\ddot{q}_{3l} + K_{196l}\ddot{q}_{4l} + K_{191l}\ddot{q}_{w} + K_{190l}\ddot{q}_{imu} + K_{192l}\ddot{q}_{torso} + K_{188l}\ddot{x}) + \mathbf{XY}_{10r}(K_{255l} + K_{215l}\ddot{\psi} + K_{218l}\ddot{q}_{1l} + K_{219l}\ddot{q}_{2l} + K_{220l}\ddot{q}_{3l} + K_{221l}\ddot{q}_{4l} + K_{216l}\ddot{q}_{w} + K_{222l}\ddot{q}_{5l} + K_{216l}\ddot{q}_{imu} + K_{217l}\ddot{q}_{torso} + \ddot{q}_{6l}c_{7l}) + \mathbf{YY}_{10r}(K_{256l} + K_{223l}\ddot{\psi} + K_{226l}\ddot{q}_{1l} + K_{227l}\ddot{q}_{2l} + K_{228l}\ddot{q}_{3l} + K_{229l}\ddot{q}_{4l} + K_{224l}\ddot{q}_{w} + K_{230l}\ddot{q}_{5l} + K_{224l}\ddot{q}_{imu} + K_{225l}\ddot{q}_{torso} + \ddot{q}_{6l}s_{7l}) - \mathbf{YZ}_{10r}(K_{206l} + \ddot{q}_{7l} + K_{172l}\ddot{\psi} + K_{175l}\ddot{q}_{1l} + K_{176l}\ddot{q}_{2l} + K_{177l}\ddot{q}_{3l} + K_{178l}\ddot{q}_{4l} + K_{173l}\ddot{q}_{w} + K_{173l}\ddot{q}_{imu} + K_{174l}\ddot{q}_{torso} + \ddot{q}_{5l}c_{6l}) + \mathbf{MZ}_{10r}(K_{257l} + K_{234l}\ddot{\psi} + K_{238l}\ddot{q}_{1l} + K_{239l}\ddot{q}_{2l} + K_{240l}\ddot{q}_{3l} + K_{241l}\ddot{q}_{4l} + K_{236l}\ddot{q}_{w} + K_{242l}\ddot{q}_{5l} + K_{243l}\ddot{q}_{6l} + K_{235l}\ddot{q}_{imu} + K_{237l}\ddot{q}_{torso} + K_{233l}\ddot{x}) & K_{212l}(K_{212l}\mathbf{XY}_{10r} + K_{213l}\mathbf{YY}_{10r} + K_{214l}\mathbf{YZ}_{10r}) - K_{213l}(K_{212l}\mathbf{XX}_{10r} + K_{213l}\mathbf{XY}_{10r} + K_{214l}\mathbf{XZ}_{10r}) + \mathbf{XZ}_{10r}(K_{255l} + K_{215l}\ddot{\psi} + K_{218l}\ddot{q}_{1l} + K_{219l}\ddot{q}_{2l} + K_{220l}\ddot{q}_{3l} + K_{221l}\ddot{q}_{4l} + K_{216l}\ddot{q}_{w} + K_{222l}\ddot{q}_{5l} + K_{216l}\ddot{q}_{imu} + K_{217l}\ddot{q}_{torso} + \ddot{q}_{6l}c_{7l}) + \mathbf{YZ}_{10r}(K_{256l} + K_{223l}\ddot{\psi} + K_{226l}\ddot{q}_{1l} + K_{227l}\ddot{q}_{2l} + K_{228l}\ddot{q}_{3l} + K_{229l}\ddot{q}_{4l} + K_{224l}\ddot{q}_{w} + K_{230l}\ddot{q}_{5l} + K_{224l}\ddot{q}_{imu} + K_{225l}\ddot{q}_{torso} + \ddot{q}_{6l}s_{7l}) - \mathbf{ZZ}_{10r}(K_{206l} + \ddot{q}_{7l} + K_{172l}\ddot{\psi} + K_{175l}\ddot{q}_{1l} + K_{176l}\ddot{q}_{2l} + K_{177l}\ddot{q}_{3l} + K_{178l}\ddot{q}_{4l} + K_{173l}\ddot{q}_{w} + K_{173l}\ddot{q}_{imu} + K_{174l}\ddot{q}_{torso} + \ddot{q}_{5l}c_{6l}) + \mathbf{MX}_{10r}(K_{258l} + K_{245l}\ddot{\psi} + K_{249l}\ddot{q}_{1l} + K_{250l}\ddot{q}_{2l} + K_{251l}\ddot{q}_{3l} + K_{252l}\ddot{q}_{4l} + K_{247l}\ddot{q}_{w} + K_{253l}\ddot{q}_{5l} + K_{254l}\ddot{q}_{6l} + K_{246l}\ddot{q}_{imu} + K_{248l}\ddot{q}_{torso} + K_{244l}\ddot{x}) - \mathbf{MY}_{10r}(K_{257l} + K_{234l}\ddot{\psi} + K_{238l}\ddot{q}_{1l} + K_{239l}\ddot{q}_{2l} + K_{240l}\ddot{q}_{3l} + K_{241l}\ddot{q}_{4l} + K_{236l}\ddot{q}_{w} + K_{242l}\ddot{q}_{5l} + K_{243l}\ddot{q}_{6l} + K_{235l}\ddot{q}_{imu} + K_{237l}\ddot{q}_{torso} + K_{233l}\ddot{x}) &  \end{matrix}\right] 
 \nonumber \\ 
D_{364l} &= - K_{188l}\mathbf{MY}_{10r} - K_{244l}\mathbf{MZ}_{10r} \nonumber \\
D_{365l} &= K_{215l}\mathbf{XX}_{10r} + K_{223l}\mathbf{XY}_{10r} - K_{172l}\mathbf{XZ}_{10r}  \nonumber \\
&- K_{189l}\mathbf{MY}_{10r} - K_{245l}\mathbf{MZ}_{10r} \nonumber \\
D_{366l} &= K_{216l}\mathbf{XX}_{10r} + K_{224l}\mathbf{XY}_{10r} - K_{173l}\mathbf{XZ}_{10r}  \nonumber \\
&- K_{190l}\mathbf{MY}_{10r} - K_{246l}\mathbf{MZ}_{10r} \nonumber \\
D_{367l} &= K_{216l}\mathbf{XX}_{10r} + K_{224l}\mathbf{XY}_{10r} - K_{173l}\mathbf{XZ}_{10r}  \nonumber \\
&- K_{191l}\mathbf{MY}_{10r} - K_{247l}\mathbf{MZ}_{10r} \nonumber \\
D_{368l} &= K_{217l}\mathbf{XX}_{10r} + K_{225l}\mathbf{XY}_{10r} - K_{174l}\mathbf{XZ}_{10r}  \nonumber \\
&- K_{192l}\mathbf{MY}_{10r} - K_{248l}\mathbf{MZ}_{10r} \nonumber \\
D_{369l} &= K_{218l}\mathbf{XX}_{10r} + K_{226l}\mathbf{XY}_{10r} - K_{175l}\mathbf{XZ}_{10r}  \nonumber \\
&- K_{193l}\mathbf{MY}_{10r} - K_{249l}\mathbf{MZ}_{10r} \nonumber \\
D_{370l} &= K_{219l}\mathbf{XX}_{10r} + K_{227l}\mathbf{XY}_{10r} - K_{176l}\mathbf{XZ}_{10r}  \nonumber \\
&- K_{194l}\mathbf{MY}_{10r} - K_{250l}\mathbf{MZ}_{10r} \nonumber \\
D_{371l} &= K_{220l}\mathbf{XX}_{10r} + K_{228l}\mathbf{XY}_{10r} - K_{177l}\mathbf{XZ}_{10r}  \nonumber \\
&- K_{195l}\mathbf{MY}_{10r} - K_{251l}\mathbf{MZ}_{10r} \nonumber \\
D_{372l} &= K_{221l}\mathbf{XX}_{10r} + K_{229l}\mathbf{XY}_{10r} - K_{178l}\mathbf{XZ}_{10r}  \nonumber \\
&- K_{196l}\mathbf{MY}_{10r} - K_{252l}\mathbf{MZ}_{10r} \nonumber \\
D_{373l} &= K_{222l}\mathbf{XX}_{10r} + K_{230l}\mathbf{XY}_{10r} - \mathbf{XZ}_{10r}c_{6l}  \nonumber \\
&- K_{253l}\mathbf{MZ}_{10r} \nonumber \\
D_{374l} &= \mathbf{XX}_{10r}c_{7l} + \mathbf{XY}_{10r}s_{7l} - K_{254l}\mathbf{MZ}_{10r} \nonumber \\
D_{375l} &= K_{255l}\mathbf{XX}_{10r} + K_{256l}\mathbf{XY}_{10r} - K_{206l}\mathbf{XZ}_{10r}  \nonumber \\
&+ K_{213l}^2\mathbf{YZ}_{10r} - K_{214l}^2\mathbf{YZ}_{10r} + K_{259l}\mathbf{MY}_{10r}  \nonumber \\
&- K_{258l}\mathbf{MZ}_{10r} - K_{212l}K_{214l}\mathbf{XY}_{10r} + K_{212l}K_{213l}\mathbf{XZ}_{10r}  \nonumber \\
&- K_{213l}K_{214l}\mathbf{YY}_{10r} + K_{213l}K_{214l}\mathbf{ZZ}_{10r} \nonumber \\
D_{376l} &= K_{188l}\mathbf{MX}_{10r} + K_{233l}\mathbf{MZ}_{10r} \nonumber \\
D_{377l} &= K_{215l}\mathbf{XY}_{10r} + K_{223l}\mathbf{YY}_{10r} - K_{172l}\mathbf{YZ}_{10r}  \nonumber \\
&+ K_{189l}\mathbf{MX}_{10r} + K_{234l}\mathbf{MZ}_{10r} \nonumber \\
D_{378l} &= K_{216l}\mathbf{XY}_{10r} + K_{224l}\mathbf{YY}_{10r} - K_{173l}\mathbf{YZ}_{10r}  \nonumber \\
&+ K_{190l}\mathbf{MX}_{10r} + K_{235l}\mathbf{MZ}_{10r} \nonumber \\
D_{379l} &= K_{216l}\mathbf{XY}_{10r} + K_{224l}\mathbf{YY}_{10r} - K_{173l}\mathbf{YZ}_{10r}  \nonumber \\
&+ K_{191l}\mathbf{MX}_{10r} + K_{236l}\mathbf{MZ}_{10r} \nonumber \\
D_{380l} &= K_{217l}\mathbf{XY}_{10r} + K_{225l}\mathbf{YY}_{10r} - K_{174l}\mathbf{YZ}_{10r}  \nonumber \\
&+ K_{192l}\mathbf{MX}_{10r} + K_{237l}\mathbf{MZ}_{10r} \nonumber \\
D_{381l} &= K_{218l}\mathbf{XY}_{10r} + K_{226l}\mathbf{YY}_{10r} - K_{175l}\mathbf{YZ}_{10r}  \nonumber \\
&+ K_{193l}\mathbf{MX}_{10r} + K_{238l}\mathbf{MZ}_{10r} \nonumber \\
D_{382l} &= K_{219l}\mathbf{XY}_{10r} + K_{227l}\mathbf{YY}_{10r} - K_{176l}\mathbf{YZ}_{10r}  \nonumber \\
&+ K_{194l}\mathbf{MX}_{10r} + K_{239l}\mathbf{MZ}_{10r} \nonumber \\
D_{383l} &= K_{220l}\mathbf{XY}_{10r} + K_{228l}\mathbf{YY}_{10r} - K_{177l}\mathbf{YZ}_{10r}  \nonumber \\
&+ K_{195l}\mathbf{MX}_{10r} + K_{240l}\mathbf{MZ}_{10r} \nonumber \\
D_{384l} &= K_{221l}\mathbf{XY}_{10r} + K_{229l}\mathbf{YY}_{10r} - K_{178l}\mathbf{YZ}_{10r}  \nonumber \\
&+ K_{196l}\mathbf{MX}_{10r} + K_{241l}\mathbf{MZ}_{10r} \nonumber \\
D_{385l} &= K_{222l}\mathbf{XY}_{10r} + K_{230l}\mathbf{YY}_{10r} - \mathbf{YZ}_{10r}c_{6l}  \nonumber \\
&+ K_{242l}\mathbf{MZ}_{10r} \nonumber \\
D_{386l} &= \mathbf{XY}_{10r}c_{7l} + \mathbf{YY}_{10r}s_{7l} + K_{243l}\mathbf{MZ}_{10r} \nonumber \\
D_{387l} &= K_{255l}\mathbf{XY}_{10r} + K_{256l}\mathbf{YY}_{10r} - K_{206l}\mathbf{YZ}_{10r}  \nonumber \\
&- K_{212l}^2\mathbf{XZ}_{10r} + K_{214l}^2\mathbf{XZ}_{10r} - K_{259l}\mathbf{MX}_{10r}  \nonumber \\
&+ K_{257l}\mathbf{MZ}_{10r} + K_{212l}K_{214l}\mathbf{XX}_{10r} + K_{213l}K_{214l}\mathbf{XY}_{10r}  \nonumber \\
&- K_{212l}K_{213l}\mathbf{YZ}_{10r} - K_{212l}K_{214l}\mathbf{ZZ}_{10r} \nonumber \\
D_{388l} &= K_{244l}\mathbf{MX}_{10r} - K_{233l}\mathbf{MY}_{10r} \nonumber \\
D_{389l} &= K_{215l}\mathbf{XZ}_{10r} + K_{223l}\mathbf{YZ}_{10r} - K_{172l}\mathbf{ZZ}_{10r}  \nonumber \\
&+ K_{245l}\mathbf{MX}_{10r} - K_{234l}\mathbf{MY}_{10r} \nonumber \\
D_{390l} &= K_{216l}\mathbf{XZ}_{10r} + K_{224l}\mathbf{YZ}_{10r} - K_{173l}\mathbf{ZZ}_{10r}  \nonumber \\
&+ K_{246l}\mathbf{MX}_{10r} - K_{235l}\mathbf{MY}_{10r} \nonumber \\
D_{391l} &= K_{216l}\mathbf{XZ}_{10r} + K_{224l}\mathbf{YZ}_{10r} - K_{173l}\mathbf{ZZ}_{10r}  \nonumber \\
&+ K_{247l}\mathbf{MX}_{10r} - K_{236l}\mathbf{MY}_{10r} \nonumber \\
D_{392l} &= K_{217l}\mathbf{XZ}_{10r} + K_{225l}\mathbf{YZ}_{10r} - K_{174l}\mathbf{ZZ}_{10r}  \nonumber \\
&+ K_{248l}\mathbf{MX}_{10r} - K_{237l}\mathbf{MY}_{10r} \nonumber \\
D_{393l} &= K_{218l}\mathbf{XZ}_{10r} + K_{226l}\mathbf{YZ}_{10r} - K_{175l}\mathbf{ZZ}_{10r}  \nonumber \\
&+ K_{249l}\mathbf{MX}_{10r} - K_{238l}\mathbf{MY}_{10r} \nonumber \\
D_{394l} &= K_{219l}\mathbf{XZ}_{10r} + K_{227l}\mathbf{YZ}_{10r} - K_{176l}\mathbf{ZZ}_{10r}  \nonumber \\
&+ K_{250l}\mathbf{MX}_{10r} - K_{239l}\mathbf{MY}_{10r} \nonumber \\
D_{395l} &= K_{220l}\mathbf{XZ}_{10r} + K_{228l}\mathbf{YZ}_{10r} - K_{177l}\mathbf{ZZ}_{10r}  \nonumber \\
&+ K_{251l}\mathbf{MX}_{10r} - K_{240l}\mathbf{MY}_{10r} \nonumber \\
D_{396l} &= K_{221l}\mathbf{XZ}_{10r} + K_{229l}\mathbf{YZ}_{10r} - K_{178l}\mathbf{ZZ}_{10r}  \nonumber \\
&+ K_{252l}\mathbf{MX}_{10r} - K_{241l}\mathbf{MY}_{10r} \nonumber \\
D_{397l} &= K_{222l}\mathbf{XZ}_{10r} + K_{230l}\mathbf{YZ}_{10r} - \mathbf{ZZ}_{10r}c_{6l}  \nonumber \\
&+ K_{253l}\mathbf{MX}_{10r} - K_{242l}\mathbf{MY}_{10r} \nonumber \\
D_{398l} &= \mathbf{XZ}_{10r}c_{7l} + \mathbf{YZ}_{10r}s_{7l} + K_{254l}\mathbf{MX}_{10r}  \nonumber \\
&- K_{243l}\mathbf{MY}_{10r} \nonumber \\
D_{399l} &= K_{255l}\mathbf{XZ}_{10r} + K_{256l}\mathbf{YZ}_{10r} - K_{206l}\mathbf{ZZ}_{10r}  \nonumber \\
&+ K_{212l}^2\mathbf{XY}_{10r} - K_{213l}^2\mathbf{XY}_{10r} + K_{258l}\mathbf{MX}_{10r}  \nonumber \\
&- K_{257l}\mathbf{MY}_{10r} - K_{212l}K_{213l}\mathbf{XX}_{10r} - K_{213l}K_{214l}\mathbf{XZ}_{10r}  \nonumber \\
&+ K_{212l}K_{213l}\mathbf{YY}_{10r} + K_{212l}K_{214l}\mathbf{YZ}_{10r} \nonumber \\
 \dot{\bar{H}}_{10l} &= \left[\begin{matrix} D_{339l} + D_{329l}\ddot{\psi} + D_{333l}\ddot{q}_{1l} + D_{334l}\ddot{q}_{2l} + D_{335l}\ddot{q}_{3l} + D_{336l}\ddot{q}_{4l} + D_{331l}\ddot{q}_{w} + D_{337l}\ddot{q}_{5l} + D_{338l}\ddot{q}_{6l} + D_{330l}\ddot{q}_{imu} + D_{332l}\ddot{q}_{torso} + D_{328l}\ddot{x} + \mathbf{MY}_{10r}\ddot{q}_{7l} & D_{351l} + D_{341l}\ddot{\psi} + D_{345l}\ddot{q}_{1l} + D_{346l}\ddot{q}_{2l} + D_{347l}\ddot{q}_{3l} + D_{348l}\ddot{q}_{4l} + D_{343l}\ddot{q}_{w} + D_{349l}\ddot{q}_{5l} + D_{350l}\ddot{q}_{6l} + D_{342l}\ddot{q}_{imu} + D_{344l}\ddot{q}_{torso} + D_{340l}\ddot{x} - \mathbf{MX}_{10r}\ddot{q}_{7l} & D_{363l} + D_{353l}\ddot{\psi} + D_{357l}\ddot{q}_{1l} + D_{358l}\ddot{q}_{2l} + D_{359l}\ddot{q}_{3l} + D_{360l}\ddot{q}_{4l} + D_{355l}\ddot{q}_{w} + D_{361l}\ddot{q}_{5l} + D_{362l}\ddot{q}_{6l} + D_{354l}\ddot{q}_{imu} + D_{356l}\ddot{q}_{torso} + D_{352l}\ddot{x} &  \end{matrix}\right] 
 \nonumber \\ 
 \bar\omega_{4r} &= {}^{4r}A_{3} \bar\omega_{3} + \dot{q}_{4r} \bar{e}_{4r} 
 \nonumber \\ 
 \bar\omega_{4r} &= \left[\begin{matrix} K_{26}c_{1r} + K_{27}s_{1r} & - K_{25} - \dot{q}_{1r} & K_{27}c_{1r} - K_{26}s_{1r} &  \end{matrix}\right] 
 \nonumber \\ 
K_{1r} &= K_{26}c_{1r} + K_{27}s_{1r} \nonumber \\
K_{2r} &= - K_{25} - \dot{q}_{1r} \nonumber \\
K_{3r} &= K_{27}c_{1r} - K_{26}s_{1r} \nonumber \\
 \bar\omega_{4r} &= \left[\begin{matrix} K_{1r} & K_{2r} & K_{3r} &  \end{matrix}\right] 
 \nonumber \\ 
 \bar\omega_{4r} &= \left[\begin{matrix} - s_{1r}(\dot{q}_{w}s_{torso} - K_{29}\dot{\psi} + \dot{q}_{imu}s_{torso}) - c_{1r}(\dot{q}_{torso} - K_{10}\dot{\psi}) & - \dot{q}_{1r} - K_{28}\dot{\psi} - \dot{q}_{w}c_{torso} - \dot{q}_{imu}c_{torso} & s_{1r}(\dot{q}_{torso} - K_{10}\dot{\psi}) - c_{1r}(\dot{q}_{w}s_{torso} - K_{29}\dot{\psi} + \dot{q}_{imu}s_{torso}) &  \end{matrix}\right] 
 \nonumber \\ 
K_{4r} &= K_{10}c_{1r} + K_{29}s_{1r} \nonumber \\
K_{5r} &= -s_{1r}s_{torso} \nonumber \\
K_{6r} &= K_{29}c_{1r} - K_{10}s_{1r} \nonumber \\
K_{7r} &= -c_{1r}s_{torso} \nonumber \\
 \bar\omega_{4r} &= \left[\begin{matrix} K_{4r}\dot{\psi} + K_{5r}\dot{q}_{w} + K_{5r}\dot{q}_{imu} - \dot{q}_{torso}c_{1r} & - \dot{q}_{1r} - K_{28}\dot{\psi} - \dot{q}_{w}c_{torso} - \dot{q}_{imu}c_{torso} & K_{6r}\dot{\psi} + K_{7r}\dot{q}_{w} + K_{7r}\dot{q}_{imu} + \dot{q}_{torso}s_{1r} &  \end{matrix}\right] 
 \nonumber \\ 
 \bar{v}_{4r} &= {}^{4r}A_{3} \left(\bar{v}_{3} + \bar\omega_{3} \times \bar{P}_{4r}\right) 
 \nonumber \\ 
 \bar{v}_{4r} &= \left[\begin{matrix} c_{1r}(K_{31} - K_{27}L_6) + s_{1r}(K_{32} + K_{25}L_5 + K_{26}L_6) & K_{27}L_5 - K_{30} & c_{1r}(K_{32} + K_{25}L_5 + K_{26}L_6) - s_{1r}(K_{31} - K_{27}L_6) &  \end{matrix}\right] 
 \nonumber \\ 
K_{8r} &= c_{1r}(K_{31} - K_{27}L_6) + s_{1r}(K_{32}  \nonumber \\
&+ K_{25}L_5 + K_{26}L_6) \nonumber \\
K_{9r} &= K_{27}L_5 - K_{30} \nonumber \\
K_{10r} &= c_{1r}(K_{32} + K_{25}L_5 + K_{26}L_6)  \nonumber \\
&- s_{1r}(K_{31} - K_{27}L_6) \nonumber \\
 \bar{v}_{4r} &= \left[\begin{matrix} K_{8r} & K_{9r} & K_{10r} &  \end{matrix}\right] 
 \nonumber \\ 
 \bar{v}_{4r} &= \left[\begin{matrix} c_{1r}(K_{37}\dot{q}_{imu} + K_{11}\dot{x} + L_4\dot{q}_{w} + L_6(\dot{q}_{w}s_{torso} - K_{29}\dot{\psi} + \dot{q}_{imu}s_{torso})) + s_{1r}(K_{39}\dot{\psi} + K_{41}\dot{q}_{w} + K_{40}\dot{q}_{imu} + K_{38}\dot{x} + L_5(K_{28}\dot{\psi} + \dot{q}_{w}c_{torso} + \dot{q}_{imu}c_{torso}) - L_6(\dot{q}_{torso} - K_{10}\dot{\psi})) & - K_{34}\dot{\psi} - K_{36}\dot{q}_{w} - K_{35}\dot{q}_{imu} - K_{33}\dot{x} - L_5(\dot{q}_{w}s_{torso} - K_{29}\dot{\psi} + \dot{q}_{imu}s_{torso}) & c_{1r}(K_{39}\dot{\psi} + K_{41}\dot{q}_{w} + K_{40}\dot{q}_{imu} + K_{38}\dot{x} + L_5(K_{28}\dot{\psi} + \dot{q}_{w}c_{torso} + \dot{q}_{imu}c_{torso}) - L_6(\dot{q}_{torso} - K_{10}\dot{\psi})) - s_{1r}(K_{37}\dot{q}_{imu} + K_{11}\dot{x} + L_4\dot{q}_{w} + L_6(\dot{q}_{w}s_{torso} - K_{29}\dot{\psi} + \dot{q}_{imu}s_{torso})) &  \end{matrix}\right] 
 \nonumber \\ 
K_{11r} &= K_{11}c_{1r} + K_{38}s_{1r} \nonumber \\
K_{12r} &= s_{1r}(K_{39} + K_{10}L_6 + K_{28}L_5)  \nonumber \\
&- K_{29}L_6c_{1r} \nonumber \\
K_{13r} &= c_{1r}(K_{37} + L_6s_{torso}) + s_{1r}(K_{40}  \nonumber \\
&+ L_5c_{torso}) \nonumber \\
K_{14r} &= s_{1r}(K_{41} + L_5c_{torso}) + c_{1r}(L_4  \nonumber \\
&+ L_6s_{torso}) \nonumber \\
K_{15r} &= -L_6s_{1r} \nonumber \\
K_{16r} &= K_{29}L_5 - K_{34} \nonumber \\
K_{17r} &= - K_{35} - L_5s_{torso} \nonumber \\
K_{18r} &= - K_{36} - L_5s_{torso} \nonumber \\
K_{19r} &= K_{38}c_{1r} - K_{11}s_{1r} \nonumber \\
K_{20r} &= c_{1r}(K_{39} + K_{10}L_6 + K_{28}L_5)  \nonumber \\
&+ K_{29}L_6s_{1r} \nonumber \\
K_{21r} &= c_{1r}(K_{40} + L_5c_{torso}) - s_{1r}(K_{37}  \nonumber \\
&+ L_6s_{torso}) \nonumber \\
K_{22r} &= c_{1r}(K_{41} + L_5c_{torso}) - s_{1r}(L_4  \nonumber \\
&+ L_6s_{torso}) \nonumber \\
K_{23r} &= -L_6c_{1r} \nonumber \\
 \bar{v}_{4r} &= \left[\begin{matrix} K_{12r}\dot{\psi} + K_{14r}\dot{q}_{w} + K_{13r}\dot{q}_{imu} + K_{15r}\dot{q}_{torso} + K_{11r}\dot{x} & K_{16r}\dot{\psi} + K_{18r}\dot{q}_{w} + K_{17r}\dot{q}_{imu} - K_{33}\dot{x} & K_{20r}\dot{\psi} + K_{22r}\dot{q}_{w} + K_{21r}\dot{q}_{imu} + K_{23r}\dot{q}_{torso} + K_{19r}\dot{x} &  \end{matrix}\right] 
 \nonumber \\ 
 \bar\alpha_{4r} &= {}^{4r}A_{3} \bar\alpha_{3} + \ddot{q}_{4r} \bar{e}_{4r} + \dot{q}_{4r} \left(\bar\omega_{4r} \times \bar{e}_{4r}\right) 
 \nonumber \\ 
 \bar\alpha_{4r} &= \left[\begin{matrix} K_{3r}\dot{q}_{1r} + s_{1r}(K_{43} + K_{29}\ddot{\psi} - \ddot{q}_{w}s_{torso} - \ddot{q}_{imu}s_{torso}) + c_{1r}(K_{19} - \ddot{q}_{torso} + K_{10}\ddot{\psi}) & - K_{42} - \ddot{q}_{1r} - K_{28}\ddot{\psi} - \ddot{q}_{w}c_{torso} - \ddot{q}_{imu}c_{torso} & c_{1r}(K_{43} + K_{29}\ddot{\psi} - \ddot{q}_{w}s_{torso} - \ddot{q}_{imu}s_{torso}) - K_{1r}\dot{q}_{1r} - s_{1r}(K_{19} - \ddot{q}_{torso} + K_{10}\ddot{\psi}) &  \end{matrix}\right] 
 \nonumber \\ 
K_{24r} &= K_{3r}\dot{q}_{1r} + K_{19}c_{1r} + K_{43}s_{1r} \nonumber \\
K_{25r} &= K_{43}c_{1r} - K_{1r}\dot{q}_{1r} - K_{19}s_{1r} \nonumber \\
 \bar\alpha_{4r} &= \left[\begin{matrix} K_{24r} + K_{4r}\ddot{\psi} + K_{5r}\ddot{q}_{w} + K_{5r}\ddot{q}_{imu} - \ddot{q}_{torso}c_{1r} & - K_{42} - \ddot{q}_{1r} - K_{28}\ddot{\psi} - \ddot{q}_{w}c_{torso} - \ddot{q}_{imu}c_{torso} & K_{25r} + K_{6r}\ddot{\psi} + K_{7r}\ddot{q}_{w} + K_{7r}\ddot{q}_{imu} + \ddot{q}_{torso}s_{1r} &  \end{matrix}\right] 
 \nonumber \\ 
 \bar{a}_{4r} &= {}^{4r}A_{3} \left(\bar{a}_{3} + \bar\alpha_{3} \times \bar{P}_{4r} + \bar\omega_{3} \times \left(\bar\omega_{3} \times \bar{P}_{4r}\right)\right) 
 \nonumber \\ 
 \bar\alpha_{4r} &= \left[\begin{matrix} c_{1r}(K_{45} + K_{37}\ddot{q}_{imu} + K_{11}\ddot{x} + L_4\ddot{q}_{w} - K_{27}^2L_5 - L_6(K_{43} + K_{29}\ddot{\psi} - \ddot{q}_{w}s_{torso} - \ddot{q}_{imu}s_{torso}) - K_{25}(K_{25}L_5 + K_{26}L_6)) + s_{1r}(K_{46} + K_{39}\ddot{\psi} + K_{41}\ddot{q}_{w} + K_{40}\ddot{q}_{imu} + K_{38}\ddot{x} + L_6(K_{19} - \ddot{q}_{torso} + K_{10}\ddot{\psi}) + L_5(K_{42} + K_{28}\ddot{\psi} + \ddot{q}_{w}c_{torso} + \ddot{q}_{imu}c_{torso}) - K_{25}K_{27}L_6 + K_{26}K_{27}L_5) & L_5(K_{43} + K_{29}\ddot{\psi} - \ddot{q}_{w}s_{torso} - \ddot{q}_{imu}s_{torso}) - K_{34}\ddot{\psi} - K_{36}\ddot{q}_{w} - K_{35}\ddot{q}_{imu} - K_{33}\ddot{x} - K_{27}^2L_6 - K_{44} - K_{26}(K_{25}L_5 + K_{26}L_6) & c_{1r}(K_{46} + K_{39}\ddot{\psi} + K_{41}\ddot{q}_{w} + K_{40}\ddot{q}_{imu} + K_{38}\ddot{x} + L_6(K_{19} - \ddot{q}_{torso} + K_{10}\ddot{\psi}) + L_5(K_{42} + K_{28}\ddot{\psi} + \ddot{q}_{w}c_{torso} + \ddot{q}_{imu}c_{torso}) - K_{25}K_{27}L_6 + K_{26}K_{27}L_5) - s_{1r}(K_{45} + K_{37}\ddot{q}_{imu} + K_{11}\ddot{x} + L_4\ddot{q}_{w} - K_{27}^2L_5 - L_6(K_{43} + K_{29}\ddot{\psi} - \ddot{q}_{w}s_{torso} - \ddot{q}_{imu}s_{torso}) - K_{25}(K_{25}L_5 + K_{26}L_6)) &  \end{matrix}\right] 
 \nonumber \\ 
K_{26r} &= K_{45}c_{1r} + K_{46}s_{1r} - K_{25}^2L_5c_{1r}  \nonumber \\
&- K_{27}^2L_5c_{1r} - K_{43}L_6c_{1r}  \nonumber \\
&+ K_{19}L_6s_{1r} + K_{42}L_5s_{1r}  \nonumber \\
&- K_{25}K_{26}L_6c_{1r} - K_{25}K_{27}L_6s_{1r}  \nonumber \\
&+ K_{26}K_{27}L_5s_{1r} \nonumber \\
K_{27r} &= K_{43}L_5 - K_{26}^2L_6 - K_{27}^2L_6  \nonumber \\
&- K_{44} - K_{25}K_{26}L_5 \nonumber \\
K_{28r} &= K_{46}c_{1r} - K_{45}s_{1r} + K_{25}^2L_5s_{1r}  \nonumber \\
&+ K_{27}^2L_5s_{1r} + K_{19}L_6c_{1r}  \nonumber \\
&+ K_{42}L_5c_{1r} + K_{43}L_6s_{1r}  \nonumber \\
&- K_{25}K_{27}L_6c_{1r} + K_{26}K_{27}L_5c_{1r}  \nonumber \\
&+ K_{25}K_{26}L_6s_{1r} \nonumber \\
 \bar{a}_{4r} &= \left[\begin{matrix} K_{26r} + K_{12r}\ddot{\psi} + K_{14r}\ddot{q}_{w} + K_{13r}\ddot{q}_{imu} + K_{15r}\ddot{q}_{torso} + K_{11r}\ddot{x} & K_{27r} + K_{16r}\ddot{\psi} + K_{18r}\ddot{q}_{w} + K_{17r}\ddot{q}_{imu} - K_{33}\ddot{x} & K_{28r} + K_{20r}\ddot{\psi} + K_{22r}\ddot{q}_{w} + K_{21r}\ddot{q}_{imu} + K_{23r}\ddot{q}_{torso} + K_{19r}\ddot{x} &  \end{matrix}\right] 
 \nonumber \\ 
 \bar{g}_{4r} &= {}^{4r}A_{3} \bar{g}_{3} 
 \nonumber \\ 
 \bar{g}_{4r} &= \left[\begin{matrix} K_{17}gc_{1r} + K_{48}gs_{1r} & -K_{47}g & K_{48}gc_{1r} - K_{17}gs_{1r} &  \end{matrix}\right] 
 \nonumber \\ 
K_{29r} &= K_{17}c_{1r} + K_{48}s_{1r} \nonumber \\
K_{30r} &= K_{48}c_{1r} - K_{17}s_{1r} \nonumber \\
 \bar{g}_{4r} &= \left[\begin{matrix} K_{29r}g & -K_{47}g & K_{30r}g &  \end{matrix}\right] 
 \nonumber \\ 
 m_{4r}\bar{S}_{4r}^{\times}\bar{g}_{4r} &= \mathbf{MS}_{4r} \times \bar{g}_{4r} 
 \nonumber \\ 
 m_{4r}\bar{S}_{4r}^{\times}\bar{g}_{4r} &= \left[\begin{matrix} K_{30r}\mathbf{MY}_{4r}g + K_{47}\mathbf{MZ}_{4r}g & K_{29r}\mathbf{MZ}_{4r}g - K_{30r}\mathbf{MX}_{4r}g & - K_{47}\mathbf{MX}_{4r}g - K_{29r}\mathbf{MY}_{4r}g &  \end{matrix}\right] 
 \nonumber \\ 
D_{1r} &= K_{30r}\mathbf{MY}_{4r} + K_{47}\mathbf{MZ}_{4r} \nonumber \\
D_{2r} &= K_{29r}\mathbf{MZ}_{4r} - K_{30r}\mathbf{MX}_{4r} \nonumber \\
D_{3r} &= - K_{47}\mathbf{MX}_{4r} - K_{29r}\mathbf{MY}_{4r} \nonumber \\
 m_{4r}\bar{S}_{4r}^{\times}\bar{g}_{4r} &= \left[\begin{matrix} D_{1r}g & D_{2r}g & D_{3r}g &  \end{matrix}\right] 
 \nonumber \\ 
 m_{4r}\bar{a}_{G(4r)} &= m_{4r}\bar{a}_{4r} + \bar\alpha_{4r} \times \mathbf{MS}_{4r} + \bar\omega_{4r} \times \left(\bar\omega_{4r} \times \mathbf{MS}_{4r}\right) 
 \nonumber \\ 
 m_{4r}\bar{a}_{G(4r)} &= \left[\begin{matrix} m_{4r}(K_{26r} + K_{12r}\ddot{\psi} + K_{14r}\ddot{q}_{w} + K_{13r}\ddot{q}_{imu} + K_{15r}\ddot{q}_{torso} + K_{11r}\ddot{x}) - \mathbf{MY}_{4r}(K_{25r} + K_{6r}\ddot{\psi} + K_{7r}\ddot{q}_{w} + K_{7r}\ddot{q}_{imu} + \ddot{q}_{torso}s_{1r}) - K_{2r}(K_{2r}\mathbf{MX}_{4r} - K_{1r}\mathbf{MY}_{4r}) - K_{3r}(K_{3r}\mathbf{MX}_{4r} - K_{1r}\mathbf{MZ}_{4r}) - \mathbf{MZ}_{4r}(K_{42} + \ddot{q}_{1r} + K_{28}\ddot{\psi} + \ddot{q}_{w}c_{torso} + \ddot{q}_{imu}c_{torso}) & \mathbf{MX}_{4r}(K_{25r} + K_{6r}\ddot{\psi} + K_{7r}\ddot{q}_{w} + K_{7r}\ddot{q}_{imu} + \ddot{q}_{torso}s_{1r}) - \mathbf{MZ}_{4r}(K_{24r} + K_{4r}\ddot{\psi} + K_{5r}\ddot{q}_{w} + K_{5r}\ddot{q}_{imu} - \ddot{q}_{torso}c_{1r}) + m_{4r}(K_{27r} + K_{16r}\ddot{\psi} + K_{18r}\ddot{q}_{w} + K_{17r}\ddot{q}_{imu} - K_{33}\ddot{x}) + K_{1r}(K_{2r}\mathbf{MX}_{4r} - K_{1r}\mathbf{MY}_{4r}) - K_{3r}(K_{3r}\mathbf{MY}_{4r} - K_{2r}\mathbf{MZ}_{4r}) & \mathbf{MY}_{4r}(K_{24r} + K_{4r}\ddot{\psi} + K_{5r}\ddot{q}_{w} + K_{5r}\ddot{q}_{imu} - \ddot{q}_{torso}c_{1r}) + m_{4r}(K_{28r} + K_{20r}\ddot{\psi} + K_{22r}\ddot{q}_{w} + K_{21r}\ddot{q}_{imu} + K_{23r}\ddot{q}_{torso} + K_{19r}\ddot{x}) + K_{1r}(K_{3r}\mathbf{MX}_{4r} - K_{1r}\mathbf{MZ}_{4r}) + K_{2r}(K_{3r}\mathbf{MY}_{4r} - K_{2r}\mathbf{MZ}_{4r}) + \mathbf{MX}_{4r}(K_{42} + \ddot{q}_{1r} + K_{28}\ddot{\psi} + \ddot{q}_{w}c_{torso} + \ddot{q}_{imu}c_{torso}) &  \end{matrix}\right] 
 \nonumber \\ 
D_{4r} &= K_{11r}m_{4r} \nonumber \\
D_{5r} &= K_{12r}m_{4r} - K_{6r}\mathbf{MY}_{4r} - K_{28}\mathbf{MZ}_{4r} \nonumber \\
D_{6r} &= K_{13r}m_{4r} - \mathbf{MZ}_{4r}c_{torso} - K_{7r}\mathbf{MY}_{4r} \nonumber \\
D_{7r} &= K_{14r}m_{4r} - \mathbf{MZ}_{4r}c_{torso} - K_{7r}\mathbf{MY}_{4r} \nonumber \\
D_{8r} &= K_{15r}m_{4r} - \mathbf{MY}_{4r}s_{1r} \nonumber \\
D_{9r} &= K_{26r}m_{4r} - K_{2r}^2\mathbf{MX}_{4r} - K_{3r}^2\mathbf{MX}_{4r}  \nonumber \\
&- K_{25r}\mathbf{MY}_{4r} - K_{42}\mathbf{MZ}_{4r} + K_{1r}K_{2r}\mathbf{MY}_{4r}  \nonumber \\
&+ K_{1r}K_{3r}\mathbf{MZ}_{4r} \nonumber \\
D_{10r} &= -K_{33}m_{4r} \nonumber \\
D_{11r} &= K_{16r}m_{4r} + K_{6r}\mathbf{MX}_{4r} - K_{4r}\mathbf{MZ}_{4r} \nonumber \\
D_{12r} &= K_{17r}m_{4r} + K_{7r}\mathbf{MX}_{4r} - K_{5r}\mathbf{MZ}_{4r} \nonumber \\
D_{13r} &= K_{18r}m_{4r} + K_{7r}\mathbf{MX}_{4r} - K_{5r}\mathbf{MZ}_{4r} \nonumber \\
D_{14r} &= \mathbf{MZ}_{4r}c_{1r} + \mathbf{MX}_{4r}s_{1r} \nonumber \\
D_{15r} &= K_{27r}m_{4r} - K_{1r}^2\mathbf{MY}_{4r} - K_{3r}^2\mathbf{MY}_{4r}  \nonumber \\
&+ K_{25r}\mathbf{MX}_{4r} - K_{24r}\mathbf{MZ}_{4r} + K_{1r}K_{2r}\mathbf{MX}_{4r}  \nonumber \\
&+ K_{2r}K_{3r}\mathbf{MZ}_{4r} \nonumber \\
D_{16r} &= K_{19r}m_{4r} \nonumber \\
D_{17r} &= K_{20r}m_{4r} + K_{28}\mathbf{MX}_{4r} + K_{4r}\mathbf{MY}_{4r} \nonumber \\
D_{18r} &= K_{21r}m_{4r} + \mathbf{MX}_{4r}c_{torso} + K_{5r}\mathbf{MY}_{4r} \nonumber \\
D_{19r} &= K_{22r}m_{4r} + \mathbf{MX}_{4r}c_{torso} + K_{5r}\mathbf{MY}_{4r} \nonumber \\
D_{20r} &= K_{23r}m_{4r} - \mathbf{MY}_{4r}c_{1r} \nonumber \\
D_{21r} &= K_{28r}m_{4r} - K_{1r}^2\mathbf{MZ}_{4r} - K_{2r}^2\mathbf{MZ}_{4r}  \nonumber \\
&+ K_{42}\mathbf{MX}_{4r} + K_{24r}\mathbf{MY}_{4r} + K_{1r}K_{3r}\mathbf{MX}_{4r}  \nonumber \\
&+ K_{2r}K_{3r}\mathbf{MY}_{4r} \nonumber \\
 m_{4r}\bar{a}_{G(4r)} &= \left[\begin{matrix} D_{9r} + D_{5r}\ddot{\psi} + D_{7r}\ddot{q}_{w} + D_{6r}\ddot{q}_{imu} + D_{8r}\ddot{q}_{torso} + D_{4r}\ddot{x} - \mathbf{MZ}_{4r}\ddot{q}_{1r} & D_{15r} + D_{11r}\ddot{\psi} + D_{13r}\ddot{q}_{w} + D_{12r}\ddot{q}_{imu} + D_{14r}\ddot{q}_{torso} + D_{10r}\ddot{x} & D_{21r} + D_{17r}\ddot{\psi} + D_{19r}\ddot{q}_{w} + D_{18r}\ddot{q}_{imu} + D_{20r}\ddot{q}_{torso} + D_{16r}\ddot{x} + \mathbf{MX}_{4r}\ddot{q}_{1r} &  \end{matrix}\right] 
 \nonumber \\ 
 \dot{\bar{H}}_{4r} &= \mathbf{MS}_{4r} \times \bar{a}_{4r} + J_{4r}\bar{\alpha}_{4r} + \bar\omega_{4r} \times J_{4r}\bar{\omega}_{4r} 
 \nonumber \\ 
 \dot{\bar{H}}_{4r} &= \left[\begin{matrix} K_{2r}(K_{1r}\mathbf{XZ}_{4r} + K_{2r}\mathbf{YZ}_{4r} + K_{3r}\mathbf{ZZ}_{4r}) - K_{3r}(K_{1r}\mathbf{XY}_{4r} + K_{2r}\mathbf{YY}_{4r} + K_{3r}\mathbf{YZ}_{4r}) - \mathbf{XY}_{4r}(K_{42} + \ddot{q}_{1r} + K_{28}\ddot{\psi} + \ddot{q}_{w}c_{torso} + \ddot{q}_{imu}c_{torso}) + \mathbf{XX}_{4r}(K_{24r} + K_{4r}\ddot{\psi} + K_{5r}\ddot{q}_{w} + K_{5r}\ddot{q}_{imu} - \ddot{q}_{torso}c_{1r}) + \mathbf{XZ}_{4r}(K_{25r} + K_{6r}\ddot{\psi} + K_{7r}\ddot{q}_{w} + K_{7r}\ddot{q}_{imu} + \ddot{q}_{torso}s_{1r}) - \mathbf{MZ}_{4r}(K_{27r} + K_{16r}\ddot{\psi} + K_{18r}\ddot{q}_{w} + K_{17r}\ddot{q}_{imu} - K_{33}\ddot{x}) + \mathbf{MY}_{4r}(K_{28r} + K_{20r}\ddot{\psi} + K_{22r}\ddot{q}_{w} + K_{21r}\ddot{q}_{imu} + K_{23r}\ddot{q}_{torso} + K_{19r}\ddot{x}) & K_{3r}(K_{1r}\mathbf{XX}_{4r} + K_{2r}\mathbf{XY}_{4r} + K_{3r}\mathbf{XZ}_{4r}) - \mathbf{YY}_{4r}(K_{42} + \ddot{q}_{1r} + K_{28}\ddot{\psi} + \ddot{q}_{w}c_{torso} + \ddot{q}_{imu}c_{torso}) - K_{1r}(K_{1r}\mathbf{XZ}_{4r} + K_{2r}\mathbf{YZ}_{4r} + K_{3r}\mathbf{ZZ}_{4r}) + \mathbf{XY}_{4r}(K_{24r} + K_{4r}\ddot{\psi} + K_{5r}\ddot{q}_{w} + K_{5r}\ddot{q}_{imu} - \ddot{q}_{torso}c_{1r}) + \mathbf{YZ}_{4r}(K_{25r} + K_{6r}\ddot{\psi} + K_{7r}\ddot{q}_{w} + K_{7r}\ddot{q}_{imu} + \ddot{q}_{torso}s_{1r}) - \mathbf{MX}_{4r}(K_{28r} + K_{20r}\ddot{\psi} + K_{22r}\ddot{q}_{w} + K_{21r}\ddot{q}_{imu} + K_{23r}\ddot{q}_{torso} + K_{19r}\ddot{x}) + \mathbf{MZ}_{4r}(K_{26r} + K_{12r}\ddot{\psi} + K_{14r}\ddot{q}_{w} + K_{13r}\ddot{q}_{imu} + K_{15r}\ddot{q}_{torso} + K_{11r}\ddot{x}) & K_{1r}(K_{1r}\mathbf{XY}_{4r} + K_{2r}\mathbf{YY}_{4r} + K_{3r}\mathbf{YZ}_{4r}) - K_{2r}(K_{1r}\mathbf{XX}_{4r} + K_{2r}\mathbf{XY}_{4r} + K_{3r}\mathbf{XZ}_{4r}) - \mathbf{YZ}_{4r}(K_{42} + \ddot{q}_{1r} + K_{28}\ddot{\psi} + \ddot{q}_{w}c_{torso} + \ddot{q}_{imu}c_{torso}) + \mathbf{XZ}_{4r}(K_{24r} + K_{4r}\ddot{\psi} + K_{5r}\ddot{q}_{w} + K_{5r}\ddot{q}_{imu} - \ddot{q}_{torso}c_{1r}) + \mathbf{ZZ}_{4r}(K_{25r} + K_{6r}\ddot{\psi} + K_{7r}\ddot{q}_{w} + K_{7r}\ddot{q}_{imu} + \ddot{q}_{torso}s_{1r}) + \mathbf{MX}_{4r}(K_{27r} + K_{16r}\ddot{\psi} + K_{18r}\ddot{q}_{w} + K_{17r}\ddot{q}_{imu} - K_{33}\ddot{x}) - \mathbf{MY}_{4r}(K_{26r} + K_{12r}\ddot{\psi} + K_{14r}\ddot{q}_{w} + K_{13r}\ddot{q}_{imu} + K_{15r}\ddot{q}_{torso} + K_{11r}\ddot{x}) &  \end{matrix}\right] 
 \nonumber \\ 
D_{22r} &= K_{19r}\mathbf{MY}_{4r} + K_{33}\mathbf{MZ}_{4r} \nonumber \\
D_{23r} &= K_{4r}\mathbf{XX}_{4r} - K_{28}\mathbf{XY}_{4r} + K_{6r}\mathbf{XZ}_{4r}  \nonumber \\
&+ K_{20r}\mathbf{MY}_{4r} - K_{16r}\mathbf{MZ}_{4r} \nonumber \\
D_{24r} &= K_{5r}\mathbf{XX}_{4r} + K_{7r}\mathbf{XZ}_{4r} - \mathbf{XY}_{4r}c_{torso}  \nonumber \\
&+ K_{21r}\mathbf{MY}_{4r} - K_{17r}\mathbf{MZ}_{4r} \nonumber \\
D_{25r} &= K_{5r}\mathbf{XX}_{4r} + K_{7r}\mathbf{XZ}_{4r} - \mathbf{XY}_{4r}c_{torso}  \nonumber \\
&+ K_{22r}\mathbf{MY}_{4r} - K_{18r}\mathbf{MZ}_{4r} \nonumber \\
D_{26r} &= \mathbf{XZ}_{4r}s_{1r} - \mathbf{XX}_{4r}c_{1r} + K_{23r}\mathbf{MY}_{4r} \nonumber \\
D_{27r} &= K_{24r}\mathbf{XX}_{4r} - K_{42}\mathbf{XY}_{4r} + K_{25r}\mathbf{XZ}_{4r}  \nonumber \\
&+ K_{2r}^2\mathbf{YZ}_{4r} - K_{3r}^2\mathbf{YZ}_{4r} + K_{28r}\mathbf{MY}_{4r}  \nonumber \\
&- K_{27r}\mathbf{MZ}_{4r} - K_{1r}K_{3r}\mathbf{XY}_{4r} + K_{1r}K_{2r}\mathbf{XZ}_{4r}  \nonumber \\
&- K_{2r}K_{3r}\mathbf{YY}_{4r} + K_{2r}K_{3r}\mathbf{ZZ}_{4r} \nonumber \\
D_{28r} &= K_{11r}\mathbf{MZ}_{4r} - K_{19r}\mathbf{MX}_{4r} \nonumber \\
D_{29r} &= K_{4r}\mathbf{XY}_{4r} - K_{28}\mathbf{YY}_{4r} + K_{6r}\mathbf{YZ}_{4r}  \nonumber \\
&- K_{20r}\mathbf{MX}_{4r} + K_{12r}\mathbf{MZ}_{4r} \nonumber \\
D_{30r} &= K_{5r}\mathbf{XY}_{4r} + K_{7r}\mathbf{YZ}_{4r} - \mathbf{YY}_{4r}c_{torso}  \nonumber \\
&- K_{21r}\mathbf{MX}_{4r} + K_{13r}\mathbf{MZ}_{4r} \nonumber \\
D_{31r} &= K_{5r}\mathbf{XY}_{4r} + K_{7r}\mathbf{YZ}_{4r} - \mathbf{YY}_{4r}c_{torso}  \nonumber \\
&- K_{22r}\mathbf{MX}_{4r} + K_{14r}\mathbf{MZ}_{4r} \nonumber \\
D_{32r} &= \mathbf{YZ}_{4r}s_{1r} - \mathbf{XY}_{4r}c_{1r} - K_{23r}\mathbf{MX}_{4r}  \nonumber \\
&+ K_{15r}\mathbf{MZ}_{4r} \nonumber \\
D_{33r} &= K_{24r}\mathbf{XY}_{4r} - K_{42}\mathbf{YY}_{4r} + K_{25r}\mathbf{YZ}_{4r}  \nonumber \\
&- K_{1r}^2\mathbf{XZ}_{4r} + K_{3r}^2\mathbf{XZ}_{4r} - K_{28r}\mathbf{MX}_{4r}  \nonumber \\
&+ K_{26r}\mathbf{MZ}_{4r} + K_{1r}K_{3r}\mathbf{XX}_{4r} + K_{2r}K_{3r}\mathbf{XY}_{4r}  \nonumber \\
&- K_{1r}K_{2r}\mathbf{YZ}_{4r} - K_{1r}K_{3r}\mathbf{ZZ}_{4r} \nonumber \\
D_{34r} &= - K_{33}\mathbf{MX}_{4r} - K_{11r}\mathbf{MY}_{4r} \nonumber \\
D_{35r} &= K_{4r}\mathbf{XZ}_{4r} - K_{28}\mathbf{YZ}_{4r} + K_{6r}\mathbf{ZZ}_{4r}  \nonumber \\
&+ K_{16r}\mathbf{MX}_{4r} - K_{12r}\mathbf{MY}_{4r} \nonumber \\
D_{36r} &= K_{5r}\mathbf{XZ}_{4r} + K_{7r}\mathbf{ZZ}_{4r} - \mathbf{YZ}_{4r}c_{torso}  \nonumber \\
&+ K_{17r}\mathbf{MX}_{4r} - K_{13r}\mathbf{MY}_{4r} \nonumber \\
D_{37r} &= K_{5r}\mathbf{XZ}_{4r} + K_{7r}\mathbf{ZZ}_{4r} - \mathbf{YZ}_{4r}c_{torso}  \nonumber \\
&+ K_{18r}\mathbf{MX}_{4r} - K_{14r}\mathbf{MY}_{4r} \nonumber \\
D_{38r} &= \mathbf{ZZ}_{4r}s_{1r} - \mathbf{XZ}_{4r}c_{1r} - K_{15r}\mathbf{MY}_{4r} \nonumber \\
D_{39r} &= K_{24r}\mathbf{XZ}_{4r} - K_{42}\mathbf{YZ}_{4r} + K_{25r}\mathbf{ZZ}_{4r}  \nonumber \\
&+ K_{1r}^2\mathbf{XY}_{4r} - K_{2r}^2\mathbf{XY}_{4r} + K_{27r}\mathbf{MX}_{4r}  \nonumber \\
&- K_{26r}\mathbf{MY}_{4r} - K_{1r}K_{2r}\mathbf{XX}_{4r} - K_{2r}K_{3r}\mathbf{XZ}_{4r}  \nonumber \\
&+ K_{1r}K_{2r}\mathbf{YY}_{4r} + K_{1r}K_{3r}\mathbf{YZ}_{4r} \nonumber \\
 \dot{\bar{H}}_{4r} &= \left[\begin{matrix} D_{9r} + D_{5r}\ddot{\psi} + D_{7r}\ddot{q}_{w} + D_{6r}\ddot{q}_{imu} + D_{8r}\ddot{q}_{torso} + D_{4r}\ddot{x} - \mathbf{MZ}_{4r}\ddot{q}_{1r} & D_{15r} + D_{11r}\ddot{\psi} + D_{13r}\ddot{q}_{w} + D_{12r}\ddot{q}_{imu} + D_{14r}\ddot{q}_{torso} + D_{10r}\ddot{x} & D_{21r} + D_{17r}\ddot{\psi} + D_{19r}\ddot{q}_{w} + D_{18r}\ddot{q}_{imu} + D_{20r}\ddot{q}_{torso} + D_{16r}\ddot{x} + \mathbf{MX}_{4r}\ddot{q}_{1r} &  \end{matrix}\right] 
 \nonumber \\ 
 \bar\omega_{5r} &= {}^{5r}A_{4r} \bar\omega_{4r} + \dot{q}_{5r} \bar{e}_{5r} 
 \nonumber \\ 
 \bar\omega_{5r} &= \left[\begin{matrix} - K_{1r} - \dot{q}_{2r} & - K_{2r}c_{2r} - K_{3r}s_{2r} & K_{3r}c_{2r} - K_{2r}s_{2r} &  \end{matrix}\right] 
 \nonumber \\ 
K_{31r} &= - K_{1r} - \dot{q}_{2r} \nonumber \\
K_{32r} &= - K_{2r}c_{2r} - K_{3r}s_{2r} \nonumber \\
K_{33r} &= K_{3r}c_{2r} - K_{2r}s_{2r} \nonumber \\
 \bar\omega_{5r} &= \left[\begin{matrix} K_{31r} & K_{32r} & K_{33r} &  \end{matrix}\right] 
 \nonumber \\ 
 \bar\omega_{5r} &= \left[\begin{matrix} \dot{q}_{torso}c_{1r} - K_{4r}\dot{\psi} - K_{5r}\dot{q}_{w} - K_{5r}\dot{q}_{imu} - \dot{q}_{2r} & c_{2r}(\dot{q}_{1r} + K_{28}\dot{\psi} + \dot{q}_{w}c_{torso} + \dot{q}_{imu}c_{torso}) - s_{2r}(K_{6r}\dot{\psi} + K_{7r}\dot{q}_{w} + K_{7r}\dot{q}_{imu} + \dot{q}_{torso}s_{1r}) & s_{2r}(\dot{q}_{1r} + K_{28}\dot{\psi} + \dot{q}_{w}c_{torso} + \dot{q}_{imu}c_{torso}) + c_{2r}(K_{6r}\dot{\psi} + K_{7r}\dot{q}_{w} + K_{7r}\dot{q}_{imu} + \dot{q}_{torso}s_{1r}) &  \end{matrix}\right] 
 \nonumber \\ 
K_{34r} &= K_{28}c_{2r} - K_{6r}s_{2r} \nonumber \\
K_{35r} &= c_{2r}c_{torso} - K_{7r}s_{2r} \nonumber \\
K_{36r} &= -s_{1r}s_{2r} \nonumber \\
K_{37r} &= K_{6r}c_{2r} + K_{28}s_{2r} \nonumber \\
K_{38r} &= c_{torso}s_{2r} + K_{7r}c_{2r} \nonumber \\
K_{39r} &= c_{2r}s_{1r} \nonumber \\
 \bar\omega_{5r} &= \left[\begin{matrix} \dot{q}_{torso}c_{1r} - K_{4r}\dot{\psi} - K_{5r}\dot{q}_{w} - K_{5r}\dot{q}_{imu} - \dot{q}_{2r} & K_{34r}\dot{\psi} + K_{35r}\dot{q}_{w} + K_{35r}\dot{q}_{imu} + K_{36r}\dot{q}_{torso} + \dot{q}_{1r}c_{2r} & K_{37r}\dot{\psi} + K_{38r}\dot{q}_{w} + K_{38r}\dot{q}_{imu} + K_{39r}\dot{q}_{torso} + \dot{q}_{1r}s_{2r} &  \end{matrix}\right] 
 \nonumber \\ 
 \bar{v}_{5r} &= {}^{5r}A_{4r} \left(\bar{v}_{4r} + \bar\omega_{4r} \times \bar{P}_{5r}\right) 
 \nonumber \\ 
 \bar{v}_{5r} &= \left[\begin{matrix} -K_{8r} & - K_{9r}c_{2r} - K_{10r}s_{2r} & K_{10r}c_{2r} - K_{9r}s_{2r} &  \end{matrix}\right] 
 \nonumber \\ 
K_{40r} &= - K_{9r}c_{2r} - K_{10r}s_{2r} \nonumber \\
K_{41r} &= K_{10r}c_{2r} - K_{9r}s_{2r} \nonumber \\
 \bar{v}_{5r} &= \left[\begin{matrix} -K_{8r} & K_{40r} & K_{41r} &  \end{matrix}\right] 
 \nonumber \\ 
 \bar{v}_{5r} &= \left[\begin{matrix} - K_{12r}\dot{\psi} - K_{14r}\dot{q}_{w} - K_{13r}\dot{q}_{imu} - K_{15r}\dot{q}_{torso} - K_{11r}\dot{x} & - c_{2r}(K_{16r}\dot{\psi} + K_{18r}\dot{q}_{w} + K_{17r}\dot{q}_{imu} - K_{33}\dot{x}) - s_{2r}(K_{20r}\dot{\psi} + K_{22r}\dot{q}_{w} + K_{21r}\dot{q}_{imu} + K_{23r}\dot{q}_{torso} + K_{19r}\dot{x}) & c_{2r}(K_{20r}\dot{\psi} + K_{22r}\dot{q}_{w} + K_{21r}\dot{q}_{imu} + K_{23r}\dot{q}_{torso} + K_{19r}\dot{x}) - s_{2r}(K_{16r}\dot{\psi} + K_{18r}\dot{q}_{w} + K_{17r}\dot{q}_{imu} - K_{33}\dot{x}) &  \end{matrix}\right] 
 \nonumber \\ 
K_{42r} &= K_{33}c_{2r} - K_{19r}s_{2r} \nonumber \\
K_{43r} &= - K_{16r}c_{2r} - K_{20r}s_{2r} \nonumber \\
K_{44r} &= - K_{17r}c_{2r} - K_{21r}s_{2r} \nonumber \\
K_{45r} &= - K_{18r}c_{2r} - K_{22r}s_{2r} \nonumber \\
K_{46r} &= -K_{23r}s_{2r} \nonumber \\
K_{47r} &= K_{19r}c_{2r} + K_{33}s_{2r} \nonumber \\
K_{48r} &= K_{20r}c_{2r} - K_{16r}s_{2r} \nonumber \\
K_{49r} &= K_{21r}c_{2r} - K_{17r}s_{2r} \nonumber \\
K_{50r} &= K_{22r}c_{2r} - K_{18r}s_{2r} \nonumber \\
K_{51r} &= K_{23r}c_{2r} \nonumber \\
 \bar{v}_{5r} &= \left[\begin{matrix} - K_{12r}\dot{\psi} - K_{14r}\dot{q}_{w} - K_{13r}\dot{q}_{imu} - K_{15r}\dot{q}_{torso} - K_{11r}\dot{x} & K_{43r}\dot{\psi} + K_{45r}\dot{q}_{w} + K_{44r}\dot{q}_{imu} + K_{46r}\dot{q}_{torso} + K_{42r}\dot{x} & K_{48r}\dot{\psi} + K_{50r}\dot{q}_{w} + K_{49r}\dot{q}_{imu} + K_{51r}\dot{q}_{torso} + K_{47r}\dot{x} &  \end{matrix}\right] 
 \nonumber \\ 
 \bar\alpha_{5r} &= {}^{5r}A_{4r} \bar\alpha_{4r} + \ddot{q}_{5r} \bar{e}_{5r} + \dot{q}_{5r} \left(\bar\omega_{5r} \times \bar{e}_{5r}\right) 
 \nonumber \\ 
 \bar\alpha_{5r} &= \left[\begin{matrix} \ddot{q}_{torso}c_{1r} - \ddot{q}_{2r} - K_{4r}\ddot{\psi} - K_{5r}\ddot{q}_{w} - K_{5r}\ddot{q}_{imu} - K_{24r} & c_{2r}(K_{42} + \ddot{q}_{1r} + K_{28}\ddot{\psi} + \ddot{q}_{w}c_{torso} + \ddot{q}_{imu}c_{torso}) - K_{33r}\dot{q}_{2r} - s_{2r}(K_{25r} + K_{6r}\ddot{\psi} + K_{7r}\ddot{q}_{w} + K_{7r}\ddot{q}_{imu} + \ddot{q}_{torso}s_{1r}) & K_{32r}\dot{q}_{2r} + s_{2r}(K_{42} + \ddot{q}_{1r} + K_{28}\ddot{\psi} + \ddot{q}_{w}c_{torso} + \ddot{q}_{imu}c_{torso}) + c_{2r}(K_{25r} + K_{6r}\ddot{\psi} + K_{7r}\ddot{q}_{w} + K_{7r}\ddot{q}_{imu} + \ddot{q}_{torso}s_{1r}) &  \end{matrix}\right] 
 \nonumber \\ 
K_{52r} &= K_{42}c_{2r} - K_{33r}\dot{q}_{2r} - K_{25r}s_{2r} \nonumber \\
K_{53r} &= K_{32r}\dot{q}_{2r} + K_{25r}c_{2r} + K_{42}s_{2r} \nonumber \\
 \bar\alpha_{5r} &= \left[\begin{matrix} \ddot{q}_{torso}c_{1r} - \ddot{q}_{2r} - K_{4r}\ddot{\psi} - K_{5r}\ddot{q}_{w} - K_{5r}\ddot{q}_{imu} - K_{24r} & K_{52r} + K_{34r}\ddot{\psi} + K_{35r}\ddot{q}_{w} + K_{35r}\ddot{q}_{imu} + K_{36r}\ddot{q}_{torso} + \ddot{q}_{1r}c_{2r} & K_{53r} + K_{37r}\ddot{\psi} + K_{38r}\ddot{q}_{w} + K_{38r}\ddot{q}_{imu} + K_{39r}\ddot{q}_{torso} + \ddot{q}_{1r}s_{2r} &  \end{matrix}\right] 
 \nonumber \\ 
 \bar{a}_{5r} &= {}^{5r}A_{4r} \left(\bar{a}_{4r} + \bar\alpha_{4r} \times \bar{P}_{5r} + \bar\omega_{4r} \times \left(\bar\omega_{4r} \times \bar{P}_{5r}\right)\right) 
 \nonumber \\ 
 \bar\alpha_{5r} &= \left[\begin{matrix} - K_{26r} - K_{12r}\ddot{\psi} - K_{14r}\ddot{q}_{w} - K_{13r}\ddot{q}_{imu} - K_{15r}\ddot{q}_{torso} - K_{11r}\ddot{x} & - c_{2r}(K_{27r} + K_{16r}\ddot{\psi} + K_{18r}\ddot{q}_{w} + K_{17r}\ddot{q}_{imu} - K_{33}\ddot{x}) - s_{2r}(K_{28r} + K_{20r}\ddot{\psi} + K_{22r}\ddot{q}_{w} + K_{21r}\ddot{q}_{imu} + K_{23r}\ddot{q}_{torso} + K_{19r}\ddot{x}) & c_{2r}(K_{28r} + K_{20r}\ddot{\psi} + K_{22r}\ddot{q}_{w} + K_{21r}\ddot{q}_{imu} + K_{23r}\ddot{q}_{torso} + K_{19r}\ddot{x}) - s_{2r}(K_{27r} + K_{16r}\ddot{\psi} + K_{18r}\ddot{q}_{w} + K_{17r}\ddot{q}_{imu} - K_{33}\ddot{x}) &  \end{matrix}\right] 
 \nonumber \\ 
K_{54r} &= - K_{27r}c_{2r} - K_{28r}s_{2r} \nonumber \\
K_{55r} &= K_{28r}c_{2r} - K_{27r}s_{2r} \nonumber \\
 \bar{a}_{5r} &= \left[\begin{matrix} - K_{26r} - K_{12r}\ddot{\psi} - K_{14r}\ddot{q}_{w} - K_{13r}\ddot{q}_{imu} - K_{15r}\ddot{q}_{torso} - K_{11r}\ddot{x} & K_{54r} + K_{43r}\ddot{\psi} + K_{45r}\ddot{q}_{w} + K_{44r}\ddot{q}_{imu} + K_{46r}\ddot{q}_{torso} + K_{42r}\ddot{x} & K_{55r} + K_{48r}\ddot{\psi} + K_{50r}\ddot{q}_{w} + K_{49r}\ddot{q}_{imu} + K_{51r}\ddot{q}_{torso} + K_{47r}\ddot{x} &  \end{matrix}\right] 
 \nonumber \\ 
 \bar{g}_{5r} &= {}^{5r}A_{4r} \bar{g}_{4r} 
 \nonumber \\ 
 \bar{g}_{5r} &= \left[\begin{matrix} -K_{29r}g & K_{47}gc_{2r} - K_{30r}gs_{2r} & K_{30r}gc_{2r} + K_{47}gs_{2r} &  \end{matrix}\right] 
 \nonumber \\ 
K_{56r} &= K_{47}c_{2r} - K_{30r}s_{2r} \nonumber \\
K_{57r} &= K_{30r}c_{2r} + K_{47}s_{2r} \nonumber \\
 \bar{g}_{5r} &= \left[\begin{matrix} -K_{29r}g & K_{56r}g & K_{57r}g &  \end{matrix}\right] 
 \nonumber \\ 
 m_{5r}\bar{S}_{5r}^{\times}\bar{g}_{5r} &= \mathbf{MS}_{5r} \times \bar{g}_{5r} 
 \nonumber \\ 
 m_{5r}\bar{S}_{5r}^{\times}\bar{g}_{5r} &= \left[\begin{matrix} K_{57r}\mathbf{MY}_{5r}g - K_{56r}\mathbf{MZ}_{5r}g & - K_{57r}\mathbf{MX}_{5r}g - K_{29r}\mathbf{MZ}_{5r}g & K_{56r}\mathbf{MX}_{5r}g + K_{29r}\mathbf{MY}_{5r}g &  \end{matrix}\right] 
 \nonumber \\ 
D_{40r} &= K_{57r}\mathbf{MY}_{5r} - K_{56r}\mathbf{MZ}_{5r} \nonumber \\
D_{41r} &= - K_{57r}\mathbf{MX}_{5r} - K_{29r}\mathbf{MZ}_{5r} \nonumber \\
D_{42r} &= K_{56r}\mathbf{MX}_{5r} + K_{29r}\mathbf{MY}_{5r} \nonumber \\
 m_{5r}\bar{S}_{5r}^{\times}\bar{g}_{5r} &= \left[\begin{matrix} D_{40r}g & D_{41r}g & D_{42r}g &  \end{matrix}\right] 
 \nonumber \\ 
 m_{5r}\bar{a}_{G(5r)} &= m_{5r}\bar{a}_{5r} + \bar\alpha_{5r} \times \mathbf{MS}_{5r} + \bar\omega_{5r} \times \left(\bar\omega_{5r} \times \mathbf{MS}_{5r}\right) 
 \nonumber \\ 
 m_{5r}\bar{a}_{G(5r)} &= \left[\begin{matrix} \mathbf{MZ}_{5r}(K_{52r} + K_{34r}\ddot{\psi} + K_{35r}\ddot{q}_{w} + K_{35r}\ddot{q}_{imu} + K_{36r}\ddot{q}_{torso} + \ddot{q}_{1r}c_{2r}) - \mathbf{MY}_{5r}(K_{53r} + K_{37r}\ddot{\psi} + K_{38r}\ddot{q}_{w} + K_{38r}\ddot{q}_{imu} + K_{39r}\ddot{q}_{torso} + \ddot{q}_{1r}s_{2r}) - m_{5r}(K_{26r} + K_{12r}\ddot{\psi} + K_{14r}\ddot{q}_{w} + K_{13r}\ddot{q}_{imu} + K_{15r}\ddot{q}_{torso} + K_{11r}\ddot{x}) - K_{32r}(K_{32r}\mathbf{MX}_{5r} - K_{31r}\mathbf{MY}_{5r}) - K_{33r}(K_{33r}\mathbf{MX}_{5r} - K_{31r}\mathbf{MZ}_{5r}) & \mathbf{MX}_{5r}(K_{53r} + K_{37r}\ddot{\psi} + K_{38r}\ddot{q}_{w} + K_{38r}\ddot{q}_{imu} + K_{39r}\ddot{q}_{torso} + \ddot{q}_{1r}s_{2r}) + \mathbf{MZ}_{5r}(K_{24r} + \ddot{q}_{2r} + K_{4r}\ddot{\psi} + K_{5r}\ddot{q}_{w} + K_{5r}\ddot{q}_{imu} - \ddot{q}_{torso}c_{1r}) + m_{5r}(K_{54r} + K_{43r}\ddot{\psi} + K_{45r}\ddot{q}_{w} + K_{44r}\ddot{q}_{imu} + K_{46r}\ddot{q}_{torso} + K_{42r}\ddot{x}) + K_{31r}(K_{32r}\mathbf{MX}_{5r} - K_{31r}\mathbf{MY}_{5r}) - K_{33r}(K_{33r}\mathbf{MY}_{5r} - K_{32r}\mathbf{MZ}_{5r}) & m_{5r}(K_{55r} + K_{48r}\ddot{\psi} + K_{50r}\ddot{q}_{w} + K_{49r}\ddot{q}_{imu} + K_{51r}\ddot{q}_{torso} + K_{47r}\ddot{x}) - \mathbf{MY}_{5r}(K_{24r} + \ddot{q}_{2r} + K_{4r}\ddot{\psi} + K_{5r}\ddot{q}_{w} + K_{5r}\ddot{q}_{imu} - \ddot{q}_{torso}c_{1r}) - \mathbf{MX}_{5r}(K_{52r} + K_{34r}\ddot{\psi} + K_{35r}\ddot{q}_{w} + K_{35r}\ddot{q}_{imu} + K_{36r}\ddot{q}_{torso} + \ddot{q}_{1r}c_{2r}) + K_{31r}(K_{33r}\mathbf{MX}_{5r} - K_{31r}\mathbf{MZ}_{5r}) + K_{32r}(K_{33r}\mathbf{MY}_{5r} - K_{32r}\mathbf{MZ}_{5r}) &  \end{matrix}\right] 
 \nonumber \\ 
D_{43r} &= -K_{11r}m_{5r} \nonumber \\
D_{44r} &= K_{34r}\mathbf{MZ}_{5r} - K_{37r}\mathbf{MY}_{5r} - K_{12r}m_{5r} \nonumber \\
D_{45r} &= K_{35r}\mathbf{MZ}_{5r} - K_{38r}\mathbf{MY}_{5r} - K_{13r}m_{5r} \nonumber \\
D_{46r} &= K_{35r}\mathbf{MZ}_{5r} - K_{38r}\mathbf{MY}_{5r} - K_{14r}m_{5r} \nonumber \\
D_{47r} &= K_{36r}\mathbf{MZ}_{5r} - K_{39r}\mathbf{MY}_{5r} - K_{15r}m_{5r} \nonumber \\
D_{48r} &= \mathbf{MZ}_{5r}c_{2r} - \mathbf{MY}_{5r}s_{2r} \nonumber \\
D_{49r} &= K_{52r}\mathbf{MZ}_{5r} - K_{32r}^2\mathbf{MX}_{5r} - K_{33r}^2\mathbf{MX}_{5r}  \nonumber \\
&- K_{53r}\mathbf{MY}_{5r} - K_{26r}m_{5r} + K_{31r}K_{32r}\mathbf{MY}_{5r}  \nonumber \\
&+ K_{31r}K_{33r}\mathbf{MZ}_{5r} \nonumber \\
D_{50r} &= K_{42r}m_{5r} \nonumber \\
D_{51r} &= K_{43r}m_{5r} + K_{37r}\mathbf{MX}_{5r} + K_{4r}\mathbf{MZ}_{5r} \nonumber \\
D_{52r} &= K_{44r}m_{5r} + K_{38r}\mathbf{MX}_{5r} + K_{5r}\mathbf{MZ}_{5r} \nonumber \\
D_{53r} &= K_{45r}m_{5r} + K_{38r}\mathbf{MX}_{5r} + K_{5r}\mathbf{MZ}_{5r} \nonumber \\
D_{54r} &= K_{46r}m_{5r} - \mathbf{MZ}_{5r}c_{1r} + K_{39r}\mathbf{MX}_{5r} \nonumber \\
D_{55r} &= \mathbf{MX}_{5r}s_{2r} \nonumber \\
D_{56r} &= K_{54r}m_{5r} - K_{31r}^2\mathbf{MY}_{5r} - K_{33r}^2\mathbf{MY}_{5r}  \nonumber \\
&+ K_{53r}\mathbf{MX}_{5r} + K_{24r}\mathbf{MZ}_{5r} + K_{31r}K_{32r}\mathbf{MX}_{5r}  \nonumber \\
&+ K_{32r}K_{33r}\mathbf{MZ}_{5r} \nonumber \\
D_{57r} &= K_{47r}m_{5r} \nonumber \\
D_{58r} &= K_{48r}m_{5r} - K_{34r}\mathbf{MX}_{5r} - K_{4r}\mathbf{MY}_{5r} \nonumber \\
D_{59r} &= K_{49r}m_{5r} - K_{35r}\mathbf{MX}_{5r} - K_{5r}\mathbf{MY}_{5r} \nonumber \\
D_{60r} &= K_{50r}m_{5r} - K_{35r}\mathbf{MX}_{5r} - K_{5r}\mathbf{MY}_{5r} \nonumber \\
D_{61r} &= K_{51r}m_{5r} + \mathbf{MY}_{5r}c_{1r} - K_{36r}\mathbf{MX}_{5r} \nonumber \\
D_{62r} &= -\mathbf{MX}_{5r}c_{2r} \nonumber \\
D_{63r} &= K_{55r}m_{5r} - K_{31r}^2\mathbf{MZ}_{5r} - K_{32r}^2\mathbf{MZ}_{5r}  \nonumber \\
&- K_{52r}\mathbf{MX}_{5r} - K_{24r}\mathbf{MY}_{5r} + K_{31r}K_{33r}\mathbf{MX}_{5r}  \nonumber \\
&+ K_{32r}K_{33r}\mathbf{MY}_{5r} \nonumber \\
 m_{5r}\bar{a}_{G(5r)} &= \left[\begin{matrix} D_{49r} + D_{44r}\ddot{\psi} + D_{48r}\ddot{q}_{1r} + D_{46r}\ddot{q}_{w} + D_{45r}\ddot{q}_{imu} + D_{47r}\ddot{q}_{torso} + D_{43r}\ddot{x} & D_{56r} + D_{51r}\ddot{\psi} + D_{55r}\ddot{q}_{1r} + D_{53r}\ddot{q}_{w} + D_{52r}\ddot{q}_{imu} + D_{54r}\ddot{q}_{torso} + D_{50r}\ddot{x} + \mathbf{MZ}_{5r}\ddot{q}_{2r} & D_{63r} + D_{58r}\ddot{\psi} + D_{62r}\ddot{q}_{1r} + D_{60r}\ddot{q}_{w} + D_{59r}\ddot{q}_{imu} + D_{61r}\ddot{q}_{torso} + D_{57r}\ddot{x} - \mathbf{MY}_{5r}\ddot{q}_{2r} &  \end{matrix}\right] 
 \nonumber \\ 
 \dot{\bar{H}}_{5r} &= \mathbf{MS}_{5r} \times \bar{a}_{5r} + J_{5r}\bar{\alpha}_{5r} + \bar\omega_{5r} \times J_{5r}\bar{\omega}_{5r} 
 \nonumber \\ 
 \dot{\bar{H}}_{5r} &= \left[\begin{matrix} K_{32r}(K_{31r}\mathbf{XZ}_{5r} + K_{32r}\mathbf{YZ}_{5r} + K_{33r}\mathbf{ZZ}_{5r}) - K_{33r}(K_{31r}\mathbf{XY}_{5r} + K_{32r}\mathbf{YY}_{5r} + K_{33r}\mathbf{YZ}_{5r}) + \mathbf{XY}_{5r}(K_{52r} + K_{34r}\ddot{\psi} + K_{35r}\ddot{q}_{w} + K_{35r}\ddot{q}_{imu} + K_{36r}\ddot{q}_{torso} + \ddot{q}_{1r}c_{2r}) + \mathbf{XZ}_{5r}(K_{53r} + K_{37r}\ddot{\psi} + K_{38r}\ddot{q}_{w} + K_{38r}\ddot{q}_{imu} + K_{39r}\ddot{q}_{torso} + \ddot{q}_{1r}s_{2r}) - \mathbf{XX}_{5r}(K_{24r} + \ddot{q}_{2r} + K_{4r}\ddot{\psi} + K_{5r}\ddot{q}_{w} + K_{5r}\ddot{q}_{imu} - \ddot{q}_{torso}c_{1r}) + \mathbf{MY}_{5r}(K_{55r} + K_{48r}\ddot{\psi} + K_{50r}\ddot{q}_{w} + K_{49r}\ddot{q}_{imu} + K_{51r}\ddot{q}_{torso} + K_{47r}\ddot{x}) - \mathbf{MZ}_{5r}(K_{54r} + K_{43r}\ddot{\psi} + K_{45r}\ddot{q}_{w} + K_{44r}\ddot{q}_{imu} + K_{46r}\ddot{q}_{torso} + K_{42r}\ddot{x}) & K_{33r}(K_{31r}\mathbf{XX}_{5r} + K_{32r}\mathbf{XY}_{5r} + K_{33r}\mathbf{XZ}_{5r}) - K_{31r}(K_{31r}\mathbf{XZ}_{5r} + K_{32r}\mathbf{YZ}_{5r} + K_{33r}\mathbf{ZZ}_{5r}) + \mathbf{YY}_{5r}(K_{52r} + K_{34r}\ddot{\psi} + K_{35r}\ddot{q}_{w} + K_{35r}\ddot{q}_{imu} + K_{36r}\ddot{q}_{torso} + \ddot{q}_{1r}c_{2r}) + \mathbf{YZ}_{5r}(K_{53r} + K_{37r}\ddot{\psi} + K_{38r}\ddot{q}_{w} + K_{38r}\ddot{q}_{imu} + K_{39r}\ddot{q}_{torso} + \ddot{q}_{1r}s_{2r}) - \mathbf{XY}_{5r}(K_{24r} + \ddot{q}_{2r} + K_{4r}\ddot{\psi} + K_{5r}\ddot{q}_{w} + K_{5r}\ddot{q}_{imu} - \ddot{q}_{torso}c_{1r}) - \mathbf{MX}_{5r}(K_{55r} + K_{48r}\ddot{\psi} + K_{50r}\ddot{q}_{w} + K_{49r}\ddot{q}_{imu} + K_{51r}\ddot{q}_{torso} + K_{47r}\ddot{x}) - \mathbf{MZ}_{5r}(K_{26r} + K_{12r}\ddot{\psi} + K_{14r}\ddot{q}_{w} + K_{13r}\ddot{q}_{imu} + K_{15r}\ddot{q}_{torso} + K_{11r}\ddot{x}) & K_{31r}(K_{31r}\mathbf{XY}_{5r} + K_{32r}\mathbf{YY}_{5r} + K_{33r}\mathbf{YZ}_{5r}) - K_{32r}(K_{31r}\mathbf{XX}_{5r} + K_{32r}\mathbf{XY}_{5r} + K_{33r}\mathbf{XZ}_{5r}) + \mathbf{YZ}_{5r}(K_{52r} + K_{34r}\ddot{\psi} + K_{35r}\ddot{q}_{w} + K_{35r}\ddot{q}_{imu} + K_{36r}\ddot{q}_{torso} + \ddot{q}_{1r}c_{2r}) + \mathbf{ZZ}_{5r}(K_{53r} + K_{37r}\ddot{\psi} + K_{38r}\ddot{q}_{w} + K_{38r}\ddot{q}_{imu} + K_{39r}\ddot{q}_{torso} + \ddot{q}_{1r}s_{2r}) - \mathbf{XZ}_{5r}(K_{24r} + \ddot{q}_{2r} + K_{4r}\ddot{\psi} + K_{5r}\ddot{q}_{w} + K_{5r}\ddot{q}_{imu} - \ddot{q}_{torso}c_{1r}) + \mathbf{MX}_{5r}(K_{54r} + K_{43r}\ddot{\psi} + K_{45r}\ddot{q}_{w} + K_{44r}\ddot{q}_{imu} + K_{46r}\ddot{q}_{torso} + K_{42r}\ddot{x}) + \mathbf{MY}_{5r}(K_{26r} + K_{12r}\ddot{\psi} + K_{14r}\ddot{q}_{w} + K_{13r}\ddot{q}_{imu} + K_{15r}\ddot{q}_{torso} + K_{11r}\ddot{x}) &  \end{matrix}\right] 
 \nonumber \\ 
D_{64r} &= K_{47r}\mathbf{MY}_{5r} - K_{42r}\mathbf{MZ}_{5r} \nonumber \\
D_{65r} &= K_{34r}\mathbf{XY}_{5r} - K_{4r}\mathbf{XX}_{5r} + K_{37r}\mathbf{XZ}_{5r}  \nonumber \\
&+ K_{48r}\mathbf{MY}_{5r} - K_{43r}\mathbf{MZ}_{5r} \nonumber \\
D_{66r} &= K_{35r}\mathbf{XY}_{5r} - K_{5r}\mathbf{XX}_{5r} + K_{38r}\mathbf{XZ}_{5r}  \nonumber \\
&+ K_{49r}\mathbf{MY}_{5r} - K_{44r}\mathbf{MZ}_{5r} \nonumber \\
D_{67r} &= K_{35r}\mathbf{XY}_{5r} - K_{5r}\mathbf{XX}_{5r} + K_{38r}\mathbf{XZ}_{5r}  \nonumber \\
&+ K_{50r}\mathbf{MY}_{5r} - K_{45r}\mathbf{MZ}_{5r} \nonumber \\
D_{68r} &= K_{36r}\mathbf{XY}_{5r} + K_{39r}\mathbf{XZ}_{5r} + \mathbf{XX}_{5r}c_{1r}  \nonumber \\
&+ K_{51r}\mathbf{MY}_{5r} - K_{46r}\mathbf{MZ}_{5r} \nonumber \\
D_{69r} &= \mathbf{XY}_{5r}c_{2r} + \mathbf{XZ}_{5r}s_{2r} \nonumber \\
D_{70r} &= K_{52r}\mathbf{XY}_{5r} - K_{24r}\mathbf{XX}_{5r} + K_{53r}\mathbf{XZ}_{5r}  \nonumber \\
&+ K_{32r}^2\mathbf{YZ}_{5r} - K_{33r}^2\mathbf{YZ}_{5r} + K_{55r}\mathbf{MY}_{5r}  \nonumber \\
&- K_{54r}\mathbf{MZ}_{5r} - K_{31r}K_{33r}\mathbf{XY}_{5r} + K_{31r}K_{32r}\mathbf{XZ}_{5r}  \nonumber \\
&- K_{32r}K_{33r}\mathbf{YY}_{5r} + K_{32r}K_{33r}\mathbf{ZZ}_{5r} \nonumber \\
D_{71r} &= - K_{47r}\mathbf{MX}_{5r} - K_{11r}\mathbf{MZ}_{5r} \nonumber \\
D_{72r} &= K_{34r}\mathbf{YY}_{5r} - K_{4r}\mathbf{XY}_{5r} + K_{37r}\mathbf{YZ}_{5r}  \nonumber \\
&- K_{48r}\mathbf{MX}_{5r} - K_{12r}\mathbf{MZ}_{5r} \nonumber \\
D_{73r} &= K_{35r}\mathbf{YY}_{5r} - K_{5r}\mathbf{XY}_{5r} + K_{38r}\mathbf{YZ}_{5r}  \nonumber \\
&- K_{49r}\mathbf{MX}_{5r} - K_{13r}\mathbf{MZ}_{5r} \nonumber \\
D_{74r} &= K_{35r}\mathbf{YY}_{5r} - K_{5r}\mathbf{XY}_{5r} + K_{38r}\mathbf{YZ}_{5r}  \nonumber \\
&- K_{50r}\mathbf{MX}_{5r} - K_{14r}\mathbf{MZ}_{5r} \nonumber \\
D_{75r} &= K_{36r}\mathbf{YY}_{5r} + K_{39r}\mathbf{YZ}_{5r} + \mathbf{XY}_{5r}c_{1r}  \nonumber \\
&- K_{51r}\mathbf{MX}_{5r} - K_{15r}\mathbf{MZ}_{5r} \nonumber \\
D_{76r} &= \mathbf{YY}_{5r}c_{2r} + \mathbf{YZ}_{5r}s_{2r} \nonumber \\
D_{77r} &= K_{52r}\mathbf{YY}_{5r} - K_{24r}\mathbf{XY}_{5r} + K_{53r}\mathbf{YZ}_{5r}  \nonumber \\
&- K_{31r}^2\mathbf{XZ}_{5r} + K_{33r}^2\mathbf{XZ}_{5r} - K_{55r}\mathbf{MX}_{5r}  \nonumber \\
&- K_{26r}\mathbf{MZ}_{5r} + K_{31r}K_{33r}\mathbf{XX}_{5r} + K_{32r}K_{33r}\mathbf{XY}_{5r}  \nonumber \\
&- K_{31r}K_{32r}\mathbf{YZ}_{5r} - K_{31r}K_{33r}\mathbf{ZZ}_{5r} \nonumber \\
D_{78r} &= K_{42r}\mathbf{MX}_{5r} + K_{11r}\mathbf{MY}_{5r} \nonumber \\
D_{79r} &= K_{34r}\mathbf{YZ}_{5r} - K_{4r}\mathbf{XZ}_{5r} + K_{37r}\mathbf{ZZ}_{5r}  \nonumber \\
&+ K_{43r}\mathbf{MX}_{5r} + K_{12r}\mathbf{MY}_{5r} \nonumber \\
D_{80r} &= K_{35r}\mathbf{YZ}_{5r} - K_{5r}\mathbf{XZ}_{5r} + K_{38r}\mathbf{ZZ}_{5r}  \nonumber \\
&+ K_{44r}\mathbf{MX}_{5r} + K_{13r}\mathbf{MY}_{5r} \nonumber \\
D_{81r} &= K_{35r}\mathbf{YZ}_{5r} - K_{5r}\mathbf{XZ}_{5r} + K_{38r}\mathbf{ZZ}_{5r}  \nonumber \\
&+ K_{45r}\mathbf{MX}_{5r} + K_{14r}\mathbf{MY}_{5r} \nonumber \\
D_{82r} &= K_{36r}\mathbf{YZ}_{5r} + K_{39r}\mathbf{ZZ}_{5r} + \mathbf{XZ}_{5r}c_{1r}  \nonumber \\
&+ K_{46r}\mathbf{MX}_{5r} + K_{15r}\mathbf{MY}_{5r} \nonumber \\
D_{83r} &= \mathbf{YZ}_{5r}c_{2r} + \mathbf{ZZ}_{5r}s_{2r} \nonumber \\
D_{84r} &= K_{52r}\mathbf{YZ}_{5r} - K_{24r}\mathbf{XZ}_{5r} + K_{53r}\mathbf{ZZ}_{5r}  \nonumber \\
&+ K_{31r}^2\mathbf{XY}_{5r} - K_{32r}^2\mathbf{XY}_{5r} + K_{54r}\mathbf{MX}_{5r}  \nonumber \\
&+ K_{26r}\mathbf{MY}_{5r} - K_{31r}K_{32r}\mathbf{XX}_{5r} - K_{32r}K_{33r}\mathbf{XZ}_{5r}  \nonumber \\
&+ K_{31r}K_{32r}\mathbf{YY}_{5r} + K_{31r}K_{33r}\mathbf{YZ}_{5r} \nonumber \\
 \dot{\bar{H}}_{5r} &= \left[\begin{matrix} D_{49r} + D_{44r}\ddot{\psi} + D_{48r}\ddot{q}_{1r} + D_{46r}\ddot{q}_{w} + D_{45r}\ddot{q}_{imu} + D_{47r}\ddot{q}_{torso} + D_{43r}\ddot{x} & D_{56r} + D_{51r}\ddot{\psi} + D_{55r}\ddot{q}_{1r} + D_{53r}\ddot{q}_{w} + D_{52r}\ddot{q}_{imu} + D_{54r}\ddot{q}_{torso} + D_{50r}\ddot{x} + \mathbf{MZ}_{5r}\ddot{q}_{2r} & D_{63r} + D_{58r}\ddot{\psi} + D_{62r}\ddot{q}_{1r} + D_{60r}\ddot{q}_{w} + D_{59r}\ddot{q}_{imu} + D_{61r}\ddot{q}_{torso} + D_{57r}\ddot{x} - \mathbf{MY}_{5r}\ddot{q}_{2r} &  \end{matrix}\right] 
 \nonumber \\ 
 \bar\omega_{6r} &= {}^{6r}A_{5r} \bar\omega_{5r} + \dot{q}_{6r} \bar{e}_{6r} 
 \nonumber \\ 
 \bar\omega_{6r} &= \left[\begin{matrix} K_{33r}s_{3r} - K_{31r}c_{3r} & - K_{32r} - \dot{q}_{3r} & K_{33r}c_{3r} + K_{31r}s_{3r} &  \end{matrix}\right] 
 \nonumber \\ 
K_{58r} &= K_{33r}s_{3r} - K_{31r}c_{3r} \nonumber \\
K_{59r} &= - K_{32r} - \dot{q}_{3r} \nonumber \\
K_{60r} &= K_{33r}c_{3r} + K_{31r}s_{3r} \nonumber \\
 \bar\omega_{6r} &= \left[\begin{matrix} K_{58r} & K_{59r} & K_{60r} &  \end{matrix}\right] 
 \nonumber \\ 
 \bar\omega_{6r} &= \left[\begin{matrix} s_{3r}(K_{37r}\dot{\psi} + K_{38r}\dot{q}_{w} + K_{38r}\dot{q}_{imu} + K_{39r}\dot{q}_{torso} + \dot{q}_{1r}s_{2r}) + c_{3r}(\dot{q}_{2r} + K_{4r}\dot{\psi} + K_{5r}\dot{q}_{w} + K_{5r}\dot{q}_{imu} - \dot{q}_{torso}c_{1r}) & - \dot{q}_{3r} - K_{34r}\dot{\psi} - K_{35r}\dot{q}_{w} - K_{35r}\dot{q}_{imu} - K_{36r}\dot{q}_{torso} - \dot{q}_{1r}c_{2r} & c_{3r}(K_{37r}\dot{\psi} + K_{38r}\dot{q}_{w} + K_{38r}\dot{q}_{imu} + K_{39r}\dot{q}_{torso} + \dot{q}_{1r}s_{2r}) - s_{3r}(\dot{q}_{2r} + K_{4r}\dot{\psi} + K_{5r}\dot{q}_{w} + K_{5r}\dot{q}_{imu} - \dot{q}_{torso}c_{1r}) &  \end{matrix}\right] 
 \nonumber \\ 
K_{61r} &= K_{4r}c_{3r} + K_{37r}s_{3r} \nonumber \\
K_{62r} &= K_{5r}c_{3r} + K_{38r}s_{3r} \nonumber \\
K_{63r} &= K_{39r}s_{3r} - c_{1r}c_{3r} \nonumber \\
K_{64r} &= s_{2r}s_{3r} \nonumber \\
K_{65r} &= K_{37r}c_{3r} - K_{4r}s_{3r} \nonumber \\
K_{66r} &= K_{38r}c_{3r} - K_{5r}s_{3r} \nonumber \\
K_{67r} &= c_{1r}s_{3r} + K_{39r}c_{3r} \nonumber \\
K_{68r} &= c_{3r}s_{2r} \nonumber \\
 \bar\omega_{6r} &= \left[\begin{matrix} K_{61r}\dot{\psi} + K_{64r}\dot{q}_{1r} + K_{62r}\dot{q}_{w} + K_{62r}\dot{q}_{imu} + K_{63r}\dot{q}_{torso} + \dot{q}_{2r}c_{3r} & - \dot{q}_{3r} - K_{34r}\dot{\psi} - K_{35r}\dot{q}_{w} - K_{35r}\dot{q}_{imu} - K_{36r}\dot{q}_{torso} - \dot{q}_{1r}c_{2r} & K_{65r}\dot{\psi} + K_{68r}\dot{q}_{1r} + K_{66r}\dot{q}_{w} + K_{66r}\dot{q}_{imu} + K_{67r}\dot{q}_{torso} - \dot{q}_{2r}s_{3r} &  \end{matrix}\right] 
 \nonumber \\ 
 \bar{v}_{6r} &= {}^{6r}A_{5r} \left(\bar{v}_{5r} + \bar\omega_{5r} \times \bar{P}_{6r}\right) 
 \nonumber \\ 
 \bar{v}_{6r} &= \left[\begin{matrix} c_{3r}(K_{8r} - K_{33r}L_7) + s_{3r}(K_{41r} - K_{31r}L_7) & -K_{40r} & c_{3r}(K_{41r} - K_{31r}L_7) - s_{3r}(K_{8r} - K_{33r}L_7) &  \end{matrix}\right] 
 \nonumber \\ 
K_{69r} &= c_{3r}(K_{8r} - K_{33r}L_7) + s_{3r}(K_{41r}  \nonumber \\
&- K_{31r}L_7) \nonumber \\
K_{70r} &= c_{3r}(K_{41r} - K_{31r}L_7) - s_{3r}(K_{8r}  \nonumber \\
&- K_{33r}L_7) \nonumber \\
 \bar{v}_{6r} &= \left[\begin{matrix} K_{69r} & -K_{40r} & K_{70r} &  \end{matrix}\right] 
 \nonumber \\ 
 \bar{v}_{6r} &= \left[\begin{matrix} s_{3r}(K_{48r}\dot{\psi} + K_{50r}\dot{q}_{w} + K_{49r}\dot{q}_{imu} + K_{51r}\dot{q}_{torso} + K_{47r}\dot{x} + L_7(\dot{q}_{2r} + K_{4r}\dot{\psi} + K_{5r}\dot{q}_{w} + K_{5r}\dot{q}_{imu} - \dot{q}_{torso}c_{1r})) + c_{3r}(K_{12r}\dot{\psi} + K_{14r}\dot{q}_{w} + K_{13r}\dot{q}_{imu} + K_{15r}\dot{q}_{torso} + K_{11r}\dot{x} - L_7(K_{37r}\dot{\psi} + K_{38r}\dot{q}_{w} + K_{38r}\dot{q}_{imu} + K_{39r}\dot{q}_{torso} + \dot{q}_{1r}s_{2r})) & - K_{43r}\dot{\psi} - K_{45r}\dot{q}_{w} - K_{44r}\dot{q}_{imu} - K_{46r}\dot{q}_{torso} - K_{42r}\dot{x} & c_{3r}(K_{48r}\dot{\psi} + K_{50r}\dot{q}_{w} + K_{49r}\dot{q}_{imu} + K_{51r}\dot{q}_{torso} + K_{47r}\dot{x} + L_7(\dot{q}_{2r} + K_{4r}\dot{\psi} + K_{5r}\dot{q}_{w} + K_{5r}\dot{q}_{imu} - \dot{q}_{torso}c_{1r})) - s_{3r}(K_{12r}\dot{\psi} + K_{14r}\dot{q}_{w} + K_{13r}\dot{q}_{imu} + K_{15r}\dot{q}_{torso} + K_{11r}\dot{x} - L_7(K_{37r}\dot{\psi} + K_{38r}\dot{q}_{w} + K_{38r}\dot{q}_{imu} + K_{39r}\dot{q}_{torso} + \dot{q}_{1r}s_{2r})) &  \end{matrix}\right] 
 \nonumber \\ 
K_{71r} &= K_{11r}c_{3r} + K_{47r}s_{3r} \nonumber \\
K_{72r} &= c_{3r}(K_{12r} - K_{37r}L_7) + s_{3r}(K_{48r}  \nonumber \\
&+ K_{4r}L_7) \nonumber \\
K_{73r} &= c_{3r}(K_{13r} - K_{38r}L_7) + s_{3r}(K_{49r}  \nonumber \\
&+ K_{5r}L_7) \nonumber \\
K_{74r} &= c_{3r}(K_{14r} - K_{38r}L_7) + s_{3r}(K_{50r}  \nonumber \\
&+ K_{5r}L_7) \nonumber \\
K_{75r} &= s_{3r}(K_{51r} - L_7c_{1r}) + c_{3r}(K_{15r}  \nonumber \\
&- K_{39r}L_7) \nonumber \\
K_{76r} &= -L_7c_{3r}s_{2r} \nonumber \\
K_{77r} &= L_7s_{3r} \nonumber \\
K_{78r} &= K_{47r}c_{3r} - K_{11r}s_{3r} \nonumber \\
K_{79r} &= c_{3r}(K_{48r} + K_{4r}L_7) - s_{3r}(K_{12r}  \nonumber \\
&- K_{37r}L_7) \nonumber \\
K_{80r} &= c_{3r}(K_{49r} + K_{5r}L_7) - s_{3r}(K_{13r}  \nonumber \\
&- K_{38r}L_7) \nonumber \\
K_{81r} &= c_{3r}(K_{50r} + K_{5r}L_7) - s_{3r}(K_{14r}  \nonumber \\
&- K_{38r}L_7) \nonumber \\
K_{82r} &= c_{3r}(K_{51r} - L_7c_{1r}) - s_{3r}(K_{15r}  \nonumber \\
&- K_{39r}L_7) \nonumber \\
K_{83r} &= L_7s_{2r}s_{3r} \nonumber \\
K_{84r} &= L_7c_{3r} \nonumber \\
 \bar{v}_{6r} &= \left[\begin{matrix} K_{72r}\dot{\psi} + K_{76r}\dot{q}_{1r} + K_{74r}\dot{q}_{w} + K_{77r}\dot{q}_{2r} + K_{73r}\dot{q}_{imu} + K_{75r}\dot{q}_{torso} + K_{71r}\dot{x} & - K_{43r}\dot{\psi} - K_{45r}\dot{q}_{w} - K_{44r}\dot{q}_{imu} - K_{46r}\dot{q}_{torso} - K_{42r}\dot{x} & K_{79r}\dot{\psi} + K_{83r}\dot{q}_{1r} + K_{81r}\dot{q}_{w} + K_{84r}\dot{q}_{2r} + K_{80r}\dot{q}_{imu} + K_{82r}\dot{q}_{torso} + K_{78r}\dot{x} &  \end{matrix}\right] 
 \nonumber \\ 
 \bar\alpha_{6r} &= {}^{6r}A_{5r} \bar\alpha_{5r} + \ddot{q}_{6r} \bar{e}_{6r} + \dot{q}_{6r} \left(\bar\omega_{6r} \times \bar{e}_{6r}\right) 
 \nonumber \\ 
 \bar\alpha_{6r} &= \left[\begin{matrix} c_{3r}(K_{24r} + \ddot{q}_{2r} + K_{4r}\ddot{\psi} + K_{5r}\ddot{q}_{w} + K_{5r}\ddot{q}_{imu} - \ddot{q}_{torso}c_{1r}) + K_{60r}\dot{q}_{3r} + s_{3r}(K_{53r} + K_{37r}\ddot{\psi} + K_{38r}\ddot{q}_{w} + K_{38r}\ddot{q}_{imu} + K_{39r}\ddot{q}_{torso} + \ddot{q}_{1r}s_{2r}) & - K_{52r} - \ddot{q}_{3r} - K_{34r}\ddot{\psi} - K_{35r}\ddot{q}_{w} - K_{35r}\ddot{q}_{imu} - K_{36r}\ddot{q}_{torso} - \ddot{q}_{1r}c_{2r} & c_{3r}(K_{53r} + K_{37r}\ddot{\psi} + K_{38r}\ddot{q}_{w} + K_{38r}\ddot{q}_{imu} + K_{39r}\ddot{q}_{torso} + \ddot{q}_{1r}s_{2r}) - s_{3r}(K_{24r} + \ddot{q}_{2r} + K_{4r}\ddot{\psi} + K_{5r}\ddot{q}_{w} + K_{5r}\ddot{q}_{imu} - \ddot{q}_{torso}c_{1r}) - K_{58r}\dot{q}_{3r} &  \end{matrix}\right] 
 \nonumber \\ 
K_{85r} &= K_{60r}\dot{q}_{3r} + K_{24r}c_{3r} + K_{53r}s_{3r} \nonumber \\
K_{86r} &= K_{53r}c_{3r} - K_{58r}\dot{q}_{3r} - K_{24r}s_{3r} \nonumber \\
 \bar\alpha_{6r} &= \left[\begin{matrix} K_{85r} + K_{61r}\ddot{\psi} + K_{64r}\ddot{q}_{1r} + K_{62r}\ddot{q}_{w} + K_{62r}\ddot{q}_{imu} + K_{63r}\ddot{q}_{torso} + \ddot{q}_{2r}c_{3r} & - K_{52r} - \ddot{q}_{3r} - K_{34r}\ddot{\psi} - K_{35r}\ddot{q}_{w} - K_{35r}\ddot{q}_{imu} - K_{36r}\ddot{q}_{torso} - \ddot{q}_{1r}c_{2r} & K_{86r} + K_{65r}\ddot{\psi} + K_{68r}\ddot{q}_{1r} + K_{66r}\ddot{q}_{w} + K_{66r}\ddot{q}_{imu} + K_{67r}\ddot{q}_{torso} - \ddot{q}_{2r}s_{3r} &  \end{matrix}\right] 
 \nonumber \\ 
 \bar{a}_{6r} &= {}^{6r}A_{5r} \left(\bar{a}_{5r} + \bar\alpha_{5r} \times \bar{P}_{6r} + \bar\omega_{5r} \times \left(\bar\omega_{5r} \times \bar{P}_{6r}\right)\right) 
 \nonumber \\ 
 \bar\alpha_{6r} &= \left[\begin{matrix} c_{3r}(K_{26r} + K_{12r}\ddot{\psi} + K_{14r}\ddot{q}_{w} + K_{13r}\ddot{q}_{imu} + K_{15r}\ddot{q}_{torso} + K_{11r}\ddot{x} - L_7(K_{53r} + K_{37r}\ddot{\psi} + K_{38r}\ddot{q}_{w} + K_{38r}\ddot{q}_{imu} + K_{39r}\ddot{q}_{torso} + \ddot{q}_{1r}s_{2r}) + K_{31r}K_{32r}L_7) + s_{3r}(K_{55r} + K_{48r}\ddot{\psi} + K_{50r}\ddot{q}_{w} + K_{49r}\ddot{q}_{imu} + K_{51r}\ddot{q}_{torso} + K_{47r}\ddot{x} + L_7(K_{24r} + \ddot{q}_{2r} + K_{4r}\ddot{\psi} + K_{5r}\ddot{q}_{w} + K_{5r}\ddot{q}_{imu} - \ddot{q}_{torso}c_{1r}) - K_{32r}K_{33r}L_7) & - K_{54r} - K_{43r}\ddot{\psi} - K_{45r}\ddot{q}_{w} - K_{44r}\ddot{q}_{imu} - K_{46r}\ddot{q}_{torso} - K_{42r}\ddot{x} - K_{31r}^2L_7 - K_{33r}^2L_7 & c_{3r}(K_{55r} + K_{48r}\ddot{\psi} + K_{50r}\ddot{q}_{w} + K_{49r}\ddot{q}_{imu} + K_{51r}\ddot{q}_{torso} + K_{47r}\ddot{x} + L_7(K_{24r} + \ddot{q}_{2r} + K_{4r}\ddot{\psi} + K_{5r}\ddot{q}_{w} + K_{5r}\ddot{q}_{imu} - \ddot{q}_{torso}c_{1r}) - K_{32r}K_{33r}L_7) - s_{3r}(K_{26r} + K_{12r}\ddot{\psi} + K_{14r}\ddot{q}_{w} + K_{13r}\ddot{q}_{imu} + K_{15r}\ddot{q}_{torso} + K_{11r}\ddot{x} - L_7(K_{53r} + K_{37r}\ddot{\psi} + K_{38r}\ddot{q}_{w} + K_{38r}\ddot{q}_{imu} + K_{39r}\ddot{q}_{torso} + \ddot{q}_{1r}s_{2r}) + K_{31r}K_{32r}L_7) &  \end{matrix}\right] 
 \nonumber \\ 
K_{87r} &= K_{26r}c_{3r} + K_{55r}s_{3r} - K_{53r}L_7c_{3r}  \nonumber \\
&+ K_{24r}L_7s_{3r} + K_{31r}K_{32r}L_7c_{3r}  \nonumber \\
&- K_{32r}K_{33r}L_7s_{3r} \nonumber \\
K_{88r} &= - K_{54r} - K_{31r}^2L_7 - K_{33r}^2L_7 \nonumber \\
K_{89r} &= K_{55r}c_{3r} - K_{26r}s_{3r} + K_{24r}L_7c_{3r}  \nonumber \\
&+ K_{53r}L_7s_{3r} - K_{32r}K_{33r}L_7c_{3r}  \nonumber \\
&- K_{31r}K_{32r}L_7s_{3r} \nonumber \\
 \bar{a}_{6r} &= \left[\begin{matrix} K_{87r} + K_{72r}\ddot{\psi} + K_{76r}\ddot{q}_{1r} + K_{74r}\ddot{q}_{w} + K_{77r}\ddot{q}_{2r} + K_{73r}\ddot{q}_{imu} + K_{75r}\ddot{q}_{torso} + K_{71r}\ddot{x} & K_{88r} - K_{43r}\ddot{\psi} - K_{45r}\ddot{q}_{w} - K_{44r}\ddot{q}_{imu} - K_{46r}\ddot{q}_{torso} - K_{42r}\ddot{x} & K_{89r} + K_{79r}\ddot{\psi} + K_{83r}\ddot{q}_{1r} + K_{81r}\ddot{q}_{w} + K_{84r}\ddot{q}_{2r} + K_{80r}\ddot{q}_{imu} + K_{82r}\ddot{q}_{torso} + K_{78r}\ddot{x} &  \end{matrix}\right] 
 \nonumber \\ 
 \bar{g}_{6r} &= {}^{6r}A_{5r} \bar{g}_{5r} 
 \nonumber \\ 
 \bar{g}_{6r} &= \left[\begin{matrix} K_{29r}gc_{3r} + K_{57r}gs_{3r} & -K_{56r}g & K_{57r}gc_{3r} - K_{29r}gs_{3r} &  \end{matrix}\right] 
 \nonumber \\ 
K_{90r} &= K_{29r}c_{3r} + K_{57r}s_{3r} \nonumber \\
K_{91r} &= K_{57r}c_{3r} - K_{29r}s_{3r} \nonumber \\
 \bar{g}_{6r} &= \left[\begin{matrix} K_{90r}g & -K_{56r}g & K_{91r}g &  \end{matrix}\right] 
 \nonumber \\ 
 m_{6r}\bar{S}_{6r}^{\times}\bar{g}_{6r} &= \mathbf{MS}_{6r} \times \bar{g}_{6r} 
 \nonumber \\ 
 m_{6r}\bar{S}_{6r}^{\times}\bar{g}_{6r} &= \left[\begin{matrix} K_{91r}\mathbf{MY}_{6r}g + K_{56r}\mathbf{MZ}_{6r}g & K_{90r}\mathbf{MZ}_{6r}g - K_{91r}\mathbf{MX}_{6r}g & - K_{56r}\mathbf{MX}_{6r}g - K_{90r}\mathbf{MY}_{6r}g &  \end{matrix}\right] 
 \nonumber \\ 
D_{85r} &= K_{91r}\mathbf{MY}_{6r} + K_{56r}\mathbf{MZ}_{6r} \nonumber \\
D_{86r} &= K_{90r}\mathbf{MZ}_{6r} - K_{91r}\mathbf{MX}_{6r} \nonumber \\
D_{87r} &= - K_{56r}\mathbf{MX}_{6r} - K_{90r}\mathbf{MY}_{6r} \nonumber \\
 m_{6r}\bar{S}_{6r}^{\times}\bar{g}_{6r} &= \left[\begin{matrix} D_{85r}g & D_{86r}g & D_{87r}g &  \end{matrix}\right] 
 \nonumber \\ 
 m_{6r}\bar{a}_{G(6r)} &= m_{6r}\bar{a}_{6r} + \bar\alpha_{6r} \times \mathbf{MS}_{6r} + \bar\omega_{6r} \times \left(\bar\omega_{6r} \times \mathbf{MS}_{6r}\right) 
 \nonumber \\ 
 m_{6r}\bar{a}_{G(6r)} &= \left[\begin{matrix} m_{6r}(K_{87r} + K_{72r}\ddot{\psi} + K_{76r}\ddot{q}_{1r} + K_{74r}\ddot{q}_{w} + K_{77r}\ddot{q}_{2r} + K_{73r}\ddot{q}_{imu} + K_{75r}\ddot{q}_{torso} + K_{71r}\ddot{x}) - \mathbf{MZ}_{6r}(K_{52r} + \ddot{q}_{3r} + K_{34r}\ddot{\psi} + K_{35r}\ddot{q}_{w} + K_{35r}\ddot{q}_{imu} + K_{36r}\ddot{q}_{torso} + \ddot{q}_{1r}c_{2r}) - \mathbf{MY}_{6r}(K_{86r} + K_{65r}\ddot{\psi} + K_{68r}\ddot{q}_{1r} + K_{66r}\ddot{q}_{w} + K_{66r}\ddot{q}_{imu} + K_{67r}\ddot{q}_{torso} - \ddot{q}_{2r}s_{3r}) - K_{59r}(K_{59r}\mathbf{MX}_{6r} - K_{58r}\mathbf{MY}_{6r}) - K_{60r}(K_{60r}\mathbf{MX}_{6r} - K_{58r}\mathbf{MZ}_{6r}) & \mathbf{MX}_{6r}(K_{86r} + K_{65r}\ddot{\psi} + K_{68r}\ddot{q}_{1r} + K_{66r}\ddot{q}_{w} + K_{66r}\ddot{q}_{imu} + K_{67r}\ddot{q}_{torso} - \ddot{q}_{2r}s_{3r}) - \mathbf{MZ}_{6r}(K_{85r} + K_{61r}\ddot{\psi} + K_{64r}\ddot{q}_{1r} + K_{62r}\ddot{q}_{w} + K_{62r}\ddot{q}_{imu} + K_{63r}\ddot{q}_{torso} + \ddot{q}_{2r}c_{3r}) - m_{6r}(K_{43r}\ddot{\psi} - K_{88r} + K_{45r}\ddot{q}_{w} + K_{44r}\ddot{q}_{imu} + K_{46r}\ddot{q}_{torso} + K_{42r}\ddot{x}) + K_{58r}(K_{59r}\mathbf{MX}_{6r} - K_{58r}\mathbf{MY}_{6r}) - K_{60r}(K_{60r}\mathbf{MY}_{6r} - K_{59r}\mathbf{MZ}_{6r}) & \mathbf{MY}_{6r}(K_{85r} + K_{61r}\ddot{\psi} + K_{64r}\ddot{q}_{1r} + K_{62r}\ddot{q}_{w} + K_{62r}\ddot{q}_{imu} + K_{63r}\ddot{q}_{torso} + \ddot{q}_{2r}c_{3r}) + \mathbf{MX}_{6r}(K_{52r} + \ddot{q}_{3r} + K_{34r}\ddot{\psi} + K_{35r}\ddot{q}_{w} + K_{35r}\ddot{q}_{imu} + K_{36r}\ddot{q}_{torso} + \ddot{q}_{1r}c_{2r}) + m_{6r}(K_{89r} + K_{79r}\ddot{\psi} + K_{83r}\ddot{q}_{1r} + K_{81r}\ddot{q}_{w} + K_{84r}\ddot{q}_{2r} + K_{80r}\ddot{q}_{imu} + K_{82r}\ddot{q}_{torso} + K_{78r}\ddot{x}) + K_{58r}(K_{60r}\mathbf{MX}_{6r} - K_{58r}\mathbf{MZ}_{6r}) + K_{59r}(K_{60r}\mathbf{MY}_{6r} - K_{59r}\mathbf{MZ}_{6r}) &  \end{matrix}\right] 
 \nonumber \\ 
D_{88r} &= K_{71r}m_{6r} \nonumber \\
D_{89r} &= K_{72r}m_{6r} - K_{65r}\mathbf{MY}_{6r} - K_{34r}\mathbf{MZ}_{6r} \nonumber \\
D_{90r} &= K_{73r}m_{6r} - K_{66r}\mathbf{MY}_{6r} - K_{35r}\mathbf{MZ}_{6r} \nonumber \\
D_{91r} &= K_{74r}m_{6r} - K_{66r}\mathbf{MY}_{6r} - K_{35r}\mathbf{MZ}_{6r} \nonumber \\
D_{92r} &= K_{75r}m_{6r} - K_{67r}\mathbf{MY}_{6r} - K_{36r}\mathbf{MZ}_{6r} \nonumber \\
D_{93r} &= K_{76r}m_{6r} - \mathbf{MZ}_{6r}c_{2r} - K_{68r}\mathbf{MY}_{6r} \nonumber \\
D_{94r} &= K_{77r}m_{6r} + \mathbf{MY}_{6r}s_{3r} \nonumber \\
D_{95r} &= K_{87r}m_{6r} - K_{59r}^2\mathbf{MX}_{6r} - K_{60r}^2\mathbf{MX}_{6r}  \nonumber \\
&- K_{86r}\mathbf{MY}_{6r} - K_{52r}\mathbf{MZ}_{6r} + K_{58r}K_{59r}\mathbf{MY}_{6r}  \nonumber \\
&+ K_{58r}K_{60r}\mathbf{MZ}_{6r} \nonumber \\
D_{96r} &= -K_{42r}m_{6r} \nonumber \\
D_{97r} &= K_{65r}\mathbf{MX}_{6r} - K_{43r}m_{6r} - K_{61r}\mathbf{MZ}_{6r} \nonumber \\
D_{98r} &= K_{66r}\mathbf{MX}_{6r} - K_{44r}m_{6r} - K_{62r}\mathbf{MZ}_{6r} \nonumber \\
D_{99r} &= K_{66r}\mathbf{MX}_{6r} - K_{45r}m_{6r} - K_{62r}\mathbf{MZ}_{6r} \nonumber \\
D_{100r} &= K_{67r}\mathbf{MX}_{6r} - K_{46r}m_{6r} - K_{63r}\mathbf{MZ}_{6r} \nonumber \\
D_{101r} &= K_{68r}\mathbf{MX}_{6r} - K_{64r}\mathbf{MZ}_{6r} \nonumber \\
D_{102r} &= - \mathbf{MZ}_{6r}c_{3r} - \mathbf{MX}_{6r}s_{3r} \nonumber \\
D_{103r} &= K_{88r}m_{6r} - K_{58r}^2\mathbf{MY}_{6r} - K_{60r}^2\mathbf{MY}_{6r}  \nonumber \\
&+ K_{86r}\mathbf{MX}_{6r} - K_{85r}\mathbf{MZ}_{6r} + K_{58r}K_{59r}\mathbf{MX}_{6r}  \nonumber \\
&+ K_{59r}K_{60r}\mathbf{MZ}_{6r} \nonumber \\
D_{104r} &= K_{78r}m_{6r} \nonumber \\
D_{105r} &= K_{79r}m_{6r} + K_{34r}\mathbf{MX}_{6r} + K_{61r}\mathbf{MY}_{6r} \nonumber \\
D_{106r} &= K_{80r}m_{6r} + K_{35r}\mathbf{MX}_{6r} + K_{62r}\mathbf{MY}_{6r} \nonumber \\
D_{107r} &= K_{81r}m_{6r} + K_{35r}\mathbf{MX}_{6r} + K_{62r}\mathbf{MY}_{6r} \nonumber \\
D_{108r} &= K_{82r}m_{6r} + K_{36r}\mathbf{MX}_{6r} + K_{63r}\mathbf{MY}_{6r} \nonumber \\
D_{109r} &= K_{83r}m_{6r} + \mathbf{MX}_{6r}c_{2r} + K_{64r}\mathbf{MY}_{6r} \nonumber \\
D_{110r} &= K_{84r}m_{6r} + \mathbf{MY}_{6r}c_{3r} \nonumber \\
D_{111r} &= K_{89r}m_{6r} - K_{58r}^2\mathbf{MZ}_{6r} - K_{59r}^2\mathbf{MZ}_{6r}  \nonumber \\
&+ K_{52r}\mathbf{MX}_{6r} + K_{85r}\mathbf{MY}_{6r} + K_{58r}K_{60r}\mathbf{MX}_{6r}  \nonumber \\
&+ K_{59r}K_{60r}\mathbf{MY}_{6r} \nonumber \\
 m_{6r}\bar{a}_{G(6r)} &= \left[\begin{matrix} D_{95r} + D_{89r}\ddot{\psi} + D_{93r}\ddot{q}_{1r} + D_{91r}\ddot{q}_{w} + D_{94r}\ddot{q}_{2r} + D_{90r}\ddot{q}_{imu} + D_{92r}\ddot{q}_{torso} + D_{88r}\ddot{x} - \mathbf{MZ}_{6r}\ddot{q}_{3r} & D_{103r} + D_{97r}\ddot{\psi} + D_{101r}\ddot{q}_{1r} + D_{99r}\ddot{q}_{w} + D_{102r}\ddot{q}_{2r} + D_{98r}\ddot{q}_{imu} + D_{100r}\ddot{q}_{torso} + D_{96r}\ddot{x} & D_{111r} + D_{105r}\ddot{\psi} + D_{109r}\ddot{q}_{1r} + D_{107r}\ddot{q}_{w} + D_{110r}\ddot{q}_{2r} + D_{106r}\ddot{q}_{imu} + D_{108r}\ddot{q}_{torso} + D_{104r}\ddot{x} + \mathbf{MX}_{6r}\ddot{q}_{3r} &  \end{matrix}\right] 
 \nonumber \\ 
 \dot{\bar{H}}_{6r} &= \mathbf{MS}_{6r} \times \bar{a}_{6r} + J_{6r}\bar{\alpha}_{6r} + \bar\omega_{6r} \times J_{6r}\bar{\omega}_{6r} 
 \nonumber \\ 
 \dot{\bar{H}}_{6r} &= \left[\begin{matrix} \mathbf{MZ}_{6r}(K_{43r}\ddot{\psi} - K_{88r} + K_{45r}\ddot{q}_{w} + K_{44r}\ddot{q}_{imu} + K_{46r}\ddot{q}_{torso} + K_{42r}\ddot{x}) - K_{60r}(K_{58r}\mathbf{XY}_{6r} + K_{59r}\mathbf{YY}_{6r} + K_{60r}\mathbf{YZ}_{6r}) + K_{59r}(K_{58r}\mathbf{XZ}_{6r} + K_{59r}\mathbf{YZ}_{6r} + K_{60r}\mathbf{ZZ}_{6r}) + \mathbf{XX}_{6r}(K_{85r} + K_{61r}\ddot{\psi} + K_{64r}\ddot{q}_{1r} + K_{62r}\ddot{q}_{w} + K_{62r}\ddot{q}_{imu} + K_{63r}\ddot{q}_{torso} + \ddot{q}_{2r}c_{3r}) + \mathbf{XZ}_{6r}(K_{86r} + K_{65r}\ddot{\psi} + K_{68r}\ddot{q}_{1r} + K_{66r}\ddot{q}_{w} + K_{66r}\ddot{q}_{imu} + K_{67r}\ddot{q}_{torso} - \ddot{q}_{2r}s_{3r}) - \mathbf{XY}_{6r}(K_{52r} + \ddot{q}_{3r} + K_{34r}\ddot{\psi} + K_{35r}\ddot{q}_{w} + K_{35r}\ddot{q}_{imu} + K_{36r}\ddot{q}_{torso} + \ddot{q}_{1r}c_{2r}) + \mathbf{MY}_{6r}(K_{89r} + K_{79r}\ddot{\psi} + K_{83r}\ddot{q}_{1r} + K_{81r}\ddot{q}_{w} + K_{84r}\ddot{q}_{2r} + K_{80r}\ddot{q}_{imu} + K_{82r}\ddot{q}_{torso} + K_{78r}\ddot{x}) & K_{60r}(K_{58r}\mathbf{XX}_{6r} + K_{59r}\mathbf{XY}_{6r} + K_{60r}\mathbf{XZ}_{6r}) - K_{58r}(K_{58r}\mathbf{XZ}_{6r} + K_{59r}\mathbf{YZ}_{6r} + K_{60r}\mathbf{ZZ}_{6r}) + \mathbf{XY}_{6r}(K_{85r} + K_{61r}\ddot{\psi} + K_{64r}\ddot{q}_{1r} + K_{62r}\ddot{q}_{w} + K_{62r}\ddot{q}_{imu} + K_{63r}\ddot{q}_{torso} + \ddot{q}_{2r}c_{3r}) + \mathbf{YZ}_{6r}(K_{86r} + K_{65r}\ddot{\psi} + K_{68r}\ddot{q}_{1r} + K_{66r}\ddot{q}_{w} + K_{66r}\ddot{q}_{imu} + K_{67r}\ddot{q}_{torso} - \ddot{q}_{2r}s_{3r}) - \mathbf{YY}_{6r}(K_{52r} + \ddot{q}_{3r} + K_{34r}\ddot{\psi} + K_{35r}\ddot{q}_{w} + K_{35r}\ddot{q}_{imu} + K_{36r}\ddot{q}_{torso} + \ddot{q}_{1r}c_{2r}) - \mathbf{MX}_{6r}(K_{89r} + K_{79r}\ddot{\psi} + K_{83r}\ddot{q}_{1r} + K_{81r}\ddot{q}_{w} + K_{84r}\ddot{q}_{2r} + K_{80r}\ddot{q}_{imu} + K_{82r}\ddot{q}_{torso} + K_{78r}\ddot{x}) + \mathbf{MZ}_{6r}(K_{87r} + K_{72r}\ddot{\psi} + K_{76r}\ddot{q}_{1r} + K_{74r}\ddot{q}_{w} + K_{77r}\ddot{q}_{2r} + K_{73r}\ddot{q}_{imu} + K_{75r}\ddot{q}_{torso} + K_{71r}\ddot{x}) & K_{58r}(K_{58r}\mathbf{XY}_{6r} + K_{59r}\mathbf{YY}_{6r} + K_{60r}\mathbf{YZ}_{6r}) - K_{59r}(K_{58r}\mathbf{XX}_{6r} + K_{59r}\mathbf{XY}_{6r} + K_{60r}\mathbf{XZ}_{6r}) - \mathbf{MX}_{6r}(K_{43r}\ddot{\psi} - K_{88r} + K_{45r}\ddot{q}_{w} + K_{44r}\ddot{q}_{imu} + K_{46r}\ddot{q}_{torso} + K_{42r}\ddot{x}) + \mathbf{XZ}_{6r}(K_{85r} + K_{61r}\ddot{\psi} + K_{64r}\ddot{q}_{1r} + K_{62r}\ddot{q}_{w} + K_{62r}\ddot{q}_{imu} + K_{63r}\ddot{q}_{torso} + \ddot{q}_{2r}c_{3r}) + \mathbf{ZZ}_{6r}(K_{86r} + K_{65r}\ddot{\psi} + K_{68r}\ddot{q}_{1r} + K_{66r}\ddot{q}_{w} + K_{66r}\ddot{q}_{imu} + K_{67r}\ddot{q}_{torso} - \ddot{q}_{2r}s_{3r}) - \mathbf{YZ}_{6r}(K_{52r} + \ddot{q}_{3r} + K_{34r}\ddot{\psi} + K_{35r}\ddot{q}_{w} + K_{35r}\ddot{q}_{imu} + K_{36r}\ddot{q}_{torso} + \ddot{q}_{1r}c_{2r}) - \mathbf{MY}_{6r}(K_{87r} + K_{72r}\ddot{\psi} + K_{76r}\ddot{q}_{1r} + K_{74r}\ddot{q}_{w} + K_{77r}\ddot{q}_{2r} + K_{73r}\ddot{q}_{imu} + K_{75r}\ddot{q}_{torso} + K_{71r}\ddot{x}) &  \end{matrix}\right] 
 \nonumber \\ 
D_{112r} &= K_{78r}\mathbf{MY}_{6r} + K_{42r}\mathbf{MZ}_{6r} \nonumber \\
D_{113r} &= K_{61r}\mathbf{XX}_{6r} - K_{34r}\mathbf{XY}_{6r} + K_{65r}\mathbf{XZ}_{6r}  \nonumber \\
&+ K_{79r}\mathbf{MY}_{6r} + K_{43r}\mathbf{MZ}_{6r} \nonumber \\
D_{114r} &= K_{62r}\mathbf{XX}_{6r} - K_{35r}\mathbf{XY}_{6r} + K_{66r}\mathbf{XZ}_{6r}  \nonumber \\
&+ K_{80r}\mathbf{MY}_{6r} + K_{44r}\mathbf{MZ}_{6r} \nonumber \\
D_{115r} &= K_{62r}\mathbf{XX}_{6r} - K_{35r}\mathbf{XY}_{6r} + K_{66r}\mathbf{XZ}_{6r}  \nonumber \\
&+ K_{81r}\mathbf{MY}_{6r} + K_{45r}\mathbf{MZ}_{6r} \nonumber \\
D_{116r} &= K_{63r}\mathbf{XX}_{6r} - K_{36r}\mathbf{XY}_{6r} + K_{67r}\mathbf{XZ}_{6r}  \nonumber \\
&+ K_{82r}\mathbf{MY}_{6r} + K_{46r}\mathbf{MZ}_{6r} \nonumber \\
D_{117r} &= K_{64r}\mathbf{XX}_{6r} + K_{68r}\mathbf{XZ}_{6r} - \mathbf{XY}_{6r}c_{2r}  \nonumber \\
&+ K_{83r}\mathbf{MY}_{6r} \nonumber \\
D_{118r} &= \mathbf{XX}_{6r}c_{3r} - \mathbf{XZ}_{6r}s_{3r} + K_{84r}\mathbf{MY}_{6r} \nonumber \\
D_{119r} &= K_{85r}\mathbf{XX}_{6r} - K_{52r}\mathbf{XY}_{6r} + K_{86r}\mathbf{XZ}_{6r}  \nonumber \\
&+ K_{59r}^2\mathbf{YZ}_{6r} - K_{60r}^2\mathbf{YZ}_{6r} + K_{89r}\mathbf{MY}_{6r}  \nonumber \\
&- K_{88r}\mathbf{MZ}_{6r} - K_{58r}K_{60r}\mathbf{XY}_{6r} + K_{58r}K_{59r}\mathbf{XZ}_{6r}  \nonumber \\
&- K_{59r}K_{60r}\mathbf{YY}_{6r} + K_{59r}K_{60r}\mathbf{ZZ}_{6r} \nonumber \\
D_{120r} &= K_{71r}\mathbf{MZ}_{6r} - K_{78r}\mathbf{MX}_{6r} \nonumber \\
D_{121r} &= K_{61r}\mathbf{XY}_{6r} - K_{34r}\mathbf{YY}_{6r} + K_{65r}\mathbf{YZ}_{6r}  \nonumber \\
&- K_{79r}\mathbf{MX}_{6r} + K_{72r}\mathbf{MZ}_{6r} \nonumber \\
D_{122r} &= K_{62r}\mathbf{XY}_{6r} - K_{35r}\mathbf{YY}_{6r} + K_{66r}\mathbf{YZ}_{6r}  \nonumber \\
&- K_{80r}\mathbf{MX}_{6r} + K_{73r}\mathbf{MZ}_{6r} \nonumber \\
D_{123r} &= K_{62r}\mathbf{XY}_{6r} - K_{35r}\mathbf{YY}_{6r} + K_{66r}\mathbf{YZ}_{6r}  \nonumber \\
&- K_{81r}\mathbf{MX}_{6r} + K_{74r}\mathbf{MZ}_{6r} \nonumber \\
D_{124r} &= K_{63r}\mathbf{XY}_{6r} - K_{36r}\mathbf{YY}_{6r} + K_{67r}\mathbf{YZ}_{6r}  \nonumber \\
&- K_{82r}\mathbf{MX}_{6r} + K_{75r}\mathbf{MZ}_{6r} \nonumber \\
D_{125r} &= K_{64r}\mathbf{XY}_{6r} + K_{68r}\mathbf{YZ}_{6r} - \mathbf{YY}_{6r}c_{2r}  \nonumber \\
&- K_{83r}\mathbf{MX}_{6r} + K_{76r}\mathbf{MZ}_{6r} \nonumber \\
D_{126r} &= \mathbf{XY}_{6r}c_{3r} - \mathbf{YZ}_{6r}s_{3r} - K_{84r}\mathbf{MX}_{6r}  \nonumber \\
&+ K_{77r}\mathbf{MZ}_{6r} \nonumber \\
D_{127r} &= K_{85r}\mathbf{XY}_{6r} - K_{52r}\mathbf{YY}_{6r} + K_{86r}\mathbf{YZ}_{6r}  \nonumber \\
&- K_{58r}^2\mathbf{XZ}_{6r} + K_{60r}^2\mathbf{XZ}_{6r} - K_{89r}\mathbf{MX}_{6r}  \nonumber \\
&+ K_{87r}\mathbf{MZ}_{6r} + K_{58r}K_{60r}\mathbf{XX}_{6r} + K_{59r}K_{60r}\mathbf{XY}_{6r}  \nonumber \\
&- K_{58r}K_{59r}\mathbf{YZ}_{6r} - K_{58r}K_{60r}\mathbf{ZZ}_{6r} \nonumber \\
D_{128r} &= - K_{42r}\mathbf{MX}_{6r} - K_{71r}\mathbf{MY}_{6r} \nonumber \\
D_{129r} &= K_{61r}\mathbf{XZ}_{6r} - K_{34r}\mathbf{YZ}_{6r} + K_{65r}\mathbf{ZZ}_{6r}  \nonumber \\
&- K_{43r}\mathbf{MX}_{6r} - K_{72r}\mathbf{MY}_{6r} \nonumber \\
D_{130r} &= K_{62r}\mathbf{XZ}_{6r} - K_{35r}\mathbf{YZ}_{6r} + K_{66r}\mathbf{ZZ}_{6r}  \nonumber \\
&- K_{44r}\mathbf{MX}_{6r} - K_{73r}\mathbf{MY}_{6r} \nonumber \\
D_{131r} &= K_{62r}\mathbf{XZ}_{6r} - K_{35r}\mathbf{YZ}_{6r} + K_{66r}\mathbf{ZZ}_{6r}  \nonumber \\
&- K_{45r}\mathbf{MX}_{6r} - K_{74r}\mathbf{MY}_{6r} \nonumber \\
D_{132r} &= K_{63r}\mathbf{XZ}_{6r} - K_{36r}\mathbf{YZ}_{6r} + K_{67r}\mathbf{ZZ}_{6r}  \nonumber \\
&- K_{46r}\mathbf{MX}_{6r} - K_{75r}\mathbf{MY}_{6r} \nonumber \\
D_{133r} &= K_{64r}\mathbf{XZ}_{6r} + K_{68r}\mathbf{ZZ}_{6r} - \mathbf{YZ}_{6r}c_{2r}  \nonumber \\
&- K_{76r}\mathbf{MY}_{6r} \nonumber \\
D_{134r} &= \mathbf{XZ}_{6r}c_{3r} - \mathbf{ZZ}_{6r}s_{3r} - K_{77r}\mathbf{MY}_{6r} \nonumber \\
D_{135r} &= K_{85r}\mathbf{XZ}_{6r} - K_{52r}\mathbf{YZ}_{6r} + K_{86r}\mathbf{ZZ}_{6r}  \nonumber \\
&+ K_{58r}^2\mathbf{XY}_{6r} - K_{59r}^2\mathbf{XY}_{6r} + K_{88r}\mathbf{MX}_{6r}  \nonumber \\
&- K_{87r}\mathbf{MY}_{6r} - K_{58r}K_{59r}\mathbf{XX}_{6r} - K_{59r}K_{60r}\mathbf{XZ}_{6r}  \nonumber \\
&+ K_{58r}K_{59r}\mathbf{YY}_{6r} + K_{58r}K_{60r}\mathbf{YZ}_{6r} \nonumber \\
 \dot{\bar{H}}_{6r} &= \left[\begin{matrix} D_{95r} + D_{89r}\ddot{\psi} + D_{93r}\ddot{q}_{1r} + D_{91r}\ddot{q}_{w} + D_{94r}\ddot{q}_{2r} + D_{90r}\ddot{q}_{imu} + D_{92r}\ddot{q}_{torso} + D_{88r}\ddot{x} - \mathbf{MZ}_{6r}\ddot{q}_{3r} & D_{103r} + D_{97r}\ddot{\psi} + D_{101r}\ddot{q}_{1r} + D_{99r}\ddot{q}_{w} + D_{102r}\ddot{q}_{2r} + D_{98r}\ddot{q}_{imu} + D_{100r}\ddot{q}_{torso} + D_{96r}\ddot{x} & D_{111r} + D_{105r}\ddot{\psi} + D_{109r}\ddot{q}_{1r} + D_{107r}\ddot{q}_{w} + D_{110r}\ddot{q}_{2r} + D_{106r}\ddot{q}_{imu} + D_{108r}\ddot{q}_{torso} + D_{104r}\ddot{x} + \mathbf{MX}_{6r}\ddot{q}_{3r} &  \end{matrix}\right] 
 \nonumber \\ 
 \bar\omega_{7r} &= {}^{7r}A_{6r} \bar\omega_{6r} + \dot{q}_{7r} \bar{e}_{7r} 
 \nonumber \\ 
 \bar\omega_{7r} &= \left[\begin{matrix} - K_{58r} - \dot{q}_{4r} & - K_{59r}c_{4r} - K_{60r}s_{4r} & K_{60r}c_{4r} - K_{59r}s_{4r} &  \end{matrix}\right] 
 \nonumber \\ 
K_{92r} &= - K_{58r} - \dot{q}_{4r} \nonumber \\
K_{93r} &= - K_{59r}c_{4r} - K_{60r}s_{4r} \nonumber \\
K_{94r} &= K_{60r}c_{4r} - K_{59r}s_{4r} \nonumber \\
 \bar\omega_{7r} &= \left[\begin{matrix} K_{92r} & K_{93r} & K_{94r} &  \end{matrix}\right] 
 \nonumber \\ 
 \bar\omega_{7r} &= \left[\begin{matrix} - \dot{q}_{4r} - K_{61r}\dot{\psi} - K_{64r}\dot{q}_{1r} - K_{62r}\dot{q}_{w} - K_{62r}\dot{q}_{imu} - K_{63r}\dot{q}_{torso} - \dot{q}_{2r}c_{3r} & c_{4r}(\dot{q}_{3r} + K_{34r}\dot{\psi} + K_{35r}\dot{q}_{w} + K_{35r}\dot{q}_{imu} + K_{36r}\dot{q}_{torso} + \dot{q}_{1r}c_{2r}) - s_{4r}(K_{65r}\dot{\psi} + K_{68r}\dot{q}_{1r} + K_{66r}\dot{q}_{w} + K_{66r}\dot{q}_{imu} + K_{67r}\dot{q}_{torso} - \dot{q}_{2r}s_{3r}) & c_{4r}(K_{65r}\dot{\psi} + K_{68r}\dot{q}_{1r} + K_{66r}\dot{q}_{w} + K_{66r}\dot{q}_{imu} + K_{67r}\dot{q}_{torso} - \dot{q}_{2r}s_{3r}) + s_{4r}(\dot{q}_{3r} + K_{34r}\dot{\psi} + K_{35r}\dot{q}_{w} + K_{35r}\dot{q}_{imu} + K_{36r}\dot{q}_{torso} + \dot{q}_{1r}c_{2r}) &  \end{matrix}\right] 
 \nonumber \\ 
K_{95r} &= K_{34r}c_{4r} - K_{65r}s_{4r} \nonumber \\
K_{96r} &= K_{35r}c_{4r} - K_{66r}s_{4r} \nonumber \\
K_{97r} &= K_{36r}c_{4r} - K_{67r}s_{4r} \nonumber \\
K_{98r} &= c_{2r}c_{4r} - K_{68r}s_{4r} \nonumber \\
K_{99r} &= s_{3r}s_{4r} \nonumber \\
K_{100r} &= K_{65r}c_{4r} + K_{34r}s_{4r} \nonumber \\
K_{101r} &= K_{66r}c_{4r} + K_{35r}s_{4r} \nonumber \\
K_{102r} &= K_{67r}c_{4r} + K_{36r}s_{4r} \nonumber \\
K_{103r} &= c_{2r}s_{4r} + K_{68r}c_{4r} \nonumber \\
K_{104r} &= -c_{4r}s_{3r} \nonumber \\
 \bar\omega_{7r} &= \left[\begin{matrix} - \dot{q}_{4r} - K_{61r}\dot{\psi} - K_{64r}\dot{q}_{1r} - K_{62r}\dot{q}_{w} - K_{62r}\dot{q}_{imu} - K_{63r}\dot{q}_{torso} - \dot{q}_{2r}c_{3r} & K_{95r}\dot{\psi} + K_{98r}\dot{q}_{1r} + K_{96r}\dot{q}_{w} + K_{99r}\dot{q}_{2r} + K_{96r}\dot{q}_{imu} + K_{97r}\dot{q}_{torso} + \dot{q}_{3r}c_{4r} & K_{100r}\dot{\psi} + K_{103r}\dot{q}_{1r} + K_{101r}\dot{q}_{w} + K_{104r}\dot{q}_{2r} + K_{101r}\dot{q}_{imu} + K_{102r}\dot{q}_{torso} + \dot{q}_{3r}s_{4r} &  \end{matrix}\right] 
 \nonumber \\ 
 \bar{v}_{7r} &= {}^{7r}A_{6r} \left(\bar{v}_{6r} + \bar\omega_{6r} \times \bar{P}_{7r}\right) 
 \nonumber \\ 
 \bar{v}_{7r} &= \left[\begin{matrix} -K_{69r} & K_{40r}c_{4r} - K_{70r}s_{4r} & K_{70r}c_{4r} + K_{40r}s_{4r} &  \end{matrix}\right] 
 \nonumber \\ 
K_{105r} &= K_{40r}c_{4r} - K_{70r}s_{4r} \nonumber \\
K_{106r} &= K_{70r}c_{4r} + K_{40r}s_{4r} \nonumber \\
 \bar{v}_{7r} &= \left[\begin{matrix} -K_{69r} & K_{105r} & K_{106r} &  \end{matrix}\right] 
 \nonumber \\ 
 \bar{v}_{7r} &= \left[\begin{matrix} - K_{72r}\dot{\psi} - K_{76r}\dot{q}_{1r} - K_{74r}\dot{q}_{w} - K_{77r}\dot{q}_{2r} - K_{73r}\dot{q}_{imu} - K_{75r}\dot{q}_{torso} - K_{71r}\dot{x} & c_{4r}(K_{43r}\dot{\psi} + K_{45r}\dot{q}_{w} + K_{44r}\dot{q}_{imu} + K_{46r}\dot{q}_{torso} + K_{42r}\dot{x}) - s_{4r}(K_{79r}\dot{\psi} + K_{83r}\dot{q}_{1r} + K_{81r}\dot{q}_{w} + K_{84r}\dot{q}_{2r} + K_{80r}\dot{q}_{imu} + K_{82r}\dot{q}_{torso} + K_{78r}\dot{x}) & s_{4r}(K_{43r}\dot{\psi} + K_{45r}\dot{q}_{w} + K_{44r}\dot{q}_{imu} + K_{46r}\dot{q}_{torso} + K_{42r}\dot{x}) + c_{4r}(K_{79r}\dot{\psi} + K_{83r}\dot{q}_{1r} + K_{81r}\dot{q}_{w} + K_{84r}\dot{q}_{2r} + K_{80r}\dot{q}_{imu} + K_{82r}\dot{q}_{torso} + K_{78r}\dot{x}) &  \end{matrix}\right] 
 \nonumber \\ 
K_{107r} &= K_{42r}c_{4r} - K_{78r}s_{4r} \nonumber \\
K_{108r} &= K_{43r}c_{4r} - K_{79r}s_{4r} \nonumber \\
K_{109r} &= K_{44r}c_{4r} - K_{80r}s_{4r} \nonumber \\
K_{110r} &= K_{45r}c_{4r} - K_{81r}s_{4r} \nonumber \\
K_{111r} &= K_{46r}c_{4r} - K_{82r}s_{4r} \nonumber \\
K_{112r} &= -K_{83r}s_{4r} \nonumber \\
K_{113r} &= -K_{84r}s_{4r} \nonumber \\
K_{114r} &= K_{78r}c_{4r} + K_{42r}s_{4r} \nonumber \\
K_{115r} &= K_{79r}c_{4r} + K_{43r}s_{4r} \nonumber \\
K_{116r} &= K_{80r}c_{4r} + K_{44r}s_{4r} \nonumber \\
K_{117r} &= K_{81r}c_{4r} + K_{45r}s_{4r} \nonumber \\
K_{118r} &= K_{82r}c_{4r} + K_{46r}s_{4r} \nonumber \\
K_{119r} &= K_{83r}c_{4r} \nonumber \\
K_{120r} &= K_{84r}c_{4r} \nonumber \\
 \bar{v}_{7r} &= \left[\begin{matrix} - K_{72r}\dot{\psi} - K_{76r}\dot{q}_{1r} - K_{74r}\dot{q}_{w} - K_{77r}\dot{q}_{2r} - K_{73r}\dot{q}_{imu} - K_{75r}\dot{q}_{torso} - K_{71r}\dot{x} & K_{108r}\dot{\psi} + K_{112r}\dot{q}_{1r} + K_{110r}\dot{q}_{w} + K_{113r}\dot{q}_{2r} + K_{109r}\dot{q}_{imu} + K_{111r}\dot{q}_{torso} + K_{107r}\dot{x} & K_{115r}\dot{\psi} + K_{119r}\dot{q}_{1r} + K_{117r}\dot{q}_{w} + K_{120r}\dot{q}_{2r} + K_{116r}\dot{q}_{imu} + K_{118r}\dot{q}_{torso} + K_{114r}\dot{x} &  \end{matrix}\right] 
 \nonumber \\ 
 \bar\alpha_{7r} &= {}^{7r}A_{6r} \bar\alpha_{6r} + \ddot{q}_{7r} \bar{e}_{7r} + \dot{q}_{7r} \left(\bar\omega_{7r} \times \bar{e}_{7r}\right) 
 \nonumber \\ 
 \bar\alpha_{7r} &= \left[\begin{matrix} - K_{85r} - \ddot{q}_{4r} - K_{61r}\ddot{\psi} - K_{64r}\ddot{q}_{1r} - K_{62r}\ddot{q}_{w} - K_{62r}\ddot{q}_{imu} - K_{63r}\ddot{q}_{torso} - \ddot{q}_{2r}c_{3r} & c_{4r}(K_{52r} + \ddot{q}_{3r} + K_{34r}\ddot{\psi} + K_{35r}\ddot{q}_{w} + K_{35r}\ddot{q}_{imu} + K_{36r}\ddot{q}_{torso} + \ddot{q}_{1r}c_{2r}) - K_{94r}\dot{q}_{4r} - s_{4r}(K_{86r} + K_{65r}\ddot{\psi} + K_{68r}\ddot{q}_{1r} + K_{66r}\ddot{q}_{w} + K_{66r}\ddot{q}_{imu} + K_{67r}\ddot{q}_{torso} - \ddot{q}_{2r}s_{3r}) & K_{93r}\dot{q}_{4r} + s_{4r}(K_{52r} + \ddot{q}_{3r} + K_{34r}\ddot{\psi} + K_{35r}\ddot{q}_{w} + K_{35r}\ddot{q}_{imu} + K_{36r}\ddot{q}_{torso} + \ddot{q}_{1r}c_{2r}) + c_{4r}(K_{86r} + K_{65r}\ddot{\psi} + K_{68r}\ddot{q}_{1r} + K_{66r}\ddot{q}_{w} + K_{66r}\ddot{q}_{imu} + K_{67r}\ddot{q}_{torso} - \ddot{q}_{2r}s_{3r}) &  \end{matrix}\right] 
 \nonumber \\ 
K_{121r} &= K_{52r}c_{4r} - K_{94r}\dot{q}_{4r} - K_{86r}s_{4r} \nonumber \\
K_{122r} &= K_{93r}\dot{q}_{4r} + K_{86r}c_{4r} + K_{52r}s_{4r} \nonumber \\
 \bar\alpha_{7r} &= \left[\begin{matrix} - K_{85r} - \ddot{q}_{4r} - K_{61r}\ddot{\psi} - K_{64r}\ddot{q}_{1r} - K_{62r}\ddot{q}_{w} - K_{62r}\ddot{q}_{imu} - K_{63r}\ddot{q}_{torso} - \ddot{q}_{2r}c_{3r} & K_{121r} + K_{95r}\ddot{\psi} + K_{98r}\ddot{q}_{1r} + K_{96r}\ddot{q}_{w} + K_{99r}\ddot{q}_{2r} + K_{96r}\ddot{q}_{imu} + K_{97r}\ddot{q}_{torso} + \ddot{q}_{3r}c_{4r} & K_{122r} + K_{100r}\ddot{\psi} + K_{103r}\ddot{q}_{1r} + K_{101r}\ddot{q}_{w} + K_{104r}\ddot{q}_{2r} + K_{101r}\ddot{q}_{imu} + K_{102r}\ddot{q}_{torso} + \ddot{q}_{3r}s_{4r} &  \end{matrix}\right] 
 \nonumber \\ 
 \bar{a}_{7r} &= {}^{7r}A_{6r} \left(\bar{a}_{6r} + \bar\alpha_{6r} \times \bar{P}_{7r} + \bar\omega_{6r} \times \left(\bar\omega_{6r} \times \bar{P}_{7r}\right)\right) 
 \nonumber \\ 
 \bar\alpha_{7r} &= \left[\begin{matrix} - K_{87r} - K_{72r}\ddot{\psi} - K_{76r}\ddot{q}_{1r} - K_{74r}\ddot{q}_{w} - K_{77r}\ddot{q}_{2r} - K_{73r}\ddot{q}_{imu} - K_{75r}\ddot{q}_{torso} - K_{71r}\ddot{x} & c_{4r}(K_{43r}\ddot{\psi} - K_{88r} + K_{45r}\ddot{q}_{w} + K_{44r}\ddot{q}_{imu} + K_{46r}\ddot{q}_{torso} + K_{42r}\ddot{x}) - s_{4r}(K_{89r} + K_{79r}\ddot{\psi} + K_{83r}\ddot{q}_{1r} + K_{81r}\ddot{q}_{w} + K_{84r}\ddot{q}_{2r} + K_{80r}\ddot{q}_{imu} + K_{82r}\ddot{q}_{torso} + K_{78r}\ddot{x}) & c_{4r}(K_{89r} + K_{79r}\ddot{\psi} + K_{83r}\ddot{q}_{1r} + K_{81r}\ddot{q}_{w} + K_{84r}\ddot{q}_{2r} + K_{80r}\ddot{q}_{imu} + K_{82r}\ddot{q}_{torso} + K_{78r}\ddot{x}) + s_{4r}(K_{43r}\ddot{\psi} - K_{88r} + K_{45r}\ddot{q}_{w} + K_{44r}\ddot{q}_{imu} + K_{46r}\ddot{q}_{torso} + K_{42r}\ddot{x}) &  \end{matrix}\right] 
 \nonumber \\ 
K_{123r} &= - K_{88r}c_{4r} - K_{89r}s_{4r} \nonumber \\
K_{124r} &= K_{89r}c_{4r} - K_{88r}s_{4r} \nonumber \\
 \bar{a}_{7r} &= \left[\begin{matrix} - K_{87r} - K_{72r}\ddot{\psi} - K_{76r}\ddot{q}_{1r} - K_{74r}\ddot{q}_{w} - K_{77r}\ddot{q}_{2r} - K_{73r}\ddot{q}_{imu} - K_{75r}\ddot{q}_{torso} - K_{71r}\ddot{x} & K_{123r} + K_{108r}\ddot{\psi} + K_{112r}\ddot{q}_{1r} + K_{110r}\ddot{q}_{w} + K_{113r}\ddot{q}_{2r} + K_{109r}\ddot{q}_{imu} + K_{111r}\ddot{q}_{torso} + K_{107r}\ddot{x} & K_{124r} + K_{115r}\ddot{\psi} + K_{119r}\ddot{q}_{1r} + K_{117r}\ddot{q}_{w} + K_{120r}\ddot{q}_{2r} + K_{116r}\ddot{q}_{imu} + K_{118r}\ddot{q}_{torso} + K_{114r}\ddot{x} &  \end{matrix}\right] 
 \nonumber \\ 
 \bar{g}_{7r} &= {}^{7r}A_{6r} \bar{g}_{6r} 
 \nonumber \\ 
 \bar{g}_{7r} &= \left[\begin{matrix} -K_{90r}g & K_{56r}gc_{4r} - K_{91r}gs_{4r} & K_{91r}gc_{4r} + K_{56r}gs_{4r} &  \end{matrix}\right] 
 \nonumber \\ 
K_{125r} &= K_{56r}c_{4r} - K_{91r}s_{4r} \nonumber \\
K_{126r} &= K_{91r}c_{4r} + K_{56r}s_{4r} \nonumber \\
 \bar{g}_{7r} &= \left[\begin{matrix} -K_{90r}g & K_{125r}g & K_{126r}g &  \end{matrix}\right] 
 \nonumber \\ 
 m_{7r}\bar{S}_{7r}^{\times}\bar{g}_{7r} &= \mathbf{MS}_{7r} \times \bar{g}_{7r} 
 \nonumber \\ 
 m_{7r}\bar{S}_{7r}^{\times}\bar{g}_{7r} &= \left[\begin{matrix} K_{126r}\mathbf{MY}_{7r}g - K_{125r}\mathbf{MZ}_{7r}g & - K_{126r}\mathbf{MX}_{7r}g - K_{90r}\mathbf{MZ}_{7r}g & K_{125r}\mathbf{MX}_{7r}g + K_{90r}\mathbf{MY}_{7r}g &  \end{matrix}\right] 
 \nonumber \\ 
D_{136r} &= K_{126r}\mathbf{MY}_{7r} - K_{125r}\mathbf{MZ}_{7r} \nonumber \\
D_{137r} &= - K_{126r}\mathbf{MX}_{7r} - K_{90r}\mathbf{MZ}_{7r} \nonumber \\
D_{138r} &= K_{125r}\mathbf{MX}_{7r} + K_{90r}\mathbf{MY}_{7r} \nonumber \\
 m_{7r}\bar{S}_{7r}^{\times}\bar{g}_{7r} &= \left[\begin{matrix} D_{136r}g & D_{137r}g & D_{138r}g &  \end{matrix}\right] 
 \nonumber \\ 
 m_{7r}\bar{a}_{G(7r)} &= m_{7r}\bar{a}_{7r} + \bar\alpha_{7r} \times \mathbf{MS}_{7r} + \bar\omega_{7r} \times \left(\bar\omega_{7r} \times \mathbf{MS}_{7r}\right) 
 \nonumber \\ 
 m_{7r}\bar{a}_{G(7r)} &= \left[\begin{matrix} \mathbf{MZ}_{7r}(K_{121r} + K_{95r}\ddot{\psi} + K_{98r}\ddot{q}_{1r} + K_{96r}\ddot{q}_{w} + K_{99r}\ddot{q}_{2r} + K_{96r}\ddot{q}_{imu} + K_{97r}\ddot{q}_{torso} + \ddot{q}_{3r}c_{4r}) - \mathbf{MY}_{7r}(K_{122r} + K_{100r}\ddot{\psi} + K_{103r}\ddot{q}_{1r} + K_{101r}\ddot{q}_{w} + K_{104r}\ddot{q}_{2r} + K_{101r}\ddot{q}_{imu} + K_{102r}\ddot{q}_{torso} + \ddot{q}_{3r}s_{4r}) - m_{7r}(K_{87r} + K_{72r}\ddot{\psi} + K_{76r}\ddot{q}_{1r} + K_{74r}\ddot{q}_{w} + K_{77r}\ddot{q}_{2r} + K_{73r}\ddot{q}_{imu} + K_{75r}\ddot{q}_{torso} + K_{71r}\ddot{x}) - K_{93r}(K_{93r}\mathbf{MX}_{7r} - K_{92r}\mathbf{MY}_{7r}) - K_{94r}(K_{94r}\mathbf{MX}_{7r} - K_{92r}\mathbf{MZ}_{7r}) & \mathbf{MX}_{7r}(K_{122r} + K_{100r}\ddot{\psi} + K_{103r}\ddot{q}_{1r} + K_{101r}\ddot{q}_{w} + K_{104r}\ddot{q}_{2r} + K_{101r}\ddot{q}_{imu} + K_{102r}\ddot{q}_{torso} + \ddot{q}_{3r}s_{4r}) + \mathbf{MZ}_{7r}(K_{85r} + \ddot{q}_{4r} + K_{61r}\ddot{\psi} + K_{64r}\ddot{q}_{1r} + K_{62r}\ddot{q}_{w} + K_{62r}\ddot{q}_{imu} + K_{63r}\ddot{q}_{torso} + \ddot{q}_{2r}c_{3r}) + m_{7r}(K_{123r} + K_{108r}\ddot{\psi} + K_{112r}\ddot{q}_{1r} + K_{110r}\ddot{q}_{w} + K_{113r}\ddot{q}_{2r} + K_{109r}\ddot{q}_{imu} + K_{111r}\ddot{q}_{torso} + K_{107r}\ddot{x}) + K_{92r}(K_{93r}\mathbf{MX}_{7r} - K_{92r}\mathbf{MY}_{7r}) - K_{94r}(K_{94r}\mathbf{MY}_{7r} - K_{93r}\mathbf{MZ}_{7r}) & m_{7r}(K_{124r} + K_{115r}\ddot{\psi} + K_{119r}\ddot{q}_{1r} + K_{117r}\ddot{q}_{w} + K_{120r}\ddot{q}_{2r} + K_{116r}\ddot{q}_{imu} + K_{118r}\ddot{q}_{torso} + K_{114r}\ddot{x}) - \mathbf{MY}_{7r}(K_{85r} + \ddot{q}_{4r} + K_{61r}\ddot{\psi} + K_{64r}\ddot{q}_{1r} + K_{62r}\ddot{q}_{w} + K_{62r}\ddot{q}_{imu} + K_{63r}\ddot{q}_{torso} + \ddot{q}_{2r}c_{3r}) - \mathbf{MX}_{7r}(K_{121r} + K_{95r}\ddot{\psi} + K_{98r}\ddot{q}_{1r} + K_{96r}\ddot{q}_{w} + K_{99r}\ddot{q}_{2r} + K_{96r}\ddot{q}_{imu} + K_{97r}\ddot{q}_{torso} + \ddot{q}_{3r}c_{4r}) + K_{92r}(K_{94r}\mathbf{MX}_{7r} - K_{92r}\mathbf{MZ}_{7r}) + K_{93r}(K_{94r}\mathbf{MY}_{7r} - K_{93r}\mathbf{MZ}_{7r}) &  \end{matrix}\right] 
 \nonumber \\ 
D_{139r} &= -K_{71r}m_{7r} \nonumber \\
D_{140r} &= K_{95r}\mathbf{MZ}_{7r} - K_{100r}\mathbf{MY}_{7r} - K_{72r}m_{7r} \nonumber \\
D_{141r} &= K_{96r}\mathbf{MZ}_{7r} - K_{101r}\mathbf{MY}_{7r} - K_{73r}m_{7r} \nonumber \\
D_{142r} &= K_{96r}\mathbf{MZ}_{7r} - K_{101r}\mathbf{MY}_{7r} - K_{74r}m_{7r} \nonumber \\
D_{143r} &= K_{97r}\mathbf{MZ}_{7r} - K_{102r}\mathbf{MY}_{7r} - K_{75r}m_{7r} \nonumber \\
D_{144r} &= K_{98r}\mathbf{MZ}_{7r} - K_{103r}\mathbf{MY}_{7r} - K_{76r}m_{7r} \nonumber \\
D_{145r} &= K_{99r}\mathbf{MZ}_{7r} - K_{104r}\mathbf{MY}_{7r} - K_{77r}m_{7r} \nonumber \\
D_{146r} &= \mathbf{MZ}_{7r}c_{4r} - \mathbf{MY}_{7r}s_{4r} \nonumber \\
D_{147r} &= K_{121r}\mathbf{MZ}_{7r} - K_{93r}^2\mathbf{MX}_{7r} - K_{94r}^2\mathbf{MX}_{7r}  \nonumber \\
&- K_{122r}\mathbf{MY}_{7r} - K_{87r}m_{7r} + K_{92r}K_{93r}\mathbf{MY}_{7r}  \nonumber \\
&+ K_{92r}K_{94r}\mathbf{MZ}_{7r} \nonumber \\
D_{148r} &= K_{107r}m_{7r} \nonumber \\
D_{149r} &= K_{108r}m_{7r} + K_{100r}\mathbf{MX}_{7r} + K_{61r}\mathbf{MZ}_{7r} \nonumber \\
D_{150r} &= K_{109r}m_{7r} + K_{101r}\mathbf{MX}_{7r} + K_{62r}\mathbf{MZ}_{7r} \nonumber \\
D_{151r} &= K_{110r}m_{7r} + K_{101r}\mathbf{MX}_{7r} + K_{62r}\mathbf{MZ}_{7r} \nonumber \\
D_{152r} &= K_{111r}m_{7r} + K_{102r}\mathbf{MX}_{7r} + K_{63r}\mathbf{MZ}_{7r} \nonumber \\
D_{153r} &= K_{112r}m_{7r} + K_{103r}\mathbf{MX}_{7r} + K_{64r}\mathbf{MZ}_{7r} \nonumber \\
D_{154r} &= K_{113r}m_{7r} + \mathbf{MZ}_{7r}c_{3r} + K_{104r}\mathbf{MX}_{7r} \nonumber \\
D_{155r} &= \mathbf{MX}_{7r}s_{4r} \nonumber \\
D_{156r} &= K_{123r}m_{7r} - K_{92r}^2\mathbf{MY}_{7r} - K_{94r}^2\mathbf{MY}_{7r}  \nonumber \\
&+ K_{122r}\mathbf{MX}_{7r} + K_{85r}\mathbf{MZ}_{7r} + K_{92r}K_{93r}\mathbf{MX}_{7r}  \nonumber \\
&+ K_{93r}K_{94r}\mathbf{MZ}_{7r} \nonumber \\
D_{157r} &= K_{114r}m_{7r} \nonumber \\
D_{158r} &= K_{115r}m_{7r} - K_{95r}\mathbf{MX}_{7r} - K_{61r}\mathbf{MY}_{7r} \nonumber \\
D_{159r} &= K_{116r}m_{7r} - K_{96r}\mathbf{MX}_{7r} - K_{62r}\mathbf{MY}_{7r} \nonumber \\
D_{160r} &= K_{117r}m_{7r} - K_{96r}\mathbf{MX}_{7r} - K_{62r}\mathbf{MY}_{7r} \nonumber \\
D_{161r} &= K_{118r}m_{7r} - K_{97r}\mathbf{MX}_{7r} - K_{63r}\mathbf{MY}_{7r} \nonumber \\
D_{162r} &= K_{119r}m_{7r} - K_{98r}\mathbf{MX}_{7r} - K_{64r}\mathbf{MY}_{7r} \nonumber \\
D_{163r} &= K_{120r}m_{7r} - \mathbf{MY}_{7r}c_{3r} - K_{99r}\mathbf{MX}_{7r} \nonumber \\
D_{164r} &= -\mathbf{MX}_{7r}c_{4r} \nonumber \\
D_{165r} &= K_{124r}m_{7r} - K_{92r}^2\mathbf{MZ}_{7r} - K_{93r}^2\mathbf{MZ}_{7r}  \nonumber \\
&- K_{121r}\mathbf{MX}_{7r} - K_{85r}\mathbf{MY}_{7r} + K_{92r}K_{94r}\mathbf{MX}_{7r}  \nonumber \\
&+ K_{93r}K_{94r}\mathbf{MY}_{7r} \nonumber \\
 m_{7r}\bar{a}_{G(7r)} &= \left[\begin{matrix} D_{147r} + D_{140r}\ddot{\psi} + D_{144r}\ddot{q}_{1r} + D_{142r}\ddot{q}_{w} + D_{145r}\ddot{q}_{2r} + D_{146r}\ddot{q}_{3r} + D_{141r}\ddot{q}_{imu} + D_{143r}\ddot{q}_{torso} + D_{139r}\ddot{x} & D_{156r} + D_{149r}\ddot{\psi} + D_{153r}\ddot{q}_{1r} + D_{151r}\ddot{q}_{w} + D_{154r}\ddot{q}_{2r} + D_{155r}\ddot{q}_{3r} + D_{150r}\ddot{q}_{imu} + D_{152r}\ddot{q}_{torso} + D_{148r}\ddot{x} + \mathbf{MZ}_{7r}\ddot{q}_{4r} & D_{165r} + D_{158r}\ddot{\psi} + D_{162r}\ddot{q}_{1r} + D_{160r}\ddot{q}_{w} + D_{163r}\ddot{q}_{2r} + D_{164r}\ddot{q}_{3r} + D_{159r}\ddot{q}_{imu} + D_{161r}\ddot{q}_{torso} + D_{157r}\ddot{x} - \mathbf{MY}_{7r}\ddot{q}_{4r} &  \end{matrix}\right] 
 \nonumber \\ 
 \dot{\bar{H}}_{7r} &= \mathbf{MS}_{7r} \times \bar{a}_{7r} + J_{7r}\bar{\alpha}_{7r} + \bar\omega_{7r} \times J_{7r}\bar{\omega}_{7r} 
 \nonumber \\ 
 \dot{\bar{H}}_{7r} &= \left[\begin{matrix} K_{93r}(K_{92r}\mathbf{XZ}_{7r} + K_{93r}\mathbf{YZ}_{7r} + K_{94r}\mathbf{ZZ}_{7r}) - K_{94r}(K_{92r}\mathbf{XY}_{7r} + K_{93r}\mathbf{YY}_{7r} + K_{94r}\mathbf{YZ}_{7r}) + \mathbf{XY}_{7r}(K_{121r} + K_{95r}\ddot{\psi} + K_{98r}\ddot{q}_{1r} + K_{96r}\ddot{q}_{w} + K_{99r}\ddot{q}_{2r} + K_{96r}\ddot{q}_{imu} + K_{97r}\ddot{q}_{torso} + \ddot{q}_{3r}c_{4r}) + \mathbf{XZ}_{7r}(K_{122r} + K_{100r}\ddot{\psi} + K_{103r}\ddot{q}_{1r} + K_{101r}\ddot{q}_{w} + K_{104r}\ddot{q}_{2r} + K_{101r}\ddot{q}_{imu} + K_{102r}\ddot{q}_{torso} + \ddot{q}_{3r}s_{4r}) - \mathbf{XX}_{7r}(K_{85r} + \ddot{q}_{4r} + K_{61r}\ddot{\psi} + K_{64r}\ddot{q}_{1r} + K_{62r}\ddot{q}_{w} + K_{62r}\ddot{q}_{imu} + K_{63r}\ddot{q}_{torso} + \ddot{q}_{2r}c_{3r}) + \mathbf{MY}_{7r}(K_{124r} + K_{115r}\ddot{\psi} + K_{119r}\ddot{q}_{1r} + K_{117r}\ddot{q}_{w} + K_{120r}\ddot{q}_{2r} + K_{116r}\ddot{q}_{imu} + K_{118r}\ddot{q}_{torso} + K_{114r}\ddot{x}) - \mathbf{MZ}_{7r}(K_{123r} + K_{108r}\ddot{\psi} + K_{112r}\ddot{q}_{1r} + K_{110r}\ddot{q}_{w} + K_{113r}\ddot{q}_{2r} + K_{109r}\ddot{q}_{imu} + K_{111r}\ddot{q}_{torso} + K_{107r}\ddot{x}) & K_{94r}(K_{92r}\mathbf{XX}_{7r} + K_{93r}\mathbf{XY}_{7r} + K_{94r}\mathbf{XZ}_{7r}) - K_{92r}(K_{92r}\mathbf{XZ}_{7r} + K_{93r}\mathbf{YZ}_{7r} + K_{94r}\mathbf{ZZ}_{7r}) + \mathbf{YY}_{7r}(K_{121r} + K_{95r}\ddot{\psi} + K_{98r}\ddot{q}_{1r} + K_{96r}\ddot{q}_{w} + K_{99r}\ddot{q}_{2r} + K_{96r}\ddot{q}_{imu} + K_{97r}\ddot{q}_{torso} + \ddot{q}_{3r}c_{4r}) + \mathbf{YZ}_{7r}(K_{122r} + K_{100r}\ddot{\psi} + K_{103r}\ddot{q}_{1r} + K_{101r}\ddot{q}_{w} + K_{104r}\ddot{q}_{2r} + K_{101r}\ddot{q}_{imu} + K_{102r}\ddot{q}_{torso} + \ddot{q}_{3r}s_{4r}) - \mathbf{XY}_{7r}(K_{85r} + \ddot{q}_{4r} + K_{61r}\ddot{\psi} + K_{64r}\ddot{q}_{1r} + K_{62r}\ddot{q}_{w} + K_{62r}\ddot{q}_{imu} + K_{63r}\ddot{q}_{torso} + \ddot{q}_{2r}c_{3r}) - \mathbf{MX}_{7r}(K_{124r} + K_{115r}\ddot{\psi} + K_{119r}\ddot{q}_{1r} + K_{117r}\ddot{q}_{w} + K_{120r}\ddot{q}_{2r} + K_{116r}\ddot{q}_{imu} + K_{118r}\ddot{q}_{torso} + K_{114r}\ddot{x}) - \mathbf{MZ}_{7r}(K_{87r} + K_{72r}\ddot{\psi} + K_{76r}\ddot{q}_{1r} + K_{74r}\ddot{q}_{w} + K_{77r}\ddot{q}_{2r} + K_{73r}\ddot{q}_{imu} + K_{75r}\ddot{q}_{torso} + K_{71r}\ddot{x}) & K_{92r}(K_{92r}\mathbf{XY}_{7r} + K_{93r}\mathbf{YY}_{7r} + K_{94r}\mathbf{YZ}_{7r}) - K_{93r}(K_{92r}\mathbf{XX}_{7r} + K_{93r}\mathbf{XY}_{7r} + K_{94r}\mathbf{XZ}_{7r}) + \mathbf{YZ}_{7r}(K_{121r} + K_{95r}\ddot{\psi} + K_{98r}\ddot{q}_{1r} + K_{96r}\ddot{q}_{w} + K_{99r}\ddot{q}_{2r} + K_{96r}\ddot{q}_{imu} + K_{97r}\ddot{q}_{torso} + \ddot{q}_{3r}c_{4r}) + \mathbf{ZZ}_{7r}(K_{122r} + K_{100r}\ddot{\psi} + K_{103r}\ddot{q}_{1r} + K_{101r}\ddot{q}_{w} + K_{104r}\ddot{q}_{2r} + K_{101r}\ddot{q}_{imu} + K_{102r}\ddot{q}_{torso} + \ddot{q}_{3r}s_{4r}) - \mathbf{XZ}_{7r}(K_{85r} + \ddot{q}_{4r} + K_{61r}\ddot{\psi} + K_{64r}\ddot{q}_{1r} + K_{62r}\ddot{q}_{w} + K_{62r}\ddot{q}_{imu} + K_{63r}\ddot{q}_{torso} + \ddot{q}_{2r}c_{3r}) + \mathbf{MX}_{7r}(K_{123r} + K_{108r}\ddot{\psi} + K_{112r}\ddot{q}_{1r} + K_{110r}\ddot{q}_{w} + K_{113r}\ddot{q}_{2r} + K_{109r}\ddot{q}_{imu} + K_{111r}\ddot{q}_{torso} + K_{107r}\ddot{x}) + \mathbf{MY}_{7r}(K_{87r} + K_{72r}\ddot{\psi} + K_{76r}\ddot{q}_{1r} + K_{74r}\ddot{q}_{w} + K_{77r}\ddot{q}_{2r} + K_{73r}\ddot{q}_{imu} + K_{75r}\ddot{q}_{torso} + K_{71r}\ddot{x}) &  \end{matrix}\right] 
 \nonumber \\ 
D_{166r} &= K_{114r}\mathbf{MY}_{7r} - K_{107r}\mathbf{MZ}_{7r} \nonumber \\
D_{167r} &= K_{95r}\mathbf{XY}_{7r} - K_{61r}\mathbf{XX}_{7r} + K_{100r}\mathbf{XZ}_{7r}  \nonumber \\
&+ K_{115r}\mathbf{MY}_{7r} - K_{108r}\mathbf{MZ}_{7r} \nonumber \\
D_{168r} &= K_{96r}\mathbf{XY}_{7r} - K_{62r}\mathbf{XX}_{7r} + K_{101r}\mathbf{XZ}_{7r}  \nonumber \\
&+ K_{116r}\mathbf{MY}_{7r} - K_{109r}\mathbf{MZ}_{7r} \nonumber \\
D_{169r} &= K_{96r}\mathbf{XY}_{7r} - K_{62r}\mathbf{XX}_{7r} + K_{101r}\mathbf{XZ}_{7r}  \nonumber \\
&+ K_{117r}\mathbf{MY}_{7r} - K_{110r}\mathbf{MZ}_{7r} \nonumber \\
D_{170r} &= K_{97r}\mathbf{XY}_{7r} - K_{63r}\mathbf{XX}_{7r} + K_{102r}\mathbf{XZ}_{7r}  \nonumber \\
&+ K_{118r}\mathbf{MY}_{7r} - K_{111r}\mathbf{MZ}_{7r} \nonumber \\
D_{171r} &= K_{98r}\mathbf{XY}_{7r} - K_{64r}\mathbf{XX}_{7r} + K_{103r}\mathbf{XZ}_{7r}  \nonumber \\
&+ K_{119r}\mathbf{MY}_{7r} - K_{112r}\mathbf{MZ}_{7r} \nonumber \\
D_{172r} &= K_{99r}\mathbf{XY}_{7r} + K_{104r}\mathbf{XZ}_{7r} - \mathbf{XX}_{7r}c_{3r}  \nonumber \\
&+ K_{120r}\mathbf{MY}_{7r} - K_{113r}\mathbf{MZ}_{7r} \nonumber \\
D_{173r} &= \mathbf{XY}_{7r}c_{4r} + \mathbf{XZ}_{7r}s_{4r} \nonumber \\
D_{174r} &= K_{121r}\mathbf{XY}_{7r} - K_{85r}\mathbf{XX}_{7r} + K_{122r}\mathbf{XZ}_{7r}  \nonumber \\
&+ K_{93r}^2\mathbf{YZ}_{7r} - K_{94r}^2\mathbf{YZ}_{7r} + K_{124r}\mathbf{MY}_{7r}  \nonumber \\
&- K_{123r}\mathbf{MZ}_{7r} - K_{92r}K_{94r}\mathbf{XY}_{7r} + K_{92r}K_{93r}\mathbf{XZ}_{7r}  \nonumber \\
&- K_{93r}K_{94r}\mathbf{YY}_{7r} + K_{93r}K_{94r}\mathbf{ZZ}_{7r} \nonumber \\
D_{175r} &= - K_{114r}\mathbf{MX}_{7r} - K_{71r}\mathbf{MZ}_{7r} \nonumber \\
D_{176r} &= K_{95r}\mathbf{YY}_{7r} - K_{61r}\mathbf{XY}_{7r} + K_{100r}\mathbf{YZ}_{7r}  \nonumber \\
&- K_{115r}\mathbf{MX}_{7r} - K_{72r}\mathbf{MZ}_{7r} \nonumber \\
D_{177r} &= K_{96r}\mathbf{YY}_{7r} - K_{62r}\mathbf{XY}_{7r} + K_{101r}\mathbf{YZ}_{7r}  \nonumber \\
&- K_{116r}\mathbf{MX}_{7r} - K_{73r}\mathbf{MZ}_{7r} \nonumber \\
D_{178r} &= K_{96r}\mathbf{YY}_{7r} - K_{62r}\mathbf{XY}_{7r} + K_{101r}\mathbf{YZ}_{7r}  \nonumber \\
&- K_{117r}\mathbf{MX}_{7r} - K_{74r}\mathbf{MZ}_{7r} \nonumber \\
D_{179r} &= K_{97r}\mathbf{YY}_{7r} - K_{63r}\mathbf{XY}_{7r} + K_{102r}\mathbf{YZ}_{7r}  \nonumber \\
&- K_{118r}\mathbf{MX}_{7r} - K_{75r}\mathbf{MZ}_{7r} \nonumber \\
D_{180r} &= K_{98r}\mathbf{YY}_{7r} - K_{64r}\mathbf{XY}_{7r} + K_{103r}\mathbf{YZ}_{7r}  \nonumber \\
&- K_{119r}\mathbf{MX}_{7r} - K_{76r}\mathbf{MZ}_{7r} \nonumber \\
D_{181r} &= K_{99r}\mathbf{YY}_{7r} + K_{104r}\mathbf{YZ}_{7r} - \mathbf{XY}_{7r}c_{3r}  \nonumber \\
&- K_{120r}\mathbf{MX}_{7r} - K_{77r}\mathbf{MZ}_{7r} \nonumber \\
D_{182r} &= \mathbf{YY}_{7r}c_{4r} + \mathbf{YZ}_{7r}s_{4r} \nonumber \\
D_{183r} &= K_{121r}\mathbf{YY}_{7r} - K_{85r}\mathbf{XY}_{7r} + K_{122r}\mathbf{YZ}_{7r}  \nonumber \\
&- K_{92r}^2\mathbf{XZ}_{7r} + K_{94r}^2\mathbf{XZ}_{7r} - K_{124r}\mathbf{MX}_{7r}  \nonumber \\
&- K_{87r}\mathbf{MZ}_{7r} + K_{92r}K_{94r}\mathbf{XX}_{7r} + K_{93r}K_{94r}\mathbf{XY}_{7r}  \nonumber \\
&- K_{92r}K_{93r}\mathbf{YZ}_{7r} - K_{92r}K_{94r}\mathbf{ZZ}_{7r} \nonumber \\
D_{184r} &= K_{107r}\mathbf{MX}_{7r} + K_{71r}\mathbf{MY}_{7r} \nonumber \\
D_{185r} &= K_{95r}\mathbf{YZ}_{7r} - K_{61r}\mathbf{XZ}_{7r} + K_{100r}\mathbf{ZZ}_{7r}  \nonumber \\
&+ K_{108r}\mathbf{MX}_{7r} + K_{72r}\mathbf{MY}_{7r} \nonumber \\
D_{186r} &= K_{96r}\mathbf{YZ}_{7r} - K_{62r}\mathbf{XZ}_{7r} + K_{101r}\mathbf{ZZ}_{7r}  \nonumber \\
&+ K_{109r}\mathbf{MX}_{7r} + K_{73r}\mathbf{MY}_{7r} \nonumber \\
D_{187r} &= K_{96r}\mathbf{YZ}_{7r} - K_{62r}\mathbf{XZ}_{7r} + K_{101r}\mathbf{ZZ}_{7r}  \nonumber \\
&+ K_{110r}\mathbf{MX}_{7r} + K_{74r}\mathbf{MY}_{7r} \nonumber \\
D_{188r} &= K_{97r}\mathbf{YZ}_{7r} - K_{63r}\mathbf{XZ}_{7r} + K_{102r}\mathbf{ZZ}_{7r}  \nonumber \\
&+ K_{111r}\mathbf{MX}_{7r} + K_{75r}\mathbf{MY}_{7r} \nonumber \\
D_{189r} &= K_{98r}\mathbf{YZ}_{7r} - K_{64r}\mathbf{XZ}_{7r} + K_{103r}\mathbf{ZZ}_{7r}  \nonumber \\
&+ K_{112r}\mathbf{MX}_{7r} + K_{76r}\mathbf{MY}_{7r} \nonumber \\
D_{190r} &= K_{99r}\mathbf{YZ}_{7r} + K_{104r}\mathbf{ZZ}_{7r} - \mathbf{XZ}_{7r}c_{3r}  \nonumber \\
&+ K_{113r}\mathbf{MX}_{7r} + K_{77r}\mathbf{MY}_{7r} \nonumber \\
D_{191r} &= \mathbf{YZ}_{7r}c_{4r} + \mathbf{ZZ}_{7r}s_{4r} \nonumber \\
D_{192r} &= K_{121r}\mathbf{YZ}_{7r} - K_{85r}\mathbf{XZ}_{7r} + K_{122r}\mathbf{ZZ}_{7r}  \nonumber \\
&+ K_{92r}^2\mathbf{XY}_{7r} - K_{93r}^2\mathbf{XY}_{7r} + K_{123r}\mathbf{MX}_{7r}  \nonumber \\
&+ K_{87r}\mathbf{MY}_{7r} - K_{92r}K_{93r}\mathbf{XX}_{7r} - K_{93r}K_{94r}\mathbf{XZ}_{7r}  \nonumber \\
&+ K_{92r}K_{93r}\mathbf{YY}_{7r} + K_{92r}K_{94r}\mathbf{YZ}_{7r} \nonumber \\
 \dot{\bar{H}}_{7r} &= \left[\begin{matrix} D_{147r} + D_{140r}\ddot{\psi} + D_{144r}\ddot{q}_{1r} + D_{142r}\ddot{q}_{w} + D_{145r}\ddot{q}_{2r} + D_{146r}\ddot{q}_{3r} + D_{141r}\ddot{q}_{imu} + D_{143r}\ddot{q}_{torso} + D_{139r}\ddot{x} & D_{156r} + D_{149r}\ddot{\psi} + D_{153r}\ddot{q}_{1r} + D_{151r}\ddot{q}_{w} + D_{154r}\ddot{q}_{2r} + D_{155r}\ddot{q}_{3r} + D_{150r}\ddot{q}_{imu} + D_{152r}\ddot{q}_{torso} + D_{148r}\ddot{x} + \mathbf{MZ}_{7r}\ddot{q}_{4r} & D_{165r} + D_{158r}\ddot{\psi} + D_{162r}\ddot{q}_{1r} + D_{160r}\ddot{q}_{w} + D_{163r}\ddot{q}_{2r} + D_{164r}\ddot{q}_{3r} + D_{159r}\ddot{q}_{imu} + D_{161r}\ddot{q}_{torso} + D_{157r}\ddot{x} - \mathbf{MY}_{7r}\ddot{q}_{4r} &  \end{matrix}\right] 
 \nonumber \\ 
 \bar\omega_{8r} &= {}^{8r}A_{7r} \bar\omega_{7r} + \dot{q}_{8r} \bar{e}_{8r} 
 \nonumber \\ 
 \bar\omega_{8r} &= \left[\begin{matrix} K_{94r}s_{5r} - K_{92r}c_{5r} & - K_{93r} - \dot{q}_{5r} & K_{94r}c_{5r} + K_{92r}s_{5r} &  \end{matrix}\right] 
 \nonumber \\ 
K_{127r} &= K_{94r}s_{5r} - K_{92r}c_{5r} \nonumber \\
K_{128r} &= - K_{93r} - \dot{q}_{5r} \nonumber \\
K_{129r} &= K_{94r}c_{5r} + K_{92r}s_{5r} \nonumber \\
 \bar\omega_{8r} &= \left[\begin{matrix} K_{127r} & K_{128r} & K_{129r} &  \end{matrix}\right] 
 \nonumber \\ 
 \bar\omega_{8r} &= \left[\begin{matrix} s_{5r}(K_{100r}\dot{\psi} + K_{103r}\dot{q}_{1r} + K_{101r}\dot{q}_{w} + K_{104r}\dot{q}_{2r} + K_{101r}\dot{q}_{imu} + K_{102r}\dot{q}_{torso} + \dot{q}_{3r}s_{4r}) + c_{5r}(\dot{q}_{4r} + K_{61r}\dot{\psi} + K_{64r}\dot{q}_{1r} + K_{62r}\dot{q}_{w} + K_{62r}\dot{q}_{imu} + K_{63r}\dot{q}_{torso} + \dot{q}_{2r}c_{3r}) & - \dot{q}_{5r} - K_{95r}\dot{\psi} - K_{98r}\dot{q}_{1r} - K_{96r}\dot{q}_{w} - K_{99r}\dot{q}_{2r} - K_{96r}\dot{q}_{imu} - K_{97r}\dot{q}_{torso} - \dot{q}_{3r}c_{4r} & c_{5r}(K_{100r}\dot{\psi} + K_{103r}\dot{q}_{1r} + K_{101r}\dot{q}_{w} + K_{104r}\dot{q}_{2r} + K_{101r}\dot{q}_{imu} + K_{102r}\dot{q}_{torso} + \dot{q}_{3r}s_{4r}) - s_{5r}(\dot{q}_{4r} + K_{61r}\dot{\psi} + K_{64r}\dot{q}_{1r} + K_{62r}\dot{q}_{w} + K_{62r}\dot{q}_{imu} + K_{63r}\dot{q}_{torso} + \dot{q}_{2r}c_{3r}) &  \end{matrix}\right] 
 \nonumber \\ 
K_{130r} &= K_{61r}c_{5r} + K_{100r}s_{5r} \nonumber \\
K_{131r} &= K_{62r}c_{5r} + K_{101r}s_{5r} \nonumber \\
K_{132r} &= K_{63r}c_{5r} + K_{102r}s_{5r} \nonumber \\
K_{133r} &= K_{64r}c_{5r} + K_{103r}s_{5r} \nonumber \\
K_{134r} &= c_{3r}c_{5r} + K_{104r}s_{5r} \nonumber \\
K_{135r} &= s_{4r}s_{5r} \nonumber \\
K_{136r} &= K_{100r}c_{5r} - K_{61r}s_{5r} \nonumber \\
K_{137r} &= K_{101r}c_{5r} - K_{62r}s_{5r} \nonumber \\
K_{138r} &= K_{102r}c_{5r} - K_{63r}s_{5r} \nonumber \\
K_{139r} &= K_{103r}c_{5r} - K_{64r}s_{5r} \nonumber \\
K_{140r} &= K_{104r}c_{5r} - c_{3r}s_{5r} \nonumber \\
K_{141r} &= c_{5r}s_{4r} \nonumber \\
 \bar\omega_{8r} &= \left[\begin{matrix} K_{130r}\dot{\psi} + K_{133r}\dot{q}_{1r} + K_{131r}\dot{q}_{w} + K_{134r}\dot{q}_{2r} + K_{135r}\dot{q}_{3r} + K_{131r}\dot{q}_{imu} + K_{132r}\dot{q}_{torso} + \dot{q}_{4r}c_{5r} & - \dot{q}_{5r} - K_{95r}\dot{\psi} - K_{98r}\dot{q}_{1r} - K_{96r}\dot{q}_{w} - K_{99r}\dot{q}_{2r} - K_{96r}\dot{q}_{imu} - K_{97r}\dot{q}_{torso} - \dot{q}_{3r}c_{4r} & K_{136r}\dot{\psi} + K_{139r}\dot{q}_{1r} + K_{137r}\dot{q}_{w} + K_{140r}\dot{q}_{2r} + K_{141r}\dot{q}_{3r} + K_{137r}\dot{q}_{imu} + K_{138r}\dot{q}_{torso} - \dot{q}_{4r}s_{5r} &  \end{matrix}\right] 
 \nonumber \\ 
 \bar{v}_{8r} &= {}^{8r}A_{7r} \left(\bar{v}_{7r} + \bar\omega_{7r} \times \bar{P}_{8r}\right) 
 \nonumber \\ 
 \bar{v}_{8r} &= \left[\begin{matrix} c_{5r}(K_{69r} - K_{94r}L_8) + s_{5r}(K_{106r} - K_{92r}L_8) & -K_{105r} & c_{5r}(K_{106r} - K_{92r}L_8) - s_{5r}(K_{69r} - K_{94r}L_8) &  \end{matrix}\right] 
 \nonumber \\ 
K_{142r} &= c_{5r}(K_{69r} - K_{94r}L_8) + s_{5r}(K_{106r}  \nonumber \\
&- K_{92r}L_8) \nonumber \\
K_{143r} &= c_{5r}(K_{106r} - K_{92r}L_8) - s_{5r}(K_{69r}  \nonumber \\
&- K_{94r}L_8) \nonumber \\
 \bar{v}_{8r} &= \left[\begin{matrix} K_{142r} & -K_{105r} & K_{143r} &  \end{matrix}\right] 
 \nonumber \\ 
 \bar{v}_{8r} &= \left[\begin{matrix} c_{5r}(K_{72r}\dot{\psi} - L_8(K_{100r}\dot{\psi} + K_{103r}\dot{q}_{1r} + K_{101r}\dot{q}_{w} + K_{104r}\dot{q}_{2r} + K_{101r}\dot{q}_{imu} + K_{102r}\dot{q}_{torso} + \dot{q}_{3r}s_{4r}) + K_{76r}\dot{q}_{1r} + K_{74r}\dot{q}_{w} + K_{77r}\dot{q}_{2r} + K_{73r}\dot{q}_{imu} + K_{75r}\dot{q}_{torso} + K_{71r}\dot{x}) + s_{5r}(K_{115r}\dot{\psi} + K_{119r}\dot{q}_{1r} + K_{117r}\dot{q}_{w} + K_{120r}\dot{q}_{2r} + K_{116r}\dot{q}_{imu} + K_{118r}\dot{q}_{torso} + K_{114r}\dot{x} + L_8(\dot{q}_{4r} + K_{61r}\dot{\psi} + K_{64r}\dot{q}_{1r} + K_{62r}\dot{q}_{w} + K_{62r}\dot{q}_{imu} + K_{63r}\dot{q}_{torso} + \dot{q}_{2r}c_{3r})) & - K_{108r}\dot{\psi} - K_{112r}\dot{q}_{1r} - K_{110r}\dot{q}_{w} - K_{113r}\dot{q}_{2r} - K_{109r}\dot{q}_{imu} - K_{111r}\dot{q}_{torso} - K_{107r}\dot{x} & c_{5r}(K_{115r}\dot{\psi} + K_{119r}\dot{q}_{1r} + K_{117r}\dot{q}_{w} + K_{120r}\dot{q}_{2r} + K_{116r}\dot{q}_{imu} + K_{118r}\dot{q}_{torso} + K_{114r}\dot{x} + L_8(\dot{q}_{4r} + K_{61r}\dot{\psi} + K_{64r}\dot{q}_{1r} + K_{62r}\dot{q}_{w} + K_{62r}\dot{q}_{imu} + K_{63r}\dot{q}_{torso} + \dot{q}_{2r}c_{3r})) - s_{5r}(K_{72r}\dot{\psi} - L_8(K_{100r}\dot{\psi} + K_{103r}\dot{q}_{1r} + K_{101r}\dot{q}_{w} + K_{104r}\dot{q}_{2r} + K_{101r}\dot{q}_{imu} + K_{102r}\dot{q}_{torso} + \dot{q}_{3r}s_{4r}) + K_{76r}\dot{q}_{1r} + K_{74r}\dot{q}_{w} + K_{77r}\dot{q}_{2r} + K_{73r}\dot{q}_{imu} + K_{75r}\dot{q}_{torso} + K_{71r}\dot{x}) &  \end{matrix}\right] 
 \nonumber \\ 
K_{144r} &= K_{71r}c_{5r} + K_{114r}s_{5r} \nonumber \\
K_{145r} &= c_{5r}(K_{72r} - K_{100r}L_8) + s_{5r}(K_{115r}  \nonumber \\
&+ K_{61r}L_8) \nonumber \\
K_{146r} &= c_{5r}(K_{73r} - K_{101r}L_8) + s_{5r}(K_{116r}  \nonumber \\
&+ K_{62r}L_8) \nonumber \\
K_{147r} &= c_{5r}(K_{74r} - K_{101r}L_8) + s_{5r}(K_{117r}  \nonumber \\
&+ K_{62r}L_8) \nonumber \\
K_{148r} &= c_{5r}(K_{75r} - K_{102r}L_8) + s_{5r}(K_{118r}  \nonumber \\
&+ K_{63r}L_8) \nonumber \\
K_{149r} &= c_{5r}(K_{76r} - K_{103r}L_8) + s_{5r}(K_{119r}  \nonumber \\
&+ K_{64r}L_8) \nonumber \\
K_{150r} &= s_{5r}(K_{120r} + L_8c_{3r}) + c_{5r}(K_{77r}  \nonumber \\
&- K_{104r}L_8) \nonumber \\
K_{151r} &= -L_8c_{5r}s_{4r} \nonumber \\
K_{152r} &= L_8s_{5r} \nonumber \\
K_{153r} &= K_{114r}c_{5r} - K_{71r}s_{5r} \nonumber \\
K_{154r} &= c_{5r}(K_{115r} + K_{61r}L_8) - s_{5r}(K_{72r}  \nonumber \\
&- K_{100r}L_8) \nonumber \\
K_{155r} &= c_{5r}(K_{116r} + K_{62r}L_8) - s_{5r}(K_{73r}  \nonumber \\
&- K_{101r}L_8) \nonumber \\
K_{156r} &= c_{5r}(K_{117r} + K_{62r}L_8) - s_{5r}(K_{74r}  \nonumber \\
&- K_{101r}L_8) \nonumber \\
K_{157r} &= c_{5r}(K_{118r} + K_{63r}L_8) - s_{5r}(K_{75r}  \nonumber \\
&- K_{102r}L_8) \nonumber \\
K_{158r} &= c_{5r}(K_{119r} + K_{64r}L_8) - s_{5r}(K_{76r}  \nonumber \\
&- K_{103r}L_8) \nonumber \\
K_{159r} &= c_{5r}(K_{120r} + L_8c_{3r}) - s_{5r}(K_{77r}  \nonumber \\
&- K_{104r}L_8) \nonumber \\
K_{160r} &= L_8s_{4r}s_{5r} \nonumber \\
K_{161r} &= L_8c_{5r} \nonumber \\
 \bar{v}_{8r} &= \left[\begin{matrix} K_{145r}\dot{\psi} + K_{149r}\dot{q}_{1r} + K_{147r}\dot{q}_{w} + K_{150r}\dot{q}_{2r} + K_{151r}\dot{q}_{3r} + K_{152r}\dot{q}_{4r} + K_{146r}\dot{q}_{imu} + K_{148r}\dot{q}_{torso} + K_{144r}\dot{x} & - K_{108r}\dot{\psi} - K_{112r}\dot{q}_{1r} - K_{110r}\dot{q}_{w} - K_{113r}\dot{q}_{2r} - K_{109r}\dot{q}_{imu} - K_{111r}\dot{q}_{torso} - K_{107r}\dot{x} & K_{154r}\dot{\psi} + K_{158r}\dot{q}_{1r} + K_{156r}\dot{q}_{w} + K_{159r}\dot{q}_{2r} + K_{160r}\dot{q}_{3r} + K_{161r}\dot{q}_{4r} + K_{155r}\dot{q}_{imu} + K_{157r}\dot{q}_{torso} + K_{153r}\dot{x} &  \end{matrix}\right] 
 \nonumber \\ 
 \bar\alpha_{8r} &= {}^{8r}A_{7r} \bar\alpha_{7r} + \ddot{q}_{8r} \bar{e}_{8r} + \dot{q}_{8r} \left(\bar\omega_{8r} \times \bar{e}_{8r}\right) 
 \nonumber \\ 
 \bar\alpha_{8r} &= \left[\begin{matrix} K_{129r}\dot{q}_{5r} + c_{5r}(K_{85r} + \ddot{q}_{4r} + K_{61r}\ddot{\psi} + K_{64r}\ddot{q}_{1r} + K_{62r}\ddot{q}_{w} + K_{62r}\ddot{q}_{imu} + K_{63r}\ddot{q}_{torso} + \ddot{q}_{2r}c_{3r}) + s_{5r}(K_{122r} + K_{100r}\ddot{\psi} + K_{103r}\ddot{q}_{1r} + K_{101r}\ddot{q}_{w} + K_{104r}\ddot{q}_{2r} + K_{101r}\ddot{q}_{imu} + K_{102r}\ddot{q}_{torso} + \ddot{q}_{3r}s_{4r}) & - K_{121r} - \ddot{q}_{5r} - K_{95r}\ddot{\psi} - K_{98r}\ddot{q}_{1r} - K_{96r}\ddot{q}_{w} - K_{99r}\ddot{q}_{2r} - K_{96r}\ddot{q}_{imu} - K_{97r}\ddot{q}_{torso} - \ddot{q}_{3r}c_{4r} & c_{5r}(K_{122r} + K_{100r}\ddot{\psi} + K_{103r}\ddot{q}_{1r} + K_{101r}\ddot{q}_{w} + K_{104r}\ddot{q}_{2r} + K_{101r}\ddot{q}_{imu} + K_{102r}\ddot{q}_{torso} + \ddot{q}_{3r}s_{4r}) - s_{5r}(K_{85r} + \ddot{q}_{4r} + K_{61r}\ddot{\psi} + K_{64r}\ddot{q}_{1r} + K_{62r}\ddot{q}_{w} + K_{62r}\ddot{q}_{imu} + K_{63r}\ddot{q}_{torso} + \ddot{q}_{2r}c_{3r}) - K_{127r}\dot{q}_{5r} &  \end{matrix}\right] 
 \nonumber \\ 
K_{162r} &= K_{129r}\dot{q}_{5r} + K_{85r}c_{5r} + K_{122r}s_{5r} \nonumber \\
K_{163r} &= K_{122r}c_{5r} - K_{127r}\dot{q}_{5r} - K_{85r}s_{5r} \nonumber \\
 \bar\alpha_{8r} &= \left[\begin{matrix} K_{162r} + K_{130r}\ddot{\psi} + K_{133r}\ddot{q}_{1r} + K_{131r}\ddot{q}_{w} + K_{134r}\ddot{q}_{2r} + K_{135r}\ddot{q}_{3r} + K_{131r}\ddot{q}_{imu} + K_{132r}\ddot{q}_{torso} + \ddot{q}_{4r}c_{5r} & - K_{121r} - \ddot{q}_{5r} - K_{95r}\ddot{\psi} - K_{98r}\ddot{q}_{1r} - K_{96r}\ddot{q}_{w} - K_{99r}\ddot{q}_{2r} - K_{96r}\ddot{q}_{imu} - K_{97r}\ddot{q}_{torso} - \ddot{q}_{3r}c_{4r} & K_{163r} + K_{136r}\ddot{\psi} + K_{139r}\ddot{q}_{1r} + K_{137r}\ddot{q}_{w} + K_{140r}\ddot{q}_{2r} + K_{141r}\ddot{q}_{3r} + K_{137r}\ddot{q}_{imu} + K_{138r}\ddot{q}_{torso} - \ddot{q}_{4r}s_{5r} &  \end{matrix}\right] 
 \nonumber \\ 
 \bar{a}_{8r} &= {}^{8r}A_{7r} \left(\bar{a}_{7r} + \bar\alpha_{7r} \times \bar{P}_{8r} + \bar\omega_{7r} \times \left(\bar\omega_{7r} \times \bar{P}_{8r}\right)\right) 
 \nonumber \\ 
 \bar\alpha_{8r} &= \left[\begin{matrix} s_{5r}(K_{124r} + K_{115r}\ddot{\psi} + K_{119r}\ddot{q}_{1r} + K_{117r}\ddot{q}_{w} + K_{120r}\ddot{q}_{2r} + K_{116r}\ddot{q}_{imu} + K_{118r}\ddot{q}_{torso} + K_{114r}\ddot{x} + L_8(K_{85r} + \ddot{q}_{4r} + K_{61r}\ddot{\psi} + K_{64r}\ddot{q}_{1r} + K_{62r}\ddot{q}_{w} + K_{62r}\ddot{q}_{imu} + K_{63r}\ddot{q}_{torso} + \ddot{q}_{2r}c_{3r}) - K_{93r}K_{94r}L_8) + c_{5r}(K_{87r} + K_{72r}\ddot{\psi} + K_{76r}\ddot{q}_{1r} + K_{74r}\ddot{q}_{w} + K_{77r}\ddot{q}_{2r} + K_{73r}\ddot{q}_{imu} + K_{75r}\ddot{q}_{torso} + K_{71r}\ddot{x} - L_8(K_{122r} + K_{100r}\ddot{\psi} + K_{103r}\ddot{q}_{1r} + K_{101r}\ddot{q}_{w} + K_{104r}\ddot{q}_{2r} + K_{101r}\ddot{q}_{imu} + K_{102r}\ddot{q}_{torso} + \ddot{q}_{3r}s_{4r}) + K_{92r}K_{93r}L_8) & - K_{123r} - K_{108r}\ddot{\psi} - K_{112r}\ddot{q}_{1r} - K_{110r}\ddot{q}_{w} - K_{113r}\ddot{q}_{2r} - K_{109r}\ddot{q}_{imu} - K_{111r}\ddot{q}_{torso} - K_{107r}\ddot{x} - K_{92r}^2L_8 - K_{94r}^2L_8 & c_{5r}(K_{124r} + K_{115r}\ddot{\psi} + K_{119r}\ddot{q}_{1r} + K_{117r}\ddot{q}_{w} + K_{120r}\ddot{q}_{2r} + K_{116r}\ddot{q}_{imu} + K_{118r}\ddot{q}_{torso} + K_{114r}\ddot{x} + L_8(K_{85r} + \ddot{q}_{4r} + K_{61r}\ddot{\psi} + K_{64r}\ddot{q}_{1r} + K_{62r}\ddot{q}_{w} + K_{62r}\ddot{q}_{imu} + K_{63r}\ddot{q}_{torso} + \ddot{q}_{2r}c_{3r}) - K_{93r}K_{94r}L_8) - s_{5r}(K_{87r} + K_{72r}\ddot{\psi} + K_{76r}\ddot{q}_{1r} + K_{74r}\ddot{q}_{w} + K_{77r}\ddot{q}_{2r} + K_{73r}\ddot{q}_{imu} + K_{75r}\ddot{q}_{torso} + K_{71r}\ddot{x} - L_8(K_{122r} + K_{100r}\ddot{\psi} + K_{103r}\ddot{q}_{1r} + K_{101r}\ddot{q}_{w} + K_{104r}\ddot{q}_{2r} + K_{101r}\ddot{q}_{imu} + K_{102r}\ddot{q}_{torso} + \ddot{q}_{3r}s_{4r}) + K_{92r}K_{93r}L_8) &  \end{matrix}\right] 
 \nonumber \\ 
K_{164r} &= K_{87r}c_{5r} + K_{124r}s_{5r} - K_{122r}L_8c_{5r}  \nonumber \\
&+ K_{85r}L_8s_{5r} + K_{92r}K_{93r}L_8c_{5r}  \nonumber \\
&- K_{93r}K_{94r}L_8s_{5r} \nonumber \\
K_{165r} &= - K_{123r} - K_{92r}^2L_8 - K_{94r}^2L_8 \nonumber \\
K_{166r} &= K_{124r}c_{5r} - K_{87r}s_{5r} + K_{85r}L_8c_{5r}  \nonumber \\
&+ K_{122r}L_8s_{5r} - K_{93r}K_{94r}L_8c_{5r}  \nonumber \\
&- K_{92r}K_{93r}L_8s_{5r} \nonumber \\
 \bar{a}_{8r} &= \left[\begin{matrix} K_{164r} + K_{145r}\ddot{\psi} + K_{149r}\ddot{q}_{1r} + K_{147r}\ddot{q}_{w} + K_{150r}\ddot{q}_{2r} + K_{151r}\ddot{q}_{3r} + K_{152r}\ddot{q}_{4r} + K_{146r}\ddot{q}_{imu} + K_{148r}\ddot{q}_{torso} + K_{144r}\ddot{x} & K_{165r} - K_{108r}\ddot{\psi} - K_{112r}\ddot{q}_{1r} - K_{110r}\ddot{q}_{w} - K_{113r}\ddot{q}_{2r} - K_{109r}\ddot{q}_{imu} - K_{111r}\ddot{q}_{torso} - K_{107r}\ddot{x} & K_{166r} + K_{154r}\ddot{\psi} + K_{158r}\ddot{q}_{1r} + K_{156r}\ddot{q}_{w} + K_{159r}\ddot{q}_{2r} + K_{160r}\ddot{q}_{3r} + K_{161r}\ddot{q}_{4r} + K_{155r}\ddot{q}_{imu} + K_{157r}\ddot{q}_{torso} + K_{153r}\ddot{x} &  \end{matrix}\right] 
 \nonumber \\ 
 \bar{g}_{8r} &= {}^{8r}A_{7r} \bar{g}_{7r} 
 \nonumber \\ 
 \bar{g}_{8r} &= \left[\begin{matrix} K_{90r}gc_{5r} + K_{126r}gs_{5r} & -K_{125r}g & K_{126r}gc_{5r} - K_{90r}gs_{5r} &  \end{matrix}\right] 
 \nonumber \\ 
K_{167r} &= K_{90r}c_{5r} + K_{126r}s_{5r} \nonumber \\
K_{168r} &= K_{126r}c_{5r} - K_{90r}s_{5r} \nonumber \\
 \bar{g}_{8r} &= \left[\begin{matrix} K_{167r}g & -K_{125r}g & K_{168r}g &  \end{matrix}\right] 
 \nonumber \\ 
 m_{8r}\bar{S}_{8r}^{\times}\bar{g}_{8r} &= \mathbf{MS}_{8r} \times \bar{g}_{8r} 
 \nonumber \\ 
 m_{8r}\bar{S}_{8r}^{\times}\bar{g}_{8r} &= \left[\begin{matrix} K_{168r}\mathbf{MY}_{8r}g + K_{125r}\mathbf{MZ}_{8r}g & K_{167r}\mathbf{MZ}_{8r}g - K_{168r}\mathbf{MX}_{8r}g & - K_{125r}\mathbf{MX}_{8r}g - K_{167r}\mathbf{MY}_{8r}g &  \end{matrix}\right] 
 \nonumber \\ 
D_{193r} &= K_{168r}\mathbf{MY}_{8r} + K_{125r}\mathbf{MZ}_{8r} \nonumber \\
D_{194r} &= K_{167r}\mathbf{MZ}_{8r} - K_{168r}\mathbf{MX}_{8r} \nonumber \\
D_{195r} &= - K_{125r}\mathbf{MX}_{8r} - K_{167r}\mathbf{MY}_{8r} \nonumber \\
 m_{8r}\bar{S}_{8r}^{\times}\bar{g}_{8r} &= \left[\begin{matrix} D_{193r}g & D_{194r}g & D_{195r}g &  \end{matrix}\right] 
 \nonumber \\ 
 m_{8r}\bar{a}_{G(8r)} &= m_{8r}\bar{a}_{8r} + \bar\alpha_{8r} \times \mathbf{MS}_{8r} + \bar\omega_{8r} \times \left(\bar\omega_{8r} \times \mathbf{MS}_{8r}\right) 
 \nonumber \\ 
 m_{8r}\bar{a}_{G(8r)} &= \left[\begin{matrix} m_{8r}(K_{164r} + K_{145r}\ddot{\psi} + K_{149r}\ddot{q}_{1r} + K_{147r}\ddot{q}_{w} + K_{150r}\ddot{q}_{2r} + K_{151r}\ddot{q}_{3r} + K_{152r}\ddot{q}_{4r} + K_{146r}\ddot{q}_{imu} + K_{148r}\ddot{q}_{torso} + K_{144r}\ddot{x}) - \mathbf{MZ}_{8r}(K_{121r} + \ddot{q}_{5r} + K_{95r}\ddot{\psi} + K_{98r}\ddot{q}_{1r} + K_{96r}\ddot{q}_{w} + K_{99r}\ddot{q}_{2r} + K_{96r}\ddot{q}_{imu} + K_{97r}\ddot{q}_{torso} + \ddot{q}_{3r}c_{4r}) - K_{128r}(K_{128r}\mathbf{MX}_{8r} - K_{127r}\mathbf{MY}_{8r}) - K_{129r}(K_{129r}\mathbf{MX}_{8r} - K_{127r}\mathbf{MZ}_{8r}) - \mathbf{MY}_{8r}(K_{163r} + K_{136r}\ddot{\psi} + K_{139r}\ddot{q}_{1r} + K_{137r}\ddot{q}_{w} + K_{140r}\ddot{q}_{2r} + K_{141r}\ddot{q}_{3r} + K_{137r}\ddot{q}_{imu} + K_{138r}\ddot{q}_{torso} - \ddot{q}_{4r}s_{5r}) & \mathbf{MX}_{8r}(K_{163r} + K_{136r}\ddot{\psi} + K_{139r}\ddot{q}_{1r} + K_{137r}\ddot{q}_{w} + K_{140r}\ddot{q}_{2r} + K_{141r}\ddot{q}_{3r} + K_{137r}\ddot{q}_{imu} + K_{138r}\ddot{q}_{torso} - \ddot{q}_{4r}s_{5r}) - \mathbf{MZ}_{8r}(K_{162r} + K_{130r}\ddot{\psi} + K_{133r}\ddot{q}_{1r} + K_{131r}\ddot{q}_{w} + K_{134r}\ddot{q}_{2r} + K_{135r}\ddot{q}_{3r} + K_{131r}\ddot{q}_{imu} + K_{132r}\ddot{q}_{torso} + \ddot{q}_{4r}c_{5r}) - m_{8r}(K_{108r}\ddot{\psi} - K_{165r} + K_{112r}\ddot{q}_{1r} + K_{110r}\ddot{q}_{w} + K_{113r}\ddot{q}_{2r} + K_{109r}\ddot{q}_{imu} + K_{111r}\ddot{q}_{torso} + K_{107r}\ddot{x}) + K_{127r}(K_{128r}\mathbf{MX}_{8r} - K_{127r}\mathbf{MY}_{8r}) - K_{129r}(K_{129r}\mathbf{MY}_{8r} - K_{128r}\mathbf{MZ}_{8r}) & \mathbf{MY}_{8r}(K_{162r} + K_{130r}\ddot{\psi} + K_{133r}\ddot{q}_{1r} + K_{131r}\ddot{q}_{w} + K_{134r}\ddot{q}_{2r} + K_{135r}\ddot{q}_{3r} + K_{131r}\ddot{q}_{imu} + K_{132r}\ddot{q}_{torso} + \ddot{q}_{4r}c_{5r}) + \mathbf{MX}_{8r}(K_{121r} + \ddot{q}_{5r} + K_{95r}\ddot{\psi} + K_{98r}\ddot{q}_{1r} + K_{96r}\ddot{q}_{w} + K_{99r}\ddot{q}_{2r} + K_{96r}\ddot{q}_{imu} + K_{97r}\ddot{q}_{torso} + \ddot{q}_{3r}c_{4r}) + K_{127r}(K_{129r}\mathbf{MX}_{8r} - K_{127r}\mathbf{MZ}_{8r}) + K_{128r}(K_{129r}\mathbf{MY}_{8r} - K_{128r}\mathbf{MZ}_{8r}) + m_{8r}(K_{166r} + K_{154r}\ddot{\psi} + K_{158r}\ddot{q}_{1r} + K_{156r}\ddot{q}_{w} + K_{159r}\ddot{q}_{2r} + K_{160r}\ddot{q}_{3r} + K_{161r}\ddot{q}_{4r} + K_{155r}\ddot{q}_{imu} + K_{157r}\ddot{q}_{torso} + K_{153r}\ddot{x}) &  \end{matrix}\right] 
 \nonumber \\ 
D_{196r} &= K_{144r}m_{8r} \nonumber \\
D_{197r} &= K_{145r}m_{8r} - K_{136r}\mathbf{MY}_{8r} - K_{95r}\mathbf{MZ}_{8r} \nonumber \\
D_{198r} &= K_{146r}m_{8r} - K_{137r}\mathbf{MY}_{8r} - K_{96r}\mathbf{MZ}_{8r} \nonumber \\
D_{199r} &= K_{147r}m_{8r} - K_{137r}\mathbf{MY}_{8r} - K_{96r}\mathbf{MZ}_{8r} \nonumber \\
D_{200r} &= K_{148r}m_{8r} - K_{138r}\mathbf{MY}_{8r} - K_{97r}\mathbf{MZ}_{8r} \nonumber \\
D_{201r} &= K_{149r}m_{8r} - K_{139r}\mathbf{MY}_{8r} - K_{98r}\mathbf{MZ}_{8r} \nonumber \\
D_{202r} &= K_{150r}m_{8r} - K_{140r}\mathbf{MY}_{8r} - K_{99r}\mathbf{MZ}_{8r} \nonumber \\
D_{203r} &= K_{151r}m_{8r} - \mathbf{MZ}_{8r}c_{4r} - K_{141r}\mathbf{MY}_{8r} \nonumber \\
D_{204r} &= K_{152r}m_{8r} + \mathbf{MY}_{8r}s_{5r} \nonumber \\
D_{205r} &= K_{164r}m_{8r} - K_{128r}^2\mathbf{MX}_{8r} - K_{129r}^2\mathbf{MX}_{8r}  \nonumber \\
&- K_{163r}\mathbf{MY}_{8r} - K_{121r}\mathbf{MZ}_{8r} + K_{127r}K_{128r}\mathbf{MY}_{8r}  \nonumber \\
&+ K_{127r}K_{129r}\mathbf{MZ}_{8r} \nonumber \\
D_{206r} &= -K_{107r}m_{8r} \nonumber \\
D_{207r} &= K_{136r}\mathbf{MX}_{8r} - K_{108r}m_{8r} - K_{130r}\mathbf{MZ}_{8r} \nonumber \\
D_{208r} &= K_{137r}\mathbf{MX}_{8r} - K_{109r}m_{8r} - K_{131r}\mathbf{MZ}_{8r} \nonumber \\
D_{209r} &= K_{137r}\mathbf{MX}_{8r} - K_{110r}m_{8r} - K_{131r}\mathbf{MZ}_{8r} \nonumber \\
D_{210r} &= K_{138r}\mathbf{MX}_{8r} - K_{111r}m_{8r} - K_{132r}\mathbf{MZ}_{8r} \nonumber \\
D_{211r} &= K_{139r}\mathbf{MX}_{8r} - K_{112r}m_{8r} - K_{133r}\mathbf{MZ}_{8r} \nonumber \\
D_{212r} &= K_{140r}\mathbf{MX}_{8r} - K_{113r}m_{8r} - K_{134r}\mathbf{MZ}_{8r} \nonumber \\
D_{213r} &= K_{141r}\mathbf{MX}_{8r} - K_{135r}\mathbf{MZ}_{8r} \nonumber \\
D_{214r} &= - \mathbf{MZ}_{8r}c_{5r} - \mathbf{MX}_{8r}s_{5r} \nonumber \\
D_{215r} &= K_{165r}m_{8r} - K_{127r}^2\mathbf{MY}_{8r} - K_{129r}^2\mathbf{MY}_{8r}  \nonumber \\
&+ K_{163r}\mathbf{MX}_{8r} - K_{162r}\mathbf{MZ}_{8r} + K_{127r}K_{128r}\mathbf{MX}_{8r}  \nonumber \\
&+ K_{128r}K_{129r}\mathbf{MZ}_{8r} \nonumber \\
D_{216r} &= K_{153r}m_{8r} \nonumber \\
D_{217r} &= K_{154r}m_{8r} + K_{95r}\mathbf{MX}_{8r} + K_{130r}\mathbf{MY}_{8r} \nonumber \\
D_{218r} &= K_{155r}m_{8r} + K_{96r}\mathbf{MX}_{8r} + K_{131r}\mathbf{MY}_{8r} \nonumber \\
D_{219r} &= K_{156r}m_{8r} + K_{96r}\mathbf{MX}_{8r} + K_{131r}\mathbf{MY}_{8r} \nonumber \\
D_{220r} &= K_{157r}m_{8r} + K_{97r}\mathbf{MX}_{8r} + K_{132r}\mathbf{MY}_{8r} \nonumber \\
D_{221r} &= K_{158r}m_{8r} + K_{98r}\mathbf{MX}_{8r} + K_{133r}\mathbf{MY}_{8r} \nonumber \\
D_{222r} &= K_{159r}m_{8r} + K_{99r}\mathbf{MX}_{8r} + K_{134r}\mathbf{MY}_{8r} \nonumber \\
D_{223r} &= K_{160r}m_{8r} + \mathbf{MX}_{8r}c_{4r} + K_{135r}\mathbf{MY}_{8r} \nonumber \\
D_{224r} &= K_{161r}m_{8r} + \mathbf{MY}_{8r}c_{5r} \nonumber \\
D_{225r} &= K_{166r}m_{8r} - K_{127r}^2\mathbf{MZ}_{8r} - K_{128r}^2\mathbf{MZ}_{8r}  \nonumber \\
&+ K_{121r}\mathbf{MX}_{8r} + K_{162r}\mathbf{MY}_{8r} + K_{127r}K_{129r}\mathbf{MX}_{8r}  \nonumber \\
&+ K_{128r}K_{129r}\mathbf{MY}_{8r} \nonumber \\
 m_{8r}\bar{a}_{G(8r)} &= \left[\begin{matrix} D_{205r} + D_{197r}\ddot{\psi} + D_{201r}\ddot{q}_{1r} + D_{199r}\ddot{q}_{w} + D_{202r}\ddot{q}_{2r} + D_{203r}\ddot{q}_{3r} + D_{204r}\ddot{q}_{4r} + D_{198r}\ddot{q}_{imu} + D_{200r}\ddot{q}_{torso} + D_{196r}\ddot{x} - \mathbf{MZ}_{8r}\ddot{q}_{5r} & D_{215r} + D_{207r}\ddot{\psi} + D_{211r}\ddot{q}_{1r} + D_{209r}\ddot{q}_{w} + D_{212r}\ddot{q}_{2r} + D_{213r}\ddot{q}_{3r} + D_{214r}\ddot{q}_{4r} + D_{208r}\ddot{q}_{imu} + D_{210r}\ddot{q}_{torso} + D_{206r}\ddot{x} & D_{225r} + D_{217r}\ddot{\psi} + D_{221r}\ddot{q}_{1r} + D_{219r}\ddot{q}_{w} + D_{222r}\ddot{q}_{2r} + D_{223r}\ddot{q}_{3r} + D_{224r}\ddot{q}_{4r} + D_{218r}\ddot{q}_{imu} + D_{220r}\ddot{q}_{torso} + D_{216r}\ddot{x} + \mathbf{MX}_{8r}\ddot{q}_{5r} &  \end{matrix}\right] 
 \nonumber \\ 
 \dot{\bar{H}}_{8r} &= \mathbf{MS}_{8r} \times \bar{a}_{8r} + J_{8r}\bar{\alpha}_{8r} + \bar\omega_{8r} \times J_{8r}\bar{\omega}_{8r} 
 \nonumber \\ 
 \dot{\bar{H}}_{8r} &= \left[\begin{matrix} K_{128r}(K_{127r}\mathbf{XZ}_{8r} + K_{128r}\mathbf{YZ}_{8r} + K_{129r}\mathbf{ZZ}_{8r}) - K_{129r}(K_{127r}\mathbf{XY}_{8r} + K_{128r}\mathbf{YY}_{8r} + K_{129r}\mathbf{YZ}_{8r}) + \mathbf{MZ}_{8r}(K_{108r}\ddot{\psi} - K_{165r} + K_{112r}\ddot{q}_{1r} + K_{110r}\ddot{q}_{w} + K_{113r}\ddot{q}_{2r} + K_{109r}\ddot{q}_{imu} + K_{111r}\ddot{q}_{torso} + K_{107r}\ddot{x}) + \mathbf{XX}_{8r}(K_{162r} + K_{130r}\ddot{\psi} + K_{133r}\ddot{q}_{1r} + K_{131r}\ddot{q}_{w} + K_{134r}\ddot{q}_{2r} + K_{135r}\ddot{q}_{3r} + K_{131r}\ddot{q}_{imu} + K_{132r}\ddot{q}_{torso} + \ddot{q}_{4r}c_{5r}) + \mathbf{XZ}_{8r}(K_{163r} + K_{136r}\ddot{\psi} + K_{139r}\ddot{q}_{1r} + K_{137r}\ddot{q}_{w} + K_{140r}\ddot{q}_{2r} + K_{141r}\ddot{q}_{3r} + K_{137r}\ddot{q}_{imu} + K_{138r}\ddot{q}_{torso} - \ddot{q}_{4r}s_{5r}) - \mathbf{XY}_{8r}(K_{121r} + \ddot{q}_{5r} + K_{95r}\ddot{\psi} + K_{98r}\ddot{q}_{1r} + K_{96r}\ddot{q}_{w} + K_{99r}\ddot{q}_{2r} + K_{96r}\ddot{q}_{imu} + K_{97r}\ddot{q}_{torso} + \ddot{q}_{3r}c_{4r}) + \mathbf{MY}_{8r}(K_{166r} + K_{154r}\ddot{\psi} + K_{158r}\ddot{q}_{1r} + K_{156r}\ddot{q}_{w} + K_{159r}\ddot{q}_{2r} + K_{160r}\ddot{q}_{3r} + K_{161r}\ddot{q}_{4r} + K_{155r}\ddot{q}_{imu} + K_{157r}\ddot{q}_{torso} + K_{153r}\ddot{x}) & K_{129r}(K_{127r}\mathbf{XX}_{8r} + K_{128r}\mathbf{XY}_{8r} + K_{129r}\mathbf{XZ}_{8r}) - K_{127r}(K_{127r}\mathbf{XZ}_{8r} + K_{128r}\mathbf{YZ}_{8r} + K_{129r}\mathbf{ZZ}_{8r}) + \mathbf{XY}_{8r}(K_{162r} + K_{130r}\ddot{\psi} + K_{133r}\ddot{q}_{1r} + K_{131r}\ddot{q}_{w} + K_{134r}\ddot{q}_{2r} + K_{135r}\ddot{q}_{3r} + K_{131r}\ddot{q}_{imu} + K_{132r}\ddot{q}_{torso} + \ddot{q}_{4r}c_{5r}) + \mathbf{YZ}_{8r}(K_{163r} + K_{136r}\ddot{\psi} + K_{139r}\ddot{q}_{1r} + K_{137r}\ddot{q}_{w} + K_{140r}\ddot{q}_{2r} + K_{141r}\ddot{q}_{3r} + K_{137r}\ddot{q}_{imu} + K_{138r}\ddot{q}_{torso} - \ddot{q}_{4r}s_{5r}) - \mathbf{YY}_{8r}(K_{121r} + \ddot{q}_{5r} + K_{95r}\ddot{\psi} + K_{98r}\ddot{q}_{1r} + K_{96r}\ddot{q}_{w} + K_{99r}\ddot{q}_{2r} + K_{96r}\ddot{q}_{imu} + K_{97r}\ddot{q}_{torso} + \ddot{q}_{3r}c_{4r}) - \mathbf{MX}_{8r}(K_{166r} + K_{154r}\ddot{\psi} + K_{158r}\ddot{q}_{1r} + K_{156r}\ddot{q}_{w} + K_{159r}\ddot{q}_{2r} + K_{160r}\ddot{q}_{3r} + K_{161r}\ddot{q}_{4r} + K_{155r}\ddot{q}_{imu} + K_{157r}\ddot{q}_{torso} + K_{153r}\ddot{x}) + \mathbf{MZ}_{8r}(K_{164r} + K_{145r}\ddot{\psi} + K_{149r}\ddot{q}_{1r} + K_{147r}\ddot{q}_{w} + K_{150r}\ddot{q}_{2r} + K_{151r}\ddot{q}_{3r} + K_{152r}\ddot{q}_{4r} + K_{146r}\ddot{q}_{imu} + K_{148r}\ddot{q}_{torso} + K_{144r}\ddot{x}) & K_{127r}(K_{127r}\mathbf{XY}_{8r} + K_{128r}\mathbf{YY}_{8r} + K_{129r}\mathbf{YZ}_{8r}) - K_{128r}(K_{127r}\mathbf{XX}_{8r} + K_{128r}\mathbf{XY}_{8r} + K_{129r}\mathbf{XZ}_{8r}) - \mathbf{MX}_{8r}(K_{108r}\ddot{\psi} - K_{165r} + K_{112r}\ddot{q}_{1r} + K_{110r}\ddot{q}_{w} + K_{113r}\ddot{q}_{2r} + K_{109r}\ddot{q}_{imu} + K_{111r}\ddot{q}_{torso} + K_{107r}\ddot{x}) + \mathbf{XZ}_{8r}(K_{162r} + K_{130r}\ddot{\psi} + K_{133r}\ddot{q}_{1r} + K_{131r}\ddot{q}_{w} + K_{134r}\ddot{q}_{2r} + K_{135r}\ddot{q}_{3r} + K_{131r}\ddot{q}_{imu} + K_{132r}\ddot{q}_{torso} + \ddot{q}_{4r}c_{5r}) + \mathbf{ZZ}_{8r}(K_{163r} + K_{136r}\ddot{\psi} + K_{139r}\ddot{q}_{1r} + K_{137r}\ddot{q}_{w} + K_{140r}\ddot{q}_{2r} + K_{141r}\ddot{q}_{3r} + K_{137r}\ddot{q}_{imu} + K_{138r}\ddot{q}_{torso} - \ddot{q}_{4r}s_{5r}) - \mathbf{YZ}_{8r}(K_{121r} + \ddot{q}_{5r} + K_{95r}\ddot{\psi} + K_{98r}\ddot{q}_{1r} + K_{96r}\ddot{q}_{w} + K_{99r}\ddot{q}_{2r} + K_{96r}\ddot{q}_{imu} + K_{97r}\ddot{q}_{torso} + \ddot{q}_{3r}c_{4r}) - \mathbf{MY}_{8r}(K_{164r} + K_{145r}\ddot{\psi} + K_{149r}\ddot{q}_{1r} + K_{147r}\ddot{q}_{w} + K_{150r}\ddot{q}_{2r} + K_{151r}\ddot{q}_{3r} + K_{152r}\ddot{q}_{4r} + K_{146r}\ddot{q}_{imu} + K_{148r}\ddot{q}_{torso} + K_{144r}\ddot{x}) &  \end{matrix}\right] 
 \nonumber \\ 
D_{226r} &= K_{153r}\mathbf{MY}_{8r} + K_{107r}\mathbf{MZ}_{8r} \nonumber \\
D_{227r} &= K_{130r}\mathbf{XX}_{8r} - K_{95r}\mathbf{XY}_{8r} + K_{136r}\mathbf{XZ}_{8r}  \nonumber \\
&+ K_{154r}\mathbf{MY}_{8r} + K_{108r}\mathbf{MZ}_{8r} \nonumber \\
D_{228r} &= K_{131r}\mathbf{XX}_{8r} - K_{96r}\mathbf{XY}_{8r} + K_{137r}\mathbf{XZ}_{8r}  \nonumber \\
&+ K_{155r}\mathbf{MY}_{8r} + K_{109r}\mathbf{MZ}_{8r} \nonumber \\
D_{229r} &= K_{131r}\mathbf{XX}_{8r} - K_{96r}\mathbf{XY}_{8r} + K_{137r}\mathbf{XZ}_{8r}  \nonumber \\
&+ K_{156r}\mathbf{MY}_{8r} + K_{110r}\mathbf{MZ}_{8r} \nonumber \\
D_{230r} &= K_{132r}\mathbf{XX}_{8r} - K_{97r}\mathbf{XY}_{8r} + K_{138r}\mathbf{XZ}_{8r}  \nonumber \\
&+ K_{157r}\mathbf{MY}_{8r} + K_{111r}\mathbf{MZ}_{8r} \nonumber \\
D_{231r} &= K_{133r}\mathbf{XX}_{8r} - K_{98r}\mathbf{XY}_{8r} + K_{139r}\mathbf{XZ}_{8r}  \nonumber \\
&+ K_{158r}\mathbf{MY}_{8r} + K_{112r}\mathbf{MZ}_{8r} \nonumber \\
D_{232r} &= K_{134r}\mathbf{XX}_{8r} - K_{99r}\mathbf{XY}_{8r} + K_{140r}\mathbf{XZ}_{8r}  \nonumber \\
&+ K_{159r}\mathbf{MY}_{8r} + K_{113r}\mathbf{MZ}_{8r} \nonumber \\
D_{233r} &= K_{135r}\mathbf{XX}_{8r} + K_{141r}\mathbf{XZ}_{8r} - \mathbf{XY}_{8r}c_{4r}  \nonumber \\
&+ K_{160r}\mathbf{MY}_{8r} \nonumber \\
D_{234r} &= \mathbf{XX}_{8r}c_{5r} - \mathbf{XZ}_{8r}s_{5r} + K_{161r}\mathbf{MY}_{8r} \nonumber \\
D_{235r} &= K_{162r}\mathbf{XX}_{8r} - K_{121r}\mathbf{XY}_{8r} + K_{163r}\mathbf{XZ}_{8r}  \nonumber \\
&+ K_{128r}^2\mathbf{YZ}_{8r} - K_{129r}^2\mathbf{YZ}_{8r} + K_{166r}\mathbf{MY}_{8r}  \nonumber \\
&- K_{165r}\mathbf{MZ}_{8r} - K_{127r}K_{129r}\mathbf{XY}_{8r} + K_{127r}K_{128r}\mathbf{XZ}_{8r}  \nonumber \\
&- K_{128r}K_{129r}\mathbf{YY}_{8r} + K_{128r}K_{129r}\mathbf{ZZ}_{8r} \nonumber \\
D_{236r} &= K_{144r}\mathbf{MZ}_{8r} - K_{153r}\mathbf{MX}_{8r} \nonumber \\
D_{237r} &= K_{130r}\mathbf{XY}_{8r} - K_{95r}\mathbf{YY}_{8r} + K_{136r}\mathbf{YZ}_{8r}  \nonumber \\
&- K_{154r}\mathbf{MX}_{8r} + K_{145r}\mathbf{MZ}_{8r} \nonumber \\
D_{238r} &= K_{131r}\mathbf{XY}_{8r} - K_{96r}\mathbf{YY}_{8r} + K_{137r}\mathbf{YZ}_{8r}  \nonumber \\
&- K_{155r}\mathbf{MX}_{8r} + K_{146r}\mathbf{MZ}_{8r} \nonumber \\
D_{239r} &= K_{131r}\mathbf{XY}_{8r} - K_{96r}\mathbf{YY}_{8r} + K_{137r}\mathbf{YZ}_{8r}  \nonumber \\
&- K_{156r}\mathbf{MX}_{8r} + K_{147r}\mathbf{MZ}_{8r} \nonumber \\
D_{240r} &= K_{132r}\mathbf{XY}_{8r} - K_{97r}\mathbf{YY}_{8r} + K_{138r}\mathbf{YZ}_{8r}  \nonumber \\
&- K_{157r}\mathbf{MX}_{8r} + K_{148r}\mathbf{MZ}_{8r} \nonumber \\
D_{241r} &= K_{133r}\mathbf{XY}_{8r} - K_{98r}\mathbf{YY}_{8r} + K_{139r}\mathbf{YZ}_{8r}  \nonumber \\
&- K_{158r}\mathbf{MX}_{8r} + K_{149r}\mathbf{MZ}_{8r} \nonumber \\
D_{242r} &= K_{134r}\mathbf{XY}_{8r} - K_{99r}\mathbf{YY}_{8r} + K_{140r}\mathbf{YZ}_{8r}  \nonumber \\
&- K_{159r}\mathbf{MX}_{8r} + K_{150r}\mathbf{MZ}_{8r} \nonumber \\
D_{243r} &= K_{135r}\mathbf{XY}_{8r} + K_{141r}\mathbf{YZ}_{8r} - \mathbf{YY}_{8r}c_{4r}  \nonumber \\
&- K_{160r}\mathbf{MX}_{8r} + K_{151r}\mathbf{MZ}_{8r} \nonumber \\
D_{244r} &= \mathbf{XY}_{8r}c_{5r} - \mathbf{YZ}_{8r}s_{5r} - K_{161r}\mathbf{MX}_{8r}  \nonumber \\
&+ K_{152r}\mathbf{MZ}_{8r} \nonumber \\
D_{245r} &= K_{162r}\mathbf{XY}_{8r} - K_{121r}\mathbf{YY}_{8r} + K_{163r}\mathbf{YZ}_{8r}  \nonumber \\
&- K_{127r}^2\mathbf{XZ}_{8r} + K_{129r}^2\mathbf{XZ}_{8r} - K_{166r}\mathbf{MX}_{8r}  \nonumber \\
&+ K_{164r}\mathbf{MZ}_{8r} + K_{127r}K_{129r}\mathbf{XX}_{8r} + K_{128r}K_{129r}\mathbf{XY}_{8r}  \nonumber \\
&- K_{127r}K_{128r}\mathbf{YZ}_{8r} - K_{127r}K_{129r}\mathbf{ZZ}_{8r} \nonumber \\
D_{246r} &= - K_{107r}\mathbf{MX}_{8r} - K_{144r}\mathbf{MY}_{8r} \nonumber \\
D_{247r} &= K_{130r}\mathbf{XZ}_{8r} - K_{95r}\mathbf{YZ}_{8r} + K_{136r}\mathbf{ZZ}_{8r}  \nonumber \\
&- K_{108r}\mathbf{MX}_{8r} - K_{145r}\mathbf{MY}_{8r} \nonumber \\
D_{248r} &= K_{131r}\mathbf{XZ}_{8r} - K_{96r}\mathbf{YZ}_{8r} + K_{137r}\mathbf{ZZ}_{8r}  \nonumber \\
&- K_{109r}\mathbf{MX}_{8r} - K_{146r}\mathbf{MY}_{8r} \nonumber \\
D_{249r} &= K_{131r}\mathbf{XZ}_{8r} - K_{96r}\mathbf{YZ}_{8r} + K_{137r}\mathbf{ZZ}_{8r}  \nonumber \\
&- K_{110r}\mathbf{MX}_{8r} - K_{147r}\mathbf{MY}_{8r} \nonumber \\
D_{250r} &= K_{132r}\mathbf{XZ}_{8r} - K_{97r}\mathbf{YZ}_{8r} + K_{138r}\mathbf{ZZ}_{8r}  \nonumber \\
&- K_{111r}\mathbf{MX}_{8r} - K_{148r}\mathbf{MY}_{8r} \nonumber \\
D_{251r} &= K_{133r}\mathbf{XZ}_{8r} - K_{98r}\mathbf{YZ}_{8r} + K_{139r}\mathbf{ZZ}_{8r}  \nonumber \\
&- K_{112r}\mathbf{MX}_{8r} - K_{149r}\mathbf{MY}_{8r} \nonumber \\
D_{252r} &= K_{134r}\mathbf{XZ}_{8r} - K_{99r}\mathbf{YZ}_{8r} + K_{140r}\mathbf{ZZ}_{8r}  \nonumber \\
&- K_{113r}\mathbf{MX}_{8r} - K_{150r}\mathbf{MY}_{8r} \nonumber \\
D_{253r} &= K_{135r}\mathbf{XZ}_{8r} + K_{141r}\mathbf{ZZ}_{8r} - \mathbf{YZ}_{8r}c_{4r}  \nonumber \\
&- K_{151r}\mathbf{MY}_{8r} \nonumber \\
D_{254r} &= \mathbf{XZ}_{8r}c_{5r} - \mathbf{ZZ}_{8r}s_{5r} - K_{152r}\mathbf{MY}_{8r} \nonumber \\
D_{255r} &= K_{162r}\mathbf{XZ}_{8r} - K_{121r}\mathbf{YZ}_{8r} + K_{163r}\mathbf{ZZ}_{8r}  \nonumber \\
&+ K_{127r}^2\mathbf{XY}_{8r} - K_{128r}^2\mathbf{XY}_{8r} + K_{165r}\mathbf{MX}_{8r}  \nonumber \\
&- K_{164r}\mathbf{MY}_{8r} - K_{127r}K_{128r}\mathbf{XX}_{8r} - K_{128r}K_{129r}\mathbf{XZ}_{8r}  \nonumber \\
&+ K_{127r}K_{128r}\mathbf{YY}_{8r} + K_{127r}K_{129r}\mathbf{YZ}_{8r} \nonumber \\
 \dot{\bar{H}}_{8r} &= \left[\begin{matrix} D_{205r} + D_{197r}\ddot{\psi} + D_{201r}\ddot{q}_{1r} + D_{199r}\ddot{q}_{w} + D_{202r}\ddot{q}_{2r} + D_{203r}\ddot{q}_{3r} + D_{204r}\ddot{q}_{4r} + D_{198r}\ddot{q}_{imu} + D_{200r}\ddot{q}_{torso} + D_{196r}\ddot{x} - \mathbf{MZ}_{8r}\ddot{q}_{5r} & D_{215r} + D_{207r}\ddot{\psi} + D_{211r}\ddot{q}_{1r} + D_{209r}\ddot{q}_{w} + D_{212r}\ddot{q}_{2r} + D_{213r}\ddot{q}_{3r} + D_{214r}\ddot{q}_{4r} + D_{208r}\ddot{q}_{imu} + D_{210r}\ddot{q}_{torso} + D_{206r}\ddot{x} & D_{225r} + D_{217r}\ddot{\psi} + D_{221r}\ddot{q}_{1r} + D_{219r}\ddot{q}_{w} + D_{222r}\ddot{q}_{2r} + D_{223r}\ddot{q}_{3r} + D_{224r}\ddot{q}_{4r} + D_{218r}\ddot{q}_{imu} + D_{220r}\ddot{q}_{torso} + D_{216r}\ddot{x} + \mathbf{MX}_{8r}\ddot{q}_{5r} &  \end{matrix}\right] 
 \nonumber \\ 
 \bar\omega_{9r} &= {}^{9r}A_{8r} \bar\omega_{8r} + \dot{q}_{9r} \bar{e}_{9r} 
 \nonumber \\ 
 \bar\omega_{9r} &= \left[\begin{matrix} - K_{127r} - \dot{q}_{6r} & - K_{128r}c_{6r} - K_{129r}s_{6r} & K_{129r}c_{6r} - K_{128r}s_{6r} &  \end{matrix}\right] 
 \nonumber \\ 
K_{169r} &= - K_{127r} - \dot{q}_{6r} \nonumber \\
K_{170r} &= - K_{128r}c_{6r} - K_{129r}s_{6r} \nonumber \\
K_{171r} &= K_{129r}c_{6r} - K_{128r}s_{6r} \nonumber \\
 \bar\omega_{9r} &= \left[\begin{matrix} K_{169r} & K_{170r} & K_{171r} &  \end{matrix}\right] 
 \nonumber \\ 
 \bar\omega_{9r} &= \left[\begin{matrix} - \dot{q}_{6r} - K_{130r}\dot{\psi} - K_{133r}\dot{q}_{1r} - K_{131r}\dot{q}_{w} - K_{134r}\dot{q}_{2r} - K_{135r}\dot{q}_{3r} - K_{131r}\dot{q}_{imu} - K_{132r}\dot{q}_{torso} - \dot{q}_{4r}c_{5r} & c_{6r}(\dot{q}_{5r} + K_{95r}\dot{\psi} + K_{98r}\dot{q}_{1r} + K_{96r}\dot{q}_{w} + K_{99r}\dot{q}_{2r} + K_{96r}\dot{q}_{imu} + K_{97r}\dot{q}_{torso} + \dot{q}_{3r}c_{4r}) - s_{6r}(K_{136r}\dot{\psi} + K_{139r}\dot{q}_{1r} + K_{137r}\dot{q}_{w} + K_{140r}\dot{q}_{2r} + K_{141r}\dot{q}_{3r} + K_{137r}\dot{q}_{imu} + K_{138r}\dot{q}_{torso} - \dot{q}_{4r}s_{5r}) & c_{6r}(K_{136r}\dot{\psi} + K_{139r}\dot{q}_{1r} + K_{137r}\dot{q}_{w} + K_{140r}\dot{q}_{2r} + K_{141r}\dot{q}_{3r} + K_{137r}\dot{q}_{imu} + K_{138r}\dot{q}_{torso} - \dot{q}_{4r}s_{5r}) + s_{6r}(\dot{q}_{5r} + K_{95r}\dot{\psi} + K_{98r}\dot{q}_{1r} + K_{96r}\dot{q}_{w} + K_{99r}\dot{q}_{2r} + K_{96r}\dot{q}_{imu} + K_{97r}\dot{q}_{torso} + \dot{q}_{3r}c_{4r}) &  \end{matrix}\right] 
 \nonumber \\ 
K_{172r} &= K_{95r}c_{6r} - K_{136r}s_{6r} \nonumber \\
K_{173r} &= K_{96r}c_{6r} - K_{137r}s_{6r} \nonumber \\
K_{174r} &= K_{97r}c_{6r} - K_{138r}s_{6r} \nonumber \\
K_{175r} &= K_{98r}c_{6r} - K_{139r}s_{6r} \nonumber \\
K_{176r} &= K_{99r}c_{6r} - K_{140r}s_{6r} \nonumber \\
K_{177r} &= c_{4r}c_{6r} - K_{141r}s_{6r} \nonumber \\
K_{178r} &= s_{5r}s_{6r} \nonumber \\
K_{179r} &= K_{136r}c_{6r} + K_{95r}s_{6r} \nonumber \\
K_{180r} &= K_{137r}c_{6r} + K_{96r}s_{6r} \nonumber \\
K_{181r} &= K_{138r}c_{6r} + K_{97r}s_{6r} \nonumber \\
K_{182r} &= K_{139r}c_{6r} + K_{98r}s_{6r} \nonumber \\
K_{183r} &= K_{140r}c_{6r} + K_{99r}s_{6r} \nonumber \\
K_{184r} &= c_{4r}s_{6r} + K_{141r}c_{6r} \nonumber \\
K_{185r} &= -c_{6r}s_{5r} \nonumber \\
 \bar\omega_{9r} &= \left[\begin{matrix} - \dot{q}_{6r} - K_{130r}\dot{\psi} - K_{133r}\dot{q}_{1r} - K_{131r}\dot{q}_{w} - K_{134r}\dot{q}_{2r} - K_{135r}\dot{q}_{3r} - K_{131r}\dot{q}_{imu} - K_{132r}\dot{q}_{torso} - \dot{q}_{4r}c_{5r} & K_{172r}\dot{\psi} + K_{175r}\dot{q}_{1r} + K_{173r}\dot{q}_{w} + K_{176r}\dot{q}_{2r} + K_{177r}\dot{q}_{3r} + K_{178r}\dot{q}_{4r} + K_{173r}\dot{q}_{imu} + K_{174r}\dot{q}_{torso} + \dot{q}_{5r}c_{6r} & K_{179r}\dot{\psi} + K_{182r}\dot{q}_{1r} + K_{180r}\dot{q}_{w} + K_{183r}\dot{q}_{2r} + K_{184r}\dot{q}_{3r} + K_{185r}\dot{q}_{4r} + K_{180r}\dot{q}_{imu} + K_{181r}\dot{q}_{torso} + \dot{q}_{5r}s_{6r} &  \end{matrix}\right] 
 \nonumber \\ 
 \bar{v}_{9r} &= {}^{9r}A_{8r} \left(\bar{v}_{8r} + \bar\omega_{8r} \times \bar{P}_{9r}\right) 
 \nonumber \\ 
 \bar{v}_{9r} &= \left[\begin{matrix} -K_{142r} & K_{105r}c_{6r} - K_{143r}s_{6r} & K_{143r}c_{6r} + K_{105r}s_{6r} &  \end{matrix}\right] 
 \nonumber \\ 
K_{186r} &= K_{105r}c_{6r} - K_{143r}s_{6r} \nonumber \\
K_{187r} &= K_{143r}c_{6r} + K_{105r}s_{6r} \nonumber \\
 \bar{v}_{9r} &= \left[\begin{matrix} -K_{142r} & K_{186r} & K_{187r} &  \end{matrix}\right] 
 \nonumber \\ 
 \bar{v}_{9r} &= \left[\begin{matrix} - K_{145r}\dot{\psi} - K_{149r}\dot{q}_{1r} - K_{147r}\dot{q}_{w} - K_{150r}\dot{q}_{2r} - K_{151r}\dot{q}_{3r} - K_{152r}\dot{q}_{4r} - K_{146r}\dot{q}_{imu} - K_{148r}\dot{q}_{torso} - K_{144r}\dot{x} & c_{6r}(K_{108r}\dot{\psi} + K_{112r}\dot{q}_{1r} + K_{110r}\dot{q}_{w} + K_{113r}\dot{q}_{2r} + K_{109r}\dot{q}_{imu} + K_{111r}\dot{q}_{torso} + K_{107r}\dot{x}) - s_{6r}(K_{154r}\dot{\psi} + K_{158r}\dot{q}_{1r} + K_{156r}\dot{q}_{w} + K_{159r}\dot{q}_{2r} + K_{160r}\dot{q}_{3r} + K_{161r}\dot{q}_{4r} + K_{155r}\dot{q}_{imu} + K_{157r}\dot{q}_{torso} + K_{153r}\dot{x}) & s_{6r}(K_{108r}\dot{\psi} + K_{112r}\dot{q}_{1r} + K_{110r}\dot{q}_{w} + K_{113r}\dot{q}_{2r} + K_{109r}\dot{q}_{imu} + K_{111r}\dot{q}_{torso} + K_{107r}\dot{x}) + c_{6r}(K_{154r}\dot{\psi} + K_{158r}\dot{q}_{1r} + K_{156r}\dot{q}_{w} + K_{159r}\dot{q}_{2r} + K_{160r}\dot{q}_{3r} + K_{161r}\dot{q}_{4r} + K_{155r}\dot{q}_{imu} + K_{157r}\dot{q}_{torso} + K_{153r}\dot{x}) &  \end{matrix}\right] 
 \nonumber \\ 
K_{188r} &= K_{107r}c_{6r} - K_{153r}s_{6r} \nonumber \\
K_{189r} &= K_{108r}c_{6r} - K_{154r}s_{6r} \nonumber \\
K_{190r} &= K_{109r}c_{6r} - K_{155r}s_{6r} \nonumber \\
K_{191r} &= K_{110r}c_{6r} - K_{156r}s_{6r} \nonumber \\
K_{192r} &= K_{111r}c_{6r} - K_{157r}s_{6r} \nonumber \\
K_{193r} &= K_{112r}c_{6r} - K_{158r}s_{6r} \nonumber \\
K_{194r} &= K_{113r}c_{6r} - K_{159r}s_{6r} \nonumber \\
K_{195r} &= -K_{160r}s_{6r} \nonumber \\
K_{196r} &= -K_{161r}s_{6r} \nonumber \\
K_{197r} &= K_{153r}c_{6r} + K_{107r}s_{6r} \nonumber \\
K_{198r} &= K_{154r}c_{6r} + K_{108r}s_{6r} \nonumber \\
K_{199r} &= K_{155r}c_{6r} + K_{109r}s_{6r} \nonumber \\
K_{200r} &= K_{156r}c_{6r} + K_{110r}s_{6r} \nonumber \\
K_{201r} &= K_{157r}c_{6r} + K_{111r}s_{6r} \nonumber \\
K_{202r} &= K_{158r}c_{6r} + K_{112r}s_{6r} \nonumber \\
K_{203r} &= K_{159r}c_{6r} + K_{113r}s_{6r} \nonumber \\
K_{204r} &= K_{160r}c_{6r} \nonumber \\
K_{205r} &= K_{161r}c_{6r} \nonumber \\
 \bar{v}_{9r} &= \left[\begin{matrix} - K_{145r}\dot{\psi} - K_{149r}\dot{q}_{1r} - K_{147r}\dot{q}_{w} - K_{150r}\dot{q}_{2r} - K_{151r}\dot{q}_{3r} - K_{152r}\dot{q}_{4r} - K_{146r}\dot{q}_{imu} - K_{148r}\dot{q}_{torso} - K_{144r}\dot{x} & K_{189r}\dot{\psi} + K_{193r}\dot{q}_{1r} + K_{191r}\dot{q}_{w} + K_{194r}\dot{q}_{2r} + K_{195r}\dot{q}_{3r} + K_{196r}\dot{q}_{4r} + K_{190r}\dot{q}_{imu} + K_{192r}\dot{q}_{torso} + K_{188r}\dot{x} & K_{198r}\dot{\psi} + K_{202r}\dot{q}_{1r} + K_{200r}\dot{q}_{w} + K_{203r}\dot{q}_{2r} + K_{204r}\dot{q}_{3r} + K_{205r}\dot{q}_{4r} + K_{199r}\dot{q}_{imu} + K_{201r}\dot{q}_{torso} + K_{197r}\dot{x} &  \end{matrix}\right] 
 \nonumber \\ 
 \bar\alpha_{9r} &= {}^{9r}A_{8r} \bar\alpha_{8r} + \ddot{q}_{9r} \bar{e}_{9r} + \dot{q}_{9r} \left(\bar\omega_{9r} \times \bar{e}_{9r}\right) 
 \nonumber \\ 
 \bar\alpha_{9r} &= \left[\begin{matrix} - K_{162r} - \ddot{q}_{6r} - K_{130r}\ddot{\psi} - K_{133r}\ddot{q}_{1r} - K_{131r}\ddot{q}_{w} - K_{134r}\ddot{q}_{2r} - K_{135r}\ddot{q}_{3r} - K_{131r}\ddot{q}_{imu} - K_{132r}\ddot{q}_{torso} - \ddot{q}_{4r}c_{5r} & c_{6r}(K_{121r} + \ddot{q}_{5r} + K_{95r}\ddot{\psi} + K_{98r}\ddot{q}_{1r} + K_{96r}\ddot{q}_{w} + K_{99r}\ddot{q}_{2r} + K_{96r}\ddot{q}_{imu} + K_{97r}\ddot{q}_{torso} + \ddot{q}_{3r}c_{4r}) - K_{171r}\dot{q}_{6r} - s_{6r}(K_{163r} + K_{136r}\ddot{\psi} + K_{139r}\ddot{q}_{1r} + K_{137r}\ddot{q}_{w} + K_{140r}\ddot{q}_{2r} + K_{141r}\ddot{q}_{3r} + K_{137r}\ddot{q}_{imu} + K_{138r}\ddot{q}_{torso} - \ddot{q}_{4r}s_{5r}) & K_{170r}\dot{q}_{6r} + s_{6r}(K_{121r} + \ddot{q}_{5r} + K_{95r}\ddot{\psi} + K_{98r}\ddot{q}_{1r} + K_{96r}\ddot{q}_{w} + K_{99r}\ddot{q}_{2r} + K_{96r}\ddot{q}_{imu} + K_{97r}\ddot{q}_{torso} + \ddot{q}_{3r}c_{4r}) + c_{6r}(K_{163r} + K_{136r}\ddot{\psi} + K_{139r}\ddot{q}_{1r} + K_{137r}\ddot{q}_{w} + K_{140r}\ddot{q}_{2r} + K_{141r}\ddot{q}_{3r} + K_{137r}\ddot{q}_{imu} + K_{138r}\ddot{q}_{torso} - \ddot{q}_{4r}s_{5r}) &  \end{matrix}\right] 
 \nonumber \\ 
K_{206r} &= K_{121r}c_{6r} - K_{171r}\dot{q}_{6r} - K_{163r}s_{6r} \nonumber \\
K_{207r} &= K_{170r}\dot{q}_{6r} + K_{163r}c_{6r} + K_{121r}s_{6r} \nonumber \\
 \bar\alpha_{9r} &= \left[\begin{matrix} - K_{162r} - \ddot{q}_{6r} - K_{130r}\ddot{\psi} - K_{133r}\ddot{q}_{1r} - K_{131r}\ddot{q}_{w} - K_{134r}\ddot{q}_{2r} - K_{135r}\ddot{q}_{3r} - K_{131r}\ddot{q}_{imu} - K_{132r}\ddot{q}_{torso} - \ddot{q}_{4r}c_{5r} & K_{206r} + K_{172r}\ddot{\psi} + K_{175r}\ddot{q}_{1r} + K_{173r}\ddot{q}_{w} + K_{176r}\ddot{q}_{2r} + K_{177r}\ddot{q}_{3r} + K_{178r}\ddot{q}_{4r} + K_{173r}\ddot{q}_{imu} + K_{174r}\ddot{q}_{torso} + \ddot{q}_{5r}c_{6r} & K_{207r} + K_{179r}\ddot{\psi} + K_{182r}\ddot{q}_{1r} + K_{180r}\ddot{q}_{w} + K_{183r}\ddot{q}_{2r} + K_{184r}\ddot{q}_{3r} + K_{185r}\ddot{q}_{4r} + K_{180r}\ddot{q}_{imu} + K_{181r}\ddot{q}_{torso} + \ddot{q}_{5r}s_{6r} &  \end{matrix}\right] 
 \nonumber \\ 
 \bar{a}_{9r} &= {}^{9r}A_{8r} \left(\bar{a}_{8r} + \bar\alpha_{8r} \times \bar{P}_{9r} + \bar\omega_{8r} \times \left(\bar\omega_{8r} \times \bar{P}_{9r}\right)\right) 
 \nonumber \\ 
 \bar\alpha_{9r} &= \left[\begin{matrix} - K_{164r} - K_{145r}\ddot{\psi} - K_{149r}\ddot{q}_{1r} - K_{147r}\ddot{q}_{w} - K_{150r}\ddot{q}_{2r} - K_{151r}\ddot{q}_{3r} - K_{152r}\ddot{q}_{4r} - K_{146r}\ddot{q}_{imu} - K_{148r}\ddot{q}_{torso} - K_{144r}\ddot{x} & c_{6r}(K_{108r}\ddot{\psi} - K_{165r} + K_{112r}\ddot{q}_{1r} + K_{110r}\ddot{q}_{w} + K_{113r}\ddot{q}_{2r} + K_{109r}\ddot{q}_{imu} + K_{111r}\ddot{q}_{torso} + K_{107r}\ddot{x}) - s_{6r}(K_{166r} + K_{154r}\ddot{\psi} + K_{158r}\ddot{q}_{1r} + K_{156r}\ddot{q}_{w} + K_{159r}\ddot{q}_{2r} + K_{160r}\ddot{q}_{3r} + K_{161r}\ddot{q}_{4r} + K_{155r}\ddot{q}_{imu} + K_{157r}\ddot{q}_{torso} + K_{153r}\ddot{x}) & c_{6r}(K_{166r} + K_{154r}\ddot{\psi} + K_{158r}\ddot{q}_{1r} + K_{156r}\ddot{q}_{w} + K_{159r}\ddot{q}_{2r} + K_{160r}\ddot{q}_{3r} + K_{161r}\ddot{q}_{4r} + K_{155r}\ddot{q}_{imu} + K_{157r}\ddot{q}_{torso} + K_{153r}\ddot{x}) + s_{6r}(K_{108r}\ddot{\psi} - K_{165r} + K_{112r}\ddot{q}_{1r} + K_{110r}\ddot{q}_{w} + K_{113r}\ddot{q}_{2r} + K_{109r}\ddot{q}_{imu} + K_{111r}\ddot{q}_{torso} + K_{107r}\ddot{x}) &  \end{matrix}\right] 
 \nonumber \\ 
K_{208r} &= - K_{165r}c_{6r} - K_{166r}s_{6r} \nonumber \\
K_{209r} &= K_{166r}c_{6r} - K_{165r}s_{6r} \nonumber \\
 \bar{a}_{9r} &= \left[\begin{matrix} - K_{164r} - K_{145r}\ddot{\psi} - K_{149r}\ddot{q}_{1r} - K_{147r}\ddot{q}_{w} - K_{150r}\ddot{q}_{2r} - K_{151r}\ddot{q}_{3r} - K_{152r}\ddot{q}_{4r} - K_{146r}\ddot{q}_{imu} - K_{148r}\ddot{q}_{torso} - K_{144r}\ddot{x} & K_{208r} + K_{189r}\ddot{\psi} + K_{193r}\ddot{q}_{1r} + K_{191r}\ddot{q}_{w} + K_{194r}\ddot{q}_{2r} + K_{195r}\ddot{q}_{3r} + K_{196r}\ddot{q}_{4r} + K_{190r}\ddot{q}_{imu} + K_{192r}\ddot{q}_{torso} + K_{188r}\ddot{x} & K_{209r} + K_{198r}\ddot{\psi} + K_{202r}\ddot{q}_{1r} + K_{200r}\ddot{q}_{w} + K_{203r}\ddot{q}_{2r} + K_{204r}\ddot{q}_{3r} + K_{205r}\ddot{q}_{4r} + K_{199r}\ddot{q}_{imu} + K_{201r}\ddot{q}_{torso} + K_{197r}\ddot{x} &  \end{matrix}\right] 
 \nonumber \\ 
 \bar{g}_{9r} &= {}^{9r}A_{8r} \bar{g}_{8r} 
 \nonumber \\ 
 \bar{g}_{9r} &= \left[\begin{matrix} -K_{167r}g & K_{125r}gc_{6r} - K_{168r}gs_{6r} & K_{168r}gc_{6r} + K_{125r}gs_{6r} &  \end{matrix}\right] 
 \nonumber \\ 
K_{210r} &= K_{125r}c_{6r} - K_{168r}s_{6r} \nonumber \\
K_{211r} &= K_{168r}c_{6r} + K_{125r}s_{6r} \nonumber \\
 \bar{g}_{9r} &= \left[\begin{matrix} -K_{167r}g & K_{210r}g & K_{211r}g &  \end{matrix}\right] 
 \nonumber \\ 
 m_{9r}\bar{S}_{9r}^{\times}\bar{g}_{9r} &= \mathbf{MS}_{9r} \times \bar{g}_{9r} 
 \nonumber \\ 
 m_{9r}\bar{S}_{9r}^{\times}\bar{g}_{9r} &= \left[\begin{matrix} K_{211r}\mathbf{MY}_{9r}g - K_{210r}\mathbf{MZ}_{9r}g & - K_{211r}\mathbf{MX}_{9r}g - K_{167r}\mathbf{MZ}_{9r}g & K_{210r}\mathbf{MX}_{9r}g + K_{167r}\mathbf{MY}_{9r}g &  \end{matrix}\right] 
 \nonumber \\ 
D_{256r} &= K_{211r}\mathbf{MY}_{9r} - K_{210r}\mathbf{MZ}_{9r} \nonumber \\
D_{257r} &= - K_{211r}\mathbf{MX}_{9r} - K_{167r}\mathbf{MZ}_{9r} \nonumber \\
D_{258r} &= K_{210r}\mathbf{MX}_{9r} + K_{167r}\mathbf{MY}_{9r} \nonumber \\
 m_{9r}\bar{S}_{9r}^{\times}\bar{g}_{9r} &= \left[\begin{matrix} D_{256r}g & D_{257r}g & D_{258r}g &  \end{matrix}\right] 
 \nonumber \\ 
 m_{9r}\bar{a}_{G(9r)} &= m_{9r}\bar{a}_{9r} + \bar\alpha_{9r} \times \mathbf{MS}_{9r} + \bar\omega_{9r} \times \left(\bar\omega_{9r} \times \mathbf{MS}_{9r}\right) 
 \nonumber \\ 
 m_{9r}\bar{a}_{G(9r)} &= \left[\begin{matrix} \mathbf{MZ}_{9r}(K_{206r} + K_{172r}\ddot{\psi} + K_{175r}\ddot{q}_{1r} + K_{173r}\ddot{q}_{w} + K_{176r}\ddot{q}_{2r} + K_{177r}\ddot{q}_{3r} + K_{178r}\ddot{q}_{4r} + K_{173r}\ddot{q}_{imu} + K_{174r}\ddot{q}_{torso} + \ddot{q}_{5r}c_{6r}) - \mathbf{MY}_{9r}(K_{207r} + K_{179r}\ddot{\psi} + K_{182r}\ddot{q}_{1r} + K_{180r}\ddot{q}_{w} + K_{183r}\ddot{q}_{2r} + K_{184r}\ddot{q}_{3r} + K_{185r}\ddot{q}_{4r} + K_{180r}\ddot{q}_{imu} + K_{181r}\ddot{q}_{torso} + \ddot{q}_{5r}s_{6r}) - K_{170r}(K_{170r}\mathbf{MX}_{9r} - K_{169r}\mathbf{MY}_{9r}) - K_{171r}(K_{171r}\mathbf{MX}_{9r} - K_{169r}\mathbf{MZ}_{9r}) - m_{9r}(K_{164r} + K_{145r}\ddot{\psi} + K_{149r}\ddot{q}_{1r} + K_{147r}\ddot{q}_{w} + K_{150r}\ddot{q}_{2r} + K_{151r}\ddot{q}_{3r} + K_{152r}\ddot{q}_{4r} + K_{146r}\ddot{q}_{imu} + K_{148r}\ddot{q}_{torso} + K_{144r}\ddot{x}) & \mathbf{MX}_{9r}(K_{207r} + K_{179r}\ddot{\psi} + K_{182r}\ddot{q}_{1r} + K_{180r}\ddot{q}_{w} + K_{183r}\ddot{q}_{2r} + K_{184r}\ddot{q}_{3r} + K_{185r}\ddot{q}_{4r} + K_{180r}\ddot{q}_{imu} + K_{181r}\ddot{q}_{torso} + \ddot{q}_{5r}s_{6r}) + \mathbf{MZ}_{9r}(K_{162r} + \ddot{q}_{6r} + K_{130r}\ddot{\psi} + K_{133r}\ddot{q}_{1r} + K_{131r}\ddot{q}_{w} + K_{134r}\ddot{q}_{2r} + K_{135r}\ddot{q}_{3r} + K_{131r}\ddot{q}_{imu} + K_{132r}\ddot{q}_{torso} + \ddot{q}_{4r}c_{5r}) + K_{169r}(K_{170r}\mathbf{MX}_{9r} - K_{169r}\mathbf{MY}_{9r}) - K_{171r}(K_{171r}\mathbf{MY}_{9r} - K_{170r}\mathbf{MZ}_{9r}) + m_{9r}(K_{208r} + K_{189r}\ddot{\psi} + K_{193r}\ddot{q}_{1r} + K_{191r}\ddot{q}_{w} + K_{194r}\ddot{q}_{2r} + K_{195r}\ddot{q}_{3r} + K_{196r}\ddot{q}_{4r} + K_{190r}\ddot{q}_{imu} + K_{192r}\ddot{q}_{torso} + K_{188r}\ddot{x}) & K_{169r}(K_{171r}\mathbf{MX}_{9r} - K_{169r}\mathbf{MZ}_{9r}) - \mathbf{MY}_{9r}(K_{162r} + \ddot{q}_{6r} + K_{130r}\ddot{\psi} + K_{133r}\ddot{q}_{1r} + K_{131r}\ddot{q}_{w} + K_{134r}\ddot{q}_{2r} + K_{135r}\ddot{q}_{3r} + K_{131r}\ddot{q}_{imu} + K_{132r}\ddot{q}_{torso} + \ddot{q}_{4r}c_{5r}) - \mathbf{MX}_{9r}(K_{206r} + K_{172r}\ddot{\psi} + K_{175r}\ddot{q}_{1r} + K_{173r}\ddot{q}_{w} + K_{176r}\ddot{q}_{2r} + K_{177r}\ddot{q}_{3r} + K_{178r}\ddot{q}_{4r} + K_{173r}\ddot{q}_{imu} + K_{174r}\ddot{q}_{torso} + \ddot{q}_{5r}c_{6r}) + K_{170r}(K_{171r}\mathbf{MY}_{9r} - K_{170r}\mathbf{MZ}_{9r}) + m_{9r}(K_{209r} + K_{198r}\ddot{\psi} + K_{202r}\ddot{q}_{1r} + K_{200r}\ddot{q}_{w} + K_{203r}\ddot{q}_{2r} + K_{204r}\ddot{q}_{3r} + K_{205r}\ddot{q}_{4r} + K_{199r}\ddot{q}_{imu} + K_{201r}\ddot{q}_{torso} + K_{197r}\ddot{x}) &  \end{matrix}\right] 
 \nonumber \\ 
D_{259r} &= -K_{144r}m_{9r} \nonumber \\
D_{260r} &= K_{172r}\mathbf{MZ}_{9r} - K_{179r}\mathbf{MY}_{9r} - K_{145r}m_{9r} \nonumber \\
D_{261r} &= K_{173r}\mathbf{MZ}_{9r} - K_{180r}\mathbf{MY}_{9r} - K_{146r}m_{9r} \nonumber \\
D_{262r} &= K_{173r}\mathbf{MZ}_{9r} - K_{180r}\mathbf{MY}_{9r} - K_{147r}m_{9r} \nonumber \\
D_{263r} &= K_{174r}\mathbf{MZ}_{9r} - K_{181r}\mathbf{MY}_{9r} - K_{148r}m_{9r} \nonumber \\
D_{264r} &= K_{175r}\mathbf{MZ}_{9r} - K_{182r}\mathbf{MY}_{9r} - K_{149r}m_{9r} \nonumber \\
D_{265r} &= K_{176r}\mathbf{MZ}_{9r} - K_{183r}\mathbf{MY}_{9r} - K_{150r}m_{9r} \nonumber \\
D_{266r} &= K_{177r}\mathbf{MZ}_{9r} - K_{184r}\mathbf{MY}_{9r} - K_{151r}m_{9r} \nonumber \\
D_{267r} &= K_{178r}\mathbf{MZ}_{9r} - K_{185r}\mathbf{MY}_{9r} - K_{152r}m_{9r} \nonumber \\
D_{268r} &= \mathbf{MZ}_{9r}c_{6r} - \mathbf{MY}_{9r}s_{6r} \nonumber \\
D_{269r} &= K_{206r}\mathbf{MZ}_{9r} - K_{170r}^2\mathbf{MX}_{9r} - K_{171r}^2\mathbf{MX}_{9r}  \nonumber \\
&- K_{207r}\mathbf{MY}_{9r} - K_{164r}m_{9r} + K_{169r}K_{170r}\mathbf{MY}_{9r}  \nonumber \\
&+ K_{169r}K_{171r}\mathbf{MZ}_{9r} \nonumber \\
D_{270r} &= K_{188r}m_{9r} \nonumber \\
D_{271r} &= K_{189r}m_{9r} + K_{179r}\mathbf{MX}_{9r} + K_{130r}\mathbf{MZ}_{9r} \nonumber \\
D_{272r} &= K_{190r}m_{9r} + K_{180r}\mathbf{MX}_{9r} + K_{131r}\mathbf{MZ}_{9r} \nonumber \\
D_{273r} &= K_{191r}m_{9r} + K_{180r}\mathbf{MX}_{9r} + K_{131r}\mathbf{MZ}_{9r} \nonumber \\
D_{274r} &= K_{192r}m_{9r} + K_{181r}\mathbf{MX}_{9r} + K_{132r}\mathbf{MZ}_{9r} \nonumber \\
D_{275r} &= K_{193r}m_{9r} + K_{182r}\mathbf{MX}_{9r} + K_{133r}\mathbf{MZ}_{9r} \nonumber \\
D_{276r} &= K_{194r}m_{9r} + K_{183r}\mathbf{MX}_{9r} + K_{134r}\mathbf{MZ}_{9r} \nonumber \\
D_{277r} &= K_{195r}m_{9r} + K_{184r}\mathbf{MX}_{9r} + K_{135r}\mathbf{MZ}_{9r} \nonumber \\
D_{278r} &= K_{196r}m_{9r} + \mathbf{MZ}_{9r}c_{5r} + K_{185r}\mathbf{MX}_{9r} \nonumber \\
D_{279r} &= \mathbf{MX}_{9r}s_{6r} \nonumber \\
D_{280r} &= K_{208r}m_{9r} - K_{169r}^2\mathbf{MY}_{9r} - K_{171r}^2\mathbf{MY}_{9r}  \nonumber \\
&+ K_{207r}\mathbf{MX}_{9r} + K_{162r}\mathbf{MZ}_{9r} + K_{169r}K_{170r}\mathbf{MX}_{9r}  \nonumber \\
&+ K_{170r}K_{171r}\mathbf{MZ}_{9r} \nonumber \\
D_{281r} &= K_{197r}m_{9r} \nonumber \\
D_{282r} &= K_{198r}m_{9r} - K_{172r}\mathbf{MX}_{9r} - K_{130r}\mathbf{MY}_{9r} \nonumber \\
D_{283r} &= K_{199r}m_{9r} - K_{173r}\mathbf{MX}_{9r} - K_{131r}\mathbf{MY}_{9r} \nonumber \\
D_{284r} &= K_{200r}m_{9r} - K_{173r}\mathbf{MX}_{9r} - K_{131r}\mathbf{MY}_{9r} \nonumber \\
D_{285r} &= K_{201r}m_{9r} - K_{174r}\mathbf{MX}_{9r} - K_{132r}\mathbf{MY}_{9r} \nonumber \\
D_{286r} &= K_{202r}m_{9r} - K_{175r}\mathbf{MX}_{9r} - K_{133r}\mathbf{MY}_{9r} \nonumber \\
D_{287r} &= K_{203r}m_{9r} - K_{176r}\mathbf{MX}_{9r} - K_{134r}\mathbf{MY}_{9r} \nonumber \\
D_{288r} &= K_{204r}m_{9r} - K_{177r}\mathbf{MX}_{9r} - K_{135r}\mathbf{MY}_{9r} \nonumber \\
D_{289r} &= K_{205r}m_{9r} - \mathbf{MY}_{9r}c_{5r} - K_{178r}\mathbf{MX}_{9r} \nonumber \\
D_{290r} &= -\mathbf{MX}_{9r}c_{6r} \nonumber \\
D_{291r} &= K_{209r}m_{9r} - K_{169r}^2\mathbf{MZ}_{9r} - K_{170r}^2\mathbf{MZ}_{9r}  \nonumber \\
&- K_{206r}\mathbf{MX}_{9r} - K_{162r}\mathbf{MY}_{9r} + K_{169r}K_{171r}\mathbf{MX}_{9r}  \nonumber \\
&+ K_{170r}K_{171r}\mathbf{MY}_{9r} \nonumber \\
 m_{9r}\bar{a}_{G(9r)} &= \left[\begin{matrix} D_{269r} + D_{260r}\ddot{\psi} + D_{264r}\ddot{q}_{1r} + D_{262r}\ddot{q}_{w} + D_{265r}\ddot{q}_{2r} + D_{266r}\ddot{q}_{3r} + D_{267r}\ddot{q}_{4r} + D_{268r}\ddot{q}_{5r} + D_{261r}\ddot{q}_{imu} + D_{263r}\ddot{q}_{torso} + D_{259r}\ddot{x} & D_{280r} + D_{271r}\ddot{\psi} + D_{275r}\ddot{q}_{1r} + D_{273r}\ddot{q}_{w} + D_{276r}\ddot{q}_{2r} + D_{277r}\ddot{q}_{3r} + D_{278r}\ddot{q}_{4r} + D_{279r}\ddot{q}_{5r} + D_{272r}\ddot{q}_{imu} + D_{274r}\ddot{q}_{torso} + D_{270r}\ddot{x} + \mathbf{MZ}_{9r}\ddot{q}_{6r} & D_{291r} + D_{282r}\ddot{\psi} + D_{286r}\ddot{q}_{1r} + D_{284r}\ddot{q}_{w} + D_{287r}\ddot{q}_{2r} + D_{288r}\ddot{q}_{3r} + D_{289r}\ddot{q}_{4r} + D_{290r}\ddot{q}_{5r} + D_{283r}\ddot{q}_{imu} + D_{285r}\ddot{q}_{torso} + D_{281r}\ddot{x} - \mathbf{MY}_{9r}\ddot{q}_{6r} &  \end{matrix}\right] 
 \nonumber \\ 
 \dot{\bar{H}}_{9r} &= \mathbf{MS}_{9r} \times \bar{a}_{9r} + J_{9r}\bar{\alpha}_{9r} + \bar\omega_{9r} \times J_{9r}\bar{\omega}_{9r} 
 \nonumber \\ 
 \dot{\bar{H}}_{9r} &= \left[\begin{matrix} K_{170r}(K_{169r}\mathbf{XZ}_{9r} + K_{170r}\mathbf{YZ}_{9r} + K_{171r}\mathbf{ZZ}_{9r}) - K_{171r}(K_{169r}\mathbf{XY}_{9r} + K_{170r}\mathbf{YY}_{9r} + K_{171r}\mathbf{YZ}_{9r}) + \mathbf{XY}_{9r}(K_{206r} + K_{172r}\ddot{\psi} + K_{175r}\ddot{q}_{1r} + K_{173r}\ddot{q}_{w} + K_{176r}\ddot{q}_{2r} + K_{177r}\ddot{q}_{3r} + K_{178r}\ddot{q}_{4r} + K_{173r}\ddot{q}_{imu} + K_{174r}\ddot{q}_{torso} + \ddot{q}_{5r}c_{6r}) + \mathbf{XZ}_{9r}(K_{207r} + K_{179r}\ddot{\psi} + K_{182r}\ddot{q}_{1r} + K_{180r}\ddot{q}_{w} + K_{183r}\ddot{q}_{2r} + K_{184r}\ddot{q}_{3r} + K_{185r}\ddot{q}_{4r} + K_{180r}\ddot{q}_{imu} + K_{181r}\ddot{q}_{torso} + \ddot{q}_{5r}s_{6r}) - \mathbf{XX}_{9r}(K_{162r} + \ddot{q}_{6r} + K_{130r}\ddot{\psi} + K_{133r}\ddot{q}_{1r} + K_{131r}\ddot{q}_{w} + K_{134r}\ddot{q}_{2r} + K_{135r}\ddot{q}_{3r} + K_{131r}\ddot{q}_{imu} + K_{132r}\ddot{q}_{torso} + \ddot{q}_{4r}c_{5r}) + \mathbf{MY}_{9r}(K_{209r} + K_{198r}\ddot{\psi} + K_{202r}\ddot{q}_{1r} + K_{200r}\ddot{q}_{w} + K_{203r}\ddot{q}_{2r} + K_{204r}\ddot{q}_{3r} + K_{205r}\ddot{q}_{4r} + K_{199r}\ddot{q}_{imu} + K_{201r}\ddot{q}_{torso} + K_{197r}\ddot{x}) - \mathbf{MZ}_{9r}(K_{208r} + K_{189r}\ddot{\psi} + K_{193r}\ddot{q}_{1r} + K_{191r}\ddot{q}_{w} + K_{194r}\ddot{q}_{2r} + K_{195r}\ddot{q}_{3r} + K_{196r}\ddot{q}_{4r} + K_{190r}\ddot{q}_{imu} + K_{192r}\ddot{q}_{torso} + K_{188r}\ddot{x}) & K_{171r}(K_{169r}\mathbf{XX}_{9r} + K_{170r}\mathbf{XY}_{9r} + K_{171r}\mathbf{XZ}_{9r}) - K_{169r}(K_{169r}\mathbf{XZ}_{9r} + K_{170r}\mathbf{YZ}_{9r} + K_{171r}\mathbf{ZZ}_{9r}) + \mathbf{YY}_{9r}(K_{206r} + K_{172r}\ddot{\psi} + K_{175r}\ddot{q}_{1r} + K_{173r}\ddot{q}_{w} + K_{176r}\ddot{q}_{2r} + K_{177r}\ddot{q}_{3r} + K_{178r}\ddot{q}_{4r} + K_{173r}\ddot{q}_{imu} + K_{174r}\ddot{q}_{torso} + \ddot{q}_{5r}c_{6r}) + \mathbf{YZ}_{9r}(K_{207r} + K_{179r}\ddot{\psi} + K_{182r}\ddot{q}_{1r} + K_{180r}\ddot{q}_{w} + K_{183r}\ddot{q}_{2r} + K_{184r}\ddot{q}_{3r} + K_{185r}\ddot{q}_{4r} + K_{180r}\ddot{q}_{imu} + K_{181r}\ddot{q}_{torso} + \ddot{q}_{5r}s_{6r}) - \mathbf{XY}_{9r}(K_{162r} + \ddot{q}_{6r} + K_{130r}\ddot{\psi} + K_{133r}\ddot{q}_{1r} + K_{131r}\ddot{q}_{w} + K_{134r}\ddot{q}_{2r} + K_{135r}\ddot{q}_{3r} + K_{131r}\ddot{q}_{imu} + K_{132r}\ddot{q}_{torso} + \ddot{q}_{4r}c_{5r}) - \mathbf{MX}_{9r}(K_{209r} + K_{198r}\ddot{\psi} + K_{202r}\ddot{q}_{1r} + K_{200r}\ddot{q}_{w} + K_{203r}\ddot{q}_{2r} + K_{204r}\ddot{q}_{3r} + K_{205r}\ddot{q}_{4r} + K_{199r}\ddot{q}_{imu} + K_{201r}\ddot{q}_{torso} + K_{197r}\ddot{x}) - \mathbf{MZ}_{9r}(K_{164r} + K_{145r}\ddot{\psi} + K_{149r}\ddot{q}_{1r} + K_{147r}\ddot{q}_{w} + K_{150r}\ddot{q}_{2r} + K_{151r}\ddot{q}_{3r} + K_{152r}\ddot{q}_{4r} + K_{146r}\ddot{q}_{imu} + K_{148r}\ddot{q}_{torso} + K_{144r}\ddot{x}) & K_{169r}(K_{169r}\mathbf{XY}_{9r} + K_{170r}\mathbf{YY}_{9r} + K_{171r}\mathbf{YZ}_{9r}) - K_{170r}(K_{169r}\mathbf{XX}_{9r} + K_{170r}\mathbf{XY}_{9r} + K_{171r}\mathbf{XZ}_{9r}) + \mathbf{YZ}_{9r}(K_{206r} + K_{172r}\ddot{\psi} + K_{175r}\ddot{q}_{1r} + K_{173r}\ddot{q}_{w} + K_{176r}\ddot{q}_{2r} + K_{177r}\ddot{q}_{3r} + K_{178r}\ddot{q}_{4r} + K_{173r}\ddot{q}_{imu} + K_{174r}\ddot{q}_{torso} + \ddot{q}_{5r}c_{6r}) + \mathbf{ZZ}_{9r}(K_{207r} + K_{179r}\ddot{\psi} + K_{182r}\ddot{q}_{1r} + K_{180r}\ddot{q}_{w} + K_{183r}\ddot{q}_{2r} + K_{184r}\ddot{q}_{3r} + K_{185r}\ddot{q}_{4r} + K_{180r}\ddot{q}_{imu} + K_{181r}\ddot{q}_{torso} + \ddot{q}_{5r}s_{6r}) - \mathbf{XZ}_{9r}(K_{162r} + \ddot{q}_{6r} + K_{130r}\ddot{\psi} + K_{133r}\ddot{q}_{1r} + K_{131r}\ddot{q}_{w} + K_{134r}\ddot{q}_{2r} + K_{135r}\ddot{q}_{3r} + K_{131r}\ddot{q}_{imu} + K_{132r}\ddot{q}_{torso} + \ddot{q}_{4r}c_{5r}) + \mathbf{MX}_{9r}(K_{208r} + K_{189r}\ddot{\psi} + K_{193r}\ddot{q}_{1r} + K_{191r}\ddot{q}_{w} + K_{194r}\ddot{q}_{2r} + K_{195r}\ddot{q}_{3r} + K_{196r}\ddot{q}_{4r} + K_{190r}\ddot{q}_{imu} + K_{192r}\ddot{q}_{torso} + K_{188r}\ddot{x}) + \mathbf{MY}_{9r}(K_{164r} + K_{145r}\ddot{\psi} + K_{149r}\ddot{q}_{1r} + K_{147r}\ddot{q}_{w} + K_{150r}\ddot{q}_{2r} + K_{151r}\ddot{q}_{3r} + K_{152r}\ddot{q}_{4r} + K_{146r}\ddot{q}_{imu} + K_{148r}\ddot{q}_{torso} + K_{144r}\ddot{x}) &  \end{matrix}\right] 
 \nonumber \\ 
D_{292r} &= K_{197r}\mathbf{MY}_{9r} - K_{188r}\mathbf{MZ}_{9r} \nonumber \\
D_{293r} &= K_{172r}\mathbf{XY}_{9r} - K_{130r}\mathbf{XX}_{9r} + K_{179r}\mathbf{XZ}_{9r}  \nonumber \\
&+ K_{198r}\mathbf{MY}_{9r} - K_{189r}\mathbf{MZ}_{9r} \nonumber \\
D_{294r} &= K_{173r}\mathbf{XY}_{9r} - K_{131r}\mathbf{XX}_{9r} + K_{180r}\mathbf{XZ}_{9r}  \nonumber \\
&+ K_{199r}\mathbf{MY}_{9r} - K_{190r}\mathbf{MZ}_{9r} \nonumber \\
D_{295r} &= K_{173r}\mathbf{XY}_{9r} - K_{131r}\mathbf{XX}_{9r} + K_{180r}\mathbf{XZ}_{9r}  \nonumber \\
&+ K_{200r}\mathbf{MY}_{9r} - K_{191r}\mathbf{MZ}_{9r} \nonumber \\
D_{296r} &= K_{174r}\mathbf{XY}_{9r} - K_{132r}\mathbf{XX}_{9r} + K_{181r}\mathbf{XZ}_{9r}  \nonumber \\
&+ K_{201r}\mathbf{MY}_{9r} - K_{192r}\mathbf{MZ}_{9r} \nonumber \\
D_{297r} &= K_{175r}\mathbf{XY}_{9r} - K_{133r}\mathbf{XX}_{9r} + K_{182r}\mathbf{XZ}_{9r}  \nonumber \\
&+ K_{202r}\mathbf{MY}_{9r} - K_{193r}\mathbf{MZ}_{9r} \nonumber \\
D_{298r} &= K_{176r}\mathbf{XY}_{9r} - K_{134r}\mathbf{XX}_{9r} + K_{183r}\mathbf{XZ}_{9r}  \nonumber \\
&+ K_{203r}\mathbf{MY}_{9r} - K_{194r}\mathbf{MZ}_{9r} \nonumber \\
D_{299r} &= K_{177r}\mathbf{XY}_{9r} - K_{135r}\mathbf{XX}_{9r} + K_{184r}\mathbf{XZ}_{9r}  \nonumber \\
&+ K_{204r}\mathbf{MY}_{9r} - K_{195r}\mathbf{MZ}_{9r} \nonumber \\
D_{300r} &= K_{178r}\mathbf{XY}_{9r} + K_{185r}\mathbf{XZ}_{9r} - \mathbf{XX}_{9r}c_{5r}  \nonumber \\
&+ K_{205r}\mathbf{MY}_{9r} - K_{196r}\mathbf{MZ}_{9r} \nonumber \\
D_{301r} &= \mathbf{XY}_{9r}c_{6r} + \mathbf{XZ}_{9r}s_{6r} \nonumber \\
D_{302r} &= K_{206r}\mathbf{XY}_{9r} - K_{162r}\mathbf{XX}_{9r} + K_{207r}\mathbf{XZ}_{9r}  \nonumber \\
&+ K_{170r}^2\mathbf{YZ}_{9r} - K_{171r}^2\mathbf{YZ}_{9r} + K_{209r}\mathbf{MY}_{9r}  \nonumber \\
&- K_{208r}\mathbf{MZ}_{9r} - K_{169r}K_{171r}\mathbf{XY}_{9r} + K_{169r}K_{170r}\mathbf{XZ}_{9r}  \nonumber \\
&- K_{170r}K_{171r}\mathbf{YY}_{9r} + K_{170r}K_{171r}\mathbf{ZZ}_{9r} \nonumber \\
D_{303r} &= - K_{197r}\mathbf{MX}_{9r} - K_{144r}\mathbf{MZ}_{9r} \nonumber \\
D_{304r} &= K_{172r}\mathbf{YY}_{9r} - K_{130r}\mathbf{XY}_{9r} + K_{179r}\mathbf{YZ}_{9r}  \nonumber \\
&- K_{198r}\mathbf{MX}_{9r} - K_{145r}\mathbf{MZ}_{9r} \nonumber \\
D_{305r} &= K_{173r}\mathbf{YY}_{9r} - K_{131r}\mathbf{XY}_{9r} + K_{180r}\mathbf{YZ}_{9r}  \nonumber \\
&- K_{199r}\mathbf{MX}_{9r} - K_{146r}\mathbf{MZ}_{9r} \nonumber \\
D_{306r} &= K_{173r}\mathbf{YY}_{9r} - K_{131r}\mathbf{XY}_{9r} + K_{180r}\mathbf{YZ}_{9r}  \nonumber \\
&- K_{200r}\mathbf{MX}_{9r} - K_{147r}\mathbf{MZ}_{9r} \nonumber \\
D_{307r} &= K_{174r}\mathbf{YY}_{9r} - K_{132r}\mathbf{XY}_{9r} + K_{181r}\mathbf{YZ}_{9r}  \nonumber \\
&- K_{201r}\mathbf{MX}_{9r} - K_{148r}\mathbf{MZ}_{9r} \nonumber \\
D_{308r} &= K_{175r}\mathbf{YY}_{9r} - K_{133r}\mathbf{XY}_{9r} + K_{182r}\mathbf{YZ}_{9r}  \nonumber \\
&- K_{202r}\mathbf{MX}_{9r} - K_{149r}\mathbf{MZ}_{9r} \nonumber \\
D_{309r} &= K_{176r}\mathbf{YY}_{9r} - K_{134r}\mathbf{XY}_{9r} + K_{183r}\mathbf{YZ}_{9r}  \nonumber \\
&- K_{203r}\mathbf{MX}_{9r} - K_{150r}\mathbf{MZ}_{9r} \nonumber \\
D_{310r} &= K_{177r}\mathbf{YY}_{9r} - K_{135r}\mathbf{XY}_{9r} + K_{184r}\mathbf{YZ}_{9r}  \nonumber \\
&- K_{204r}\mathbf{MX}_{9r} - K_{151r}\mathbf{MZ}_{9r} \nonumber \\
D_{311r} &= K_{178r}\mathbf{YY}_{9r} + K_{185r}\mathbf{YZ}_{9r} - \mathbf{XY}_{9r}c_{5r}  \nonumber \\
&- K_{205r}\mathbf{MX}_{9r} - K_{152r}\mathbf{MZ}_{9r} \nonumber \\
D_{312r} &= \mathbf{YY}_{9r}c_{6r} + \mathbf{YZ}_{9r}s_{6r} \nonumber \\
D_{313r} &= K_{206r}\mathbf{YY}_{9r} - K_{162r}\mathbf{XY}_{9r} + K_{207r}\mathbf{YZ}_{9r}  \nonumber \\
&- K_{169r}^2\mathbf{XZ}_{9r} + K_{171r}^2\mathbf{XZ}_{9r} - K_{209r}\mathbf{MX}_{9r}  \nonumber \\
&- K_{164r}\mathbf{MZ}_{9r} + K_{169r}K_{171r}\mathbf{XX}_{9r} + K_{170r}K_{171r}\mathbf{XY}_{9r}  \nonumber \\
&- K_{169r}K_{170r}\mathbf{YZ}_{9r} - K_{169r}K_{171r}\mathbf{ZZ}_{9r} \nonumber \\
D_{314r} &= K_{188r}\mathbf{MX}_{9r} + K_{144r}\mathbf{MY}_{9r} \nonumber \\
D_{315r} &= K_{172r}\mathbf{YZ}_{9r} - K_{130r}\mathbf{XZ}_{9r} + K_{179r}\mathbf{ZZ}_{9r}  \nonumber \\
&+ K_{189r}\mathbf{MX}_{9r} + K_{145r}\mathbf{MY}_{9r} \nonumber \\
D_{316r} &= K_{173r}\mathbf{YZ}_{9r} - K_{131r}\mathbf{XZ}_{9r} + K_{180r}\mathbf{ZZ}_{9r}  \nonumber \\
&+ K_{190r}\mathbf{MX}_{9r} + K_{146r}\mathbf{MY}_{9r} \nonumber \\
D_{317r} &= K_{173r}\mathbf{YZ}_{9r} - K_{131r}\mathbf{XZ}_{9r} + K_{180r}\mathbf{ZZ}_{9r}  \nonumber \\
&+ K_{191r}\mathbf{MX}_{9r} + K_{147r}\mathbf{MY}_{9r} \nonumber \\
D_{318r} &= K_{174r}\mathbf{YZ}_{9r} - K_{132r}\mathbf{XZ}_{9r} + K_{181r}\mathbf{ZZ}_{9r}  \nonumber \\
&+ K_{192r}\mathbf{MX}_{9r} + K_{148r}\mathbf{MY}_{9r} \nonumber \\
D_{319r} &= K_{175r}\mathbf{YZ}_{9r} - K_{133r}\mathbf{XZ}_{9r} + K_{182r}\mathbf{ZZ}_{9r}  \nonumber \\
&+ K_{193r}\mathbf{MX}_{9r} + K_{149r}\mathbf{MY}_{9r} \nonumber \\
D_{320r} &= K_{176r}\mathbf{YZ}_{9r} - K_{134r}\mathbf{XZ}_{9r} + K_{183r}\mathbf{ZZ}_{9r}  \nonumber \\
&+ K_{194r}\mathbf{MX}_{9r} + K_{150r}\mathbf{MY}_{9r} \nonumber \\
D_{321r} &= K_{177r}\mathbf{YZ}_{9r} - K_{135r}\mathbf{XZ}_{9r} + K_{184r}\mathbf{ZZ}_{9r}  \nonumber \\
&+ K_{195r}\mathbf{MX}_{9r} + K_{151r}\mathbf{MY}_{9r} \nonumber \\
D_{322r} &= K_{178r}\mathbf{YZ}_{9r} + K_{185r}\mathbf{ZZ}_{9r} - \mathbf{XZ}_{9r}c_{5r}  \nonumber \\
&+ K_{196r}\mathbf{MX}_{9r} + K_{152r}\mathbf{MY}_{9r} \nonumber \\
D_{323r} &= \mathbf{YZ}_{9r}c_{6r} + \mathbf{ZZ}_{9r}s_{6r} \nonumber \\
D_{324r} &= K_{206r}\mathbf{YZ}_{9r} - K_{162r}\mathbf{XZ}_{9r} + K_{207r}\mathbf{ZZ}_{9r}  \nonumber \\
&+ K_{169r}^2\mathbf{XY}_{9r} - K_{170r}^2\mathbf{XY}_{9r} + K_{208r}\mathbf{MX}_{9r}  \nonumber \\
&+ K_{164r}\mathbf{MY}_{9r} - K_{169r}K_{170r}\mathbf{XX}_{9r} - K_{170r}K_{171r}\mathbf{XZ}_{9r}  \nonumber \\
&+ K_{169r}K_{170r}\mathbf{YY}_{9r} + K_{169r}K_{171r}\mathbf{YZ}_{9r} \nonumber \\
 \dot{\bar{H}}_{9r} &= \left[\begin{matrix} D_{269r} + D_{260r}\ddot{\psi} + D_{264r}\ddot{q}_{1r} + D_{262r}\ddot{q}_{w} + D_{265r}\ddot{q}_{2r} + D_{266r}\ddot{q}_{3r} + D_{267r}\ddot{q}_{4r} + D_{268r}\ddot{q}_{5r} + D_{261r}\ddot{q}_{imu} + D_{263r}\ddot{q}_{torso} + D_{259r}\ddot{x} & D_{280r} + D_{271r}\ddot{\psi} + D_{275r}\ddot{q}_{1r} + D_{273r}\ddot{q}_{w} + D_{276r}\ddot{q}_{2r} + D_{277r}\ddot{q}_{3r} + D_{278r}\ddot{q}_{4r} + D_{279r}\ddot{q}_{5r} + D_{272r}\ddot{q}_{imu} + D_{274r}\ddot{q}_{torso} + D_{270r}\ddot{x} + \mathbf{MZ}_{9r}\ddot{q}_{6r} & D_{291r} + D_{282r}\ddot{\psi} + D_{286r}\ddot{q}_{1r} + D_{284r}\ddot{q}_{w} + D_{287r}\ddot{q}_{2r} + D_{288r}\ddot{q}_{3r} + D_{289r}\ddot{q}_{4r} + D_{290r}\ddot{q}_{5r} + D_{283r}\ddot{q}_{imu} + D_{285r}\ddot{q}_{torso} + D_{281r}\ddot{x} - \mathbf{MY}_{9r}\ddot{q}_{6r} &  \end{matrix}\right] 
 \nonumber \\ 
 \bar\omega_{10r} &= {}^{10r}A_{9r} \bar\omega_{9r} + \dot{q}_{10r} \bar{e}_{10r} 
 \nonumber \\ 
 \bar\omega_{10r} &= \left[\begin{matrix} K_{171r}s_{7r} - K_{169r}c_{7r} & - K_{171r}c_{7r} - K_{169r}s_{7r} & - K_{170r} - \dot{q}_{7r} &  \end{matrix}\right] 
 \nonumber \\ 
K_{212r} &= K_{171r}s_{7r} - K_{169r}c_{7r} \nonumber \\
K_{213r} &= - K_{171r}c_{7r} - K_{169r}s_{7r} \nonumber \\
K_{214r} &= - K_{170r} - \dot{q}_{7r} \nonumber \\
 \bar\omega_{10r} &= \left[\begin{matrix} K_{212r} & K_{213r} & K_{214r} &  \end{matrix}\right] 
 \nonumber \\ 
 \bar\omega_{10r} &= \left[\begin{matrix} s_{7r}(K_{179r}\dot{\psi} + K_{182r}\dot{q}_{1r} + K_{180r}\dot{q}_{w} + K_{183r}\dot{q}_{2r} + K_{184r}\dot{q}_{3r} + K_{185r}\dot{q}_{4r} + K_{180r}\dot{q}_{imu} + K_{181r}\dot{q}_{torso} + \dot{q}_{5r}s_{6r}) + c_{7r}(\dot{q}_{6r} + K_{130r}\dot{\psi} + K_{133r}\dot{q}_{1r} + K_{131r}\dot{q}_{w} + K_{134r}\dot{q}_{2r} + K_{135r}\dot{q}_{3r} + K_{131r}\dot{q}_{imu} + K_{132r}\dot{q}_{torso} + \dot{q}_{4r}c_{5r}) & s_{7r}(\dot{q}_{6r} + K_{130r}\dot{\psi} + K_{133r}\dot{q}_{1r} + K_{131r}\dot{q}_{w} + K_{134r}\dot{q}_{2r} + K_{135r}\dot{q}_{3r} + K_{131r}\dot{q}_{imu} + K_{132r}\dot{q}_{torso} + \dot{q}_{4r}c_{5r}) - c_{7r}(K_{179r}\dot{\psi} + K_{182r}\dot{q}_{1r} + K_{180r}\dot{q}_{w} + K_{183r}\dot{q}_{2r} + K_{184r}\dot{q}_{3r} + K_{185r}\dot{q}_{4r} + K_{180r}\dot{q}_{imu} + K_{181r}\dot{q}_{torso} + \dot{q}_{5r}s_{6r}) & - \dot{q}_{7r} - K_{172r}\dot{\psi} - K_{175r}\dot{q}_{1r} - K_{173r}\dot{q}_{w} - K_{176r}\dot{q}_{2r} - K_{177r}\dot{q}_{3r} - K_{178r}\dot{q}_{4r} - K_{173r}\dot{q}_{imu} - K_{174r}\dot{q}_{torso} - \dot{q}_{5r}c_{6r} &  \end{matrix}\right] 
 \nonumber \\ 
K_{215r} &= K_{130r}c_{7r} + K_{179r}s_{7r} \nonumber \\
K_{216r} &= K_{131r}c_{7r} + K_{180r}s_{7r} \nonumber \\
K_{217r} &= K_{132r}c_{7r} + K_{181r}s_{7r} \nonumber \\
K_{218r} &= K_{133r}c_{7r} + K_{182r}s_{7r} \nonumber \\
K_{219r} &= K_{134r}c_{7r} + K_{183r}s_{7r} \nonumber \\
K_{220r} &= K_{135r}c_{7r} + K_{184r}s_{7r} \nonumber \\
K_{221r} &= c_{5r}c_{7r} + K_{185r}s_{7r} \nonumber \\
K_{222r} &= s_{6r}s_{7r} \nonumber \\
K_{223r} &= K_{130r}s_{7r} - K_{179r}c_{7r} \nonumber \\
K_{224r} &= K_{131r}s_{7r} - K_{180r}c_{7r} \nonumber \\
K_{225r} &= K_{132r}s_{7r} - K_{181r}c_{7r} \nonumber \\
K_{226r} &= K_{133r}s_{7r} - K_{182r}c_{7r} \nonumber \\
K_{227r} &= K_{134r}s_{7r} - K_{183r}c_{7r} \nonumber \\
K_{228r} &= K_{135r}s_{7r} - K_{184r}c_{7r} \nonumber \\
K_{229r} &= c_{5r}s_{7r} - K_{185r}c_{7r} \nonumber \\
K_{230r} &= -c_{7r}s_{6r} \nonumber \\
 \bar\omega_{10r} &= \left[\begin{matrix} K_{215r}\dot{\psi} + K_{218r}\dot{q}_{1r} + K_{216r}\dot{q}_{w} + K_{219r}\dot{q}_{2r} + K_{220r}\dot{q}_{3r} + K_{221r}\dot{q}_{4r} + K_{222r}\dot{q}_{5r} + K_{216r}\dot{q}_{imu} + K_{217r}\dot{q}_{torso} + \dot{q}_{6r}c_{7r} & K_{223r}\dot{\psi} + K_{226r}\dot{q}_{1r} + K_{224r}\dot{q}_{w} + K_{227r}\dot{q}_{2r} + K_{228r}\dot{q}_{3r} + K_{229r}\dot{q}_{4r} + K_{230r}\dot{q}_{5r} + K_{224r}\dot{q}_{imu} + K_{225r}\dot{q}_{torso} + \dot{q}_{6r}s_{7r} & - \dot{q}_{7r} - K_{172r}\dot{\psi} - K_{175r}\dot{q}_{1r} - K_{173r}\dot{q}_{w} - K_{176r}\dot{q}_{2r} - K_{177r}\dot{q}_{3r} - K_{178r}\dot{q}_{4r} - K_{173r}\dot{q}_{imu} - K_{174r}\dot{q}_{torso} - \dot{q}_{5r}c_{6r} &  \end{matrix}\right] 
 \nonumber \\ 
 \bar{v}_{10r} &= {}^{10r}A_{9r} \left(\bar{v}_{9r} + \bar\omega_{9r} \times \bar{P}_{10r}\right) 
 \nonumber \\ 
 \bar{v}_{10r} &= \left[\begin{matrix} c_{7r}(K_{142r} - K_{171r}L_9) + s_{7r}(K_{187r} - K_{169r}L_9) & s_{7r}(K_{142r} - K_{171r}L_9) - c_{7r}(K_{187r} - K_{169r}L_9) & -K_{186r} &  \end{matrix}\right] 
 \nonumber \\ 
K_{231r} &= c_{7r}(K_{142r} - K_{171r}L_9) + s_{7r}(K_{187r}  \nonumber \\
&- K_{169r}L_9) \nonumber \\
K_{232r} &= s_{7r}(K_{142r} - K_{171r}L_9) - c_{7r}(K_{187r}  \nonumber \\
&- K_{169r}L_9) \nonumber \\
 \bar{v}_{10r} &= \left[\begin{matrix} K_{231r} & K_{232r} & -K_{186r} &  \end{matrix}\right] 
 \nonumber \\ 
 \bar{v}_{10r} &= \left[\begin{matrix} c_{7r}(K_{145r}\dot{\psi} + K_{149r}\dot{q}_{1r} + K_{147r}\dot{q}_{w} + K_{150r}\dot{q}_{2r} + K_{151r}\dot{q}_{3r} + K_{152r}\dot{q}_{4r} + K_{146r}\dot{q}_{imu} + K_{148r}\dot{q}_{torso} + K_{144r}\dot{x} - L_9(K_{179r}\dot{\psi} + K_{182r}\dot{q}_{1r} + K_{180r}\dot{q}_{w} + K_{183r}\dot{q}_{2r} + K_{184r}\dot{q}_{3r} + K_{185r}\dot{q}_{4r} + K_{180r}\dot{q}_{imu} + K_{181r}\dot{q}_{torso} + \dot{q}_{5r}s_{6r})) + s_{7r}(K_{198r}\dot{\psi} + K_{202r}\dot{q}_{1r} + K_{200r}\dot{q}_{w} + K_{203r}\dot{q}_{2r} + K_{204r}\dot{q}_{3r} + K_{205r}\dot{q}_{4r} + K_{199r}\dot{q}_{imu} + K_{201r}\dot{q}_{torso} + K_{197r}\dot{x} + L_9(\dot{q}_{6r} + K_{130r}\dot{\psi} + K_{133r}\dot{q}_{1r} + K_{131r}\dot{q}_{w} + K_{134r}\dot{q}_{2r} + K_{135r}\dot{q}_{3r} + K_{131r}\dot{q}_{imu} + K_{132r}\dot{q}_{torso} + \dot{q}_{4r}c_{5r})) & s_{7r}(K_{145r}\dot{\psi} + K_{149r}\dot{q}_{1r} + K_{147r}\dot{q}_{w} + K_{150r}\dot{q}_{2r} + K_{151r}\dot{q}_{3r} + K_{152r}\dot{q}_{4r} + K_{146r}\dot{q}_{imu} + K_{148r}\dot{q}_{torso} + K_{144r}\dot{x} - L_9(K_{179r}\dot{\psi} + K_{182r}\dot{q}_{1r} + K_{180r}\dot{q}_{w} + K_{183r}\dot{q}_{2r} + K_{184r}\dot{q}_{3r} + K_{185r}\dot{q}_{4r} + K_{180r}\dot{q}_{imu} + K_{181r}\dot{q}_{torso} + \dot{q}_{5r}s_{6r})) - c_{7r}(K_{198r}\dot{\psi} + K_{202r}\dot{q}_{1r} + K_{200r}\dot{q}_{w} + K_{203r}\dot{q}_{2r} + K_{204r}\dot{q}_{3r} + K_{205r}\dot{q}_{4r} + K_{199r}\dot{q}_{imu} + K_{201r}\dot{q}_{torso} + K_{197r}\dot{x} + L_9(\dot{q}_{6r} + K_{130r}\dot{\psi} + K_{133r}\dot{q}_{1r} + K_{131r}\dot{q}_{w} + K_{134r}\dot{q}_{2r} + K_{135r}\dot{q}_{3r} + K_{131r}\dot{q}_{imu} + K_{132r}\dot{q}_{torso} + \dot{q}_{4r}c_{5r})) & - K_{189r}\dot{\psi} - K_{193r}\dot{q}_{1r} - K_{191r}\dot{q}_{w} - K_{194r}\dot{q}_{2r} - K_{195r}\dot{q}_{3r} - K_{196r}\dot{q}_{4r} - K_{190r}\dot{q}_{imu} - K_{192r}\dot{q}_{torso} - K_{188r}\dot{x} &  \end{matrix}\right] 
 \nonumber \\ 
K_{233r} &= K_{144r}c_{7r} + K_{197r}s_{7r} \nonumber \\
K_{234r} &= c_{7r}(K_{145r} - K_{179r}L_9) + s_{7r}(K_{198r}  \nonumber \\
&+ K_{130r}L_9) \nonumber \\
K_{235r} &= c_{7r}(K_{146r} - K_{180r}L_9) + s_{7r}(K_{199r}  \nonumber \\
&+ K_{131r}L_9) \nonumber \\
K_{236r} &= c_{7r}(K_{147r} - K_{180r}L_9) + s_{7r}(K_{200r}  \nonumber \\
&+ K_{131r}L_9) \nonumber \\
K_{237r} &= c_{7r}(K_{148r} - K_{181r}L_9) + s_{7r}(K_{201r}  \nonumber \\
&+ K_{132r}L_9) \nonumber \\
K_{238r} &= c_{7r}(K_{149r} - K_{182r}L_9) + s_{7r}(K_{202r}  \nonumber \\
&+ K_{133r}L_9) \nonumber \\
K_{239r} &= c_{7r}(K_{150r} - K_{183r}L_9) + s_{7r}(K_{203r}  \nonumber \\
&+ K_{134r}L_9) \nonumber \\
K_{240r} &= c_{7r}(K_{151r} - K_{184r}L_9) + s_{7r}(K_{204r}  \nonumber \\
&+ K_{135r}L_9) \nonumber \\
K_{241r} &= s_{7r}(K_{205r} + L_9c_{5r}) + c_{7r}(K_{152r}  \nonumber \\
&- K_{185r}L_9) \nonumber \\
K_{242r} &= -L_9c_{7r}s_{6r} \nonumber \\
K_{243r} &= L_9s_{7r} \nonumber \\
K_{244r} &= K_{144r}s_{7r} - K_{197r}c_{7r} \nonumber \\
K_{245r} &= s_{7r}(K_{145r} - K_{179r}L_9) - c_{7r}(K_{198r}  \nonumber \\
&+ K_{130r}L_9) \nonumber \\
K_{246r} &= s_{7r}(K_{146r} - K_{180r}L_9) - c_{7r}(K_{199r}  \nonumber \\
&+ K_{131r}L_9) \nonumber \\
K_{247r} &= s_{7r}(K_{147r} - K_{180r}L_9) - c_{7r}(K_{200r}  \nonumber \\
&+ K_{131r}L_9) \nonumber \\
K_{248r} &= s_{7r}(K_{148r} - K_{181r}L_9) - c_{7r}(K_{201r}  \nonumber \\
&+ K_{132r}L_9) \nonumber \\
K_{249r} &= s_{7r}(K_{149r} - K_{182r}L_9) - c_{7r}(K_{202r}  \nonumber \\
&+ K_{133r}L_9) \nonumber \\
K_{250r} &= s_{7r}(K_{150r} - K_{183r}L_9) - c_{7r}(K_{203r}  \nonumber \\
&+ K_{134r}L_9) \nonumber \\
K_{251r} &= s_{7r}(K_{151r} - K_{184r}L_9) - c_{7r}(K_{204r}  \nonumber \\
&+ K_{135r}L_9) \nonumber \\
K_{252r} &= s_{7r}(K_{152r} - K_{185r}L_9) - c_{7r}(K_{205r}  \nonumber \\
&+ L_9c_{5r}) \nonumber \\
K_{253r} &= -L_9s_{6r}s_{7r} \nonumber \\
K_{254r} &= -L_9c_{7r} \nonumber \\
 \bar{v}_{10r} &= \left[\begin{matrix} K_{234r}\dot{\psi} + K_{238r}\dot{q}_{1r} + K_{236r}\dot{q}_{w} + K_{239r}\dot{q}_{2r} + K_{240r}\dot{q}_{3r} + K_{241r}\dot{q}_{4r} + K_{242r}\dot{q}_{5r} + K_{243r}\dot{q}_{6r} + K_{235r}\dot{q}_{imu} + K_{237r}\dot{q}_{torso} + K_{233r}\dot{x} & K_{245r}\dot{\psi} + K_{249r}\dot{q}_{1r} + K_{247r}\dot{q}_{w} + K_{250r}\dot{q}_{2r} + K_{251r}\dot{q}_{3r} + K_{252r}\dot{q}_{4r} + K_{253r}\dot{q}_{5r} + K_{254r}\dot{q}_{6r} + K_{246r}\dot{q}_{imu} + K_{248r}\dot{q}_{torso} + K_{244r}\dot{x} & - K_{189r}\dot{\psi} - K_{193r}\dot{q}_{1r} - K_{191r}\dot{q}_{w} - K_{194r}\dot{q}_{2r} - K_{195r}\dot{q}_{3r} - K_{196r}\dot{q}_{4r} - K_{190r}\dot{q}_{imu} - K_{192r}\dot{q}_{torso} - K_{188r}\dot{x} &  \end{matrix}\right] 
 \nonumber \\ 
 \bar\alpha_{10r} &= {}^{10r}A_{9r} \bar\alpha_{9r} + \ddot{q}_{10r} \bar{e}_{10r} + \dot{q}_{10r} \left(\bar\omega_{10r} \times \bar{e}_{10r}\right) 
 \nonumber \\ 
 \bar\alpha_{10r} &= \left[\begin{matrix} s_{7r}(K_{207r} + K_{179r}\ddot{\psi} + K_{182r}\ddot{q}_{1r} + K_{180r}\ddot{q}_{w} + K_{183r}\ddot{q}_{2r} + K_{184r}\ddot{q}_{3r} + K_{185r}\ddot{q}_{4r} + K_{180r}\ddot{q}_{imu} + K_{181r}\ddot{q}_{torso} + \ddot{q}_{5r}s_{6r}) - K_{213r}\dot{q}_{7r} + c_{7r}(K_{162r} + \ddot{q}_{6r} + K_{130r}\ddot{\psi} + K_{133r}\ddot{q}_{1r} + K_{131r}\ddot{q}_{w} + K_{134r}\ddot{q}_{2r} + K_{135r}\ddot{q}_{3r} + K_{131r}\ddot{q}_{imu} + K_{132r}\ddot{q}_{torso} + \ddot{q}_{4r}c_{5r}) & K_{212r}\dot{q}_{7r} + s_{7r}(K_{162r} + \ddot{q}_{6r} + K_{130r}\ddot{\psi} + K_{133r}\ddot{q}_{1r} + K_{131r}\ddot{q}_{w} + K_{134r}\ddot{q}_{2r} + K_{135r}\ddot{q}_{3r} + K_{131r}\ddot{q}_{imu} + K_{132r}\ddot{q}_{torso} + \ddot{q}_{4r}c_{5r}) - c_{7r}(K_{207r} + K_{179r}\ddot{\psi} + K_{182r}\ddot{q}_{1r} + K_{180r}\ddot{q}_{w} + K_{183r}\ddot{q}_{2r} + K_{184r}\ddot{q}_{3r} + K_{185r}\ddot{q}_{4r} + K_{180r}\ddot{q}_{imu} + K_{181r}\ddot{q}_{torso} + \ddot{q}_{5r}s_{6r}) & - K_{206r} - \ddot{q}_{7r} - K_{172r}\ddot{\psi} - K_{175r}\ddot{q}_{1r} - K_{173r}\ddot{q}_{w} - K_{176r}\ddot{q}_{2r} - K_{177r}\ddot{q}_{3r} - K_{178r}\ddot{q}_{4r} - K_{173r}\ddot{q}_{imu} - K_{174r}\ddot{q}_{torso} - \ddot{q}_{5r}c_{6r} &  \end{matrix}\right] 
 \nonumber \\ 
K_{255r} &= K_{162r}c_{7r} - K_{213r}\dot{q}_{7r} + K_{207r}s_{7r} \nonumber \\
K_{256r} &= K_{212r}\dot{q}_{7r} - K_{207r}c_{7r} + K_{162r}s_{7r} \nonumber \\
 \bar\alpha_{10r} &= \left[\begin{matrix} K_{255r} + K_{215r}\ddot{\psi} + K_{218r}\ddot{q}_{1r} + K_{216r}\ddot{q}_{w} + K_{219r}\ddot{q}_{2r} + K_{220r}\ddot{q}_{3r} + K_{221r}\ddot{q}_{4r} + K_{222r}\ddot{q}_{5r} + K_{216r}\ddot{q}_{imu} + K_{217r}\ddot{q}_{torso} + \ddot{q}_{6r}c_{7r} & K_{256r} + K_{223r}\ddot{\psi} + K_{226r}\ddot{q}_{1r} + K_{224r}\ddot{q}_{w} + K_{227r}\ddot{q}_{2r} + K_{228r}\ddot{q}_{3r} + K_{229r}\ddot{q}_{4r} + K_{230r}\ddot{q}_{5r} + K_{224r}\ddot{q}_{imu} + K_{225r}\ddot{q}_{torso} + \ddot{q}_{6r}s_{7r} & - K_{206r} - \ddot{q}_{7r} - K_{172r}\ddot{\psi} - K_{175r}\ddot{q}_{1r} - K_{173r}\ddot{q}_{w} - K_{176r}\ddot{q}_{2r} - K_{177r}\ddot{q}_{3r} - K_{178r}\ddot{q}_{4r} - K_{173r}\ddot{q}_{imu} - K_{174r}\ddot{q}_{torso} - \ddot{q}_{5r}c_{6r} &  \end{matrix}\right] 
 \nonumber \\ 
 \bar{a}_{10r} &= {}^{10r}A_{9r} \left(\bar{a}_{9r} + \bar\alpha_{9r} \times \bar{P}_{10r} + \bar\omega_{9r} \times \left(\bar\omega_{9r} \times \bar{P}_{10r}\right)\right) 
 \nonumber \\ 
 \bar\alpha_{10r} &= \left[\begin{matrix} c_{7r}(K_{164r} + K_{145r}\ddot{\psi} + K_{149r}\ddot{q}_{1r} + K_{147r}\ddot{q}_{w} + K_{150r}\ddot{q}_{2r} + K_{151r}\ddot{q}_{3r} + K_{152r}\ddot{q}_{4r} + K_{146r}\ddot{q}_{imu} + K_{148r}\ddot{q}_{torso} + K_{144r}\ddot{x} - L_9(K_{207r} + K_{179r}\ddot{\psi} + K_{182r}\ddot{q}_{1r} + K_{180r}\ddot{q}_{w} + K_{183r}\ddot{q}_{2r} + K_{184r}\ddot{q}_{3r} + K_{185r}\ddot{q}_{4r} + K_{180r}\ddot{q}_{imu} + K_{181r}\ddot{q}_{torso} + \ddot{q}_{5r}s_{6r}) + K_{169r}K_{170r}L_9) + s_{7r}(K_{209r} + K_{198r}\ddot{\psi} + K_{202r}\ddot{q}_{1r} + K_{200r}\ddot{q}_{w} + K_{203r}\ddot{q}_{2r} + K_{204r}\ddot{q}_{3r} + K_{205r}\ddot{q}_{4r} + K_{199r}\ddot{q}_{imu} + K_{201r}\ddot{q}_{torso} + K_{197r}\ddot{x} + L_9(K_{162r} + \ddot{q}_{6r} + K_{130r}\ddot{\psi} + K_{133r}\ddot{q}_{1r} + K_{131r}\ddot{q}_{w} + K_{134r}\ddot{q}_{2r} + K_{135r}\ddot{q}_{3r} + K_{131r}\ddot{q}_{imu} + K_{132r}\ddot{q}_{torso} + \ddot{q}_{4r}c_{5r}) - K_{170r}K_{171r}L_9) & s_{7r}(K_{164r} + K_{145r}\ddot{\psi} + K_{149r}\ddot{q}_{1r} + K_{147r}\ddot{q}_{w} + K_{150r}\ddot{q}_{2r} + K_{151r}\ddot{q}_{3r} + K_{152r}\ddot{q}_{4r} + K_{146r}\ddot{q}_{imu} + K_{148r}\ddot{q}_{torso} + K_{144r}\ddot{x} - L_9(K_{207r} + K_{179r}\ddot{\psi} + K_{182r}\ddot{q}_{1r} + K_{180r}\ddot{q}_{w} + K_{183r}\ddot{q}_{2r} + K_{184r}\ddot{q}_{3r} + K_{185r}\ddot{q}_{4r} + K_{180r}\ddot{q}_{imu} + K_{181r}\ddot{q}_{torso} + \ddot{q}_{5r}s_{6r}) + K_{169r}K_{170r}L_9) - c_{7r}(K_{209r} + K_{198r}\ddot{\psi} + K_{202r}\ddot{q}_{1r} + K_{200r}\ddot{q}_{w} + K_{203r}\ddot{q}_{2r} + K_{204r}\ddot{q}_{3r} + K_{205r}\ddot{q}_{4r} + K_{199r}\ddot{q}_{imu} + K_{201r}\ddot{q}_{torso} + K_{197r}\ddot{x} + L_9(K_{162r} + \ddot{q}_{6r} + K_{130r}\ddot{\psi} + K_{133r}\ddot{q}_{1r} + K_{131r}\ddot{q}_{w} + K_{134r}\ddot{q}_{2r} + K_{135r}\ddot{q}_{3r} + K_{131r}\ddot{q}_{imu} + K_{132r}\ddot{q}_{torso} + \ddot{q}_{4r}c_{5r}) - K_{170r}K_{171r}L_9) & - K_{208r} - K_{189r}\ddot{\psi} - K_{193r}\ddot{q}_{1r} - K_{191r}\ddot{q}_{w} - K_{194r}\ddot{q}_{2r} - K_{195r}\ddot{q}_{3r} - K_{196r}\ddot{q}_{4r} - K_{190r}\ddot{q}_{imu} - K_{192r}\ddot{q}_{torso} - K_{188r}\ddot{x} - K_{169r}^2L_9 - K_{171r}^2L_9 &  \end{matrix}\right] 
 \nonumber \\ 
K_{257r} &= K_{164r}c_{7r} + K_{209r}s_{7r} - K_{207r}L_9c_{7r}  \nonumber \\
&+ K_{162r}L_9s_{7r} + K_{169r}K_{170r}L_9c_{7r}  \nonumber \\
&- K_{170r}K_{171r}L_9s_{7r} \nonumber \\
K_{258r} &= K_{164r}s_{7r} - K_{209r}c_{7r} - K_{162r}L_9c_{7r}  \nonumber \\
&- K_{207r}L_9s_{7r} + K_{170r}K_{171r}L_9c_{7r}  \nonumber \\
&+ K_{169r}K_{170r}L_9s_{7r} \nonumber \\
K_{259r} &= - K_{208r} - K_{169r}^2L_9 - K_{171r}^2L_9 \nonumber \\
 \bar{a}_{10r} &= \left[\begin{matrix} K_{257r} + K_{234r}\ddot{\psi} + K_{238r}\ddot{q}_{1r} + K_{236r}\ddot{q}_{w} + K_{239r}\ddot{q}_{2r} + K_{240r}\ddot{q}_{3r} + K_{241r}\ddot{q}_{4r} + K_{242r}\ddot{q}_{5r} + K_{243r}\ddot{q}_{6r} + K_{235r}\ddot{q}_{imu} + K_{237r}\ddot{q}_{torso} + K_{233r}\ddot{x} & K_{258r} + K_{245r}\ddot{\psi} + K_{249r}\ddot{q}_{1r} + K_{247r}\ddot{q}_{w} + K_{250r}\ddot{q}_{2r} + K_{251r}\ddot{q}_{3r} + K_{252r}\ddot{q}_{4r} + K_{253r}\ddot{q}_{5r} + K_{254r}\ddot{q}_{6r} + K_{246r}\ddot{q}_{imu} + K_{248r}\ddot{q}_{torso} + K_{244r}\ddot{x} & K_{259r} - K_{189r}\ddot{\psi} - K_{193r}\ddot{q}_{1r} - K_{191r}\ddot{q}_{w} - K_{194r}\ddot{q}_{2r} - K_{195r}\ddot{q}_{3r} - K_{196r}\ddot{q}_{4r} - K_{190r}\ddot{q}_{imu} - K_{192r}\ddot{q}_{torso} - K_{188r}\ddot{x} &  \end{matrix}\right] 
 \nonumber \\ 
 \bar{g}_{10r} &= {}^{10r}A_{9r} \bar{g}_{9r} 
 \nonumber \\ 
 \bar{g}_{10r} &= \left[\begin{matrix} K_{167r}gc_{7r} + K_{211r}gs_{7r} & K_{167r}gs_{7r} - K_{211r}gc_{7r} & -K_{210r}g &  \end{matrix}\right] 
 \nonumber \\ 
K_{260r} &= K_{167r}c_{7r} + K_{211r}s_{7r} \nonumber \\
K_{261r} &= K_{167r}s_{7r} - K_{211r}c_{7r} \nonumber \\
 \bar{g}_{10r} &= \left[\begin{matrix} K_{260r}g & K_{261r}g & -K_{210r}g &  \end{matrix}\right] 
 \nonumber \\ 
 m_{10r}\bar{S}_{10r}^{\times}\bar{g}_{10r} &= \mathbf{MS}_{10r} \times \bar{g}_{10r} 
 \nonumber \\ 
 m_{10r}\bar{S}_{10r}^{\times}\bar{g}_{10r} &= \left[\begin{matrix} - K_{210r}\mathbf{MY}_{10r}g - K_{261r}\mathbf{MZ}_{10r}g & K_{210r}\mathbf{MX}_{10r}g + K_{260r}\mathbf{MZ}_{10r}g & K_{261r}\mathbf{MX}_{10r}g - K_{260r}\mathbf{MY}_{10r}g &  \end{matrix}\right] 
 \nonumber \\ 
D_{325r} &= - K_{210r}\mathbf{MY}_{10r} - K_{261r}\mathbf{MZ}_{10r} \nonumber \\
D_{326r} &= K_{210r}\mathbf{MX}_{10r} + K_{260r}\mathbf{MZ}_{10r} \nonumber \\
D_{327r} &= K_{261r}\mathbf{MX}_{10r} - K_{260r}\mathbf{MY}_{10r} \nonumber \\
 m_{10r}\bar{S}_{10r}^{\times}\bar{g}_{10r} &= \left[\begin{matrix} D_{325r}g & D_{326r}g & D_{327r}g &  \end{matrix}\right] 
 \nonumber \\ 
 m_{10r}\bar{a}_{G(10r)} &= m_{10r}\bar{a}_{10r} + \bar\alpha_{10r} \times \mathbf{MS}_{10r} + \bar\omega_{10r} \times \left(\bar\omega_{10r} \times \mathbf{MS}_{10r}\right) 
 \nonumber \\ 
 m_{10r}\bar{a}_{G(10r)} &= \left[\begin{matrix} m_{10r}(K_{257r} + K_{234r}\ddot{\psi} + K_{238r}\ddot{q}_{1r} + K_{236r}\ddot{q}_{w} + K_{239r}\ddot{q}_{2r} + K_{240r}\ddot{q}_{3r} + K_{241r}\ddot{q}_{4r} + K_{242r}\ddot{q}_{5r} + K_{243r}\ddot{q}_{6r} + K_{235r}\ddot{q}_{imu} + K_{237r}\ddot{q}_{torso} + K_{233r}\ddot{x}) + \mathbf{MZ}_{10r}(K_{256r} + K_{223r}\ddot{\psi} + K_{226r}\ddot{q}_{1r} + K_{224r}\ddot{q}_{w} + K_{227r}\ddot{q}_{2r} + K_{228r}\ddot{q}_{3r} + K_{229r}\ddot{q}_{4r} + K_{230r}\ddot{q}_{5r} + K_{224r}\ddot{q}_{imu} + K_{225r}\ddot{q}_{torso} + \ddot{q}_{6r}s_{7r}) + \mathbf{MY}_{10r}(K_{206r} + \ddot{q}_{7r} + K_{172r}\ddot{\psi} + K_{175r}\ddot{q}_{1r} + K_{173r}\ddot{q}_{w} + K_{176r}\ddot{q}_{2r} + K_{177r}\ddot{q}_{3r} + K_{178r}\ddot{q}_{4r} + K_{173r}\ddot{q}_{imu} + K_{174r}\ddot{q}_{torso} + \ddot{q}_{5r}c_{6r}) - K_{213r}(K_{213r}\mathbf{MX}_{10r} - K_{212r}\mathbf{MY}_{10r}) - K_{214r}(K_{214r}\mathbf{MX}_{10r} - K_{212r}\mathbf{MZ}_{10r}) & m_{10r}(K_{258r} + K_{245r}\ddot{\psi} + K_{249r}\ddot{q}_{1r} + K_{247r}\ddot{q}_{w} + K_{250r}\ddot{q}_{2r} + K_{251r}\ddot{q}_{3r} + K_{252r}\ddot{q}_{4r} + K_{253r}\ddot{q}_{5r} + K_{254r}\ddot{q}_{6r} + K_{246r}\ddot{q}_{imu} + K_{248r}\ddot{q}_{torso} + K_{244r}\ddot{x}) - \mathbf{MZ}_{10r}(K_{255r} + K_{215r}\ddot{\psi} + K_{218r}\ddot{q}_{1r} + K_{216r}\ddot{q}_{w} + K_{219r}\ddot{q}_{2r} + K_{220r}\ddot{q}_{3r} + K_{221r}\ddot{q}_{4r} + K_{222r}\ddot{q}_{5r} + K_{216r}\ddot{q}_{imu} + K_{217r}\ddot{q}_{torso} + \ddot{q}_{6r}c_{7r}) - \mathbf{MX}_{10r}(K_{206r} + \ddot{q}_{7r} + K_{172r}\ddot{\psi} + K_{175r}\ddot{q}_{1r} + K_{173r}\ddot{q}_{w} + K_{176r}\ddot{q}_{2r} + K_{177r}\ddot{q}_{3r} + K_{178r}\ddot{q}_{4r} + K_{173r}\ddot{q}_{imu} + K_{174r}\ddot{q}_{torso} + \ddot{q}_{5r}c_{6r}) + K_{212r}(K_{213r}\mathbf{MX}_{10r} - K_{212r}\mathbf{MY}_{10r}) - K_{214r}(K_{214r}\mathbf{MY}_{10r} - K_{213r}\mathbf{MZ}_{10r}) & \mathbf{MY}_{10r}(K_{255r} + K_{215r}\ddot{\psi} + K_{218r}\ddot{q}_{1r} + K_{216r}\ddot{q}_{w} + K_{219r}\ddot{q}_{2r} + K_{220r}\ddot{q}_{3r} + K_{221r}\ddot{q}_{4r} + K_{222r}\ddot{q}_{5r} + K_{216r}\ddot{q}_{imu} + K_{217r}\ddot{q}_{torso} + \ddot{q}_{6r}c_{7r}) - m_{10r}(K_{189r}\ddot{\psi} - K_{259r} + K_{193r}\ddot{q}_{1r} + K_{191r}\ddot{q}_{w} + K_{194r}\ddot{q}_{2r} + K_{195r}\ddot{q}_{3r} + K_{196r}\ddot{q}_{4r} + K_{190r}\ddot{q}_{imu} + K_{192r}\ddot{q}_{torso} + K_{188r}\ddot{x}) - \mathbf{MX}_{10r}(K_{256r} + K_{223r}\ddot{\psi} + K_{226r}\ddot{q}_{1r} + K_{224r}\ddot{q}_{w} + K_{227r}\ddot{q}_{2r} + K_{228r}\ddot{q}_{3r} + K_{229r}\ddot{q}_{4r} + K_{230r}\ddot{q}_{5r} + K_{224r}\ddot{q}_{imu} + K_{225r}\ddot{q}_{torso} + \ddot{q}_{6r}s_{7r}) + K_{212r}(K_{214r}\mathbf{MX}_{10r} - K_{212r}\mathbf{MZ}_{10r}) + K_{213r}(K_{214r}\mathbf{MY}_{10r} - K_{213r}\mathbf{MZ}_{10r}) &  \end{matrix}\right] 
 \nonumber \\ 
D_{328r} &= K_{233r}m_{10r} \nonumber \\
D_{329r} &= K_{234r}m_{10r} + K_{172r}\mathbf{MY}_{10r} + K_{223r}\mathbf{MZ}_{10r} \nonumber \\
D_{330r} &= K_{235r}m_{10r} + K_{173r}\mathbf{MY}_{10r} + K_{224r}\mathbf{MZ}_{10r} \nonumber \\
D_{331r} &= K_{236r}m_{10r} + K_{173r}\mathbf{MY}_{10r} + K_{224r}\mathbf{MZ}_{10r} \nonumber \\
D_{332r} &= K_{237r}m_{10r} + K_{174r}\mathbf{MY}_{10r} + K_{225r}\mathbf{MZ}_{10r} \nonumber \\
D_{333r} &= K_{238r}m_{10r} + K_{175r}\mathbf{MY}_{10r} + K_{226r}\mathbf{MZ}_{10r} \nonumber \\
D_{334r} &= K_{239r}m_{10r} + K_{176r}\mathbf{MY}_{10r} + K_{227r}\mathbf{MZ}_{10r} \nonumber \\
D_{335r} &= K_{240r}m_{10r} + K_{177r}\mathbf{MY}_{10r} + K_{228r}\mathbf{MZ}_{10r} \nonumber \\
D_{336r} &= K_{241r}m_{10r} + K_{178r}\mathbf{MY}_{10r} + K_{229r}\mathbf{MZ}_{10r} \nonumber \\
D_{337r} &= K_{242r}m_{10r} + \mathbf{MY}_{10r}c_{6r} + K_{230r}\mathbf{MZ}_{10r} \nonumber \\
D_{338r} &= K_{243r}m_{10r} + \mathbf{MZ}_{10r}s_{7r} \nonumber \\
D_{339r} &= K_{257r}m_{10r} - K_{213r}^2\mathbf{MX}_{10r} - K_{214r}^2\mathbf{MX}_{10r}  \nonumber \\
&+ K_{206r}\mathbf{MY}_{10r} + K_{256r}\mathbf{MZ}_{10r} + K_{212r}K_{213r}\mathbf{MY}_{10r}  \nonumber \\
&+ K_{212r}K_{214r}\mathbf{MZ}_{10r} \nonumber \\
D_{340r} &= K_{244r}m_{10r} \nonumber \\
D_{341r} &= K_{245r}m_{10r} - K_{172r}\mathbf{MX}_{10r} - K_{215r}\mathbf{MZ}_{10r} \nonumber \\
D_{342r} &= K_{246r}m_{10r} - K_{173r}\mathbf{MX}_{10r} - K_{216r}\mathbf{MZ}_{10r} \nonumber \\
D_{343r} &= K_{247r}m_{10r} - K_{173r}\mathbf{MX}_{10r} - K_{216r}\mathbf{MZ}_{10r} \nonumber \\
D_{344r} &= K_{248r}m_{10r} - K_{174r}\mathbf{MX}_{10r} - K_{217r}\mathbf{MZ}_{10r} \nonumber \\
D_{345r} &= K_{249r}m_{10r} - K_{175r}\mathbf{MX}_{10r} - K_{218r}\mathbf{MZ}_{10r} \nonumber \\
D_{346r} &= K_{250r}m_{10r} - K_{176r}\mathbf{MX}_{10r} - K_{219r}\mathbf{MZ}_{10r} \nonumber \\
D_{347r} &= K_{251r}m_{10r} - K_{177r}\mathbf{MX}_{10r} - K_{220r}\mathbf{MZ}_{10r} \nonumber \\
D_{348r} &= K_{252r}m_{10r} - K_{178r}\mathbf{MX}_{10r} - K_{221r}\mathbf{MZ}_{10r} \nonumber \\
D_{349r} &= K_{253r}m_{10r} - \mathbf{MX}_{10r}c_{6r} - K_{222r}\mathbf{MZ}_{10r} \nonumber \\
D_{350r} &= K_{254r}m_{10r} - \mathbf{MZ}_{10r}c_{7r} \nonumber \\
D_{351r} &= K_{258r}m_{10r} - K_{212r}^2\mathbf{MY}_{10r} - K_{214r}^2\mathbf{MY}_{10r}  \nonumber \\
&- K_{206r}\mathbf{MX}_{10r} - K_{255r}\mathbf{MZ}_{10r} + K_{212r}K_{213r}\mathbf{MX}_{10r}  \nonumber \\
&+ K_{213r}K_{214r}\mathbf{MZ}_{10r} \nonumber \\
D_{352r} &= -K_{188r}m_{10r} \nonumber \\
D_{353r} &= K_{215r}\mathbf{MY}_{10r} - K_{223r}\mathbf{MX}_{10r} - K_{189r}m_{10r} \nonumber \\
D_{354r} &= K_{216r}\mathbf{MY}_{10r} - K_{224r}\mathbf{MX}_{10r} - K_{190r}m_{10r} \nonumber \\
D_{355r} &= K_{216r}\mathbf{MY}_{10r} - K_{224r}\mathbf{MX}_{10r} - K_{191r}m_{10r} \nonumber \\
D_{356r} &= K_{217r}\mathbf{MY}_{10r} - K_{225r}\mathbf{MX}_{10r} - K_{192r}m_{10r} \nonumber \\
D_{357r} &= K_{218r}\mathbf{MY}_{10r} - K_{226r}\mathbf{MX}_{10r} - K_{193r}m_{10r} \nonumber \\
D_{358r} &= K_{219r}\mathbf{MY}_{10r} - K_{227r}\mathbf{MX}_{10r} - K_{194r}m_{10r} \nonumber \\
D_{359r} &= K_{220r}\mathbf{MY}_{10r} - K_{228r}\mathbf{MX}_{10r} - K_{195r}m_{10r} \nonumber \\
D_{360r} &= K_{221r}\mathbf{MY}_{10r} - K_{229r}\mathbf{MX}_{10r} - K_{196r}m_{10r} \nonumber \\
D_{361r} &= K_{222r}\mathbf{MY}_{10r} - K_{230r}\mathbf{MX}_{10r} \nonumber \\
D_{362r} &= \mathbf{MY}_{10r}c_{7r} - \mathbf{MX}_{10r}s_{7r} \nonumber \\
D_{363r} &= K_{259r}m_{10r} - K_{212r}^2\mathbf{MZ}_{10r} - K_{213r}^2\mathbf{MZ}_{10r}  \nonumber \\
&- K_{256r}\mathbf{MX}_{10r} + K_{255r}\mathbf{MY}_{10r} + K_{212r}K_{214r}\mathbf{MX}_{10r}  \nonumber \\
&+ K_{213r}K_{214r}\mathbf{MY}_{10r} \nonumber \\
 m_{10r}\bar{a}_{G(10r)} &= \left[\begin{matrix} D_{339r} + D_{329r}\ddot{\psi} + D_{333r}\ddot{q}_{1r} + D_{331r}\ddot{q}_{w} + D_{334r}\ddot{q}_{2r} + D_{335r}\ddot{q}_{3r} + D_{336r}\ddot{q}_{4r} + D_{337r}\ddot{q}_{5r} + D_{338r}\ddot{q}_{6r} + D_{330r}\ddot{q}_{imu} + D_{332r}\ddot{q}_{torso} + D_{328r}\ddot{x} + \mathbf{MY}_{10r}\ddot{q}_{7r} & D_{351r} + D_{341r}\ddot{\psi} + D_{345r}\ddot{q}_{1r} + D_{343r}\ddot{q}_{w} + D_{346r}\ddot{q}_{2r} + D_{347r}\ddot{q}_{3r} + D_{348r}\ddot{q}_{4r} + D_{349r}\ddot{q}_{5r} + D_{350r}\ddot{q}_{6r} + D_{342r}\ddot{q}_{imu} + D_{344r}\ddot{q}_{torso} + D_{340r}\ddot{x} - \mathbf{MX}_{10r}\ddot{q}_{7r} & D_{363r} + D_{353r}\ddot{\psi} + D_{357r}\ddot{q}_{1r} + D_{355r}\ddot{q}_{w} + D_{358r}\ddot{q}_{2r} + D_{359r}\ddot{q}_{3r} + D_{360r}\ddot{q}_{4r} + D_{361r}\ddot{q}_{5r} + D_{362r}\ddot{q}_{6r} + D_{354r}\ddot{q}_{imu} + D_{356r}\ddot{q}_{torso} + D_{352r}\ddot{x} &  \end{matrix}\right] 
 \nonumber \\ 
 \dot{\bar{H}}_{10r} &= \mathbf{MS}_{10r} \times \bar{a}_{10r} + J_{10r}\bar{\alpha}_{10r} + \bar\omega_{10r} \times J_{10r}\bar{\omega}_{10r} 
 \nonumber \\ 
 \dot{\bar{H}}_{10r} &= \left[\begin{matrix} K_{213r}(K_{212r}\mathbf{XZ}_{10r} + K_{213r}\mathbf{YZ}_{10r} + K_{214r}\mathbf{ZZ}_{10r}) - K_{214r}(K_{212r}\mathbf{XY}_{10r} + K_{213r}\mathbf{YY}_{10r} + K_{214r}\mathbf{YZ}_{10r}) - \mathbf{MY}_{10r}(K_{189r}\ddot{\psi} - K_{259r} + K_{193r}\ddot{q}_{1r} + K_{191r}\ddot{q}_{w} + K_{194r}\ddot{q}_{2r} + K_{195r}\ddot{q}_{3r} + K_{196r}\ddot{q}_{4r} + K_{190r}\ddot{q}_{imu} + K_{192r}\ddot{q}_{torso} + K_{188r}\ddot{x}) + \mathbf{XX}_{10r}(K_{255r} + K_{215r}\ddot{\psi} + K_{218r}\ddot{q}_{1r} + K_{216r}\ddot{q}_{w} + K_{219r}\ddot{q}_{2r} + K_{220r}\ddot{q}_{3r} + K_{221r}\ddot{q}_{4r} + K_{222r}\ddot{q}_{5r} + K_{216r}\ddot{q}_{imu} + K_{217r}\ddot{q}_{torso} + \ddot{q}_{6r}c_{7r}) + \mathbf{XY}_{10r}(K_{256r} + K_{223r}\ddot{\psi} + K_{226r}\ddot{q}_{1r} + K_{224r}\ddot{q}_{w} + K_{227r}\ddot{q}_{2r} + K_{228r}\ddot{q}_{3r} + K_{229r}\ddot{q}_{4r} + K_{230r}\ddot{q}_{5r} + K_{224r}\ddot{q}_{imu} + K_{225r}\ddot{q}_{torso} + \ddot{q}_{6r}s_{7r}) - \mathbf{XZ}_{10r}(K_{206r} + \ddot{q}_{7r} + K_{172r}\ddot{\psi} + K_{175r}\ddot{q}_{1r} + K_{173r}\ddot{q}_{w} + K_{176r}\ddot{q}_{2r} + K_{177r}\ddot{q}_{3r} + K_{178r}\ddot{q}_{4r} + K_{173r}\ddot{q}_{imu} + K_{174r}\ddot{q}_{torso} + \ddot{q}_{5r}c_{6r}) - \mathbf{MZ}_{10r}(K_{258r} + K_{245r}\ddot{\psi} + K_{249r}\ddot{q}_{1r} + K_{247r}\ddot{q}_{w} + K_{250r}\ddot{q}_{2r} + K_{251r}\ddot{q}_{3r} + K_{252r}\ddot{q}_{4r} + K_{253r}\ddot{q}_{5r} + K_{254r}\ddot{q}_{6r} + K_{246r}\ddot{q}_{imu} + K_{248r}\ddot{q}_{torso} + K_{244r}\ddot{x}) & K_{214r}(K_{212r}\mathbf{XX}_{10r} + K_{213r}\mathbf{XY}_{10r} + K_{214r}\mathbf{XZ}_{10r}) - K_{212r}(K_{212r}\mathbf{XZ}_{10r} + K_{213r}\mathbf{YZ}_{10r} + K_{214r}\mathbf{ZZ}_{10r}) + \mathbf{MX}_{10r}(K_{189r}\ddot{\psi} - K_{259r} + K_{193r}\ddot{q}_{1r} + K_{191r}\ddot{q}_{w} + K_{194r}\ddot{q}_{2r} + K_{195r}\ddot{q}_{3r} + K_{196r}\ddot{q}_{4r} + K_{190r}\ddot{q}_{imu} + K_{192r}\ddot{q}_{torso} + K_{188r}\ddot{x}) + \mathbf{XY}_{10r}(K_{255r} + K_{215r}\ddot{\psi} + K_{218r}\ddot{q}_{1r} + K_{216r}\ddot{q}_{w} + K_{219r}\ddot{q}_{2r} + K_{220r}\ddot{q}_{3r} + K_{221r}\ddot{q}_{4r} + K_{222r}\ddot{q}_{5r} + K_{216r}\ddot{q}_{imu} + K_{217r}\ddot{q}_{torso} + \ddot{q}_{6r}c_{7r}) + \mathbf{YY}_{10r}(K_{256r} + K_{223r}\ddot{\psi} + K_{226r}\ddot{q}_{1r} + K_{224r}\ddot{q}_{w} + K_{227r}\ddot{q}_{2r} + K_{228r}\ddot{q}_{3r} + K_{229r}\ddot{q}_{4r} + K_{230r}\ddot{q}_{5r} + K_{224r}\ddot{q}_{imu} + K_{225r}\ddot{q}_{torso} + \ddot{q}_{6r}s_{7r}) - \mathbf{YZ}_{10r}(K_{206r} + \ddot{q}_{7r} + K_{172r}\ddot{\psi} + K_{175r}\ddot{q}_{1r} + K_{173r}\ddot{q}_{w} + K_{176r}\ddot{q}_{2r} + K_{177r}\ddot{q}_{3r} + K_{178r}\ddot{q}_{4r} + K_{173r}\ddot{q}_{imu} + K_{174r}\ddot{q}_{torso} + \ddot{q}_{5r}c_{6r}) + \mathbf{MZ}_{10r}(K_{257r} + K_{234r}\ddot{\psi} + K_{238r}\ddot{q}_{1r} + K_{236r}\ddot{q}_{w} + K_{239r}\ddot{q}_{2r} + K_{240r}\ddot{q}_{3r} + K_{241r}\ddot{q}_{4r} + K_{242r}\ddot{q}_{5r} + K_{243r}\ddot{q}_{6r} + K_{235r}\ddot{q}_{imu} + K_{237r}\ddot{q}_{torso} + K_{233r}\ddot{x}) & K_{212r}(K_{212r}\mathbf{XY}_{10r} + K_{213r}\mathbf{YY}_{10r} + K_{214r}\mathbf{YZ}_{10r}) - K_{213r}(K_{212r}\mathbf{XX}_{10r} + K_{213r}\mathbf{XY}_{10r} + K_{214r}\mathbf{XZ}_{10r}) + \mathbf{XZ}_{10r}(K_{255r} + K_{215r}\ddot{\psi} + K_{218r}\ddot{q}_{1r} + K_{216r}\ddot{q}_{w} + K_{219r}\ddot{q}_{2r} + K_{220r}\ddot{q}_{3r} + K_{221r}\ddot{q}_{4r} + K_{222r}\ddot{q}_{5r} + K_{216r}\ddot{q}_{imu} + K_{217r}\ddot{q}_{torso} + \ddot{q}_{6r}c_{7r}) + \mathbf{YZ}_{10r}(K_{256r} + K_{223r}\ddot{\psi} + K_{226r}\ddot{q}_{1r} + K_{224r}\ddot{q}_{w} + K_{227r}\ddot{q}_{2r} + K_{228r}\ddot{q}_{3r} + K_{229r}\ddot{q}_{4r} + K_{230r}\ddot{q}_{5r} + K_{224r}\ddot{q}_{imu} + K_{225r}\ddot{q}_{torso} + \ddot{q}_{6r}s_{7r}) - \mathbf{ZZ}_{10r}(K_{206r} + \ddot{q}_{7r} + K_{172r}\ddot{\psi} + K_{175r}\ddot{q}_{1r} + K_{173r}\ddot{q}_{w} + K_{176r}\ddot{q}_{2r} + K_{177r}\ddot{q}_{3r} + K_{178r}\ddot{q}_{4r} + K_{173r}\ddot{q}_{imu} + K_{174r}\ddot{q}_{torso} + \ddot{q}_{5r}c_{6r}) + \mathbf{MX}_{10r}(K_{258r} + K_{245r}\ddot{\psi} + K_{249r}\ddot{q}_{1r} + K_{247r}\ddot{q}_{w} + K_{250r}\ddot{q}_{2r} + K_{251r}\ddot{q}_{3r} + K_{252r}\ddot{q}_{4r} + K_{253r}\ddot{q}_{5r} + K_{254r}\ddot{q}_{6r} + K_{246r}\ddot{q}_{imu} + K_{248r}\ddot{q}_{torso} + K_{244r}\ddot{x}) - \mathbf{MY}_{10r}(K_{257r} + K_{234r}\ddot{\psi} + K_{238r}\ddot{q}_{1r} + K_{236r}\ddot{q}_{w} + K_{239r}\ddot{q}_{2r} + K_{240r}\ddot{q}_{3r} + K_{241r}\ddot{q}_{4r} + K_{242r}\ddot{q}_{5r} + K_{243r}\ddot{q}_{6r} + K_{235r}\ddot{q}_{imu} + K_{237r}\ddot{q}_{torso} + K_{233r}\ddot{x}) &  \end{matrix}\right] 
 \nonumber \\ 
D_{364r} &= - K_{188r}\mathbf{MY}_{10r} - K_{244r}\mathbf{MZ}_{10r} \nonumber \\
D_{365r} &= K_{215r}\mathbf{XX}_{10r} + K_{223r}\mathbf{XY}_{10r} - K_{172r}\mathbf{XZ}_{10r}  \nonumber \\
&- K_{189r}\mathbf{MY}_{10r} - K_{245r}\mathbf{MZ}_{10r} \nonumber \\
D_{366r} &= K_{216r}\mathbf{XX}_{10r} + K_{224r}\mathbf{XY}_{10r} - K_{173r}\mathbf{XZ}_{10r}  \nonumber \\
&- K_{190r}\mathbf{MY}_{10r} - K_{246r}\mathbf{MZ}_{10r} \nonumber \\
D_{367r} &= K_{216r}\mathbf{XX}_{10r} + K_{224r}\mathbf{XY}_{10r} - K_{173r}\mathbf{XZ}_{10r}  \nonumber \\
&- K_{191r}\mathbf{MY}_{10r} - K_{247r}\mathbf{MZ}_{10r} \nonumber \\
D_{368r} &= K_{217r}\mathbf{XX}_{10r} + K_{225r}\mathbf{XY}_{10r} - K_{174r}\mathbf{XZ}_{10r}  \nonumber \\
&- K_{192r}\mathbf{MY}_{10r} - K_{248r}\mathbf{MZ}_{10r} \nonumber \\
D_{369r} &= K_{218r}\mathbf{XX}_{10r} + K_{226r}\mathbf{XY}_{10r} - K_{175r}\mathbf{XZ}_{10r}  \nonumber \\
&- K_{193r}\mathbf{MY}_{10r} - K_{249r}\mathbf{MZ}_{10r} \nonumber \\
D_{370r} &= K_{219r}\mathbf{XX}_{10r} + K_{227r}\mathbf{XY}_{10r} - K_{176r}\mathbf{XZ}_{10r}  \nonumber \\
&- K_{194r}\mathbf{MY}_{10r} - K_{250r}\mathbf{MZ}_{10r} \nonumber \\
D_{371r} &= K_{220r}\mathbf{XX}_{10r} + K_{228r}\mathbf{XY}_{10r} - K_{177r}\mathbf{XZ}_{10r}  \nonumber \\
&- K_{195r}\mathbf{MY}_{10r} - K_{251r}\mathbf{MZ}_{10r} \nonumber \\
D_{372r} &= K_{221r}\mathbf{XX}_{10r} + K_{229r}\mathbf{XY}_{10r} - K_{178r}\mathbf{XZ}_{10r}  \nonumber \\
&- K_{196r}\mathbf{MY}_{10r} - K_{252r}\mathbf{MZ}_{10r} \nonumber \\
D_{373r} &= K_{222r}\mathbf{XX}_{10r} + K_{230r}\mathbf{XY}_{10r} - \mathbf{XZ}_{10r}c_{6r}  \nonumber \\
&- K_{253r}\mathbf{MZ}_{10r} \nonumber \\
D_{374r} &= \mathbf{XX}_{10r}c_{7r} + \mathbf{XY}_{10r}s_{7r} - K_{254r}\mathbf{MZ}_{10r} \nonumber \\
D_{375r} &= K_{255r}\mathbf{XX}_{10r} + K_{256r}\mathbf{XY}_{10r} - K_{206r}\mathbf{XZ}_{10r}  \nonumber \\
&+ K_{213r}^2\mathbf{YZ}_{10r} - K_{214r}^2\mathbf{YZ}_{10r} + K_{259r}\mathbf{MY}_{10r}  \nonumber \\
&- K_{258r}\mathbf{MZ}_{10r} - K_{212r}K_{214r}\mathbf{XY}_{10r} + K_{212r}K_{213r}\mathbf{XZ}_{10r}  \nonumber \\
&- K_{213r}K_{214r}\mathbf{YY}_{10r} + K_{213r}K_{214r}\mathbf{ZZ}_{10r} \nonumber \\
D_{376r} &= K_{188r}\mathbf{MX}_{10r} + K_{233r}\mathbf{MZ}_{10r} \nonumber \\
D_{377r} &= K_{215r}\mathbf{XY}_{10r} + K_{223r}\mathbf{YY}_{10r} - K_{172r}\mathbf{YZ}_{10r}  \nonumber \\
&+ K_{189r}\mathbf{MX}_{10r} + K_{234r}\mathbf{MZ}_{10r} \nonumber \\
D_{378r} &= K_{216r}\mathbf{XY}_{10r} + K_{224r}\mathbf{YY}_{10r} - K_{173r}\mathbf{YZ}_{10r}  \nonumber \\
&+ K_{190r}\mathbf{MX}_{10r} + K_{235r}\mathbf{MZ}_{10r} \nonumber \\
D_{379r} &= K_{216r}\mathbf{XY}_{10r} + K_{224r}\mathbf{YY}_{10r} - K_{173r}\mathbf{YZ}_{10r}  \nonumber \\
&+ K_{191r}\mathbf{MX}_{10r} + K_{236r}\mathbf{MZ}_{10r} \nonumber \\
D_{380r} &= K_{217r}\mathbf{XY}_{10r} + K_{225r}\mathbf{YY}_{10r} - K_{174r}\mathbf{YZ}_{10r}  \nonumber \\
&+ K_{192r}\mathbf{MX}_{10r} + K_{237r}\mathbf{MZ}_{10r} \nonumber \\
D_{381r} &= K_{218r}\mathbf{XY}_{10r} + K_{226r}\mathbf{YY}_{10r} - K_{175r}\mathbf{YZ}_{10r}  \nonumber \\
&+ K_{193r}\mathbf{MX}_{10r} + K_{238r}\mathbf{MZ}_{10r} \nonumber \\
D_{382r} &= K_{219r}\mathbf{XY}_{10r} + K_{227r}\mathbf{YY}_{10r} - K_{176r}\mathbf{YZ}_{10r}  \nonumber \\
&+ K_{194r}\mathbf{MX}_{10r} + K_{239r}\mathbf{MZ}_{10r} \nonumber \\
D_{383r} &= K_{220r}\mathbf{XY}_{10r} + K_{228r}\mathbf{YY}_{10r} - K_{177r}\mathbf{YZ}_{10r}  \nonumber \\
&+ K_{195r}\mathbf{MX}_{10r} + K_{240r}\mathbf{MZ}_{10r} \nonumber \\
D_{384r} &= K_{221r}\mathbf{XY}_{10r} + K_{229r}\mathbf{YY}_{10r} - K_{178r}\mathbf{YZ}_{10r}  \nonumber \\
&+ K_{196r}\mathbf{MX}_{10r} + K_{241r}\mathbf{MZ}_{10r} \nonumber \\
D_{385r} &= K_{222r}\mathbf{XY}_{10r} + K_{230r}\mathbf{YY}_{10r} - \mathbf{YZ}_{10r}c_{6r}  \nonumber \\
&+ K_{242r}\mathbf{MZ}_{10r} \nonumber \\
D_{386r} &= \mathbf{XY}_{10r}c_{7r} + \mathbf{YY}_{10r}s_{7r} + K_{243r}\mathbf{MZ}_{10r} \nonumber \\
D_{387r} &= K_{255r}\mathbf{XY}_{10r} + K_{256r}\mathbf{YY}_{10r} - K_{206r}\mathbf{YZ}_{10r}  \nonumber \\
&- K_{212r}^2\mathbf{XZ}_{10r} + K_{214r}^2\mathbf{XZ}_{10r} - K_{259r}\mathbf{MX}_{10r}  \nonumber \\
&+ K_{257r}\mathbf{MZ}_{10r} + K_{212r}K_{214r}\mathbf{XX}_{10r} + K_{213r}K_{214r}\mathbf{XY}_{10r}  \nonumber \\
&- K_{212r}K_{213r}\mathbf{YZ}_{10r} - K_{212r}K_{214r}\mathbf{ZZ}_{10r} \nonumber \\
D_{388r} &= K_{244r}\mathbf{MX}_{10r} - K_{233r}\mathbf{MY}_{10r} \nonumber \\
D_{389r} &= K_{215r}\mathbf{XZ}_{10r} + K_{223r}\mathbf{YZ}_{10r} - K_{172r}\mathbf{ZZ}_{10r}  \nonumber \\
&+ K_{245r}\mathbf{MX}_{10r} - K_{234r}\mathbf{MY}_{10r} \nonumber \\
D_{390r} &= K_{216r}\mathbf{XZ}_{10r} + K_{224r}\mathbf{YZ}_{10r} - K_{173r}\mathbf{ZZ}_{10r}  \nonumber \\
&+ K_{246r}\mathbf{MX}_{10r} - K_{235r}\mathbf{MY}_{10r} \nonumber \\
D_{391r} &= K_{216r}\mathbf{XZ}_{10r} + K_{224r}\mathbf{YZ}_{10r} - K_{173r}\mathbf{ZZ}_{10r}  \nonumber \\
&+ K_{247r}\mathbf{MX}_{10r} - K_{236r}\mathbf{MY}_{10r} \nonumber \\
D_{392r} &= K_{217r}\mathbf{XZ}_{10r} + K_{225r}\mathbf{YZ}_{10r} - K_{174r}\mathbf{ZZ}_{10r}  \nonumber \\
&+ K_{248r}\mathbf{MX}_{10r} - K_{237r}\mathbf{MY}_{10r} \nonumber \\
D_{393r} &= K_{218r}\mathbf{XZ}_{10r} + K_{226r}\mathbf{YZ}_{10r} - K_{175r}\mathbf{ZZ}_{10r}  \nonumber \\
&+ K_{249r}\mathbf{MX}_{10r} - K_{238r}\mathbf{MY}_{10r} \nonumber \\
D_{394r} &= K_{219r}\mathbf{XZ}_{10r} + K_{227r}\mathbf{YZ}_{10r} - K_{176r}\mathbf{ZZ}_{10r}  \nonumber \\
&+ K_{250r}\mathbf{MX}_{10r} - K_{239r}\mathbf{MY}_{10r} \nonumber \\
D_{395r} &= K_{220r}\mathbf{XZ}_{10r} + K_{228r}\mathbf{YZ}_{10r} - K_{177r}\mathbf{ZZ}_{10r}  \nonumber \\
&+ K_{251r}\mathbf{MX}_{10r} - K_{240r}\mathbf{MY}_{10r} \nonumber \\
D_{396r} &= K_{221r}\mathbf{XZ}_{10r} + K_{229r}\mathbf{YZ}_{10r} - K_{178r}\mathbf{ZZ}_{10r}  \nonumber \\
&+ K_{252r}\mathbf{MX}_{10r} - K_{241r}\mathbf{MY}_{10r} \nonumber \\
D_{397r} &= K_{222r}\mathbf{XZ}_{10r} + K_{230r}\mathbf{YZ}_{10r} - \mathbf{ZZ}_{10r}c_{6r}  \nonumber \\
&+ K_{253r}\mathbf{MX}_{10r} - K_{242r}\mathbf{MY}_{10r} \nonumber \\
D_{398r} &= \mathbf{XZ}_{10r}c_{7r} + \mathbf{YZ}_{10r}s_{7r} + K_{254r}\mathbf{MX}_{10r}  \nonumber \\
&- K_{243r}\mathbf{MY}_{10r} \nonumber \\
D_{399r} &= K_{255r}\mathbf{XZ}_{10r} + K_{256r}\mathbf{YZ}_{10r} - K_{206r}\mathbf{ZZ}_{10r}  \nonumber \\
&+ K_{212r}^2\mathbf{XY}_{10r} - K_{213r}^2\mathbf{XY}_{10r} + K_{258r}\mathbf{MX}_{10r}  \nonumber \\
&- K_{257r}\mathbf{MY}_{10r} - K_{212r}K_{213r}\mathbf{XX}_{10r} - K_{213r}K_{214r}\mathbf{XZ}_{10r}  \nonumber \\
&+ K_{212r}K_{213r}\mathbf{YY}_{10r} + K_{212r}K_{214r}\mathbf{YZ}_{10r} \nonumber \\
 \dot{\bar{H}}_{10r} &= \left[\begin{matrix} D_{339r} + D_{329r}\ddot{\psi} + D_{333r}\ddot{q}_{1r} + D_{331r}\ddot{q}_{w} + D_{334r}\ddot{q}_{2r} + D_{335r}\ddot{q}_{3r} + D_{336r}\ddot{q}_{4r} + D_{337r}\ddot{q}_{5r} + D_{338r}\ddot{q}_{6r} + D_{330r}\ddot{q}_{imu} + D_{332r}\ddot{q}_{torso} + D_{328r}\ddot{x} + \mathbf{MY}_{10r}\ddot{q}_{7r} & D_{351r} + D_{341r}\ddot{\psi} + D_{345r}\ddot{q}_{1r} + D_{343r}\ddot{q}_{w} + D_{346r}\ddot{q}_{2r} + D_{347r}\ddot{q}_{3r} + D_{348r}\ddot{q}_{4r} + D_{349r}\ddot{q}_{5r} + D_{350r}\ddot{q}_{6r} + D_{342r}\ddot{q}_{imu} + D_{344r}\ddot{q}_{torso} + D_{340r}\ddot{x} - \mathbf{MX}_{10r}\ddot{q}_{7r} & D_{363r} + D_{353r}\ddot{\psi} + D_{357r}\ddot{q}_{1r} + D_{355r}\ddot{q}_{w} + D_{358r}\ddot{q}_{2r} + D_{359r}\ddot{q}_{3r} + D_{360r}\ddot{q}_{4r} + D_{361r}\ddot{q}_{5r} + D_{362r}\ddot{q}_{6r} + D_{354r}\ddot{q}_{imu} + D_{356r}\ddot{q}_{torso} + D_{352r}\ddot{x} &  \end{matrix}\right] 
 \nonumber \\ 
\end{align}


\bibliographystyle{plain}
\bibliography{reference}

\end{document}
