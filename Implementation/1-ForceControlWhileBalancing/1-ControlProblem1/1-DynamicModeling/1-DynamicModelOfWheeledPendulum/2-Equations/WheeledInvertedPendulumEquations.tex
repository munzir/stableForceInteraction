\documentclass[a4paper,10pt]{article}
\usepackage[utf8]{inputenc}
\usepackage{graphicx}
\usepackage[thinlines]{easytable}
\usepackage{enumitem}
\usepackage{amsmath}

%opening
\title{Equations Of Motion Of a Wheeled Inverted Pendulum}
\author{Munzir Zafar}

\begin{document}

\maketitle

In an earlier report \cite{munzir2013balancing} we had derived using Newton-Euler method the equations of
motion of a wheeled inverted pendulum. But this model did not take into account the spin motion of the
robot. Before attempting to derive the new equations including the spin, we will look at the equations
derived in the existing literature.

\section{Using the equation from \cite{kim2005dynamic}}

The equations derived in \cite{kim2005dynamic} for the motion of two-wheeled inverted pendulum robot are:
\begin{align}
 &3(m_c+m_s)\ddot{x}-m_sdcos\phi\ddot{\phi}+m_sdsin\phi({\dot{\phi}}^2+{\dot{\psi}}^2)=-\frac{\alpha_3+\beta_3}{R} \label{eq1}\\
 &\left\lbrace(3L^2+1/2R^2)m_c+m_sd^2sin^2\phi+I_2\right\rbrace\ddot{\psi}+m_sd^2sin\phi cos\phi\dot{\psi}\dot{\phi}=\frac{L}{R}(\alpha_3-\beta_3) \label{eq2}\\
 &m_sdcos\phi\ddot{x}+(-m_sd^2-I_3)\ddot{\phi}+m_sd^2sin\phi cos\phi {\dot{\phi}}^2+m_sgdsin\phi=\alpha_3+\beta_3 \label{eq3}
\end{align} where,

$\dot{x}$ is the heading speed of the robot 

$\phi$ is the rotation of the C.G. about $n_3$-directional 

$\psi$ is the heading angle (angle between $n_1$ and the world frame) 

$\alpha_3$ is the torque of the left wheel 

$\beta_3$ is the torque of the right wheel 

$m_c$ is the mass of the wheel 

$m_s$ is the mass of the body 

$d$ is the distance between wheel axis to C.G. 

$R$ is the radius of the wheel 

$L$ is the half distance between wheels 

$I_2$ is the $n_2$-directional rotational inertia of the body 

$I_3$ is the $n_3$-directional rotational inertia of the body 

$n_2$ is the unit vector pointing vertically upwards 

$n_3$ is the unit vector pointing from the left wheel to the right wheel \newline

The equations that we had derived in our earlier report \cite{munzir2013balancing} did not include
the rotation about the vertical axis of the robot. One of the equations above represent that motion.
We will try to first match the equations we had derived with the equations listed above as a sanity
check. Then we will write down the thrid equation using the variables that we had used in our 
earlier report. Then we will attempt to derive the equation using Newton-Euler method. Equations from
our earlier report are listed here:

\begin{align}
 &[(m+M)r+I_w/r+\eta^2I_m/r]\ddot{x}+(mrlcos\theta-\eta^2I_m)\ddot{\theta} = K_fu-\tau_f+F_{ext}rcos\theta+mrl\dot{\theta}^2sin\theta \label{eq4}\\
 &(mlcos\theta - \eta^2I_m/r)\ddot{x} + (ml^2 + I + \eta^2I_m)\ddot{\theta} = - K_fu + \tau_f + F_{ext}l + mglsin\theta \label{eq5}
\end{align} where,

$m$ is the mass of the body

$M$ is the mass of the wheels

$r$ is the radius of the wheel

$I_w$ is the inertia of the wheel

$\eta$ is the gear ratio of the motor

$I_m$ is the motor inertia

$\dot{x}$ is the heading speed

$l$ is the distance between wheel axis and the C.G.

$\theta$ is the rotation of the C.G. about the wheel axis

$K_f$ is the torque to current ratio of the motor

$\tau_f$ is the frictional torque on the wheels

$F_{ext}$ is the external force being applied at the C.G. perpendicular to $l$

$I$ is the inertia of the robot about the wheel axis \newline

Using the symbols used in equations \ref{eq4}-\ref{eq5}, we re-write the equations \ref{eq1}-\ref{eq3}:
\begin{align}
 &3(M+m)\ddot{x}-mlcos\theta\ddot{\theta}+mlsin\theta({\dot{\theta}}^2+{\dot{\psi}}^2)=-\frac{K_f(u_1+u_2)}{r} \label{eq6} \\
 &\left\lbrace(3L^2+1/2r^2)M+ml^2sin^2\theta+I_2\right\rbrace\ddot{\psi}+ml^2sin\theta cos\theta\dot{\psi}\dot{\theta}=\frac{L}{r}K_f(u_1-u_2) \label{eq7} \\
 &mlcos\theta\ddot{x}+(-ml^2-I)\ddot{\theta}+ml^2sin\theta cos\theta {\dot{\theta}}^2+mglsin\theta=K_f(u_1+u_2) \label{eq8}
\end{align} where, we have retained the symbols for quantities that do not appear in equations 
\ref{eq4}-\ref{eq5} i.e. $\psi$ and $I_2$ which represent the heading direction and the inertia 
about the vertical axis respectively.

In equations \ref{eq6}-\ref{eq8}, we make the following observations:
\begin{enumerate}[label=(\roman*)]
 \item Equations \ref{eq6} and \ref{eq8} are the equivalents of the equations \ref{eq4} and \ref{eq5} respectively
 \item The equations \ref{eq6} and \ref{eq8} ignore the effects of wheel inertia $I_w$, the motor inertia $I_m$, frictional
 torque $\tau_f$ at the wheel motor and the external force $F_{ext}$, all of which were considered in equations \ref{eq4} and \ref{eq5}
 \item The terms containing $\ddot{x}$ in equations \ref{eq4} and \ref{eq5} appear with opposite signs in equations \ref{eq6} and \ref{eq8} \label{item:ddotx}
 \item The term $mrlsin\theta{\dot\theta}^2$ of equation \ref{eq4} appears as $mrlsin\theta({\dot\theta}^2+{\dot\psi}^2)$ in equation \ref{eq6} \label{item:psi}
 \item Equation \ref{eq4} does not contain the coefficient $3$ with the $\ddot{x}$ term which appears in equation \ref{eq6} \label{item:coeff3}
 \item Equation \ref{eq5} does not contain the term $ml^2sin\theta cos\theta {\dot{\theta}}^2$ which appears in equation \ref{eq8} \label{item:corriolis}
\end{enumerate}

 Point number \ref{item:ddotx} may be explained by assuming that the two derivations assumed $x$ increases in different directions.
Point number \ref{item:psi} may be explained by the fact that equations \ref{eq4} and \ref{eq5} assume constant heading direction i.e. $\dot\psi=0$
But the last two points are not easy to explain. An interesting fact regarding point regarding \ref{item:coeff3} is that the paper \cite{kim2005dynamic}
changes the term from $3(m_c+m_s)$ in the original equation that we cited aboce to $3m_c+m_s$ in the later equations of the same paper. It appears that
the latter expression is more accurate and basically $3m_c$ represents the mass of three wheels of equal mass, one of which is a supporting wheel.
The paper discusses supporting wheels at length, so it won't be surprise. That solves the mystery of the second last point. What remains now is to
discuss the very last point. Since we see that there is a typo done in the earlier equation, we can expect this term to have a typo, in that it
is missing a $\dot\psi$ term. If this was true we will safely assume that the reason this term isn't present in our earlier analysis (i.e. equations 
\ref{eq4} and \ref{eq5}) is because we assumed constant heading direction i.e. $\dot{\psi}=0$.

\section{Using equations from \cite{li2012advanced}}
In the book \cite{li2012advanced} following equations of motion are derived:

\begin{align}
 &\left(M+2M_w+m+2\frac{I_w}{r^2}\right)\dot{v}+ml\ddot{\alpha}cos\alpha-ml{\dot{\alpha}}^2sin\alpha = \frac{\tau_l}{r}+\frac{\tau_r}{r}+d_l+d_r \label{eq9} \\
 &\left(I_p+2\left(M_w+\frac{I_w}{r^2}\right)d^2\right)\dot{\omega}=2d\left(\frac{\tau_l}{r}-\frac{\tau_r}{r}+d_l-d_r\right) \label{eq10} \\
 &ml\dot{v}cos\alpha+\left(ml^2+I_M\right)\ddot{\alpha}-mglsin\alpha=0 \label{eq11}
\end{align} where,

$M$ is the mass of the platform

$M_w$ is the mass of one wheel	

$m$ is the mass of the robot

$I_w$ is the inertia of one wheel

$r$ is the radius of one wheel

$v$ is the heading speed of the wheel

$l$ is the distance from C.G. to wheel axis

$\alpha$ is the rotation of C.G. about the wheel axis

$\tau_l$ is the torque applied by left wheel motor

$\tau_r$ is the torque applied by right wheem motor

$d_l$ is the external force acting on the left wheel

$d_r$ is the external force acting on the right wheel
\newline
Now, replacing these variables with the ones we had used in \cite{munzir2013balancing}, 
equations, \ref{eq9}-\ref{eq11} become:

\begin{align}
 &\left(M_p+M+m+\frac{I_w}{r^2}\right)\ddot{x}+ml\ddot{\theta}cos\theta-ml{\dot{\theta}}^2sin\theta = \frac{K_fu_1}{r}+\frac{K_fu_2}{r}+d_l+d_r \label{eq12} \\
 &\left(I_z+\left(M+\frac{I_w}{r^2}\right)L^2\right)\ddot{\psi}=2L\left(\frac{K_fu_1}{r}-\frac{K_fu_2}{r}+d_l-d_r\right) \label{eq13} \\
 &ml\ddot{x}cos\theta+\left(ml^2+I\right)\ddot{\theta}-mglsin\theta=0 \label{eq14}
\end{align}

Following observation are made:
\begin{enumerate}[label=(\roman*)]
 \item Equations \ref{eq12} and \ref{eq14} are the equivalents of equations \ref{eq4} and \ref{eq5} respectively
 \item Equation \ref{eq13} represents spin
 \item There is no difference between equation \ref{eq12} and \ref{eq4} except that $(a)$ equation \ref{eq4} considers
 motor inertia $I_m$ while \ref{eq12} does not and $(b)$ eq \ref{eq12} consider platform as different from the pendulum while 
 krang has no such thing as a platform so the additional mass term $M_p$ in equation \ref{eq12} is not there in \ref{eq4}
 \item There is no difference between equation \ref{eq12} and \ref{eq4} except that $(a)$ equation \ref{eq4} considers
 motor inertia $I_m$ while \ref{eq12} does not and $(b)$ the effect of counter torque on the pendulum is not considered
 in the eq \ref{eq12} as the pendulum does not experience the countertorque due to it being on a platform and not 
 directly attached to the motor so we don't see any term on the right hand side of eq \ref{eq12}
 \item Equation \ref{eq13} when compared with eq \ref{eq7} that represented spin in the previous section we see that
 the coefficient of $\ddot{\psi}$ was a funtion of $\theta$ there but here it is not. The reason is that $I_z$ term
 in eq \ref{eq13} is actually a function of $\theta$. That function is written over there but not here.
 \item Also there is a $\dot{\psi}$ term in equation \ref{eq7} that is not there in eq \ref{eq13}. It seems like equation \ref{eq7}
 makes more sense as a non-zero $\dot{\theta}$ will introduce a corriolis force in the system that is apparently not taken into
 account by equation \ref{eq13}
\end{enumerate}

\section{Comparing \cite{kim2005dynamic} and \cite{li2012advanced}}

It appears that the analysis done by \cite{kim2005dynamic} is more correct with regards to understanding of the dynamics,
only that it seems to have introduced some typos and thus can't be trusted blindly. The analysis in \cite{li2012advanced}
on the other hand is very cleanly explained and does not contain typos, it is weak in representation of all dynamic effects
in the system. The way we will move forward is by using the expressions for velocities that are more completely derived
in \cite{kim2005dynamic} and use the detailed procedures explained in \cite{li2012advanced} to come up with expressions
of dynamics that are useful for our purposes.


\section{Deriving the Dynamic Model for our robot}

In \cite{kim2005dynamic} the expressions we have for the kinematics of the system are as follows:

Angular velocity of the body and the velocity at the center of gravity in the body are governed as follows:
\begin{align} \label{eq:kinem1} \begin{split}
 {}^F\mathbf{\omega}^S &= u_2\mathbf{n}_2 + u_3\mathbf{n}_3 \\
 {}^F\mathbf{v}^{S^C} &= u_1\mathbf{n}_1  \\
 {}^F\mathbf{v}^{S^*} &= {}^F\mathbf{v}^{S^C} + {}^F\mathbf{\omega}^S\times \mathbf{d}\\
  &= (u_1-u_3dcos\phi)\mathbf{n}_1 - u_3dsin\phi\mathbf{n}_2 + u_2dsin\phi\mathbf{n}_3  
\end{split} \end{align} 

The angular velocities of each wheel and the velocity at the center of each wheel are governed as follows:
\begin{align} \label{eq:kinem2} \begin{split}
 &{}^F\mathbf{\omega}^{C_1} = \left(-\frac{1}{R}u_1+\frac{L}{R}u_2\right)\mathbf{n}_3+u_2\mathbf{n}_2 \\
 &{}^F\mathbf{v}^{C_1^*} = (u_1-u_2L)\mathbf{n}_1 
\end{split} \end{align}  
\begin{align} \label{eq:kinem3} \begin{split}
 &{}^F\mathbf{\omega}^{C_2} = \left(-\frac{1}{R}u_1-\frac{L}{R}u_2\right)\mathbf{n}_3+u_2\mathbf{n}_2 \\
 &{}^F\mathbf{v}^{C_2^*} = (u_1-u_2L)\mathbf{n}_1 
\end{split} \end{align} 

The angular accelerations for each left and right wheel are governed as follows:
\begin{align} \label{eq:kinem4} \begin{split}
 &{}^F\mathbf{\alpha}^S=\dot{u}_2\mathbf{n}_2+\dot{u}_3\mathbf{n}_3 \\
 &{}^F\mathbf{\alpha}^{C_1}=\left(-\frac{1}{R}u_1u_2+\frac{L}{R}u_2^2\right)\mathbf{n}_1+\dot{u}_2\mathbf{n}_2 + 
 \left(-\frac{1}{R}\dot{u}_1+\frac{L}{R}\dot{u}_2\right)\mathbf{n}_3 \\
 &{}^F\mathbf{\alpha}^{C_2}=\left(-\frac{1}{R}u_1u_2-\frac{L}{R}u_2^2\right)\mathbf{n}_1+\dot{u}_2\mathbf{n}_2 + 
 \left(-\frac{1}{R}\dot{u}_1-\frac{L}{R}\dot{u}_2\right)\mathbf{n}_3 
\end{split} \end{align} 

The acceleration of the robot's body and the acceleration at the center of each wheel are governed as follows:
\begin{align} \label{eq:kinem5} \begin{split}
 {}^F\mathbf{a}^{S^*} &= \frac{d{}^F\mathbf{v}^{S^C}}{dt}+{}^F\mathbf{\alpha}^S\times \overline{S^CS^*}+{}^F\mathbf{\omega}^S\times \left({}^F\mathbf{\omega}^S\times \overline{S^CS^*}\right) \\
 &=\begin{cases}
    \dot{u}_1-\dot{u}_3dcos\phi+\left(u_2^2+u_3^2\right)dsin\phi &\mathbf{n}_1 \\
    -\dot{u}_3dsin\phi-u_3^2dcos\phi &\mathbf{n}_2 \\
    +\dot{u}_2dsin\phi+u_2u_3dcos\phi &\mathbf{n}_3 
   \end{cases} \\
 {}^F\mathbf{a}^{C_1^*} &= \frac{d{}^F\mathbf{v}^{S^C}}{dt}+{}^F\mathbf{\alpha}^{S_C}\times \overline{S^CC_1^*}+{}^F\mathbf{\omega}^{S_C}\times \left({}^F\mathbf{\omega}^{S_C}\times \overline{S^CC_1^*}\right) \\
 &=\begin{cases}
    \dot{u}_1-L\dot{u}_2 &\mathbf{n}_1 \\
    +0 &\mathbf{n}_2 \\
    +Lu_2^2 &\mathbf{n}_3 
   \end{cases} \\
 {}^F\mathbf{a}^{C_2^*} &= \frac{d{}^F\mathbf{v}^{S^C}}{dt}+{}^F\mathbf{\alpha}^{S_C}\times \overline{S^CC_2^*}+{}^F\mathbf{\omega}^{S_C}\times \left({}^F\mathbf{\omega}^{S_C}\times \overline{S^CC_2^*}\right) \\
 &=\begin{cases}
    \dot{u}_1+L\dot{u}_2 &\mathbf{n}_1 \\
    +0 &\mathbf{n}_2 \\
    -Lu_2^2 &\mathbf{n}_3 
   \end{cases} 
\end{split} \end{align} where,
$F$, $S$, $C_1$ and $C_2$ denote world, main body, left wheel and right wheel frames
$S^C$ denotes the center of mass of the main body
$\mathbf{n}_1$, $\mathbf{n}_2$, $\mathbf{n}_3$ are unit vectors along forward, upward and left-to-right-wheel directions respectively
$u_1$, $u_2$, $u_3$ denote $x$ (forward motion), $\psi$ (spin motion) and $\phi$ (tilt of main body) 
respectively and they are positive along/about the unit vectors $\mathbf{n}_1$, $\mathbf{n}_2$, $\mathbf{n}_3$
$\mathbf{d}$ is the vector from wheel axis center to the center of mass of the main body
$L$ is the distance between the wheels
$R$ is the radius of the wheels
${}^F\mathbf{\omega}^S$
${}^F\mathbf{\omega}^{C_1}$
${}^F\mathbf{\omega}^{C_2}$
${}^F\mathbf{v}^{S^C}$
${}^F\mathbf{v}^{S^*}$
${}^F\mathbf{v}^{C_1^*}$
${}^F\mathbf{v}^{C_2^*}$
${}^F\mathbf{\alpha}^S$
${}^F\mathbf{\alpha}^{C_1}$
${}^F\mathbf{\alpha}^{C_2}$
${}^F\mathbf{a}^{S^*}$
${}^F\mathbf{a}^{C_1^*}$
${}^F\mathbf{a}^{C_2^*}$

\bibliographystyle{plain}
\bibliography{reference}

\end{document}


















