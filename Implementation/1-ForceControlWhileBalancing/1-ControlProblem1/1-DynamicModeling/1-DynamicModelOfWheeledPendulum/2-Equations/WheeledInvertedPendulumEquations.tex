\documentclass[a4paper,10pt]{article}
\usepackage[utf8]{inputenc}
\usepackage{graphicx}
\usepackage[thinlines]{easytable}
\usepackage{enumitem}
\usepackage{amsmath}

%opening
\title{Equations Of Motion Of a Wheeled Inverted Pendulum}
\author{Munzir Zafar}

\begin{document}

\maketitle

The equations derived in \cite{kim2005dynamic} for the motion of two-wheeled inverted pendulum robot are:
\begin{align}
 &3(m_c+m_s)\ddot{x}-m_sdcos\phi\ddot{\phi}+m_sdsin\phi({\dot{\phi}}^2+{\dot{\psi}}^2)=-\frac{\alpha_3+\beta_3}{R} \label{eq1}\\
 &\left\lbrace(3L^2+1/2R^2)m_c+m_sd^2sin^2\phi+I_2\right\rbrace\ddot{\psi}+m_sd^2sin\phi cos\phi\dot{\psi}\dot{\phi}=\frac{L}{R}(\alpha_3-\beta_3) \label{eq2}\\
 &m_sdcos\phi\ddot{x}+(-m_sd^2-I_3)\ddot{\phi}+m_sd^2sin\phi cos\phi {\dot{\phi}}^2+m_sgdsin\phi=\alpha_3+\beta_3 \label{eq3}
\end{align} where,

$\dot{x}$ is the heading speed of the robot 

$\phi$ is the rotation of the C.G. about $n_3$-directional 

$\psi$ is the heading angle (angle between $n_1$ and the world frame) 

$\alpha_3$ is the torque of the left wheel 

$\beta_3$ is the torque of the right wheel 

$m_c$ is the mass of the wheel 

$m_s$ is the mass of the body 

$d$ is the distance between wheel axis to C.G. 

$R$ is the radius of the wheel 

$L$ is the half distance between wheels 

$I_2$ is the $n_2$-directional rotational inertia of the body 

$I_3$ is the $n_3$-directional rotational inertia of the body 

$n_2$ is the unit vector pointing vertically upwards 

$n_3$ is the unit vector pointing from the left wheel to the right wheel \newline

The equations that we had derived in our earlier report \cite{munzir2013balancing} did not include
the rotation about the vertical axis of the robot. One of the equations above represent that motion.
We will try to first match the equations we had derived with the equations listed above as a sanity
check. Then we will write down the thrid equation using the variables that we had used in our 
earlier report. Then we will attempt to derive the equation using Newton-Euler method. Equations from
our earlier report are listed here:

\begin{align}
 &[(m+M)r+I_w/r+\eta^2I_m/r]\ddot{x}+(mrlcos\theta-\eta^2I_m)\ddot{\theta} = K_fu-\tau_f+F_{ext}rcos\theta+mrl\dot{\theta}^2sin\theta \label{eq4}\\
 &(mlcos\theta - \eta^2I_m/r)\ddot{x} + (ml^2 + I + \eta^2I_m)\ddot{\theta} = - K_fu + \tau_f + F_{ext}l + mglsin\theta \label{eq5}
\end{align} where,

$m$ is the mass of the body

$M$ is the mass of the wheels

$r$ is the radius of the wheel

$I_w$ is the inertia of the wheel

$\eta$ is the gear ratio of the motor

$I_m$ is the motor inertia

$\dot{x}$ is the heading speed

$l$ is the distance between wheel axis and the C.G.

$\theta$ is the rotation of the C.G. about the wheel axis

$K_f$ is the torque to current ratio of the motor

$\tau_f$ is the frictional torque on the wheels

$F_{ext}$ is the external force being applied at the C.G. perpendicular to $l$

$I$ is the inertia of the robot about the wheel axis \newline

Using the symbols used in equations \ref{eq4}-\ref{eq5}, we re-write the equations \ref{eq1}-\ref{eq3}:
\begin{align}
 &3(M+m)\ddot{x}-mlcos\theta\ddot{\theta}+mlsin\theta({\dot{\theta}}^2+{\dot{\psi}}^2)=-\frac{K_f(u_1+u_2)}{r} \label{eq6} \\
 &\left\lbrace(3L^2+1/2r^2)M+ml^2sin^2\theta+I_2\right\rbrace\ddot{\psi}+ml^2sin\theta cos\theta\dot{\psi}\dot{\theta}=\frac{L}{r}K_f(u_1-u_2) \label{eq7} \\
 &mlcos\theta\ddot{x}+(-ml^2-I)\ddot{\theta}+ml^2sin\theta cos\theta {\dot{\theta}}^2+mglsin\theta=K_f(u_1+u_2) \label{eq8}
\end{align} where, we have retained the symbols for quantities that do not appear in equations 
\ref{eq4}-\ref{eq5} i.e. $\psi$ and $I_2$ which represent the heading direction and the inertia 
about the vertical axis respectively.

In equations \ref{eq6}-\ref{eq8}, we make the following observations:
\begin{enumerate}[label=(\roman*)]
 \item Equations \ref{eq6} and \ref{eq8} are the equivalents of the equations \ref{eq4} and \ref{eq5} respectively
 \item The equations \ref{eq6} and \ref{eq8} ignore the effects of wheel inertia $I_w$, the motor inertia $I_m$, frictional
 torque $\tau_f$ at the wheel motor and the external force $F_{ext}$, all of which were considered in equations \ref{eq4} and \ref{eq5}
 \item The terms containing $\ddot{x}$ in equations \ref{eq4} and \ref{eq5} appear with opposite signs in equations \ref{eq6} and \ref{eq8} \label{item:ddotx}
 \item The term $mrlsin\theta{\dot\theta}^2$ of equation \ref{eq4} appears as $mrlsin\theta({\dot\theta}^2+{\dot\psi}^2)$ in equation \ref{eq6} \label{item:psi}
 \item Equation \ref{eq4} does not contain the coefficient $3$ with the $\ddot{x}$ term which appears in equation \ref{eq6} \label{item:coeff3}
 \item Equation \ref{eq5} does not contain the term $ml^2sin\theta cos\theta {\dot{\theta}}^2$ which appears in equation \ref{eq8} \label{item:corriolis}
\end{enumerate}

 Point number \ref{item:ddotx} may be explained by assuming that the two derivations assumed $x$ increases in different directions.
Point number \ref{item:psi} may be explained by the fact that equations \ref{eq4} and \ref{eq5} assume constant heading direction i.e. $\dot\psi=0$
But the last two points are not easy to explain. An interesting fact regarding point regarding \ref{item:coeff3} is that the paper \cite{kim2005dynamic}
changes the term from $3(m_c+m_s)$ in the original equation that we cited aboce to $3m_c+m_s$ in the later equations of the same paper. It appears that
the latter expression is more accurate and basically $3m_c$ represents the mass of three wheels of equal mass, one of which is a supporting wheel.
The paper discusses supporting wheels at length, so it won't be surprise. That solves the mystery of the second last point. What remains now is to
discuss the very last point. Since we see that there is a typo done in the earlier equation, we can expect this term to have a typo, in that it
is missing a $\dot\psi$ term. If this was true we will safely assume that the reason this term isn't present in our earlier analysis (i.e. equations 
\ref{eq4} and \ref{eq5}) is because we assumed constant heading direction i.e. $\dot{\psi}=0$.

\bibliographystyle{plain}
\bibliography{reference}

\end{document}
